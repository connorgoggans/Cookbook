\documentclass[letterpaper]{recipePMG}

%\usepackage{lmodern}
\usepackage{microtype}
\usepackage{float, array}
\usepackage{bookman}
\usepackage[T1]{fontenc}
\usepackage{nicefrac}
\usepackage {graphicx}
\usepackage{siunitx}
\usepackage{xfrac}
%\sisetup{quotient-mode = fraction}
\sisetup{parse-numbers = false}
\usepackage[margin=1.25in]{geometry} 
\usepackage[pdftex,breaklinks]{hyperref}
\hypersetup{colorlinks=true,linkcolor=blue}


% Include the following to enable nicer-looking tables
\usepackage{booktabs}
\usepackage{tabularx}
% Add some padding between caption and table toprule
\setlength{\abovetopsep}{2ex}
% Add some vertical padding between table elements
\renewcommand{\arraystretch}{1.2}


\newcommand{\bsi}[2]{%
  \fontencoding{T1}\fontfamily{cmr}\fontseries{m}\fontshape{n}%
  \fontsize{#1}{#2}\selectfont}

\renewcommand{\inghead}{\textbf{Ingredients}:\ }

\renewcommand{\equhead}{\textbf{Equipment}:\ }

\renewcommand{\rechead}{\centering\bsi{24.88pt}{30pt}}

\renewcommand{\deg}{$^\circ$}

\newcommand{\sixteenth}{\nicefrac{1}{16} \,}
\newcommand{\eighth}{\nicefrac{1}{8} \,}
\newcommand{\sixth}{\nicefrac{1}{6} \,}
\newcommand{\quarter}{\nicefrac{1}{4} \,}
\newcommand{\third}{\nicefrac{1}{3} \,}
\newcommand{\threeeights}{\nicefrac{3}{8} \,}
\newcommand{\half}{\nicefrac{1}{2} \,}
\newcommand{\twothirds}{\nicefrac{2}{3} \,}
\newcommand{\threequarters}{\nicefrac{3}{4} \,}
\newcommand{\threehalves}{\nicefrac{3}{2} \,}
\newcommand{\threesixteenths}{\nicefrac{3}{16} \,}
\newcommand{\fiveeights}{\nicefrac{7}{8} \,}
\newcommand{\seveneights}{\nicefrac{7}{8} \,}

\makeatletter
\renewcommand*\l@subsubsection{\@dottedtocline{3}{3em}{0em}}
\makeatother

\setlength\parindent{0pt}
\setlength\parskip{2ex plus 0.5ex}

\newcommand{\HRule}[1]{\rule{\linewidth}{#1}} 	% Horizontal rule

\makeatletter							% Title
\def\printtitle{%						
    {\centering \@title\par}}
\makeatother									

\makeatletter							% Author
\def\printauthor{%					
    {\centering \large \@author}}				
\makeatother	

\title{	\normalsize \textsc{Personal Cooking Notes} 	% Subtitle of the document
		 	\\[2.0cm]													% 2cm spacing
			\HRule{0.5pt} \\										% Upper rule
			\LARGE \textbf{\uppercase{Collected Recipes}}	% Title
			\HRule{2pt} \\ [0.5cm]								% Lower rule + 0.5cm spacing
			\normalsize \today									% Todays date
		}

\author{
		Paul M. Goggans\\	
        \texttt{paul.goggans@gmail.com} \\
}

\begin{document}
\frontmatter
\thispagestyle{empty}
\printtitle									% Print the title data as defined above
  	\vfill
\printauthor								% Print the author data as defined above

%
\chapter{Preface}
This is a book of recipes that I have collected from many sources and adapted to suit the equipment I own, the techniques I am comfortable with, the ingredients that are available where I live, and, most importantly, my own taste.  Some of these recipes are from cookbooks, some are from magazines and newspapers, some are from TV shows, and some are from family and friends. Many of the recently included recipes are from web sites such as The New York Times Cooking site. In many cases I don't know or have forgotten the origin of the recipes. Where I know the source of a recipe I have included it. A very few of these recipes are actually my creations, mostly, these are my recipes only in the sense that I like them and make them regularly enough so that I wanted to record them in a single place.  A few of the recipes are given exactly as I found them, but most of the recipes are changed in some way. These changes could be to the ingredients or proportions or cooking times or techniques used.  Most commonly I have changed the instructions to make them more specific and/or to make the result more pleasing to me. 

Mainly this is a place for me to record my cooking notes so they don't get lost. If you find it useful too, so much the better.

\tableofcontents\label{TOC}

\mainmatter
\chapter{Salads}
\recipe{Romaine, Orange, and Pecan Salad}
\label{RomaineOrangePecanSalad}


\ingredients{2 romaine lettuce hearts;
3 tablespoon olive oil;
\half cup whole pecans;
4 oranges peeled and cut into rings or 2 grapefruits cut into supremes;
1 small red onion sliced thin and separated into rings;
\half cup olive oil;
2 tablespoon red wine vinegar;
1 tablespoon lemon juice;
2 teaspoon Dijon mustard;
\half teaspoon sherry;
salt and pepper to taste.}



Prepare the lettuce by separating and washing the leaves.  Place the wet leaves on a clean towel and then roll up the towel and place it in the refrigerator to allow the leaves to become firm and crisp.  When the lettuce is crisp (after 30 minutes or so in the refrigerator), remove the large end of the center stems, if any, and then tear the leaves into bite size pieces.  Do not use the removed stems in the salad.

Saut\'{e} the pecans in olive oil until they are medium brown and then cool on absorbent paper.
  
Prepare the orange slices or grapefruit supremes. Remove the seeds from the orange slices if using. Peel the onion and slice into very thin rings.  Slicing the onion can be done with patience and a very sharp knife but is easier to accomplish using a mandolin. 

The remaining ingredients are mixed together to make the vinaigrette dressing. Fresh herbs such as dill, tarragon, basil, thyme and oregano can be added to the vinaigrette as desired.  If using, pick one of the above and add 1\half teaspoons to the dressing.  When the dressing is ready toss it with the prepared lettuce.

Arrange the salad on individual salad plates or in a single large bowl as desired by first placing the dressed romaine and then topping the romaine with the orange slices or grapefruit supremes, red onion rings, and saut\'{e}ed pecans.



Yields: Six servings.

\hyperref[TOC]{Table of Contents}
\newpage
\recipe{Curried Carrot Raisin Salad}
\label{CurriedCarrotRaisinSalad}
\ingredients{10 tablespoons \hyperref[MyMayonnaise]{homemade mayonnaise};
2 teaspoons \hyperref[CurryPowder]{homemade curry powder};
2 teaspoons kosher salt;
\quarter teaspoon cayenne;
juice of one lemon;
1 pound carrots, peeled and shredded;
2 cups good-quality raisins;
1 cup loosely packed fresh flat-leaf parsley leaves, thinly sliced.}

In a medium bowl, mix the mayonnaise with the curry powder, salt, cayenne, and lemon juice. 

Fold in the carrots, raisins, and parsley. 

Refrigerate for an hour up to 4 hours to allow the flavors to develop.

Source: ``Down South'' by Donald Link.

%\hyperref[TOC]{Table of Contents}
\newpage
\recipe{Brownwood-Style Coleslaw}
\label{BrownwoodColeslaw}

\ingredients{1 cup \hyperref[PoppySeedDressing]{homemade poppy seed dressing};
1 cup apple cider vinegar;
1 teaspoon salt;
1 teaspoon freshly ground black pepper;
1 large head purple or green cabbage, cored, halved, sliced \half to \threequarters inch thick, and chopped;
2 cups of whole seedless grapes (black for green cabbage, green for purple cabbage);
2 Granny Smith apples, peeled and chopped;
\threequarters cup toasted pecans;
6 green onions with both the white and green parts chopped.}

Make dressing by whisking the poppy seed dressing with the vinegar and then seasoning with salt and pepper. Cover and refrigerate for at least 1 hour.

To make slaw, toss the cabbage, grapes, apples, and pecans in a large bowl. Pour the \twothirds of the the dressing over the slaw and mix gently. Add more dressing if necessary to completely dress the slaw but do not over dress.  Add the green onions and mix again.

Serve immediately or cover and refrigerate.

Source: John Willingham from John Willingham's World Champion Bar-B-Q.

%\hyperref[TOC]{Corky's Coleslaw}
\newpage
\recipe{Corky's Coleslaw}
\label{CorkysColeslaw}

\ingredients{1 medium-size green cabbage;
2 medium-size carrots;
1 green bell pepper;
2 tablespoons grated onion;
2 cups \hyperref[MyMayonnaise]{homemade mayonnaise};
\threequarters cup granulated sugar;
\quarter cup Dijon mustard;
\quarter cup cider vinegar;
2 tablespoons celery seeds;
1 teaspoon salt;
\eighth teaspoon white pepper.}

Remove tough outer leaves from cabbage, then core and shred. Peel and grate carrots. Core, seed, and finely dice green pepper.

Place cabbage, carrots, green pepper and onion in a large bowl.

In another bowl, mix together all remaining ingredients. Pour over the vegetables and toss well to combine.

Cover the coleslaw and refrigerate for 3 to 4 hours to allow the flavors to meld before serving.

Serves 6.

Source: The Commercial Appeal Memphis.


\newpage
\recipe{Hattie B's Black-Eyed Pea Salad}
\label{BlackEyedPeaSalad}

\ingredients{\\
\\
\textsc{Pepper Vinaigrette --}
\quarter cup champagne vinegar;
\quarter cup malt vinegar;
\half cup extra virgin olive oil;
\half teaspoon minced fresh thyme;
\half teaspoon finely chopped fresh parsley;
1\half teaspoons black pepper, freshly ground;
\half teaspoon fine sea salt;
1 red bell pepper, diced small;
1 green bell pepper, diced small;
1 yellow bell pepper, diced small;
2 scallions, finely sliced;
\half teaspoon freshly roasted garlic, minced.\\
\\
\textsc{Black-Eyed Peas --}
2 cups dried black-eyed peas;
4 strips of bacon;
6 cups of chicken stock;
\half teaspoon fine sea salt.
}

Start by making the vinaigrette. Whisk together the vinegars, olive oil, thyme, parsley, black pepper, half the salt, all the peppers, scallions, and garlic.  Let sit in the refrigerator overnight.

In a large pot, add enough water to cover the peas by about 3-4 inches. Let sit overnight while the peppers are marinating. 

The next day, drain the peas in a colander and set aside.  In a large pot over medium-high heat, render the bacon, leaving it whole so it can be easily removed at the end of the recipe.  Once the bacon is rendered, add the chicken stock and bring to a simmer.

Once simmering, add the peas and cook on low for about 25 minutes. The peas should be tender but not mushy.  Strain the peas,  remove the strips of bacon, and place in a stainless steel or glass mixing bowl.

While the peas are hot, toss in the vinaigrette and add \half teaspoon of salt.  Taste and adjust seasonings.

Chill completely before serving.




%\hyperref[TOC]{Table of Contents}
\newpage
\recipe{Poppy-Seed Dressing}
\label{PoppySeedDressing}

\ingredients{1\half cups sugar;
    2 teaspoons dry mustard;
    2 teaspoons salt;
    \twothirds cup vinegar;
    3 tablespoons onion juice (see note after recipe);
    2 cups canola oil;
    3 tablespoons poppy seeds.}

Make the onion juice by grating a large white onion using the fine grating disk on a food processor. Use a fine mesh strainer over a bowl to strain the juice.  Press on the grated onion to extract the maximum amount of juice.   

In the bowl of a food processor fitted with a blade, mix sugar, mustard, salt, and vinegar. Add onion juice and pulse to stir it in thoroughly. With the machine running add oil slowly until all the oil is incorporated and a thick emulsion is formed.  Add poppy seeds and pulse to stir them in thoroughly. Store in a cool place, or in the refrigerator. Makes 3-1/2 cups.

The dressing is delicious on fruit salads of any kind. 

Source: Helen Corbitt 

%\hyperref[TOC]{Table of Contents}
\newpage
\recipe{Crawfish Salad}
\label{CrawfishSalad}


\ingredients{
\half cup \hyperref[MyMayonnaise]{homemade mayonnaise};
\quarter cup Creole mustard;
2 Tablespoons fresh lemon juice;
1 teaspoon minced parsley, plus more for garnish;
Hot sauce, preferably Tabasco, to taste;
Kosher salt and freshly ground black pepper, to taste;
1 pound cooked, peeled crawfish tails, roughly chopped;
1 cup frozen peas, defrosted;
6 scallions, minced;
2 \hyperref[SoftHardEggs]{hard-cooked eggs}, peeled and diced;
2 stalks celery, minced.}

Whisk mayonnaise, mustard, lemon juice, parsley, hot sauce, salt, and pepper in a bowl. Stir in crawfish, peas, scallions, eggs, and celery. 

To serve as salad, fill four large bowls with torn tender lettuce leaves of your choice then top with crawfish salad and garnish with parsley.  Can also be used in a bread roll instead of lobster salad.

Serves 4

From the Saveur website


\chapter{Vegetables}

\recipe{Greens}

\ingredients{3 bunches mustard greens;
4 bunches turnip greens;
2 cups water;
\half pound salt pork;
\half stick of butter or 2 tablespoons of rendered bacon fat;
salt and black pepper to taste.}

Cut the salt pork into batons.  Add pork to water in the stock pot set over high heat.  When water comes to a boil reduce to a simmer and cook for 30 minutes or so.

While the pork stock is simmering, prepare the greens for cooking by removing crushed or yellow leaves and cutting off excess stems with the roots.  Wash greens thoroughly in plenty of cold water. Lift leaves from container of water and let sand and grit settle to bottom. Wash through several changes of water until greens are completely clean.

Bring pork stock to a boil then add the greens and stir.  When the greens are wilted, reduce heat and simmer until tender. When tender check the amount of liquid. Pour off and save all but around a cup of liquid.  Add butter or bacon fat and stir.

%\hyperref[TOC]{Table of Contents}
\newpage
\recipe{Brooklyn-Style Collard Greens}

\label{BrooklynCollards}


\ingredients{1 smoked turkey wing or 2 medium ham hocks;
3 to 4 bunches collard greens (about four pounds)
salt;
\quarter cup apple cider vinegar;
3 garlic cloves, smashed;
\quarter teaspoon sugar;
1 medium onion, sliced;
\half teaspoon crushed red pepper, more to taste.}

Place turkey wing or hocks in a very large pot and add just 
enough water to cover. Bring to a boil, turn heat to medium 
and simmer until water is reduced by about half.

Meanwhile, plunge greens into a sink full of lightly salted cold water, 
drain and then rinse well with cold fresh water. Trim or remove biggest stems. 
Place five or six leaves on top of one another and roll like a cigar, lengthwise. 
Cut collards into inch-wide ribbons, then, keeping collards rolled,
cut ribbons in half.

Turn heat under pot to high. Add vinegar, one teaspoon salt and the garlic,
then add as many greens as pot will hold. Wait until greens cook down, 
then add remaining greens. Turn heat to a simmer, cover and cook until greens
 for 30 minutes, stirring occasionally.

Add sugar, onion and crushed red pepper, cover again and continue 
to simmer until tender, another 15 to 30 minutes; 
time will vary depending on toughness of greens. Serve.

Yield: 4 to 6 servings.

Source: The New York Times, December 30, 2009.

%\hyperref[TOC]{Table of Contents}
\newpage
\recipe{White Beans with Collards}

\ingredients{1 pound large navy beans (washed and picked over for stones);
2 quarts water;
1 large onion chopped;
1 small (grocery store) bunch of collard greens;
1 pound smoked ham (cubed);
3 cloves garlic (chopped);
1 tablespoon table salt;
1 tablespoon white pepper;
1 whole red pepper;
\half cup olive oil;
1 bay leave.}

Placed washed beans, water, and chopped onion in a large Dutch oven
or small stock pot (at least 6 qts.). Bring beans to boil and then boil for 1 hour.

Remove the stems from the greens using a sharp knife to cut the
green portion of the leaves from both sides of the stem.  Wash the green
portions of the leaves very well to remove all sand and dirt (this may
require moving the leaves back an forth between sinks of clean water).
Take two or three leaf halves and roll into a cigar shape (roll from front
to back of leave). Cut the roll in half along the axis and then perpendicular
to the axis in one inch intervals.

When beans have cooked for 1 hour, add greens to beans along with ham,
garlic, salt, white pepper, and whole red pepper. Stir well, adding olive oil
and bay leave.

Cover pot and let simmer for 30 minutes until both greens and beans are tender.

Source: Leah Chase from The Dooky Chase Cookbook

Notes: In place of the whole red pepper I used a chopped red bell pepper
and added a few teaspoons of very hot hot-sauce. I also added around
3 tablespoons of cider vinegar because I decided it was needed to balance
the richness of the olive oil.  I used smoked ham but I think the dish would be
even better with the ham replaced with smoked pork shoulder.

%\hyperref[TOC]{Table of Contents}
\newpage
\recipe{Ale-Braised Collard Greens With Smoked Ham Hock}

\ingredients{2 tablespoons extra-virgin olive oil;
1 medium yellow onion, diced;
2 garlic cloves, minced;
Kosher salt, as needed;
\half teaspoon red pepper flakes;
1 tablespoon dark brown sugar;
12 ounces American amber ale;
\half cup apple cider vinegar;
1 smoked ham hock;
3 bunches (about 3 pounds) collard greens, thoroughly washed, stems removed, cut into 2-inch pieces;
Black pepper, as needed}

Heat oil in a large stockpot over medium heat. Add onion and garlic and saut\'{e}, stirring occasionally, until onion is softened and just starting to color, 10 to 12 minutes.

Add 1 teaspoon salt, the red pepper flakes and the brown sugar; stir to combine. Add beer and cook, scraping up any browned bits from bottom of pan. Raise heat to high and bring to a boil, then reduce heat to low and simmer for 3 minutes.

Add 2 cups water, the apple cider vinegar, the ham hock and the collard greens; stir to combine. Cover pot, raise heat to high, and bring to a rolling boil. Stir collards thoroughly to incorporate flavors, then reduce heat to low and simmer, stirring every 30 minutes, until collards reach desired tenderness, at least 30 minutes but preferably up to 2 hours. Remove ham hock; pull off and chop meat and return to pan, or discard if desired. Season with salt and pepper.

Source: The New York Times.

\newpage

\recipe{Braised Collards With Peanut Butter and Hot Honey on Corn Bread}

\ingredients{
2 tablespoons neutral oil;
2  yellow onions, finely diced;
2 mild red chili peppers, sliced in rings;
2 garlic cloves, thinly sliced;
2 tablespoons tomato paste;
2 tablespoons apple  cider vinegar;
2 pounds (around 2 bunches) collard greens, thoroughly washed, stems removed, cut into 2-inch pieces;
3 cups water;
\half cup peanut butter;
1 recipe of \hyperref[cornbread]{corn bread} made in four small skillets; 2 tablespoons Frank's RedHot Sauce;
2 tablespoons honey; \half cup roasted peanuts; kosher salt;
black pepper, freshly ground.}

Heat the oil in a large, heavy pot over medium-high heat. Add the onions, season with salt, and cook, stirring occasionally, until the onions pick up some color, about 5 minutes. Reduce the heat and continue cooking until the onions are totally soft, about 20 minutes.

Add the sliced chili and cook for a couple of minutes to soften. Add the garlic and cook a few minutes more until it starts to soften. Stir in the tomato paste, coating the vegetables with it, and cook until it darkens a few shades, 3 to 4 minutes. Add the cider vinegar and use it to help scrape up browned bits from the bottom of the pot.

Add the collards, a handful at a time, wilting them between additions. Pour 2 cups of water into the pot, season with a pinch of salt, cover, and simmer the greens for 10 minutes.

Uncover the pot and stir. Add the peanut butter and swirl until it dissolves into the pot liquor. Add the remaining 1 cup of water to the pot, cover, and simmer the greens until they are tinder, about 30 minutes longer. Season with salt, if needed, and pepper. At this point the collards can be cooled and refrigerated up to 3 days.

When ready to serve, reheat the collards and make one recipe of \hyperref[cornbread]{corn bread} in four small cast-iron skillets. While the corn bread is in the oven, make the hot honey by stirring the hot sauce and honey in a small bowl until smooth. Season with a pinch each of salt and pepper then reserve for serving.

When the cornbread comes out of the oven it is time to assemble the dish. Place a cornbread round each in the bottom of four large shallow bowls. Spoon the collards and juices over the cornbread and drizzle with the hot honey. Sprinkle with peanuts and serve.

Source: Modified from a recipe in Lucky Peach Presents Power Vegetables!

\newpage

\recipe{Spaghetti Squash Marinara}

\ingredients{1 whole spaghetti squash, about 1\threequarters pound;
kosher salt;
freshly ground black pepper;
\half cup extra-virgin olive oil;
4 springs of fresh thyme;
1 cup Marinara sauce;
4 ounces fresh mozzarella cheese;
chile oil}

Preheat the oven to 450\deg F. Cut the squash in half lengthwise and scoop out seeds with a sturdy spoon.

Arrange the squash halves, cut side up, on a rimmed baking sheet. Season squash with salt and pepper, drizzle with olive oil, and top each half with two sprigs of thyme.  Roast the  squash halves until fork tender, about 30 minutes.

Remove squash from the oven and discard the thyme. Fill squash cavities nearly to the top with marinara.  Using your fingers, pull apart the mozzarella into chunks and scatter them on top of the marinara. Return the squash to the oven and roast until marinara starts to bubble and cheese in melted, about 5 more minutes. Remove from the oven and drizzle with chili oil.

Source: Andrew Ticer and Michael Hudman from Collards \& Carbonara

%\hyperref[TOC]{Table of Contents}
\newpage
\recipe{Squash Casserole}

\equilpment{Blu Skillet Ironware 13 inch carbon-steel French-skillet;  Black Le Creuset \#30 cast-iron baking-dish}

\ingredients{1\half pounds summer squash;
2 tablespoons butter;
1 small onion, finely diced;
3 tablespoons all-purpose flour;
1 cup milk;
\half cup reserved squash cooking liquid;
\half teaspoon salt;
\eighth teaspoon freshly ground black pepper;
1 roasted and peeled red bell-pepper, diced;
2 eggs, beaten;
1 cup (3 ounces) graded extra sharp cheddar cheese;
\half cup dry homemade bread crumbs;
\half cup (1\half ounces) graded extra sharp cheddar cheese.}

Preheat oven to 350\deg F.

Wash and slice squash then add to a small amount of boiling salted water in a pan. Cover pan an bring water back to a boil. When water returns to a boil, lower heat and continue cooking until squash in tender.  Drain squash reserving cooking liquid.

While squash is cooking, melt butter in a skillet. Add onions and cook over low heat until transparent and tender. Add flour and stir to make a smooth paste. Heat milk in microwave until just below boiling but do not boil. Gradually add milk to paste, stirring with each addition of milk to prevent lumping. Add \half cup of reserved squash cooking liquid and stir until smooth.

Add salt, pepper, diced red bell pepper, drained squash, and beaten eggs to the mixture. Stir until well blended. Stir in 1 cup graded cheddar cheese.

Pour mixture into a buttered 1\half quart baking-dish or casserole. Bake for 30 minutes.

Mix remaining \half cup of cheese with bread crumbs and sprinkle over top. Return to oven and bake for 15 minutes longer.


Source: Based on recipe in the 1980 version of The Auburn Cookbook. 

%\hyperref[TOC]{Table of Contents}
\newpage
\recipe{Brown Rice For Future Use}

\ingredients{2-pound package brown rice (907 grams, around 5 cups); 
8 cups water;
3 tablespoon unsalted butter;
1 tablespoon kosher salt}

Preheat the oven to 375\deg F.

Place the rice into a 4 quart 15x10x2 inch Pyrex glass baking dish.

Bring the water, butter, and salt just to a boil in a kettle or covered saucepan. Once the water boils, pour it over the rice, stir to combine, and cover the dish tightly with heavy-duty aluminum foil. Bake on the middle rack of the oven for 1 hour.

After 1 hour, remove cover and fluff the rice with a fork. 

Makes around 15 cups of cooked rice that can be conveniently portioned into 5 3-cup freezer containers and frozen for future use.

Source: Based on a recipe by Alton Brown.

\newpage
\recipe{Beets Baked in Foil}

\ingredients{8 medium beets of nearly equal size and shape, 1\half to 2 pounds.}

Preheat oven to 400\deg F.

Trim greens and tap root from beets then wash well. Wrap beets individually in foil and put on a  half sheet pan. Beets should be tightly wrapped. 

Bake, undisturbed for 30 to 45 minutes. When done a thin bladed knife should slip easily through a test beet.

Let beets cool to room temperature, remove foil, and then remove peels from beets with a paper towel. To avoid staining hands wear kitchen gloves. To avoid staining cutting board, peel beets on the sheet pan.

To serve hot, slice beets and reheat in butter. Finish with a squeeze of lemon juice.

Source: Based on a recipe in  ``How to Cook Everything'' by Mark Bittman.

\newpage
\recipe{Roasted Broccoli}

\ingredients{1\half pounds broccoli crowns, around 2 large crowns;
3 tablespoons extra-virgin olive oil;
Kosher salt;
freshly ground black pepper.}

Position oven rack in center of oven and heat to 450\deg F. Line a half sheet pan with heavy-duty aluminium foil.

Prepare the broccoli by first removing any leaves from the crowns. Then cut away the individual florets so that the stem ends is no more that \half inch thick. Cut florets that are too large in half. Trim fibrous outer layer from stems and then cut in \half inch coins.

Place broccoli on sheet pan, drizzle with oil, season with salt and pepper, and toss to coat. Spread into an even layer placing cut side of any split florets down.  Roast for 15 minutes then flip pieces and roast for an additional 5 minutes.

Excellent served cold or room temperature with vinaigrette added just before serving.  

Source: Based on a recipe in ``All About Roasting'' by Molly Stevens.


\newpage
\recipe{French-Fried Potatoes}
\label{FrenchFriedPotatoes}
\equilpment{R \& V Works Cajun Fryer Model FF1.}
\ingredients{8 ounces Russet Potatoes; peanut oil to fill fryer; salt.}

Peel the potatoes and cut each into \quarter-inch thick slices. Stack a few slices at a time and cut the slices lengthwise into \quarter-inch-wide strips or sticks.  Drop the potatoes into cold water and drain almost immediately. Pat the potatoes completely dry.

Add potatoes to oil in fryer heated to 325\deg F.  Blanch them for 2 minutes.

Drain them well and transfer potatoes to a pan lined with absorbent paper.

Finish the potatoes in 375\deg F oil until they are golden brown and cooked through. Drain them well.

Salt to taste away from the fryer and serve immediately. 

Yields: Two servings

Source: Adapted from recipes in `` The New Professional Chef'' by The Culinary Institute of America and in the New York Times by Pierre Franey.
\newpage
\recipe{Spinach Cheddar Quiche}
\label{SpinichCheddarQuiche}

\ingredients{One recipe \hyperref[FlakyPieCrust]{Flaky Pie Crust}; 20 ounces frozen chopped spinach, thawed; 6 eggs; 6 cups shredded aged sharp cheddar cheese; 2 cups whole milk.}

Preheat oven to 350\deg F and Meanwhile, make one Flaky Pie Crust recipe.

To make the filling, squeeze as much excess water from the spinach as humanly possible (use dish towels) and then combine all remaining ingredients in a large bowl.


Roll out crusts for two pies. Divide the spinach and cheddar mixture between each crust and bake for 60 minutes.

\newpage
\recipe{Zucchini Pancakes (Mucver)}

\ingredients{3 medium zucchini (about 1 pound), shredded;
Salt;
freshly ground black pepper;
3 large eggs, beaten;
\half cup all-purpose flour;
1 tablespoon extra virgin olive oil;
1 cup crumbled feta cheese;
3 scallions, finely chopped;
\third cup finely chopped dill;
1 teaspoon baking powder;
4 to 6 tablespoons vegetable oil, more as needed;
\twothirds cup plain yogurt;
2 cloves garlic, finely chopped;
\half teaspoon salt.}

Preheat oven to 250\deg F. Place zucchini in a colander over a bowl, and mix with \half teaspoon salt. Allow to drain for five minutes. Transfer to a cloth kitchen towel, and squeeze hard to extract as much moisture as possible. Squeeze a second time; volume will shrink to about half the original.

In a large mixing bowl, combine zucchini and eggs. Using a fork, mix well. Add flour, \half teaspoon salt, olive oil, feta, scallions, dill and \half teaspoon black pepper. Mix well, add baking powder, and mix again.

Place a cast iron skillet or other heavy skillet over medium heat. Add 2 tablespoons vegetable oil and heat until shimmering. Place heaping tablespoons of zucchini batter in pan several inches apart, allowing room to spread. Flatten them with a spatula if necessary; pancakes should be about \threeeights inch thick and about 3 inches in diameter. Fry until golden on one side, then turn and fry again until golden on other side. Repeat once or twice, frying about 5 to 6 minutes total, so pancakes get quite crisp. Transfer to a plate lined with paper towels, and keep warm in oven. Continue frying remaining batter, adding more oil to pan as needed. Serve hot.

For yogurt sauce: In a small bowl, combine yogurt, garlic and salt. Mix well, and serve on the side or on pancakes.

Source: Aytekin Yar from NYT Cooking

\newpage

\recipe{French Lentils With Garlic and Thyme}

\ingredients{3 tablespoons olive oil;
    1 onion, peeled and finely chopped;
    2 cloves garlic, peeled and finely chopped;
    1 carrot, peeled and finely chopped; 
    2\quarter cups Le Puy lentils;
    1 teaspoon dried or fresh thyme;
    3 bay leaves;
    1 tablespoon kosher salt.}

Place a large saucepan over medium heat and add oil. When hot, add chopped vegetables and saut\'{e} until softened, 5 to 10 minutes.


Add 6 cups water, lentils, thyme, bay leaves and salt. Bring to a boil, then reduce to a fast simmer.

Simmer lentils until they are tender and have absorbed most of the water, 20 to 25 minutes. If necessary, drain any excess water after lentils have cooked. Serve immediately, or allow them to cool and reheat later. Excellent with a large dollop of plain yogurt or sour cream.  

Source: Nigella Lawson from NYT Cooking

\chapter{Soups, Stews and Gumbos}

\recipe{Beef, Bean, and Barley Stew}

\ingredients{ 1 tablespoon extra-virgin olive oil;
2 pounds beef chuck roast, cut into \half-inch cubes;
3 teaspoons Kosher salt;
\quarter teaspoon fresh ground black pepper;
1 medium onion, quartered and sliced;
1 stalk celery, diced;
4 cups water;
14.5-ounce can diced tomatoes;
\half cup pearled barley, uncooked;
15-ounce can cooked kidney beans, drained and rinsed;
2 bunches collards; cleaned and cut into \half-inch squares or 24-ounce bag frozen kale;
2 teaspoons finely chopped fresh rosemary;
1 teaspoon dried thyme.}

Preheat oven to 325\deg F.

Cut roast into cubes. If using fresh collards then clean, remove stems and chop leaves (see \hyperref[BrooklynCollards]{Brooklyn Style Collard Greens}.)

Toss beef cubes with salt and black pepper. Heat olive oil in cast iron gumbo pot until smoking.  Add half the beef cubes and brown on all sides. Remove browned beef to a plate.  Add additional oil and bring pot back to high heat and brown second half of beef cubes.  

Return beef cubes from plate to pot. 
Add onion, celery, water, tomatoes, barley, kidney beans, collards or kale, rosemary, and thyme.

Bring pot to boil then cover pot and place in oven.  Bake for 1 hour.


\newpage
 
\recipe{Coconut-Curry Chickpea Pumpkin Stew}

\ingredients{3 tablespoons neutral oil, such as sunflower or canola;
1 large onion, chopped;
2 jalape\~{n}os, seeded or not, thinly sliced;
1 bay leaf;
1 knob ginger (about 1 inch), minced;
4 garlic cloves, minced;
1 \half teaspoons garam masala;
1 teaspoon ground cumin;
\half teaspoon ground turmeric;
2 (15-ounce) cans chickpeas, rinsed;
1 (13.5-ounce) can full-fat coconut milk;
1 (13.5-ounce) can pumpkin pur\'{e}e;
1\half teaspoons fine sea salt, more as needed;
\threequarters cup chopped cilantro, more for serving;
2 to 3 tablespoons fresh lime juice, plus wedges for serving;
Cooked rice or couscous, for serving}

Heat oil in a large skillet over medium-high heat. Stir in onion, jalape\~{n}os and bay leaf. Cook, stirring occasionally, until onion is golden on the edges, about 8 minutes.

Add ginger and garlic and cook until fragrant, about 2 minutes, stirring frequently. Stir in garam masala, cumin and turmeric; cook for an additional 30 seconds.

Stir in chickpeas, coconut milk, pumpkin, 1 \half cup water and 1 \half teaspoons salt. Bring to a simmer and continue to simmer for 10 minutes, stirring occasionally, to let the flavors meld. (Add more water if it starts to look too thick.) Stir in cilantro and lime juice to taste. Taste and add more salt if necessary.

Serve over rice or couscous, and top with more cilantro and lime wedges on the side.

Serves 6 and takes around an hour to make start to finish.

Source: Melissa Clark from the NYT

\newpage

\recipe{Red Lentil Soup with Lemon}

\ingredients{3 tablespoons olive oil, more for drizzling;
1 large onion, chopped;
2 garlic cloves, minced;
1 tablespoon tomato paste;
1 teaspoon ground cumin;
\quarter teaspoon kosher salt, more to taste;
\quarter teaspoon ground black pepper;
Pinch of ground chile powder or cayenne, more to taste;
1 quart chicken or vegetable broth;
1 cup red lentils;
1 large carrot, peeled and diced;
Juice of \half lemon, more to taste,
3 tablespoons chopped fresh cilantro.} 

In a large pot, heat 3 tablespoons oil over high heat until hot and shimmering. Add onion and garlic, and saut\'{e} until golden, about 4 minutes.

Stir in tomato paste, cumin, salt, black pepper and chili powder or cayenne, and saut\'{e} for 2 minutes longer.

Add broth, lentils and carrot. Bring to a simmer, then partially cover pot and turn heat to medium-low. Simmer until lentils are soft, about 30 minutes. Taste and add salt if necessary.

Using an immersion or regular blender or a food processor, pur\'{e}e half the soup then add it back to pot. Soup should be somewhat chunky.

Reheat soup if necessary, then stir in lemon juice and cilantro. Serve soup drizzled with good olive oil and dusted lightly with chili powder if desired.

Source: Melissa Clark from the NYT

%\hyperref[TOC]{Table of Contents}

\newpage
\recipe{Butter Bean and Ham Soup}
\label{ButterBeanAndHamSoup}
\equilpment{10-quart stock pot; 8-quart pot or food-container;  immersion blender.}

\ingredients{1 ham bone from Honey Baked Ham; 4 quarts water; 2 pounds Camellia Brand dried large limas; 2 large yellow or white onions, chopped; 4 cloves garlic, finely chopped; 2 bay leaves; 1 tablespoon \hyperref[CreoleSeasoning]{Creole seasoning}; olive oil; salt and black pepper to taste.
}


\textsc{This recipe works best if the ham stock is made the day before the soup and allowed to cool overnight in the refrigerator.}

Cut most of the ham off of the ham bone and reserve.  Place the ham bone in the stock pot and add the water.  Bring to a boil and then simmer covered for 1 to 2 hours until the bone separates into two pieces and the attached ham falls off the bone. Remove pot from heat and fish out the bones.  Place the pot in the refrigerator overnight to cool.

In the morning skim off the congealed fat from the top of the ham stock. Strain the stock into  the food-container. Dump out the strainer onto a large cutting board then pick out the ham from the fat and gristle. Shred the ham and add it to the food-container.  

Add olive oil to cleaned stock pot and place over medium heat.  When oil is hot add onions and garlic with a good pinch of salt. Cook stirring occasionally until the onions are lightly browned. While the onions are cooking dice the reserved ham into \quarter inch cubes.  Add olive oil to cover the bottom of a large skillet and use to brown the ham cubes. Add the ham stock and ham cubes to the onions. Sort and rinse the beans and add to the stock along with the bay leaves and Creole seasoning. Bring pot to a boil for 10 minutes. Reduce heat, cover and simmer, stirring occasionally for about 1\half hours until the beans are tender. 

With a ladle, remove some beans and liquid and place in the container that came with the immersion blender.  Pur\'{e}e with the immersion blender until smooth and then add back to pot. Stir to combine and evaluate the thickness.  Continue to pur\'{e}e beans until the desired thickness is achieved. Season the soup to taste with salt and black pepper.

\newpage

\recipe{Vegetable Broth Base}
\label{VegetableBroth}

\ingredients{2 leeks, white and light green parts only, chopped and washed thoroughly (2\half cups or 5 ounces);
2 carrots, peeled and cut into \half-inch pieces (\twothirds cup or 3 ounces);
\half small celery root, peeled and cut into \half-inch pieces (\threequarters cup or 3 ounces);
\half cup (\half ounce) parsley leaves and thin stems;
3 tablespoons dried minced onions;
2 tablespoons Diamond Crystal kosher salt;
1\half tablespoons tomato paste;
3 tablespoons soy sauce.}

For the best balance of flavors, measure the prepped vegetables by weight. Kosher salt aids in grinding the vegetables. The broth base contains enough salt to keep it from freezing solid, making it easy to remove 1 tablespoon at a time. To make 1 cup of broth, stir 1 tablespoon of fresh or frozen broth base into 1 cup of boiling water. If particle-free broth is desired, let the broth steep for 5 minutes and then strain it through a fine-mesh strainer.

Process leeks, carrots, celery root, parsley, minced onions, and salt in food processor, scraping down sides of bowl frequently, until paste is as fine as possible, 3 to 4 minutes. Add tomato paste and process for 1 minute, scraping down sides of bowl every 20 seconds. Add soy sauce and continue to process 1 minute longer. Transfer mixture to airtight container and tap firmly on counter to remove air bubbles. Press small piece of parchment paper flush against surface of mixture and cover. Freeze for up to 6 months.

To make broth for use, mix 1 tablespoon of base for each cup of boiling water. For example, to make a quart of stock mix 4 tablespoons of base with 1 quart of boiling water.  To make 7 quarts of broth, mix an entire recipe of base with 7 quarts of boiling water.

Source: America's Test Kitchen


%\hyperref[TOC]{Table of Contents}
\newpage
\recipe{Buttermilk Chicken Stew}
\ingredients{1 fryer, cut into pieces or 6 boneless thighs and 3 boneless half breast;
1 tablespoon olive oil;
2 tablespoons butter;
1 medium onion, chopped fine;
1 pound can whole tomatoes, drained;
1\quarter cups buttermilk;
\quarter teaspoon sugar;
\half teaspoon salt;
\eighth teaspoon black pepper;
dash of pepper sauce;
\half cup finely chopped green onions;
\quarter cup flat leaf parsley, chopped;
1\half teaspoons dried dill weed;
\quarter teaspoon lemon juice;
1 cup sour cream.}

Add butter and oil to a large cast iron pot over medium heat.  When hot, add chicken pieces an cook until brown.  

When chicken is browned all over, add onion and saut\'{e} over low heat until soft, about 5 minutes.

Add tomatoes, buttermilk, sugar, salt, pepper, and pepper sauce and simmer mixture for about 25 minutes.

After 25 minutes, add the green onions, parsley, and dill then cook, uncovered for 5 minutes.  Add lemon juice and sour cream and continue to cook until just heated.

Serve with French bread and a glass of dry rose.  Serves 4.

Source: Leon E. Soniat, Jr. from La Bouche Creole.

%\hyperref[TOC]{Table of Contents}
\newpage
\recipe{Smoked Pulled Pork, Collard Greens, and Blackeye Peas Gumbo}

\ingredients{\\
\\
\textsc{Roux --}
\twothirds cup vegetable oil or bacon fat;
\threequarters cup flour;
2 medium onions, diced;
1 medium green bell pepper, diced;
1 rib celery, diced;
3 cloves garlic, minced;
8 cups chicken stock, 1 cup reserved.\\
\\
\textsc{Gumbo --}
2 bay leaves;
3 slices hardwood smoked bacon;
1 bunch collard greens, washed and sliced into ribbons;
1 white onion, chopped;
1 teaspoon white vinegar;
2 teaspoons sugar;
2 teaspoons Louisiana-style hot sauce;
2 teaspoons salt;
1 teaspoon freshly ground black pepper;
2 cups cooked blackeye peas prepared from fresh, frozen or dryed blackeye peas;
1 pounds smoked, pulled pork (not with barbecue sauce), shredded;
1 tablespoon chopped fresh thyme;
1 tablespoon \hyperref[CreoleSeasoning]{Creole seasoning}.}

Cook blackeyed peas and then hold drained peas in the refrigerator. Dice vegetables and mince garlic for the roux.

In an 8 cup glass measuring cup combine the flour and the oil or bacon fat.  Make a very \hyperref[BrownRoux]{dark roux} using the microwave oven. When roux is a deep chocolate color, add the celery, onions, bell pepper, and garlic to the measuring cup and and saut\'{e} in the roux until tender, around 5 minutes. The roux will continue to darken, and the vegetables will caramelize.
 

Transfer roux to a large cast-iron Dutch oven. Add 7 cups of stock and bay leaves to the roux mixture. Simmer for an hour, stirring often. Skim any fat off the top. The roux mixture will seem very thin at this point but this is necessary to account for the  large amount of ingredients added in the next steps.

While stock is simmering, fry bacon in a tall sided skillet or shallow Dutch oven until crisp. Remove bacon and drain on paper towels, leaving bacon drippings in skillet or oven.

Wash collard greens in several changes of water to remove all grit than remove stems and slice leaves into ribbons.  Chop remaining onion then add collard greens and chopped onion to bacon drippings in skillet, and saut\'{e} until wilted. Crumble bacon and combine it with cooked greens/onion mixture.

After the roux mixture has simmered for an hour, to Dutch oven, add vinegar, sugar, hot sauce, and salt and pepper to taste. Then add greens mixture, blackeye peas, pork, thyme, and Creole seasoning. If the gumbo is too thick then add some of the reserved stock.

Return to a simmer and adjust seasonings to taste.

Serve over hot cooked rice with French bread.

Yield:  10 servings.

Source: Adapted from recipe found on Camellia brand beans website.
%(http://www.camelliabrand.com)

%\hyperref[TOC]{Table of Contents}
\newpage
\recipe{Duck and Andouille Gumbo}

\ingredients{2 frozen ducks, thawed;
2 pounds andouille sausage cut into 1/4 inch rounds;
2 medium onions, finely chopped;
1 bell pepper, finely chopped;
4 stalks celery, finely chopped;
1 cup canola oil;
1 cup flour;
4 cloves garlic, finely chopped;
3 bay leaves;
\quarter teaspoon ground allspice;
\quarter teaspoon cayenne pepper;
1 teaspoon basil;
\half teaspoon powdered cloves;
\half teaspoon poultry seasoning;
1 tablespoon salt;
1 teaspoon ground black pepper;
4 tablespoons Worcestershire sauce;
\half teaspoon Tabasco sauce;
4 (10\half ounce) cans of Campbells beef consomm\'e;
3 consomm\'e cans of water.} 


Prepare duck by pricking skin all over and then cooking in a glass tabletop convection oven.  Brown ducks one at a time. Start with the oven at 300 \deg F for 30 minutes to render the fat under the skin and then increase to 450 \deg F and cook until browned. Let both ducks cool and then remove meat from the bones and reserve.
Be sure to save the rendered duck fat for another use or, if you live dangerously, to make the roux. This step can be done well in advance.

Prepare the sausage, onions, bell pepper, celery, and garlic.

Combine the consomm\'e and water in a stockpot and bring to boil. Reduce heat and let simmer.

In a heavy skillet brown the sausage then drain on paper towels.

In an 8 cup glass measuring cup combine the flour and the oil and/or duck fat.  Make a very \hyperref[BrownRoux]{dark roux} using the microwave oven. When roux is ready cook the celery, onions, and pepper in the roux until it is tender. Transfer roux to a large cast-iron Dutch oven. Very carefully add hot stock to the roux stirring vigorously until roux is smooth and thin.

Add duck, sausage, roux and all remaining ingredients except for the pepper sauce and green onions. Partially cover and continue simmering until duck is very tender around 2\half hours.

Remove from heat and add pepper sauce.  Taste for seasoning and add salt if needed.  Stir in green onions and serve over rice.

Source: Leon E. Soniat, Jr. from La Bouche Creole.

%\hyperref[TOC]{Table of Contents}
\newpage
\recipe{Leftover-Turkey Gumbo}
\label{TurkeyGumbo}
\ingredients{Leftover turkey carcass with enough leg and thigh meat to make 3 to 4 cups cubed -- a smoked turkey is best.\\
\\
\textsc{Turkey Bone Broth --}
3 ribs celery, roughly chopped;
2 medium onions, roughly chopped; 
4 garlic cloves, smashed; 
1 tablespoon whole black peppercorns;
1 tablespoon thyme leaves;
4 bay leaves;
1 gallon water.\\
\\
\textsc{Gumbo --}
\threequarters cup vegetable oil;
\threequarters cup all-purpose flour;
2 cups chopped onions
\half cup chopped red bell peppers;
\half cup chopped celery;
1 tablespoon minced garlic;
1 tablespoon fresh thyme leaves, chopped fine;
3 quarts turkey bone broth;
3 tablespoons or so Kitchen Bouquet, optional;
4 cups turkey meat, torn into bite-sized pieces from the turkey carcass and/or leftovers;
\half pound andouille, cut into \quarter inch rounds then quartered;
salt; freshly ground pepper; ground cayenne; hot sauce; flat-leaf parsley, chopped;
green onions, thinly sliced.
}

Remove as much meat as possible from the turkey carcass; tear into bite-sized pieces. Refrigerate the meat until you are ready to make the gumbo.

To make the turkey bone broth, break up carcass and place pieces on a foil-lined half sheet-pan. Roast at 425 \deg F until bones are browned.

Place the browned carcass pieces in a large stockpot. Scrape all the browned bits and fond from the sheet-pan and add to pot. Add the celery, onions, garlic, thyme, peppercorns and bay leaves. Add one gallon of water.  Add more water if necessary to cover carcass pieces. 

Bring to a boil, then reduce the heat so that the broth is at a bare simmer. Cook for at least 2 hours and up to overnight. If simmering for an extended time, partially or fully cover the pot. 

Strain the broth through a large colander and discard the solids. Refrigerate until the fat has solidified and can be easily removed. Should yield 3 quarts of stock.  The gumbo uses 1\half to 2 quarts of stock. The extra stock can be frozen and used for another purpose (\hyperref[TurkeyGravy]{turkey gravy}).

To make the gumbo, chop and have ready the onions, peppers, celery and garlic. While preparing the vegetables, heat 2 quarts of bone broth in a large saucepan. 

In a heavy skillet brown the sausage then drain on paper towels. When the sausage has cooled, cut the rounds into quarters.

In an 8 cup glass measuring cup combine the flour and the oil.  Using the recipe given elsewhere in these notes, make a very \hyperref[BrownRoux]{dark roux} using the microwave oven. When roux is a deep chocolate color, add the celery, onions, bell pepper, garlic and thyme to the measuring cup, stir until blended, and cook in the roux until the onion is tender (2 minutes + 1 minute + 1 minute + 1 minute). If any oil as risen to the top, pour this off.

Transfer roux to a large cast-iron Dutch oven. Very carefully add 6 cups of the hot broth to the roux stirring vigorously until roux is smooth and thin. Add Kitchen Bouquet to achieve desired color. Bring to a boil then reduce the heat to medium-low and simmer uncovered for 45 minutes. Add the reserved turkey meat and sausage to the gumbo.  If the gumbo is too thick then add up to 2 additional cups of hot stock to thin it. Simmer for 20 minutes.

Season to taste with salt, pepper and cayenne; let simmer another 10 minutes. Check the seasoning and adjust if needed.

Place a mound of cooked rice in each soup bowl, surround with the gumbo, and sprinkle with parsley and green onions.

Source: Modified from a recipe by Pableaux Johnson found at \url{http://illinoistimes.com/article-16457-turkey-bone-gumbo.html}.

%\hyperref[TOC]{Table of Contents}
\newpage

\recipe{Brown Roux}
\label{BrownRoux}

\ingredients{flour; canola oil or well rendered bacon fat; onions, chopped; bell peppers, chopped; celery, chopped; garlic, minced; Kitchen Bouquet Browning and Seasoning Sauce (optional).}

\textsc{I leaned to make roux in a microwave oven from the book Tout de Suite \`{a} la Microwave by Jean K. Durkee. In 1977 when this book was published the maximum output power of home microwave ovens was 600 - 650 watts. The output power of current microwave ovens is 1000 - 1200 watts with 1000 watts being the most common value. The instructions here are for a 1000 watt oven, however, ovens vary enough that you will have to play with the cook times in this recipe and find values that work with your oven.}

Combine equal amounts  of flour and oil or fat in an 8 cup Pyrex measuring cup (typically \twothirds cup of each or use the amounts specified in your recipe). Stir well to combine with a sturdy metal whisk or a long handled wooden spoon.

Microwave on high for 5 minutes.  Carefully remove measuring cup from microwave using a dish towel and stir well being very careful not to splatter any of the hot roux on yourself (it's called Cajun napalm for a reason). The roux should be light brown at this point. Return measuring cup to the microwave and cook on high for 2 minutes then remove again and stir. Return measuring cup to the microwave and cook on high for 1 minute then remove again and stir. Continue this process of cooking for 1 minute and stirring until roux is the desired color. Three or four 1 minute cooking intervals is usually enough to get a dark brown roux. If a very dark roux is desired it may be necessary to reduce the cooking interval to 30 seconds when the roux is dark but not dark enough to avoid burning the roux. 

If you burn your roux as evidenced by black specks and smoke then toss it out and start over.  Since a roux can be made so quickly in the microwave you haven't lost much time.  After a some experimentation and practice with your microwave you will be able to turn out perfectly dark roux every time.

Once the flour and fat mixture is browned to your satisfaction you can proceed to cook the onions, bell peppers, celery and garlic specified in your recipe in the microwave by adding them to the measuring cup or you can proceed as described below.

Transfer roux to a large cast iron Dutch oven and add the onions, bell peppers, celery, and garlic in the amounts specified in your recipe. Saut\'{e} vegetables for around 5 minutes. The roux will continue to darken and the vegetables will caramelize. If any oil has risen to the top, pour this off.

Add stock to roux mixture and stir well.  It the color is not dark enough to suit you then add a tablespoon or more of Kitchen Bouquet Browning and Seasoning Sauce to achieve the desired color. Professional kitchens add caramelized sugar to color their roux so this is not cheating. If they can do it then you can too. Kitchen Bouquet is made with caramel and caramelized vegetables so it is actually a better way to add color since it adds flavor too.

\newpage

\recipe{Red Curry Lentils With Sweet Potatoes and Spinach}
\label{RedCurryLentils}

\ingredients{3 tablespoons olive oil;
1 pound sweet potatoes (about 2 medium sweet potatoes), peeled and cut into \threequarters inch cubes;
1 medium yellow onion, chopped;
3 tablespoons Thai red curry paste;
3 garlic cloves, minced (about 1 tablespoon);
1 (1-inch) piece fresh ginger, peeled and grated (about 1 tablespoon);
1 red chile, such as Fresno or serrano, halved, seeds and ribs removed, then minced;
1 teaspoon ground turmeric;
1 cup red lentils, rinsed;
4 cups low-sodium vegetable stock;
2 teaspoons kosher salt, plus more to taste;
1 (13-ounce) can full-fat coconut milk;
1 (4- to 5-ounce) bag baby spinach;
\half lime, juiced;
Fresh cilantro leaves, for serving;
Toasted unsweetened coconut flakes, for serving (optional).}

In a Dutch oven or pot, heat 2 tablespoons olive oil over medium- high. Add the sweet potatoes and cook, stirring occasionally, until browned all over, 5 to 7 minutes. Transfer the browned sweet potatoes to a plate and set aside.

Add the remaining 1 tablespoon olive oil to the pot and set the heat to medium-low. Add the onion and cook, stirring occasionally, until translucent, 4 to 6 minutes. Add the curry paste, garlic, ginger, chile and turmeric, and cook until fragrant, about 1 minute.

Add the lentils, stock, salt and browned sweet potatoes to the pot and bring to a boil over high. Lower the heat and simmer, uncovered, stirring occasionally, until the lentils are just tender, 20 to 25 minutes.

Add the coconut milk and simmer, stirring occasionally, until the liquid has reduced and the lentils are creamy and falling apart, 15 to 20 minutes.

Add the spinach and stir until just wilted, 2 to 3 minutes. Off the heat, stir in the lime juice and season with salt to taste.

Divide among shallow bowls and top with cilantro and coconut flakes, if using.

Source: Lidey Heuck from The New York Times Cooking

\chapter{Fish and Seafood}

\recipe{Grilled Pickled Shrimp}
\label{GrilledPickledShrimp}

\ingredients{1 to 1\half pound shrimp;
\twothirds cup of shrimp stock made from shrimp shells;
1 medium onion chopped into bite-size squares;
2 tablespoons crab boil;
\twothirds cup of olive oil;
\twothirds cup of white wine vinegar;
\quarter cup of lemon juice;
2 tablespoons capers;
1 teaspoon sugar;
\half teaspoon salt;
\quarter teaspoon hot sauce
1 cup pimiento-stuffed green olives.}
 
Peal shrimp and remove tails. Devein the shrimp if you are fussy.

Make shrimp stock by simmering shells and tails in \twothirds cup of water for 10 minutes. Strain out the shells and tails and reserve the stock.  

Combine the shrimp stock, the onion and the crab boil in a small sauce pan. Bring the liquid to a boil, reduce heat, and simmer for around 5 minutes. Remove from heat and add all remaining ingredients except the olives.  Pour the warm marinade over the shrimp, mix well, and marinate from 8 to 24 hours refrigerated.

Drain the shrimp and the onions and alternate them on skewers with the green olives. Skewer one end of the shrimp, slide on an olive, skewer the other end of the shrimp then place an onion square between shrimp on the skewer. The skewers can be returned to the marinade until just before cooking.

Grill shrimp skewers over medium-hot ashen-gray coals for a minute or two, until the shrimp are just firm. 

Serves 4 as a main course.

Source: Cheryl Alters Jamison and Bill Jamison from Texas Home Cooking.  

I like to serve these shrimp on top of yellow rice.  I strain some of the capers and onions from the marinade and reserve a little liquid from the marinade before adding the raw shrimp.  I add the reserved onions, capers, and marinade liquid to the yellow rice.  This recipe is ideal for entertaining because it requires very little last minute preparation.

\newpage

\recipe{Crispy Coconut-Crusted Fish Fingers}
\label{CoconutFishFingers}
\ingredients{\\
\\
\textsc{Dipping Sauce -- }\third cup apricot preserves; 1 tablespoon apple cider vinegar; 1 teaspoon peeled and grated fresh ginger; \quarter teaspoon lemon or lime zest, finely grated (optional), \eighth teaspoon chipotle chile powder (optional); 1 whole scallion, thinly sliced; 1 to 2 teaspoons of sambal oelek.\\
\\
\textsc{Fish Fingers --} 10 to 12 ounces cod or keta salmon fillets, pin bones and skin removed; \half cup panko breadcrumbs; \half cup shredded unsweetened coconut; 1 large egg beaten with 1 tablespoon water; all-purpose flour, salt and pepper, neutral oil for pan frying.
}

For dipping sauce, in a small sauce pan over medium-high heat, combine apricot preserves,  vinegar, ginger, and \quarter cup of water. Bring to a boil then reduce the heat to low and simmer until slightly reduced, 1 to 2 minutes. Stir in the scallion and sambal then season with salt to taste. Transfer the sauce to a small serving bowl and reserve.

For the fish fingers, begin by patting the fish dry with paper towels and cutting it into 1-inch wide strips. Sprinkle the fish with salt and let it sit for a few minutes while you prepare the coating.

Chose three shallow bowls and set up a three bowl dredging system. Put flour in the first bowl and egg/water mixture in the second bowl. In the third bowl, combine the panko and coconut and then season the mixture with salt and pepper.

Rinse the fish fingers to remove the excess salt and then dry them well with paper towels. One at a time, coat the fish in flour, tapping off any excess, then dip in the egg wash. Finally, add the fish to the panko mixture and turn to coat the finger, pressing down so that the coating adheres to the fish.  Place the coated fish finger on a plate and then repeat the coating process with the remaining fish.

Add \half inch of oil to a medium skillet and place on medium-high heat.  When the oil reaches 400\deg F, add the fish fingers (in batches if necessary) and fry, turning once or twice until golden.  This should take 2 or 3 minutes per side. Drain on paper towels and sprinkle lightly with salt.

Transfer fried fish to a platter and serve immediately accompanied with the dipping sauce.

Serves: 2

Source: Grace Parisi of Sika Salmon Shares



\newpage

\recipe{Catfish Courtbouillon}

\ingredients{\half pound shrimp;
1\half pounds  catfish fillets;
5 tablespoons flour;
5 tablespoons olive oil;
2 cups chopped onion;
2 cups chopped celery;
\threequarters cup chopped bell peppers;
3 cloves garlic, minced;
\half cup chopped green onions;
\half cup chopped parsley;
1 cup dry white wine;
28 oz. can whole tomatoes, drained and broken up by hand;
\half small can tomato paste;
2 teaspoons salt;
\half teaspoon black pepper;
4 bay leaves;
\half teaspoon thyme;
2 tablespoons lemon juice;
2 teaspoons more or less hot sauce, to taste.}

Peal shrimp reserving heads, shells, and tails. Put shrimp in refrigerator and hold for use at the end of the recipe.  Simmer reserved shrimp heads, shells, and tails in 2 cups water for 10 minutes. Strain and reserve shrimp stock.

Combine flour and oil in glass measuring cup and proceed to make a dark brown roux in the microwave. Stir onions, celery, bell peppers, and garlic into the roux and then microwave for a few more minutes until the onions are soft.

Transfer roux mixture to a cast iron pot. Mix in tomatoes and tomato paste then cook, stirring constantly, about five minutes.  Add shrimp stock, wine, parsley, green onions, lemon juice, bay leaves, thyme, salt, pepper, and hot sauce. Simmer for about 45 minutes. Can stop at this point and then reheat courtbouillon before continuing recipe.

Add fish and simmer for 10 minutes or so until fish flakes when tested with a fork. If desired the shrimp can be added 5 or so minutes after the fish so that shrimp and fish are done at the same time. Serve with rice.

%\hyperref[TOC]{Table of Contents}
\newpage
\recipe{Catfish Pecan with Lemon Thyme Butter}


\ingredients{3 cups (10 ounces) pecan halves;
1\half cups all-purpose flour;
Creole seafood seasoning, to taste;
1 medium egg;
1 cup milk;
6 catfish fillets, 5 to 7 ounces each;
12 tablespoons (1\half sticks) butter;
3 lemons, cut in half;
1 tablespoon Worcestershire sauce;
6 large sprigs fresh thyme;
Kosher salt and freshly ground pepper to taste.}

Place half the pecans, the flour, and the Creole
seasoning in the work bowl of a food processor, and process until
finely ground. Transfer the pecan flour to a large bowl.

Whisk the egg in a large mixing bowl and add the milk. Season both
sides of the fish fillets with Creole seasoning. On e at a time,
place the fillets in the egg wash.

Remove one fillet from the egg wash, letting any excess fluid
drain back into the bowl. Dredge the fillet in the pecan flour and
coat both sides, shaking off any excess. Transfer to a dry sheet
pan, and repeat with the remaining fillets.

Place a large saut\'e an over high heat and add 2 tablespoons of the
butter. Heat for about 2 minute, or until the butter is completely
melted and starts to bubble. Place three fish fillets in the pan,
skin side up, and cook for 30 seconds. Reduce the heat to medium,
and cook for another 1\threequarters to 2 minutes, or until the fillets are
evenly brown and crisp. Turn the fish over and cook on the second
side for 2 to 2\half minutes, or until the fish is firm to the
touch and an even brown. The most important factor in determining
the ideal cooking time is the thickness of the fillets you are
using.

Remove the fish, place on a baking rack, wipe the pan clean with a
paper towel, and 2 tablespoons of butter, and repeat with the
remaining pieces of fish.

When a all the fish fillets are cooked, wipe the pan, and return
the heat to high. Melt the remaining 8 tablespoons of butter and,
just as the butter turns brown, and the remaining 1\half cups of
pecans and saut\'e or 2 to 3 minutes or until the nuts are toasted,
stirring occasionally. Put the lemons face down in the pan, first
squeezing a little juice from each piece. Add the Worcestershire
and the fresh thyme, season with salt and pepper, and cook for 30
seconds more or until thyme starts to wilt and becomes very
aromatic.

Place one fish fillet and a lemon piece on each of six dinner
plates, spoon some pecan better around each piece of the fish, and
use the wilted thyme to garnish the piece.

%Alternate instructions:
%
%For two large catfish filets, combine 1/2 C flour and 1/2 C pecan
%halves in the food processor with 1 t creole seasoning and grind
%until fine.  Mix one egg with milk.  Season both sides of the
%catfish and then put in egg wash.  From the egg wash press the
%filets into the pecan/flour mixture making sure the filets are
%well coated. Saut\'{e} catfish over medium heat in butter (not too hot
%or the crust will burn) for about 2 min. per side.  Take off fish
%and add more butter to the pan with whole pecan halves.  Toast the
%pecans. When brown add lemon halves cut side down to brown over
%low heat.  Add fresh thyme and season with salt and pepper. Finish
%pan sauce with Worcestershire sauce after it has cooked for a moment.
%To serve, put catfish on plate, put lemon halve cut side up on
%plate, put thyme sprig on top of catfish, scatter pecan halves
%over catfish, pour pan sauce around plate and scatter chopped
%green onions around plate.

%\hyperref[TOC]{Table of Contents}
\newpage
\recipe{Thin Fried Catfish with Slaw and Tartar Sauce}

\ingredients{catfish fillets;
buttermilk;
hot sauce;
yellow cornmeal;
flour;
peanut oil for frying;
Napa cabbage;
red cabbage;
red onion;
celery seeds;
salt;
black pepper;
cane vinegar;
\half cup homemade mayonnaise;
1 tablespoon prepared  horseradish;
2 tablespoons Creole mustard;
2 tablespoons chopped homemade sweet pickle;
1 teaspoons creole seasoning;
\quarter to \half cup chopped green onions.}

Cut 1 ounce portions of catfish.  Cut the catfish at an angle to get
thin pieces.  Put catfish in buttermilk and add hot sauce (2 or 3
tablespoons to 1 cup of buttermilk).  Make a cornmeal crust for the catfish by
combining equal parts of yellow cornmeal and shifted flour.  Shake
buttermilk off of catfish and press fish in cornmeal flour
mixture.  Press hard on both sides to flatten catfish pieces.  Fry
catfish pieces at 350\deg F in peanut oil. Season catfish with creole
seasoning when it comes out of the oil. 

Make a slaw with thinly cut Napa cabbage, red cabbage and red
onion.  Add celery seeds, salt, pepper and cane vinegar.

To make Tartar sauce, mix mayonnaise, horseradish,  Creole mustard,
chopped homemade sweet pickle, 
Creole seasoning and chopped green onions.  Season with salt and pepper.

To serve, put slaw on plate and top with catfish.  Drizzle tartar
sauce around plate.

%\hyperref[TOC]{Table of Contents}
\newpage
\recipe{Smoked Catfish P\^{a}t\'{e}}

\ingredients{1\half pounds catfish filets;
mesquite or pecan wood chips for smoking;
1 envelope unflavored gelatin;
\quarter cup cold water;
\half cup boiling water;
\half cup homemade mayonnaise;
1 tablespoon lemon juice;
1 tablespoon graded onion;
1 large clove garlic, pressed;
1 tablespoon hot sauce;
2 tablespoons finely chopped fresh basil;
1 teaspoon salt;
\quarter cup roasted red bell pepper strips;
1 cup heavy cream.}

Cook catfish on a barbecue grill using wood chips to obtain a strongly smoked flavor. The catfish should be cooked to the point where it flakes easily. Finely flake 2 cups of the smoked catfish to use in the recipe and set aside.

Soften the gelatin in the cold water in a large mixing bowl. Stir in the boiling water and whisk the mixture slowly until the gelatin dissolves. Cool to room temperature.

Whisk in the mayonnaise, lemon juice, graded onion, pressed garlic, hot sauce, basal and salt. Stir to blend completely and refrigerate  until mixture begins to thicken slightly, about 15 to 20 minutes.

Fold in the flaked catfish and bell pepper strips. In a separate bowl, whip the cream until is is thickened to peaks and fluffy.  Fold gently but completely into the catfish mixture. Transfer to a 6 to 8 cup mold (a bread pan works fine). Cover and chill for at least 4 hours.

Unmold on a platter and garnish with parsley, watercress, or other greenery. Serve with toast, black bread, or crackers.

%\hyperref[TOC]{Table of Contents}
\newpage
\recipe{Stove Top Smoked Salmon}

\ingredients{1 pound (more or less) farm raised salmon filet;
\twothirds cup light brown sugar;
\third cup Kosher salt;
2 or 3 Tablespoons cherry wood chips}

Combine brown sugar and salt in a glass casserole dish.  

Wash filet under running water and then pat completely dry using paper towels.

Spread a \quarter inch layer of the sugar and salt mixture on the bottom of glass casserole leaving the remaining mixture at the edges of the dish. Place the filet skin side down on the layer of curing mixture then completely cover the salmon flesh with the remaining mixture.  Cover the casserole with plastic wrap and refrigerate for at least one hour. 

Prepare the stove top smoker by adding cherry wood chips to bottom. Add filet to smoker rack, close cover of smoker and place smoker of medium high burner.  Smoke for 25 to 35 minutes depending on the thickness of the filet. When done the internal temperature should be around 140 \deg F and the filets should have a smokey brown appearance. 

Remove from smoker and cool for a few minutes if serving hot. Loosely wrap in foil and refrigerate if serving cold.

Excellent served warm over a green salad with grapefruit supremes and a creamy grapefruit vinaigrette dressing. Also excellent served in the traditional way as an appetizer with cream cheese, red onion, capers, and toasted bagels or sour dough bread.

%\hyperref[TOC]{Table of Contents}
\newpage
\recipe{Crawfish Monica}

\ingredients{1 pound crawfish tails; 
1 stick butter;
1 pint half and half;
1 bunch green onions, green and white parts chopped;
5 cloves garlic (or up to 10 to taste), chopped;
1-2 tablespoons Creole Seasoning;
1 pound Rotelli}

Cook pasta according to the directions on the package.  Drain,
then rinse under cool water.  Drain again, thoroughly.

Melt the butter in a large pot and saut\'{e} onions and garlic for 3
minutes.  Add the crawfish and saut\'{e} for 2 minutes.  Add the
half-and-half, then add several big pinches of Creole seasoning,
tasting before the next pinch until you think it's right.

Cook for 5 - 10 minutes over medium heat until the sauce thickens.
Add the pasta and toss well.  Let it sit for 10 minutes or so
over very low heat, stirring often.  Serve immediately, with lots
of French bread and a nice dry white wine.


%\hyperref[TOC]{Table of Contents}
\newpage

\recipe{Shrimp in Cashew-Yogurt Sauce}
\ingredients{1 to 1\quarter pounds large (16-20 count) frozen shrimp, shell-on and deveined;
5 tablespoons unsalted butter;
1 medium onion;
2 cloves garlic;
\half teaspoon chili powder, plus more as needed;
\half teaspoon ground cumin;
1 teaspoon ground turmeric;
\half teaspoon kosher salt, or more as needed;
1 teaspoon freshly ground black pepper; 
1 cup roasted, unsalted cashews;
2 cups plain Greek yogurt, preferably full-fat}

Place the shrimp (to taste) in a bowl of tap water and let them defrost a bit while you prep the butter. When you can, pull off the tails and reserve.

Melt the butter in the pot, over medium-low heat. As soon as white milk solids form on the surface, skim off and discard them. The butter should be mostly golden and clear; you have clarified it enough for this dish.

Add the reserved shrimp tails; cook for about 10 minutes, stirring once or twice. (You are infusing the butter with shrimp flavor!)

Meanwhile, peel the shrimp; it's okay if they are not fully defrosted. Cut the onion into small dice. Mince the garlic.

Remove the shrimp tails from the pot and discard, then stir in the onion and garlic. Increase the heat to medium; cook for 3 or 4 minutes until just softened, stirring a few times, then add the chili powder, cumin, turmeric, salt and pepper, stirring to incorporate. Add half the cashews.

Add the yogurt, stirring to form a thick sauce; cook for 2 or 3 minutes, then reduce the heat to low.
 
Use an immersion blender to puree the mixture right in the pot.

Drain the shrimp and add them to the pot, along with the remaining cashews. Increase the heat to medium; cook for about 4 minutes, stirring to make sure all the shrimp is pink and no longer translucent. Taste and add more salt and/or chili powder, as needed.

Divide among wide, shallow bowls. Sprinkle each portion with a little chili powder, and serve warm.

Makes four servings when served over rice. Attractive when topped with strips of roasted red bell pepper.

Source: Washington Post

%\hyperref[TOC]{Table of Contents}

\newpage

\recipe{Salmon Burgers}

\ingredients{1\quarter pounds (20 ounces) skinless center-cut salmon fillet, pin bones removed;
2 tablespoons Dijon mustard;
1 tablespoon \hyperref[MyMayonnaise]{mayonnaise};
1 tablespoon lemon juice;
\half teaspoon grated lemon zest;
pinch of cayenne pepper;
2 scallions, chopped;
2 tablespoons plus 2 cups (or more) Panko  breadcrumbs;
Kosher salt and freshly ground black pepper;
2 tablespoons extra-virgin olive oil, plus more for brushing;
4 brioche buns, split;
\hyperref[TartarSauce]{Tartar sauce} and arugula, for topping.}

Cut three-quarters (15 ounces) of the salmon into \quarter inch pieces. Put in a large
bowl. Cut the remaining 5 ounces of the salmon into chunk. Transfer the chunks to a food processor along with the mustard, mayonnaise, lemon juice, lemon zest and cayenne. Pulse to make a paste. Add the pureed salmon mixture to the bowl with the diced salmon. Add the scallions, 2 tablespoons Panko, \half teaspoon salt, and black pepper to taste. Gently mix until just combined.

Line a quarter sheet pan with parchment paper or oiled foil.
Weigh the salmon mixture and divide it into 4 equal weight mounds on the sheet pan. Using a 4-inch diameter ring-mold form the mounds into patties.  Loosely cover the sheet pan with plastic wrap and then refrigerate the patties for at least 30 minutes.

Spread the remaining 2 cups Panko on a plate. Press both sides of the salmon patties in the Panko. Use more Panko if necessary to cover evenly cover both sides of all the patties. Heat the olive oil in a large nonstick or cast-iron skillet over medium-high heat. Add the patties (in batches if necessary) and cook until browned on the bottom, 3 to 4 minutes, adjusting the heat if necessary. Turn and cook until the other side is browned and the patties feel springy in the center, 3 to 4 more minutes. Transfer to a paper towel-lined plate to drain; season with salt.

Meanwhile, arrange the buns, cut-side up, on a broiler pan and broil
until toasted, 1 to 2 minutes. This works best in a toaster oven.

Place the patties on the buns topped with tartar sauce and arugula. Serve with a hearty salad like \hyperref[BrownwoodColeslaw]{Brownwood-Style Coleslaw}, \hyperref[BlackEyedPeaSalad]{Hattie B's Black-Eyed Pea Salad} or \hyperref[CurriedCarrotRaisinSalad]{Curried Carrot Raisin Salad}.

Yield: Four large burgers -- enough to serve 4 for dinner.  The same amount of salmon mixture can be made into 6 burgers if smaller serving are desired. 



Source: Food Network (Perfect Salmon Burger)

\chapter{Poultry}
\recipe{French Chicken in a Pot}

\ingredients{1 small onion diced;
1 rib celery;
6 whole garlic cloves;
1 bay leave;
1 sprig rosemary;
4\half to 5 pound natural/free-range/organic chicken;
2 teaspoons Kosher salt;
fresh ground black pepper;
olive oil;
juice of \half lemon.}

Preheat oven to 250 \deg F. Dice onion and celery. Peal garlic cloves.

Fold back chicken's wing tips. Completely dry chicken with paper towels and
then season with salt and pepper.

Heat enough olive oil to cover the bottom of a Dutch oven till near smoking.
Using long tongs place the chicken breast side down in the Dutch oven.
Add bay leave, celery, rosemary, onion and pressed garlic cloves
on either side of chicken.

Brown breast side for around 5 minutes then turn to brown back for some what longer
(around 8 minutes). Occasionally stir the vegetables so they don't burn.

Turn off heat, put foil over Dutch oven to seal, put on lid and
place Dutch oven in preheated oven.

Bake for 80 to 110 minutes until breast is 160-165 \deg F and dark meat is 175 \deg F.

Let chicken rest under foil tent for 20 minutes before carving.

Strain juices from pan into a fat separator.

Pour fat-free juices into a small sauce pan and season with lemon juice,
salt and pepper. Reheat juices over low heat.

When chicken has rested, carve off wings and leg/thigh then separate
the leg and thigh. Remove breast halves each in one piece then
slice across from keel side to side side. Arrange  meat on a platter.

Pour some the heated juice over the meat and reserve some juice to serve on the side.

\newpage
\recipe{Korean Fried Chicken Wings}
\label{KoreanWings}
\equilpment{Microplane Classic Zester Grater; R \& V Works Cajun Fryer Model FF1.}
\ingredients{1 tablespoon toasted
sesame oil;
1 teaspoon garlic, grated to paste;
1 teaspoon grated fresh ginger;
1\threequarters cups water;
3 tablespoons sugar;
2-3 tablespoons gochujang;
1 tablespoon soy sauce;
1 cup all-purpose flour;
3 tablespoons cornstarch;
3 pounds chicken wings, cut at joints, wingtips discarded;
peanut oil to fill fryer.}

 For a complete meal, serve with steamed white rice and a slaw.

Combine sesame oil, garlic, and ginger in large bowl and microwave until mixture is bubbly and garlic and ginger are fragrant but not browned, 40 to 60 seconds. Whisk in \quarter cup water, sugar, gochujang, and soy sauce until smooth; set aside.

Whisk flour, cornstarch, and remaining 1\half cups water in second large bowl until smooth.

Heat vegetable oil in fryer to 350\deg F. Place wings in batter and stir to coat. Using tongs, remove wings from batter one at a time, allowing any excess batter to drip back into bowl, and add to hot oil. When adding a wing to the oil, hold the wing in the oil with the tongs for a few seconds to allow the coating to set before releasing the wing into the fryer. Cook, stirring occasionally to prevent wings from sticking, until coating is light golden and beginning to crisp, about 7 minutes. Raise fryer basket and let wings rest for 5 minutes.

Heat oil to 375\deg F.  Return all wings to oil and cook, stirring occasionally, until deep golden brown and very crispy, about 7 minutes. Raise fryer basket and let stand for 2 minutes. Transfer wings to reserved sauce and toss until coated. Transfer to platter, let stand for 2 minutes to allow coating to set and then serve.

A generous meal for two.

Source: America's Test Kitchen recipe modified to use a R \& V Works Cajun Fryer. 

\newpage
\recipe{Marinated Game Hens in Mustard and Tarragon Sauce}

\ingredients{4 game hens;
\half cup olive oil;
\half teaspoon dried or 1\half teaspoons fresh tarragon;
2 tablespoons chopped shallot;
2 tablespoons chopped shallot
\half cup white wine or tarragon vinegar;
\half cup white vermouth;
2 tablespoons lemon or lime; 
juice strained roasting pan juices;
2 cups chicken stock;
1 cup heavy cream;
1 tablespoons Dijon mustard;
\half teaspoon tarragon or to taste;
3 or 4 tablespoons butter.}

\textsc{I learned to make this from my friend, Rosa Rajkovic, during my time in Albuquerque, NM.  The reduction sauce that accompanies the game hens is very rich.  As a result, I usually serve this with a simple garnish of steamed baby carrots and/or broccoli on the serving platter and nothing but this and bread for the main course.  I have made several variations of this recipe.  These include replacing the vermouth with red port or Madeira and replacing the white vinegar with red.  Instead of roasting the marinated game hens they can be grilled and the sauce made without the pan juices. }

Spilt the game hens with poultry shears (removing the backbone) and marinate overnight in a mixture of olive oil, tarragon, and shallot.

Arrange the split hens meat side up in a deep roasting pan.  Be sure that the hens do not touch.  Roast the hens for 35 to 45 minutes in a 375\deg F oven.  Check the drum sticks for doneness.

Transfer the hens to a covered warming pan and keep warm in the oven.  Strain the juice from the cooking pan and reserve for use in the sauce in the next step.

The remaining ingredients and the pan juices are used to make the sauce.  Because this is a reduction sauce, all the remaining ingredients should be prepared and measured before beginning the sauce.

In a wide, non-reactive skillet placed over high heat, add the vinegar, vermouth, and chopped shallots and boil until nearly dry.

Add the lemon or lime juice and the pan juices and reduce by one half.

Add the chicken stock to the pan and reduce by one half again.

Add the cream to the sauce and cook down to the consistency of maple syrup.

Finish the sauce by adding the mustard, tarragon, and butter.

Remove the hens from the warming pan and place on the serving platter.  Surround the hens with the broccoli and carrots and drizzle the sauce over the hens.  Serve the remaining sauce in a sauce boat.

 Serves 8 (half a game hen per person is usually enough if a meal with a vegetable garnish, salad, bread, and desert is served).

\newpage
\recipe{Oven-Roasted Turkey Meatloaf}

\ingredients{2 pounds 85\% lean 15\% fat ground-turkey;
\half pound bell peppers, diced;
\half pound yellow onions, diced;
1\half cup oatmeal;
1 cup breadcrumbs;
\half cup BBQ sauce;
\half cup garlic, minced;
4 eggs;
1\half teaspoon chicken base;
1\half teaspoon salt;
1\half teaspoon freshly ground black pepper; BBQ sauce for service (optional).}

% Origional recipe
%\ingredients{1\quarter pound 85\% lean 15\% fat ground-turkey;
%\quarter pound bell peppers, diced;
%\quarter pound yellow onions, diced;
%\threequarters cup oatmeal;
%\half cup breadcrumbs;
%\quarter cup BBQ sauce;
%\quarter cup garlic, minced;
%2 eggs;
%\threequarters teaspoon chicken base;
%\threequarters teaspoon salt;
%\threequarters teaspoon freshly ground black pepper.}

\textsc{For this recipe I suggest making a Memphis style BBQ sauce such as \hyperref[MemphisMopBBQSauce]{Memphis Mop BBQ Sauce}. Asian style BBQ sauce also works well with this recipe. If you are in a hurry then store bought BBQ sauce works too although I would avoid sauces that are very smokey and/or very sweet.}

\textsc{This recipe makes two loafs.  Unless you are serving 8 people I suggest freezing one loaf and serving one now.  Wrapped tightly in foil these loafs freeze very well.}

Preheat oven to 350\deg F.

Combine all ingredients but do not over mix.

Line two loaf pans with parchment paper and spray with non-stick cooking spray.

Fill pans with mixture, pack well to prevent air pockets. As an alternative, roll up mixture in parchment paper, twist the ends to make a compact bundle, and then place seam side down in loaf pans.

Place pans in oven cook 45 minutes to an hour.

Top slices of meatloaf with BBQ sauce and serve with Macaroni and cheese and Southern cooked greens.

Source: Modified from a recipe used at Isaac Hayes' restaurant in Memphis (Short lived but great while it lasted). 

Each loaf makes 4 servings.

\newpage
\recipe{Roasted Chicken Proven\c{c}al}

\ingredients{4 chicken leg quarters;
2 teaspoons kosher salt;
1 teaspoon freshly ground black pepper;
\half to \threequarters cup all-purpose flour;
3 tablespoons olive oil;
2 tablespoons herbes de Provence;
1 lemon, quartered;
8-10 cloves garlic, peeled;
4-6 medium-size shallots, peeled and halved;
\third cup dry vermouth;
4 sprigs of thyme, for serving.}

Preheat oven to 400\deg F. Season the chicken with salt and pepper. Put the flour in a shallow pan, and lightly dredge the chicken in it, shaking the pieces to remove excess flour.

Swirl the oil in a large roasting pan and place the floured chicken in it. Season the chicken with the herbes de Provence. Arrange the lemons, garlic cloves and shallots around the chicken, and then add the vermouth to the pan.

Put the pan in the oven,and roast for 25 to 30 minutes,then baste it with the pan juices. Continue roasting for an additional 25 to 30 minutes, or until the chicken is very crisp and the meat cooked through.

Serve in the pan or on a warmed platter, garnished with the thyme.

Yield: 4 servings

Source: The New York Times Magazine, April 1, 2015

\newpage
\recipe{Gai Yang (Thai-Style Grilled Chicken)}
\label{GaiYang}
\ingredients{6 medium cloves garlic;
12 sprigs cilantro, including thick stalks;
3 tablespoons palm or light brown sugar;
2 teaspoons ground white pepper;
1 teaspoon ground coriander;
2 tablespoons fish sauce;
2 teaspoons dark soy sauce;
2 (3-inch) segments lemongrass (optional);
1 whole chicken, spine removed, split in half along the breast bone;
1 recipe \hyperref[ThaiDippingSauce]{Thai-Style Sweet Chili Dipping Sauce} (recipe elsewhere in book)}

Combine garlic, cilantro, sugar, white pepper, coriander, fish sauce, dark soy sauce, and lemongrass (if using) in the bowl of a food processor. Process until a rough paste is formed. Set aside.

Place chicken halves flat on a cutting board. Insert two metal skewers into each chicken, running parallel through the legs and the breasts. Transfer chicken to a casserole dish that just fits them. Rub on all surfaces with marinade. Cover, transfer to refrigerator, and allow to marinate for at least 2 hours and up to over night.

When ready to cook, light one chimney full of charcoal. When all the charcoal is lit and covered with gray ash, pour out and spread the coals evenly over half of coal grate. Set cooking grate in place, cover gill and allow to preheat for 5 minutes. Clean and oil the grilling grate.

Place chicken skin-side up on cooler side of grill with legs facing towards hotter side. Cover grill with vents on lid open and aligned over the chicken. Open bottom vents of grill. Cook until instant read thermometer inserted into deepest part of breast registers 140\deg F, about 45 minutes. Carefully flip chicken and place skin-side-down on cooler side of grill with breasts pointed towards hotter side. Cover and cook until skin is crisp and instant read thermometer inserted into deepest part of breast registers 145 to 150\deg F, about 3 minutes longer (be careful, the skin is very prone to burning). Do not leave the lid off for longer than it takes to check temperature or chicken will burn.

Transfer chicken to a cutting board and allow to rest for 5 minutes. Carve and serve with dipping sauce.

Yield: 3 or 4 servings

Source: SeriousEats.com

\newpage
\recipe{Thai-Style Sweet Chili Dipping Sauce}
\label{ThaiDippingSauce} 
\ingredients{10 to 12 red Thai bird chilies, finely minced OR 3 tablespoons of Sambal Oelek;
4 cloves garlic, finely minced (about 4 teaspoons);
\quarter cup Asian fish sauce;
\half cup light brown sugar;
3 tablespoons distilled white vinegar;
2 tablespoons fresh lime juice}
 
Combine all ingredients in a small sealable container. Shake well, set aside, and allow sugar to dissolve, shaking every 10 minutes or so. When sugar is fully dissolved, store in refrigerator until ready to use. Sauce will keep in the refrigerator for up to 3 weeks.

Sauce used with \hyperref[GaiYang]{Gai Yang} (Thai-Style Grilled Chicken).

 
Source: SeriousEats.com

\newpage
\recipe{Slow Roasted Turkey Breasts}

\ingredients{whole turkey breast with bones attached;
1 cup kosher salt;
1 cup sugar}

If the turkey breast is frozen defrost it in the refrigerator for a day or two until completely thawed. Carefully cut the two breast off the bone while leaving the skin covering the breasts intact. Reserve the bones for stock making.  The breasts should weight from 3 to 4\half pounds each.  

In a sauce pan bring 2 quarts of water to boil with the salt and sugar. Pour brine into brining container and add ice to make the total volume 4 quarts.  Submerge the turkey breast in the brine and refrigerate from 12 to 24 hours.

Preheat the oven to 250\deg F.  Remove the breasts from the brine and wrap each one four times in plastic wrap and once in aluminum foil. Place the breast on a wire rack in a half sheet pan. Add water to reach just below the rack. Cook until the internal temperature reaches 135\deg F, 2 to 3 hours. Remove turkey from oven and then, without removing the wrapping, submerge the turkey in an ice bath for 5 minutes. Remove foil, plastic wrap, and skin.

One option for serving the roasted turkey breast is to cut 1 inch thick slices and brown both sides in butter using a cast iron skillet.

\newpage
\recipe{Garlicky Chicken With Lemon-Anchovy Sauce}

\ingredients{1\quarter pounds boneless, skinless chicken thighs (4 to 5 thighs);
1 teaspoon coarse kosher salt;
Freshly ground black pepper;
6 garlic cloves, smashed and peeled;
\quarter cup extra-virgin olive oil;
5 anchovy fillets;
2 tablespoons drained capers, patted dry;
1 large pinch chile flakes;
1 lemon, halved;
Fresh chopped parsley, for serving.}

Heat oven to 350\deg F. Season the chicken thighs with salt and pepper and let rest while you prepare the anchovy-garlic oil. Mince one of the garlic cloves and set it aside for later. In a large, ovenproof skillet over medium-high heat, add the oil. When the oil is hot, add the 5 smashed whole garlic cloves, the anchovies, capers and chile. Let cook, stirring with a wooden spoon to break up the anchovies, until the garlic browns around the edges and the anchovies dissolve, 3 to 5 minutes.

Add the chicken thighs and cook until nicely browned on one side, 5 to 7 minutes. Flip the thighs, place the pan in the oven and cook another 5 to 10 minutes, until the chicken is cooked through.

When chicken is done, transfer thighs to a plate (be careful, as the pan handle will be hot). Place skillet back on the heat and add minced garlic and the juice of one lemon half. Cook for about 30 seconds, scraping up the browned bits on the bottom of the pan. Return chicken to the pan and cook it in the sauce for another 15 to 30 seconds.

Source: Melissa Clark from the NYT

\newpage
\recipe{Roasted Spatchcocked Chicken}

\ingredients{3 to 4 pound whole chicken;
1 tablespoons kosher salt;
\half tablespoon fresh ground black pepper;
2 tablespoons pure olive oil;
2 tablespoons unsalted butter;
juice of one lemon.}

Preheat oven to 400\deg F. 

To spatchcock chicken first remove the back bone by cutting down both sides of the bone with heavy kitchen shears. Turn over chicken so that skin side is up and press hard on the breast to flatten the chicken. Although not necessary, the sternum can be removed to make the chicken very flat as this helps with the browning step. Cut off the wing tips and secure the ends of the legs by inserting them into slits cut in the skin at the bottom of the breast. 

Mix the salt and pepper together in a small bowl. Carefully dry the chicken with paper towels and then heavily season both sides with the salt and pepper mix.

Heat a large cast-iron skillet over high heat. When hot add olive oil and then butter. After the butter stops foaming, add chicken to the skillet skin-side down. Let the chicken skin brown without moving the chicken for approximately 3 minutes. When the chicken is brown, carefully turn the chicken over. Avoid breaking the skin while turning the chicken. Transfer the skillet with the chicken to the oven.

Roast chicken until until an instant-read thermometer inserted into the thickest part reaches 165 \deg F. Transfer chicken to a cutting board, cover loosely with foil, and let stand 10 minutes.

Add lemon juice and, optionally, an additional tablespoon butter to the pan, swirling to combine. Cut chicken into pieces and serve immediately drizzled with the pan sauce.

\newpage

\recipe{Peruvian-Style Grilled Chicken}

\ingredients{\\
\\
\textsc{Green Sauce --}
3 whole jalape\~{n}o chiles, roughly chopped;1 tablespoon  aj\'{i} amarillo pepper paste; 1 cup fresh cilantro leaves (28g); 2 medium cloves garlic; \half cup  mayonnaise; \quarter cup  sour cream; 2 teaspoons fresh juice; 1 teaspoon distilled white vinegar; 2 tablespoons extra-virgin olive oil; Kosher salt and freshly ground black pepper;\\
\\
\textsc{Chicken --}
1 whole chicken, 3\half to 4 pounds; 4 teaspoons  kosher salt; 2 tablespoons ground cumin; 2 tablespoons  paprika; 1 teaspoon  freshly ground black pepper; 3 medium cloves garlic, minced (about 1 tablespoon); 2 tablespoons  white vinegar; 2 tablespoons vegetable or canola oil.
}

To make the sauce combine jalape\~{n}os, aj\'{i} amarillo (if using), cilantro, garlic, mayonnaise, sour cream, lime juice, and vinegar in the jar of a blender. Blend on high speed, scraping down sides as necessary, until smooth. With blender running, slowly drizzle in olive oil. Season to taste with salt and pepper. Sauce will be quite loose at this point, but will thicken as it sits. Transfer to a sealed container and refrigerate until ready to use. 

To prepare the chicken, pat chicken dry with paper towels and place on a large cutting board, breast side down. Using sharp kitchen shears, remove backbone by cutting along either side of it. Turn chicken over and lay out flat. Press firmly on breast to flatten chicken. For added stability, run a metal or wooden skewer horizontally through chicken, entering through one thigh, going through both breast halves, and exiting through other thigh. Tuck wing tips behind back.

Combine salt, cumin, paprika, pepper, garlic, vinegar, and oil in a small bowl and massage with fingertips until homogeneous. Spread mixture evenly over all surfaces of chicken. 

Light a chimney full of charcoal. When all charcoal is lit and covered with gray ash, pour out and spread coals evenly over half of coal grate. Alternatively, set half the burners of a gas grill to high heat. Set cooking grate in place, cover grill, and allow to preheat for 5 minutes. Clean and oil grilling grate. 

Place chicken, skin side up, on cooler side of grill, with legs facing toward hotter side. Cover grill, with vents on lid open and aligned over chicken. Open bottom vents of grill. Cook until an instant-read thermometer inserted into thickest part of breast registers 110\deg F (43\deg C). Carefully flip chicken and place, skin side down, on hotter side of grill, with breasts pointed toward cooler side. Press down firmly with a wide, stiff spatula to ensure good contact between bird and grill grates. Cover and cook until skin is crisp and an instant-read thermometer inserted into thickest part of breast registers 145 to 150\deg F (63 to 66\deg C), about 10 minutes longer. If chicken threatens to burn before temperature is achieved, carefully slide to cooler side of grill, cover, and continue to cook until done. Do not leave the lid off for longer than it takes to check temperature, or chicken will burn.

Transfer chicken to a cutting board and allow to rest for 5 to 10 minutes. Carve and serve with sauce. 

Source:  J. Kenji L\'{o}pez-Alt in Serious Eats

\chapter{Meat}

\recipe{Eye Round Roast}

\ingredients{3\half to 4 pound eye round roast;
4 teaspoons kosher salt;
freshly ground black pepper;
vegetable oil.}

Rub salt over entire surface of roast then wrap tightly in plastic wrap and 
place in refrigerator for 18 to 24 hours.

Preheat oven to 225\deg F.

Dry the surface of the roast very well with paper towels then coat with olive oil and 
a generous amount of black pepper.

Heat a cast iron skillet to a high temperature then add roast. When the first side is
very brown, turn the roast and brown the next side.  Continue turning and 
browning until the entire roast is very brown.  The ends are browned by holding 
the roast upright with tongs.  If the browning is done correctly there is a good 
deal of smoke and splattering -- it's best if you have a hood over your cooktop.

Transfer skillet with roast to the oven and cook until the roast is 130\deg F. Check temperature after 60 minutes. Total cook time is 75 to 90 minutes depending on size and initial internal temperature of the roast.

Remove roast and let rest on cutting board under a foil tent for at least 15 minutes 
before carving. Makes great cold roast beef for sandwiches.

Source: Modified from an America's Test Kitchen recipe.

\newpage

\recipe{Sichuan Pork Burgers}

\ingredients{\\
\\ 
\textsc{Saut\'{e}ed Onions --}
2 tablespoons pure olive oil;
2 large yellow onions, halved and thinly sliced (about 5 cups);\\
\\
\textsc{Burgers --}
1 tablespoon canola or other vegetable oil;
1 tablespoon garlic paste;
1 pound ground pork;
2 tablespoons gochujang;
\half tablespoon ground Sichuan peppercorns;
1 teaspoon gochugaru chili flakes;
1 tablespoon soy sauce;
1 tablespoon sugar;
Kosher salt;\\
\\
\textsc{Acompaniments --}
yellow prepared mustard;
planked dill pickles;
OR
kimchi;
fresh cabbage, shredded;
mayonnaise;\\
\\
\textsc{Buns --}
4 \hyperref[HamburgerBuns]{homemade hamburger buns};
8 tablespoons softened butter.
}

In a large skillet saut\'{e} onions in 2 tablespoons of oil for about 15 minutes until they are light golden.  Remove the onions from the skillet and wipe clean in preparation for cooking the burgers.

Grate garlic to paste using a Microplane. Grind Sichuan peppercorns in a spice grinder. 

In a bowl mix the burger ingredients (except oil) by hand.  Weigh and divide burger mixture into quarters then form into fairly thin burgers.  Add oil to hot skillet and then cook burgers until well done and browned on both sides.  When cooked set burgers aside to rest covered with foil.

Coat cut sides of buns with butter and then brown lightly in the skillet or a grill pan.

Serve burgers with saut\'{e}ed onions, yellow mustard and planked dill pickles.

Inspired by a Luck Peach recipe for Sichuan Pork Ragu. Thanks to Jack Williams of Water Valley Mississippi for the idea of making burgers using the same ingredients.

\newpage
\recipe{Beef, Turkey, and Andouille Meat Loaf}

\ingredients{\\
\\
\textsc{Loaf --}
2 tablespoons olive oil;
1\half cups onion, finely chopped;
\half cup chopped red bell peppers;
2 teaspoons minced garlic;
1 tablespoon Worcestershire sauce;
2 cups fresh bread crumbs;
\half cup milk;
2 large eggs, lightly beaten;
\half pound andouille, coarsely chopped in food processor;
1 pound 85\% lean ground beef;
1 pound ground turkey;
\half teaspoon dried sage;
1 teaspoon dried thyme;
1 teaspoon freshly ground black pepper;
1 teaspoon salt;
\\
\\
\textsc{Glaze --}
4 tablespoons (2 ounces) Dijon mustard;
2 tablespoons dark brown sugar;
2 teaspoons apple cider vinegar.}

Preheat oven to 350\deg F. In a food processor make bread crumbs from 2 or 3 slices of hearty white bread.  Empty food processor and reserve bread crumbs. Cut andouille in large chunks and place in food processor. Pulse food processor until andouille is coarsely chopped.

Heat oil in a large skillet over medium-high heat and add onions, bell peppers, garlic and a pinch of salt. Cover and cook  stirring occasionally until soft, around 5 minutes. Remove from heat and then add sage, thyme, pepper and salt. Stir to combine.  In as small bowl combine bread crumbs, milk, and Worcestershire sauce.  Once combined add beaten eggs and stir in. 

Put on food service nitrile gloves. Place chopped andouille, ground beef and ground turkey in a large bowl.  Mix the meats by hand until roughly homogeneous. Add onion mixture and bread crumb mixture to the meat mixture and knead by hand until well blended. Line a half sheet pan with foil and spray a metal loaf pan with cooking spray. Uniformly pack the meat loaf mixture into the loaf pan to form a loaf about 10 by 4 by 4 inches.  Turn out the loaf onto the sheet pan and remove the loaf pan. It may be necessary to slam the loaf pan onto the sheet pan to dislodge the meat mixture. Place loaf in oven and bake for 30 minutes.

While the loaf is in the oven combine the mustard, brown sugar and vinegar to make a glaze. After the first 30 minutes of baking, brush the glaze all over the loaf.

Bake the loaf for 30 to 45 minutes more until the internal temperature is 155\deg F.  Let loaf rest for 10 to 20 minutes, loosely covered with foil, before slicing and serving.

Source: Adapted from a recipe by Bruce Aidells and Denis Kelly in The Complete Meat Cookbook.

\newpage

\recipe{Beef Pot Roast With Tomato Sauce\ and Pearl Onions}
\ingredients{3 pound chuck roast;
salt and freshly ground black pepper;
2 tablespoons olive oil;
1 medium onion, finely chopped;
3 tablespoons of cider vinegar;
6 ounce can tomato paste;
1\half cups water;
\half teaspoon ground allspice;
\half teaspoon ground cumin seeds;
\quarter teaspoon sugar;
\threequarters cup frozen pealed pearl onions;
1 clove garlic, unpeeled;
2 bay leaves.}

Season roast with salt and pepper. Heat oil in a pressure cooker until very hot. Add roast and brown well on all sides.

Remove roast from cooker, add chopped yellow onions and cook until wilted.

Add vinegar, tomato paste, water, allspice, cumin and sugar to cooker then stir to mix well. Return roast to cooker and top with garlic, bay leaves, and pearl onions.

Close cooker and cook at 15 psi for 30 minutes. Release pressure and remove lid. Pick out garlic and bay leaves and discard. Remove roast from cooker and slice.

Serve roast with the tomato sauce and pearl onions. Excellent with spaghetti dressed with the tomato sauce and pearl onions then topped with slices of roast.

\newpage
\recipe{Garlic-Lime Grilled Pork Tenderloin Steaks}

\ingredients{2 (1-pound) pork tenderloins, trimmed;
1 tablespoon grated lime zest plus 1/4 cup juice (2 limes);
4 garlic cloves, minced;
4 teaspoons honey;
2 teaspoons fish sauce
\threequarters teaspoon salt;
\half teaspoon pepper; 
\half cup vegetable oil;
4 teaspoons mayonnaise;
1 tablespoon chopped fresh cilantro;
Flake sea salt (optional)}

Slice each tenderloin in half crosswise (perpendicular to the long axis) to create 4 steaks total. Pound each half to 3/4-inch thickness. Using sharp knife, cut 1/8-inch-deep slits spaced 1/2 inch apart in crosshatch pattern on both sides of steaks.

Whisk lime zest and juice, garlic, honey, fish sauce, salt, and pepper together in large bowl. Whisking constantly, slowly drizzle oil into lime mixture until smooth and slightly thickened. Transfer 1/2 cup lime mixture to small bowl and whisk in mayonnaise; set aside sauce. Add steaks to bowl with remaining marinade and toss thoroughly to coat; transfer steaks and marinade to large zipper-lock bag, press out as much air as possible, and seal bag. Let steaks sit at room temperature for 45 minutes.

Completely open the bottom vents of a charcoal grill. Light a large chimney starter filled with charcoal briquettes (6 quarts). When top coals are partially covered with ash, pour evenly over half of grill. Set cooking grate in place, cover, and open lid vent completely. Heat grill until hot, about 5 minutes.

Clean and oil cooking grate. Remove steaks from marinade (do not pat dry) and place over hotter part of grill. Cook, uncovered, until well browned on first side, 3 to 4 minutes. Flip steaks and cook until well browned on second side, 3 to 4 minutes. Transfer steaks to cooler part of grill, with wider end of each steak facing hotter part of grill. Cover and cook until meat registers 140 degrees, 3 to 8 minutes longer (remove steaks as they come to temperature). Transfer steaks to carving board and let rest for 5 minutes.

While steaks rest, microwave reserved sauce until warm, 15 to 30 seconds; stir in cilantro. Slice steaks against grain into 1/2-inch-thick slices. Drizzle with half of sauce; sprinkle with sea salt, if using; and serve, passing remaining sauce separately.

Yield: 4 to 6 servings

Source: America's Test Kitchen Season 15

\newpage
\recipe{Momofuku's Bo Ssam}

\ingredients{\textsc{Pork Butt --}
1 whole bone-in pork butt or picnic ham (8 to 10 pounds);
1 cup white sugar;
1 cup plus 1 tablespoon kosher salt;
7 tablespoons brown sugar;\\
\\
\textsc{Ginger-Scallion Sauce --}
2\half cups thinly sliced scallions, both green and white parts;
\half cup peeled, minced fresh ginger;
\quarter cup neutral oil;
1\half teaspoons light soy sauce;
1 scant teaspoon sherry vinegar;
\half teaspoon kosher salt, or to taste;\\
\\
\textsc{Ssam Sauce --}
2 tablespoons fermented bean-and-chili paste (ssamjang);
1 tablespoon chili paste (kochujang);
\half cup sherry vinegar;
\half cup neutral oil;\\
\\
\textsc{Accompaniments --}
2 cups plain white rice, cooked;
3 heads bibb lettuce, leaves separated, washed and dried;
1 dozen or more fresh oysters (optional);
Kimchi}


Place the pork in a large, shallow bowl. Mix the white sugar and 1 cup of the salt together in another bowl, then rub the mixture all over the meat. Cover it with plastic wrap and place in the refrigerator for at least 6 hours, or overnight.
    
When you're ready to cook, heat oven to 300\deg F. Remove pork from refrigerator and discard any juices. Place the pork in a roasting pan and set in the oven and cook for approximately 6 hours, or until it collapses, yielding easily to the tines of a fork. (After the first hour, baste hourly with pan juices.) At this point, you may remove the meat from the oven and allow it to rest for up to an hour.
    
Meanwhile, make the ginger-scallion sauce. In a large bowl, combine the scallions with the rest of the ingredients. Mix well and taste, adding salt if needed.
    
Make the ssam sauce. In a medium bowl, combine the chili pastes with the vinegar and oil, and mix well.
    
Prepare rice, wash lettuce and, if using, shuck the oysters. Put kimchi and sauces into serving bowls.
    
When your accompaniments are prepared and you are ready to serve the food, turn oven to 500\deg F. In a small bowl, stir together the remaining tablespoon of salt with the brown sugar. Rub this mixture all over the cooked pork. Place in oven for approximately 10 to 15 minutes, or until a dark caramel crust has developed on the meat. Serve hot, with the accompaniments.

Yield: 6 to 10 servings

Source: Adapted by the New York Times from ``Momofuku'' by David Chang and Peter Meehan.



\newpage
\recipe{Cottage Pie with Cauliflower Topping}
\equilpment{immersion blender; food processor, 7 quart cast-iron dutch oven; 9 inch by 12 inch baking dish (Le Creuset 40) or 6 5-inch ramekins;}

\ingredients{\\
\\
\textsc{Topping --}
1 medium head cauliflower, cut into large pieces;
2 tablespoons extra-virgin olive oil; 
\threequarters teaspoon salt; 
\quarter teaspoon white pepper;
15 ounce can cannellini beans, drained and rinsed.\\
\\
\textsc{Filling --}
1 large onion, pealed and quartered;
2 cloves garlic, pealed;
1 large fennel bulb, cut into large pieces;
1 6-ounce can tomato paste;
\half cup water;
1 \half teaspoons salt;
\quarter teaspoon black pepper;
1 tablespoon cayenne-based hot sauce;
1 tablespoon Worcestershire sauce;
2 teaspoons Maggi Liquid Seasoning;
1 teaspoon extra-virgin olive oil;
1 \half pounds 90\% lean ground beef;
8 ounces button or cremini mushrooms, sliced.
}

Preheat oven to 375\deg F.

To make topping, place cauliflower in a pot and then add water just to cover. Bring pot to boil over high heat, then reduce the heat to medium, and cook until cauliflower is very tender, about 10 minutes. Drain cauliflower, return it to the pot, and add the olive oil, salt, pepper, and beans.  Pur\'{e}e with an immersion blender until smooth. Set aside until assembly.

To make filling, first make two components that are added during the cooking phase. For the first component, place onion, garlic, and fennel in a food processor and pulse until finely chopped.  For the second component, stir the tomato paste and water together in a small bowl. Add the salt, pepper, hot sauce, Worcestershire sauce, Maggi seasoning and whisk to combine. Now heat olive oil in a large cast-iron dutch oven.  When hot add beef and brown.  When beef is well browned add onion mixture and mushrooms and cook until mushrooms are softened but not mushy.  Add tomato paste mixture and cook for a few more minutes to combine.

To assemble, transfer beef mixture to the baking dish and top with pureed cauliflower mixture.  Bake for 20 to 30 minutes until the casserole is bubbling.

Makes 6 servings. 

Source: Adapted from a recipe in Always Hungry? by David Ludwig

\newpage

\recipe{Steak with Fish Sauce}

\ingredients{1 thin-cut boneless strip steak, \half inch thick (about 7 to 8 ounces);
1 pinch salt and pepper;
1 tablespoon grapeseed oil (or other oil with a high smoke point);
1 small handful green beans (optional);
1 lime (\half juiced,\half wedged for serving);
1 teaspoon fish sauce;
\half teaspoon dark brown sugar;
1 tablespoon cold butter.}

\textsc{With the green beans, an excellent quick meal for one.}
    
Pat steak dry with paper towel and season both sides with salt and pepper. Heat a large skillet until smoking slightly, then add oil followed by the steak. Pressing down with tongs, cook for 2 minutes on the first side and 1 minute on the second side, or until medium-rare.
    
Plate the steak, then throw in green beans and toss in the fat, cooking for about 1 minute and 30 seconds. Plate the beans alongside the resting steak.
    
Take the pan off the heat, then add lime juice, fish sauce, brown sugar, and butter. Stir until the butter is incorporated. Pour the sauce all over the steak and beans and serve with the remaining lime wedges. Enjoy with a glass of red wine.




\chapter{Pasta}
\recipe{Lemon-Basil Orzotto}
\ingredients{2 tablespoons extra-virgin olive oil;
1 cup diced onion;
1\half cups orzo or pearl barley;
\half cup dry white wine;
3 cups chicken stock;
\half cup frozen petite green peas;
\third cup grated Parmesan cheese;
2 tablespoons chiffonade fresh basil (or 2 teaspoons pesto);
1 teaspoon lemon zest;
\quarter cup heavy cream;
Juice of 1 lemon;
Salt and freshly ground black pepper;
Optional: 2 boneless skinless chicken thighs cut in a \half inch dice.}

In a heavy-bottomed medium saucepan, heat the olive oil over medium-high heat. 
Add the onion and saut\'{e} until fragrant and translucent. Add the orzo and toast for 
2 minutes, stirring occasionally. Add the wine and cook until absorbed.

Gradually add the chicken stock, stirring frequently. Bring to a simmer, 
lower the heat, and cover. If using the optional chicken add it at this point. 
Cook for 8 to 10 minutes, until the liquid is almost absorbed and orzo is tender. 
Remove from the heat.

Stir in peas, Parmesan, fresh basil, lemon zest, heavy cream, and lemon juice. 
Season the orzo with salt and pepper, to taste, and serve.

Source: Melissa d'Arabian 

\newpage
\recipe{Marinara}

\ingredients{260 grams yellow onion, large dice;
160 grams carrots, medium dice;
18 grams garlic;
20 grams (1\half tablespoons) olive oil;
1 large can (794 grams) Italian crushed tomatoes;
extra-virgin olive oil;
salt.}

Mince the onions, carrots, and garlic in food processor.

Add the olive oil to the base of a pressure cooker, and saut\'{e} the minced vegetables over medium heat until the onions become translucent, about 4 minutes.

Stir the tomatoes into the saut\'{e}ed vegetables.

Pressure-cook the vegetables at 15 psi for 45 minutes. Start timing as soon as full pressure has been reached.

Depressurize cooker, then season the sauce to taste with extra-virgin olive oil and plenty of salt.

From Modernist Cuisine at Home.

\newpage
\recipe{Butter Tomato Sauce}

\ingredients{2 cups peeled, seeded, and roughly chopped plum tomatoes, in addition to enough of their juices to cover or a 28-ounce can of San Marzano whole peeled tomatoes;
5 tablespoons butter;
1 onion, peeled and cut in half;
Salt.}

Combine the tomatoes, their juices, the butter and the onion halves in a saucepan. Add a pinch or two of salt.

Place over medium heat and bring to a simmer. Cook, uncovered, for about 45 minutes. Stir occasionally, mashing any large pieces of tomato with a spoon. Add salt as needed.

Discard the onion then emulsify sauce in pan using an immersion blender. 

Yields: 2 cups -- enough for one pound of pasta and 4 servings

Source: Marcella Hazan via the New York Times

\newpage
\recipe{Mississippi Butter-Tomato Sauce}
\label{MStomatoSauce}



\equilpment{13\quarter quart Le Creuset Dutch Oven.}

\ingredients{1 25-pound box end-of-summer Mississippi tomatoes;
6 sticks plus 2 tablespoons (50 tablespoons) butter;
10 medium-sized yellow-onions, peeled and cut in half;
1\half teaspoons table salt.}

\textsc{In her classic recipe for Butter-Tomato Sauce, Marcella Hazan uses plum tomatoes.  This is a sensible choice since plum tomatoes have a high flesh to juice ratio and are intended for cooking. In Mississippi plum potatoes are not common at farmers' markets. However, at the end of the summer Mississippi has abundant and beautiful tomatoes that are perfect for eating fresh. These tomatoes can be used to make an excellent Butter-Tomato Sauce but because of their high juice to flesh ratio additional steps are required to remove their excess juice before they can be used for sauce making.}

Using a large pot of boiling water and an ice bath, peel the tomatoes.  When the peeled tomatoes are cool enough to handle cut them into quarters removing any green stem or green flesh at the same time. Squeeze each quarter by hand to remove the seeds and excess juice saving the juice and seeds separately from the tomato flesh. This process should yield 20 cups (5 quarts) of tomato flesh (adjust the a mounts of the other ingredients proportionally if you have more or less tomato flesh). Strain the juice from the seeds and save in case additional liquid is needed later in the recipe.

Add the tomato quarters to a Dutch oven or stock pot of at least 12 quart capacity. Check the liquid level and, if necessary, add reserved strained juice to bring the level of the liquid to just below the level of the tomato solids. Combine the butter, salt and the onion halves with the tomatoes.

Place over medium heat and bring to a simmer. Cook, uncovered, for about 45 minutes. Stir occasionally, mashing any large pieces of tomato with a spoon. Taste and add salt as needed.

Discard the onion then emulsify sauce using an immersion blender. Package sauce in 2-cup portions and freeze for later use.

Yields: Ten 2-cup portions. Each portion is enough for one pound of pasta and 4 servings.

\newpage
\recipe{Lemon-Butter Pasta with Parmesan}

\ingredients{
    8 tablespoons (1 stick) unsalted butter;
    Finely grated zest from 2 large lemons (2 tablespoons);
    Kosher salt;
    1 pound fettuccine;
    1 cup freshly grated Parmesan cheese, plus more for optional garnish;
    Freshly ground black pepper
}

In a Dutch oven or skillet large enough to hold all the pasta, melt the butter over medium heat. Add the lemon zest and swirl to mix. Remove from heat.

Bring a large pot of salted water to a boil, then add the pasta and cook until al dente, about 10 minutes. Drain the pasta, reserving 1\half cups of the pasta water.

Return the butter to medium heat, then whisk in the reserved pasta water until combined, about 2 minutes. Add the Parmesan in a couple of handfuls, whisking until emulsified, 3 to 4 minutes. Season to taste with salt and pepper. Add the pasta to the sauce; toss and stir until the noodles are glossed with sauce, 3 to 5 minutes.

Serve with more black pepper and Parmesan.

Yields: 4 to 6 servings

Source: Ali Slagle from The Washington Post

\chapter{Pizza}
\recipe{Neapolitan Pizza Dough}

\ingredients{500 grams Antico Caputo brand 00 wheat flour;
310 grams water;
10 grams salt;
10 grams agave syrup;
2.5 grams (1\half teaspoons) Bob's Red Mill brand vital wheat gluten;
2.5 grams (\threequarters teaspoon) Active dry yeast;
Neutral oil as needed.}

Mix flour, water, salt, agave syrup, and yeast in the bowl of a stand mixer with a dough hook until fully incorporated.

Mix on medium speed (4 on KitchenAid mixer) for 5 minutes.

Let the dough rest in the bowl for 10 minutes at room temperature, and then mix again for another 5 minutes at medium speed.
Allowing the gluten to relax between kneading creates an especially smooth, stretchy dough.

Transfer the dough to a well-floured surface, and cut into four equal weight chunks. Stretch and roll the dough into smooth, even balls.
Stretching the surface layer develops a network of gluten that traps air inside the ball, resulting in a lighter, more blistered crust.

Coat the balls lightly with oil, cover them with plastic wrap, and allow them to rise at room temperature for 1 hour before using. For a more complex flavor,
refrigerate the dough balls overnight, and then let them rest at room temperature for 1 hour before using.

Yields enough dough for four 12-14 inch pizzas.  The dough keeps for 3 days when refrigerated in plastic wrap or up to 3 months when frozen.

Source: Modernist Cuisine at Home.

\newpage
\recipe{Pizza Sauce}

\ingredients{50 grams (\quarter cup) chopped garlic;
50 grams (\half cup) olive oil;
1 large can (794 grams) Italian crushed tomatoes;
extra-virgin olive oil;
salt.}


Add the olive oil to the base of a pressure cooker and then fry garlic until golden brown, about 5 minutes.

Stir the tomatoes into cooker.

Pressure-cook the vegetables at 15 psi for 45 minutes. Start timing as soon as full pressure has been reached.

Depressurize cooker, then season the sauce to taste with extra-virgin olive oil and plenty of salt.

Source: Modernist Cuisine at Home.

\recipe{Basic Pizza}
\label{BasicPizza}
\equilpment{pizza steel; pizza peal}

\ingredients{1 ball, approximately 200 grams, Neapolitan Pizza Dough; \half cup Pizza Sauce; 2 ounces fresh mozzarella; 1\third cup shredded low-moister whole-milk mozzarella; 2 tablespoons grated hard cheese; 2 teaspoons Mediterranean oregano; wheat flour or course ground corn meal for peal; extra virgin olive oil, basil oil, or paprika oil (optional)}

Place a oven rack in the middle of the oven with no rack above it.  Place the pizza steel on the rack and then set oven to 550\deg F. After the oven heats to 550\deg F preheat the oven and steel for at least one hour.

While the steel heats, press as much moister as possible from the fresh mozzarella by wrapping it with a clean cloth or paper towel and placing a brick, can or other heavy object on top of it and leaving it to press for a while.

\newpage
\recipe{Basil Oil}
\label{BasilOil}
\equilpment{immersion blender}

\ingredients{\half cup extra virgin olive oil; 10 to 15 large fresh basil leaves, zest and juice from \half lemon, salt, freshly ground black pepper}

Combine all the ingredients except the salt an pepper in the container of an immersion blender.  Blend until completely liquefied then season to taste with salt and pepper. Transfer to a squeeze bottle for use. Keeps in the refrigerator for several weeks.


\recipe{Paprika Oil}

\ingredients{\third cup extra-virgin olive oil;
1\half teaspoons smoked (hot or sweet) paprika;
\half teaspoon red-pepper flakes.}

Heat olive oil in a small skillet over the lowest heat on your smallest burner. Add the paprika and red-pepper flakes and cook, stirring frequently, just until toasted and flavors bloom, 1 to 2 minutes. Strain oil, discarding flakes.

\chapter{Dessert}


\recipe{Buttermilk Chess Pie}

\ingredients{1/2 cup butter;
2 cup sugar;
2 tablespoons all-purpose flour;
1 teaspoon vanilla;
5 large egg, lightly beaten;
2/3 cup buttermilk;
1 unbaked 9-inch pastry shell}

\textsc{This pie was the winner of the first-ever Oxford, Mississippi Mid-Town
Farmers' Market pie contest. The winning cook, Ms. Izetta
Barrington, submitted this recipe for her pie.}

In mixing bowl combine sugar and flour.
Add eggs and buttermilk, stirring until blended.
Stir in butter and vanilla.

Pour mixture into the pastry shell.
Bake in 350\deg F oven for 45 minutes, or until set.
Cool on a wire rack.


\newpage



\recipe{Chocolate Chess Pie}

\ingredients{\half cup butter;
1 cup sugar;
\quarter cup flour;
1 teaspoon vanilla;
\eighth teaspoon salt;
3 egg yolks;
1\half ounces unsweetened chocolate or 4\half tablespoons cocoa + 1\half tablespoons butter ;
\half cup evaporated milk;
1 pastry pie shell.}

In mixing bowl combine butter and sugar and beat well.

Add flour, vanilla, melted chocolate or cocoa plus additional melted butter, salt, and beat in egg yolks and milk.

Pour mixture in the pie shell and bake in a 425\deg F oven for 10 minutes.  Reduce the temperature to 300\deg F and cook for an additional 20 to 25 minutes.  The pie should be firm in the center so that a tooth pick inserted in the center comes out clean.

Serve cool topped with whipped cream. Can be made without the coco or chocolate for a traditional chess pie.

\newpage
\recipe{Blueberry Pie Filling}

\ingredients{10 to 11 tablespoons minute tapioca;
3 teaspoons ground cinnamon;
1 teaspoon salt;
2\half cups sugar;
8 cups blueberries (if frozen thaw temperature first if fresh, crush);  
8 tablespoons butter, melted;
3 teaspoons vanilla extract.}

Mix together flour, cinnamon, salt and sugar.

Add blueberries and toss lightly.

Stir in melted butter and vanilla, mixing until berries are well coated.

Make in a 9-inch pie pan with a bottom and top crust.

Line bottom  rack of oven with foil to catch drips.

Bake in preheated 450\deg F oven for 15 minutes, then reduce temperature to 375\deg F
and bake 25 to 30 minutes more.

Yields filling for two 9-inch pies.

\newpage
\recipe{Flaky Pie Crust}
\label{FlakyPieCrust}

\ingredients{3 cups unbleached all-purpose flour;
1 teaspoon salt;
\quarter teaspoon baking powder;
1\quarter cups shortening;
1 egg;
4 to 6 tablespoons ice cold water;
1 tablespoon white vinegar.}

Combine dry ingredients and cut in shortening.

In a separate bowl, beat together egg, 4 tablespoons water, and vinegar.

Add egg mixture to flower mixture and blend together well.

Add 1 to 2 more tablespoons of water if needed.

Wrap dough and chill before rolling out.

Divide dough into quarters and roll out on floured surface or between wax paper
to desired size.

Yields four 9-inch crust.

\newpage

\recipe{Pound Cake}

\ingredients{2 sticks butter;
2 cups sugar;
6 whole eggs;
2 cups all-purpose flour;
dash of salt;
2 teaspoons vanilla extract;
1 teaspoon almond extract}

Before beginning, bring all ingredients to room temperature. Preheat oven to 350\deg F. 

Beat the butter in the bowl of a stand mixer. Add the sugar and salt and beat thoroughly. Add \third cup flour and one egg at a time, beating thoroughly after each addition until all the flour and eggs are added.  Add vanilla and almond extracts and mix until blended.

Pour into a Bundt pan. Line the pan with slivered almonds if desired.

Bake for one hour.

Source: Lucie Krueger
\newpage


\recipe{Scandinavian Almond Cake}

\ingredients{1\quarter cup sugar;
1 egg;
1\half teaspoon almond extract;
\twothirds cups milk;
1\quarter cup flour;
\half teaspoon baking powder;
1 stick melted butter.}

\textsc{This recipe uses a special fluted half-cylinder pan.  Mine is aluminum 
and was made in Germany. Due to the ridges in cake pan, slicing is a 'piece of cake'.}

Beat together sugar, egg, extract and milk. Stir in flour and baking powder.
Then add butter and mix well.

Spray cake pan with cooking spray. Sprinkle slivered almonds into bottom of pan, if desired, then  pour batter into a greased pan. 

Bake at 350\deg F for 40 to 50 minutes or till edges are golden brown. 

Cool cake in pan completely before removing. To remove place inverted 
Almond Cake Plate to top of cake pan and then flip. Dust with confectionery 
sugar if desired.



 

\newpage
\recipe{Nanaimo Bars}

\ingredients{\half cup butter;
5 teaspoons cocoa;
2 teaspoons vanilla extract;
1 egg, beaten;
;
2 cups graham cracker crumbs;
\half cup nuts, chopped;
1 cup shredded coconut;
;
\half cup butter;
2 Tablespoons milk;
2 teaspoons vanilla extract;
2 cups confectioners sugar;
2 teaspoons vanilla pudding mix;
;
2 teaspoons butter;
4 squares semi sweet chocolate, melted}

Combine first five ingredients in a saucepan, mix and then heat until slightly thick. To this add the graham cracker crumbs, nuts and coconut.
Mix well and pack in a pan and chill in refrigerator until cool.

Make a second layer by combining second \half cup of butter, milk, 2 teaspoons of vanilla extract, vanilla pudding mix. Spread on first layer then return to refrigerator until chilled. Melt chocolate and butter together, mix and pour over second layer then chill until firm.

Cut into bars and serve.

\newpage
\recipe{Blueberry Crisp}
\ingredients{\\
\\
\textsc{Filling --} 4 generous cups blueberries, cleaned and stemmed;
\half cup sugar;
\quarter cup unbleached all-purpose flour;
\quarter teaspoon salt;
2 teaspoons lemon juice;\\
\\
\textsc{Topping --} 1\half cups unbleached all-purpose flour;
\quarter teaspoon salt;
\half cup sugar;
1\quarter sticks butter, melted;
1 cup pecans, chopped.}

Grease and flour a 9-inch pie pan. Preheat the oven to 350\deg F.

To make filling, put the berries in the pie pan. Mix the sugar, flour, salt, and lemon juice together, and sprinkle this mixture over the berries. 

For the topping, stir together the flour, salt, sugar, melted butter, and nuts in a medium mixing bowl. Sprinkle the topping over the fruit.

Bake the crisp for 45 to 50 minutes, until the top is golden and the filling is bubbly. Cool slightly, then serve warm, with vanilla ice cream or whipped cream.

Source: The King Arthur Flour Baker's Companion.

\newpage
\recipe{Pear and Almond Tart}
\label{PearTart}
\equilpment{Deep steel tart pan approximately 9" in diameter at the top and 2" deep}
\ingredients{\\
\\
\textsc{Poached Pears --} 3 ripe but firm pears;
\half cup (100 grams) sugar;
2 cups water.\\
\\
\textsc{Tart Pastry --} 1\quarter cups all purpose flour;
pinch of salt;
\half cup (1 stick) unsalted butter, cut into \quarter inch slices;
1 to 3 tablespoons ice cold water.\\
\\
\textsc{Almond Filling --} 1 stick (\half cup or 100 grams) unsalted butter, softened or melted;
1 cup (100 grams) almond flour;
\half cup (100 grams) sugar;
1 egg;
1 tablespoon all purpose flour;
1 tablespoon dark rum or kirsch.}

To poach the pears first make a syrup by bring the water and sugar to boil in a saucepan. Peel pears, cut in half, then remove stem and seeds. Poach pears in simmering syrup until cooked through but still firm enough to hold together. Remove pear halves from syrup and allow to cool. Cook down syrup until reduced to a glaze then cool and reserve glaze. 

Preheat the oven to 425\deg F.

To make the tart pastry, place the flour and salt in the bowl of a food processor fitted with the steel blade. Cut in the butter in slices, and buzz several times until the mixture is uniform and resembles coarse meal.  Continue to process in quick spurts as you add water, 1 tablespoon at at time. As soon as the dough adheres to itself when pinched together, stop adding water, turn it out onto a floured surface, and push it together into a ball. 

Butter the tart pan then roll the dough into an 11-inch or so circle. Drape the dough over your rolling pin and the lift the dough into the tart pan, nudging it gently into the corners, and use your hands to form an even edge all the way around. Prick the dough all over with the tines of a fork. Wrap tightly and store in the refrigerator until assembly. 

To make the filling, us a mixer to mix butter, sugar, and almond flour, until the dough is smooth, add the rum, then the beaten egg. Work until the mixture is homogeneous, then add the flour and mix completely.

To assemble tart place poached pears radially around the center of the tart pan with the cut sides down and the stems pointed at the center. Pour the filling evenly around the pears leaving most of the pear exposed. Bake for 45 minutes then remove from oven and pour reserved glaze evenly over tart.  The tart will absorb the glaze after a few minutes so you can continue to pour over glaze until it is all used.  Let cool to room temperature before removing from the pan and serving.

\newpage

\recipe{Mighty Mousse}

\ingredients{1\half cups semisweet chocolate chips;
1 tablespoon rum or cognac;
2 large eggs;
1\half cups of half and half}

Heat half and half in sauce pan until scalding hot.

Combine all ingredients except half and half in a blender and mix briefly.

Add scalding hot half and half and blend thoroughly.

Pour into dishes, cover with plastic wrap or foil and chill until set.

Serve garnished with whipped cream and mint or shaved chocolate.  It is also good garnished with cherries that result from making \hyperref[CherryLiqueur]{Cherry Liqueur}.

Source: Too Busy to Cook column of Bon Appetite magazine

Serves: 6

\recipe{Poppy Seed Cake}

\ingredients{\half pound butter; 1\half cups sugar; 4 eggs, yolks and whites separated; \half pint sour cream; 2 cups flour, shifted; 3 teaspoons vanilla extract; 2 ounces dry poppy seeds.}

Preheat oven to 350\deg F. In a medium mixing bowl, beat the egg whites to stiff peaks.

In a large mixing bowl, cream the butter, sugar and egg yolks. Add sour cream and incorporate.  To this add the flour, vanilla extract and poppy seeds and mix well. Carefully fold in the egg whites.

Transfer the mixture to a Bundt pan and bake for about 1 hour.


\chapter{Jellies, Conserves, and Pickles}

\recipe{Sweet Pickles}
\label{SweetPickles}

\ingredients{7 pounds of cucumbers (or yellow squash or green tomatoes) sliced 3/8" thick;
2 gallons water;
1 cup pickling lime (food grade calcium hydroxide);
\half gallon White House vinegar;
6 pounds sugar;
2 tablespoons table salt (not iodized);
1 tablespoon pickling spice;
1 teaspoon whole cloves.}


Place cucumbers in a solution made from the water and pickling lime. 
Soak cucumbers in the solution for 24 hour stirring often.

Drain and rinse cucumbers and then soak for 4 hours in clean water.  
Drain and rinse cucumbers again.

Make a pickling solution by heating the remaining ingredients (do not boil).

Place the cucumbers in the pickling solution and heat to just simmering.  
Let cool for 4 hours then repeat simmering and cooling until pickles are clear.

Bring pickles to a simmer again and then hold at a simmer for 35 minutes.

Seal hot pickles in sterilized jars. Yields 12 pints.

\newpage
\recipe{Pear Brandy Conserve}

\ingredients{3\half cup ground pears;
\half cup slivered almonds;
\half cup seedless golden raisins;
\half teaspoon butter (to reduce foaming);
\quarter teaspoon cinnamon;
\quarter teaspoon grated lemon rind;
2 tablespoons lemon juice;
5 cups sugar;
1 box Sure Jell brand fruit pectin;
\half cup brandy.}


Peal and core pears then grind with a meat grinder or
grate using a food processor.

Combine ground pears, almonds, butter, raisins, lemon
rind, lemon juice and cinnamon in a tall 8-quart saucepan (6
inches tall and 9 inches in diameter).

Measure sugar and set it aside.

Mix Sure Jell into pear mixture then cook on high, stirring
constantly, until it reaches a full rolling boil. A full rolling
boil cannot be stirred down. It may be necessary to add some water
(up to a cup or so) to the mixture.

After the mixture has reached a full rolling boil, add the pre-measured sugar all at once.

While continuing to stir, bring the mixture back to a
rolling boil and then cook at a rolling boil for one additional
minute.

Remove mixture from heat, add brandy and seal in sterilized jars.

Makes seven half pints (7 cups).  Do not attempt to double this
recipe as it will not boil correctly.  If you want to make more
that seven half pints, boil the amount for seven half pints
repeatedly. Be sure to clean spoon and pan of all sugar between
batches.

\newpage
\recipe{Muscadine Hull Sauce}

\ingredients{very ripe muscadines to yield 19 \threequarters pounds of hulls and pulp once the seeds are removed;
18 pounds sugar;
1 quart cider vinegar;
6 tablespoons ground cinnamon;
3 tablespoons ground mace;
3 tablespoons ground cloves.}

Wash and pulp grapes, separating hulls and pulp.  Cook hulls until tender then grind hulls using a meat grinder with the coarse plate.  Cook pulp until tender and seeds start to separate. Completely the separate of the seeds from pulp by running the pulp through a colander to the remove seeds.  Mix ground hulls and deseeded pulp.  Weigh out 19 \threequarters pounds of hulls and pulp mixture.  Mix in vinegar, cinnamon, mace and cloves.

Boil the mixture until it reads 220\deg F on a jelly and candy thermometer.

Seal in sterilized jars. Yields 28 pints.

For preserves leave out vinegar and spices and add water to replace the vinegar.

\newpage
\recipe{Pepper Jelly}

\ingredients{2\half cups chopped bell peppers;
\half cup chopped hot peppers;
2 cups water;
6\half cups sugar;
2 cups apple cider vinegar;
\quarter teaspoon salt;
2 bottles Certo brand liquid pectin;
green or red liquid food coloring.}

Boil bell peppers and hot peppers until tender then mash the pulp and strain it through a jelly bag and reserve liquid. 

Combine 2 cups of reserved liquid, sugar, vinegar, and salt then boil for 5 minutes.

Take off heat and stir in 2 bottles of liquid pectin and green or red color.  Return to stove and bring to boil.  

Pour quickly into hot sterilized jelly glasses and cover with paraffin.

Yields 10-11 jelly glasses.


\chapter{Breads}
\recipe{Whole Wheat Sandwich Bread}

\equilpment{KitchenAid Mixer; Two 1\quarter quart Le Creuset Stoneware Deep-Dish Loaf Pans.}

\ingredients{6 to 6\half cups whole wheat flour;
2\half cups warm water between 105-110\deg F;
1\half tablespoons instant active dry yeast; 
\third cup honey;
\third cup oil;
2\half teaspoons salt.}

Combine water, yeast and 2 cups of the flour in the mixer's bowl. Set aside to sponge for 15-20 minutes, until risen and bubbly.

Add honey, oil, and  salt to the bowl.  Mix using the flat beater while adding a total of 4 cups of flour in \quarter cup increments. After adding the flour, continue mixing until the dough starts to clean sides of bowl. Change to dough hook and knead 6 to 7 minutes. Add a few tablespoons of flour at a time if dough sticks to sides.

Form into two loaves and place in greased loaf pans. Allow to rise in a warm place for about 60 minutes, until 1/2 to 1-inch above pans. Preheat oven to 350\deg F ten minutes before rising time is done.

Bake for 30 minutes, rotating halfway through if needed. Remove loaves from pans and return to oven for an additional 10 minutes to brown.

Allow  loaves to completely cool on racks before slicing.

Source: Adapted from a recipe on the website ``An Oregon Cottage.''

\newpage
\recipe{Sweet Potato Biscuits}

\ingredients{1 cup sweet potato puree;
2 cups self rising flour;
4 tablespoons of vegetable shortening or lard;
buttermilk.}

Preheat oven to 425\deg F.

Mix together sweet potato puree, flour, and shortening then add just enough buttermilk to make the right consistency for a soft dough.

Turn out dough and knead for about 30 seconds.

Roll out dough to the desired thickness on a lightly floured cloth or board.

Cut with a floured biscuit cutter and place on a greased baking sheet.

Cook until brown and done.  Makes around 10 biscuits.

\newpage
\recipe{Southern Cornbread Sticks}
\label{cornbread}    

\ingredients{\\
\textsc{Self-Rising White Cornmeal Mix} 
1\half cups regular white cornmeal;
6 tablespoons all-purpose flour;
2 tablespoons baking powder;
1 teaspoon granulated salt.\\
\\
\textsc{Cornbread Batter}
1 egg, beaten;
1\threequarters cups buttermilk;
\quarter cup melted Crisco shortening;
2 cups Self-Rising White Cornmeal Mix.}


Preheat oven to 450\deg F. Grease two cast iron cornstick pans very well with room 
temperature Crisco and place in the oven to heat.

In a large bowl, add cornmeal mix ingredients and combine. Instead of homemade cornmeal mix you can substitute 2 cups of Martha Whites Self-Rising White Cornmeal Mix.

Combine egg, buttermilk and corn meal mix and mix well.  
To this add the melted shortening and mix well again.

When the shortening in pans is beginning to smoke, remove pans from oven 
and fill each mold space \twothirds full.

Return pans to oven and bake cornsticks for 15 to 20 minutes
until they are golden brown.

Recipe will work with a cast iron muffin pan using the same instructions 
yielding 12 muffins.  The recipe will also work with a 8 or 9 inch cast iron 
skillet, however the baking time is then 20 to 25 minutes.

Yields 16 cornsticks.
\newpage

\recipe{Hamburger Buns}

\label{HamburgerBuns}

\ingredients{\threequarters cup lukewarm water;
2 tablespoons butter;
1 large egg;
3 \half cups King Arthur Unbleached All-Purpose Flour;
\quarter cup sugar;
1\quarter teaspoons salt;
1 tablespoon instant yeast}

In a heavy duty mixer with the dough hook attached, mix and then knead all of the dough ingredients to make a soft, smooth dough.

Cover the dough, and let it rise for 1 to 2 hours, or until it's nearly doubled in bulk.

Gently deflate the dough, and divide it into 8 pieces. Shape each piece into a round ball; flatten to about 3 inches across. Place the buns on a silicon baking mat in a half sheet pan, cover, and let rise for about an hour, until noticeably puffy.

Bake the buns in a preheated 375\deg F oven for 15 to 18 minutes, until golden.
Cool the buns on a rack.

Brushing buns with melted butter will give them a soft, light golden crust. Brushing with an egg-white wash (1 egg white beaten with 1/4 cup water) will give them a shinier, darker crust. For seeded buns, brush with the egg wash; it'll make the seeds adhere. And, feel free to add the extra yolk to the dough, reserving the white for the wash.

Yield: 8 large buns. 


\chapter{Breakfast}

\recipe{Banana Waffles}

\ingredients{2 cups sifted flour;
3 teaspoons baking powder;
1 teaspoon salt;
2 eggs;
1\quarter cups buttermilk;
\half cup melted butter;
1 cup mashed very ripe bananas.}

These are best with bananas that have become over ripe and then put in the freezer for a few days. Defrost bananas before using.

Preheat oven to 200\deg F. Preheat waffle iron. In the bowl of a stand mixer, sift together flour, baking powder, and salt. 

Beat eggs slightly and add buttermilk. Add egg mixture to flour mixture in bowl of stand mixer and mix well.

Add melted butter and well-mashed bananas and mix in completely.

Add 1 cup of batter to iron for Belgian waffles. Close iron and then cook until brown and crisp. Place cooked waffles on oven grates to further crisp and keep warm while making the remaining waffles.

Makes 5 waffles. 

\newpage
\recipe{Granola} 

\equilpment{2 half sheet-pans; wide heavy-duty aluminum-foil; heavy meat-pounder.}

\ingredients{\twothirds cup cane or maple syrup;
7 fluid ounces vegetable oil;
2 tablespoons vanilla extract;
1 teaspoons salt;
1 32-ounce bag regular rolled oats;
1 12-ounce bag chopped pecans or almonds;
1 cup unsweetened dried coconut flakes (optional).}


Preheat oven to 350\deg F. In a large bowl mix together the first five ingredients.  
To this add the oats, nuts, and coconut (if using) and mix thoroughly. 

Line two half sheet-pans with foil, and then spray foil lightly with cooking spray. Distribute the mixture evenly between the pans, and then force the mixture into a compact layer using a 
large flat pressing tool such as a meat-pounder. Cook for 40 minutes reordering and turning the sheet pans once.

Cool the mixture for one hour then break into large pieces. At this point you can
add 3 cups of raisins or other dried fruit if desired. 

\recipe{Oats in a Jar}
\ingredients{\half cup oats; 1 date, roughly chopped; 1\half tablespoons dried cranberries; 1 tablespoon shredded coconut; 1 tablespoon chi seeds; 1 generous tablespoon blanched almonds; \threequarters cup milk.}

Combine ingredients in a pint jar then cover with lid and store overnight in refrigerator before eating.

\newpage
\recipe{Yogurt}
\ingredients{6 cups 1 percent milk; \third cup nonfat dry milk; one package of yogurt starter or \third cup yogurt reserved from previous batch.}

Starting with milk from the refrigerator, microwave the milk in an 8 cup glass measuring cup for 10 to 12 minutes until it is 180\deg F. 
Remove from microwave and whisk in dry milk.

Cover the measuring cup with a dishtowel and leave to cool at room temperature for around one hour until the milk mixture cools to between 110 and 100\deg F. Whisk in starter or reserved yogurt. Place in yogurt maker and process for 8 hours. At the end of 8 hours transfer yogurt to storage container and refrigerate. 

\newpage

\recipe{Black Bean Tofu Hash}
\ingredients{1 tablespoon extra-virgin olive oil;
1 clove garlic, minced;
16-ounce box extra-firm tofu, drained and gently pressed;
1 tablespoon chili powder;
\half teaspoon table salt;
\quarter teaspoon ground black pepper;
2 tablespoons water;
15-ounce can black beans, drained;
\half cup chopped fresh  cilantro.}

\textsc{For a quick and easy breakfast serve with sliced avocado, sour cream, and salsa or \hyperref[RancheroSauce]{Ranchero sauce}.}

Heat the olive oil in a large skillet over medium heat.  When the oil begins to ripple add the garlic and cook until lightly browned, around 30 seconds. Crumble in the tofu and sprinkle with the chili powder, cumin, salt and pepper. Saut\'{e} for 2 to 3 minutes.  Add water, stirring to allow the tofu to absorb the seasonings and water. Stir in the black beans and saut\'{e} for 2 or 3 minutes more. Add the cilantro and adjust the seasonings to taste.

Makes 4 servings.

Source: Adapted from a recipe in Always Hungry? by David Ludwig

\newpage
\recipe{Baked Egg Squares} 

\equilpment{8-inch square metal baking pan (must be metal); quarter sheet pan.}

\ingredients{8 large eggs; \quarter teaspoon table salt; \twothirds cup water; vegetable oil spray.}

Adjust oven rack to middle position and heat oven to 300 degrees. Whisk eggs and salt in large bowl until well combined. Whisk in \twothirds cup water. Spray 8-inch square baking pan with vegetable oil spray. Pour egg mixture into prepared pan and set pan on rimmed baking sheet. Add 1\half cups water to sheet. Transfer sheet to oven and bake until eggs are fully set, 35 to 40 minutes, rotating pan halfway through baking. 

Remove pan from sheet, transfer to wire rack, and let cool for 10 minutes. When the eggs have cooled, run a knife around edges of pan and, using dish towel or oven mitts, invert eggs onto cutting board (if eggs stick to pan, tap bottom of pan firmly to dislodge). Cut into 4 equal squares. 

For hot applications like sandwiches, serve immediately. Alternatively, let the eggs cool completely after cutting them, stack them in an airtight container, and refrigerate them for up to three days. To reheat, arrange the egg squares on a large plate and microwave at 50 percent power until they're warm, 2 to 3 minutes.

These baked eggs can also be served a room temperature. For example, cut the squares into 4 pieces and top with a mayonnaise/pickle combination to make a deviled egg variation.

Source: Taken from the America's Test Kitchen recipe for Egg, Kimchi, and Avocado Sandwiches.

\chapter{ From New Mexico}

\recipe{Green Chile Sauce}
\ingredients{\quarter cup olive oil;
1 clove garlic, minced;
\half cup onion, minced;
1 tablespoon all purpose flour;
1 cup pork stock, chicken stock, or water;
1 cup diced green chile (2 4-ounce cans if fresh or frozen is not available);
salt.}

Saut\'{e} garlic and onions in olive oil using a heavy saucepan.  When onions and garlic are translucent but not browned, blend in flour and stir with a wooden spoon. Cook flour long enough to remove the raw taste but do not brown.  When flour is cooked add stock or water and diced green chile to the pan. Bring to boil and then reduce to a simmer. Stir frequently until sauce is thickened to your satisfaction, about 5 minutes. Add salt to taste.

\recipe{Red Chile Sauce}

\ingredients{4 New Mexican red chile pods;
3 tablespoons olive oil;
1 clove garlic, minced;
2 tablespoons all purpose flour;
2 cups pork stock, chicken stock, or water; salt.}

Begin by making red chile powder from the chile pods. Wash the pods to remove any dust or dirt from the exterior then tear open and remove the stem, seeds and veins. Tear the remain flesh into large pieces and toast in a heavy skillet.  Be careful not to burn the chile. Place toasted chile pieces in a spice grinder or blender and grind to a powder. Measure \half cup of powder to use in this recipe and save the rest for another use.

Saut\'{e} the garlic in olive oil for a minute or so then blend in flour with a  wooden spoon. Cook flour long enough to remove the raw taste but do not brown. When flour is cooked add the reserved \half cup of chili powder and blend in.  Blend in stock or water and cook to desired consistency. Add salt to taste.

\newpage

\recipe{Ranchero Sauce}
\label{RancheroSauce}
\ingredients{1 yellow bell pepper, cored and seeded;
1 pablano pepper, stemmed and seeded;
2 to 4 jalape\~{n}os, stemmed and seeded;
1 large clove garlic;
1 large onion, cut into large chunks;
\quarter cup extra virgin olive oil;
1 teaspoon salt;
\half teaspoon ground black pepper;
1 tablespoon dried Mexican oregano;
\half teaspoon New Mexican red chile powder;
1 4-ounce can mild green chiles;
2 14.5-ounce cans diced fire-roasted tomatoes;
cayenne pepper as needed (optional).}

Place the bell pepper, pablano pepper, jalape\~{n}os, garlic and onions is a food processor and process until finely minced.

Heat the oil in a deep skillet over medium heat.  Add the pepper and onion mixture, the salt, black pepper, oregano and powdered chili. Saut\'{e} until onion is soft, about 5 minutes. Add the green chiles and tomatoes. Reduce the heat to medium-low and simmer for 10 to 15 minutes. Taste for heat and add cayenne pepper a pinch at at time as needed to achieve your desired heat level.

Blend with an immersion blender until the sauce is still a bit chunky but well combined then simmer for 5 minutes more. 

Makes 4\half cups.

Source: Adapted from a recipe in Always Hungry? by David Ludwig

\recipe{Steaks with Green Chilies and Mushrooms}
\ingredients{1 medium onion finely chopped;
2 tablespoon olive oil;
1 cup roasted, peeled, and chopped fresh New Mexico green chilies;
\quarter teaspoon dried oregano;
\quarter teaspoon salt;
1 teaspoon hot sauce;
4 large fresh mushrooms sliced thin;
4 tablespoon butter;
2 steaks (as large as 14- to 15-ounce);
wood chips for smoking on the grill.}

The first 6 ingredients are for the green chili sauce.  To make the sauce, first saut\'{e} the onions in the olive oil.  When the onions are soft, add the remaining sauce ingredients and cook for 5 minutes.  Keep the sauce warm in oven set to low.

Saut\'{e} the mushrooms in butter until soft.  Remove from pan and keep warm with sauce.

Grill steaks using wood chips.

Transfer steaks to plates, divide mushrooms over the top and cover with green chili sauce.

From the Pink Adobe restaurant in Santa Fe, New Mexico.

\newpage

\recipe{Pink Adobe's Green Chili Stew}

\ingredients{2 tablespoons olive oil;
    2 pounds boneless pork shoulder cut into 1-inch cubes;
    \half cup chopped yellow onion;
    2 clove garlic, minced;
    \quarter cup all-purpose flour;
    2 cups chopped tomatoes (peeled, fresh);
    2 cups chopped green chilies (roasted, peeled and chopped fresh hatch chilies);
    1 jalapeno pepper, chopped;
    1 teaspoon salt;
    \half teaspoon freshly ground black pepper;
    \half teaspoon granulated sugar;
    1 cup chicken stock.}

Heat olive oil in 4-quart Dutch oven with cover. Add pork and cook until lightly browned. Add onion and garlic and stir with meat. Add flour and stir 1-2 minutes.Add tomatoes, green chilies, jalapeno , salt, pepper and sugar. Mix to incorporate. Add broth. Lower heat. Cover pot and simmer for 1-1\half hours until meat is tender. Serve with flour tortillas.

Yields: 6 servings

From the Pink Adobe restaurant in Santa Fe, New Mexico. 

\newpage
    
\recipe{Chicken Enchiladas with Sour Cream}

\ingredients{2 whole chicken breasts;
2 cups water;
1 teaspoon salt;
1 small carrot, chopped coarse or whole;
1 small onion, quartered;
2 sprigs parsley;
1 stalk celery, quartered;
2 cups grated cheddar cheese;
1 pint sour cream;
4 tablespoons olive oil;
1 clove garlic, chopped;
1 16-oz can chopped green chiles (I had to use four 4 oz. cans of chopped green chiles, which got a little pricey they were Ortega brand, Anaheim chiles);
5 medium ripe tomatoes, peeled and chopped;
2 medium onions, chopped fine;
\half teaspoon dried oregano;
\half teaspoon salt;
1 cup chicken broth (from poaching);
Canola oil for frying;
12 corn tortillas (the small taco kind).}

To poach chicken: Put chicken in a large pot and cover with water. Bring water to a boil and add all other ingredients. Cover pot and simmer for about 45 minutes; chicken should be cooked through (meat will look white and have a flakey texture). Take off heat and let chicken cool in broth.

Meanwhile, peel tomatoes: This is actually very easy, as long as you don't try to do it with a knife. Fill a large pot with water and bring it to a boil. Once it's boiling, add the tomatoes, whole. Cook for about 5 minutes, until skins start to peel off. Let the tomatoes cool in the water; when you remove them, you should be able to just pull the skins off with your hands. After peeling, chop roughly.

Meanwhile meanwhile (while the water is boiling for the tomatoes and the chicken is poaching), prep other items (this is when I grated the cheese, chopped the garlic and onions, and opened my cans of chiles).

Now back to the chicken: Strain chicken and vegetables, reserving the broth. ``Shred'' the chicken, pulling it apart with the grain into relatively small flakes of meat (this will make sense once you have the poached chicken). Mix chicken with half (1 cup) of the grated cheese and all the sour cream to the shredded chicken and mix thoroughly. Set aside.
  
Around now would be a good time to preheat the oven to 350\deg F.
  

Prepare green chile sauce: heat olive oil in a medium pot. Add garlic and sauté until golden. Add chiles, tomatoes, onions, oregano, and salt. Add broth and cook over low heat until liquid is reduced by half (about 15-20 minutes). Set aside.

Okay, enchilada assembly time! This is where it gets fun. Set up as assembly line with a frying pan on the stove, the bowl of chicken mix, the green chile sauce, a flat surface to work on (I used a cutting board with paper towels covering it to absorb the oil), and the pan the enchiladas are eventually going to land in.

a.    Then heat the oil in a frying pan. I worked 2 tortillas at a time -- you could do more if your cutting board is bigger than mine. Fry each tortilla individually, just for a few seconds on each side until it puffs slightly. Remove with tongs or a spatula and let drain on the paper towel covered cutting board (turn it over at some point so both sides can drain).

b.    Divide the chicken mix in half, and then divide each half into six parts, roughly. Spoon 1/12 of chicken mixture onto tortilla and top with 2 tbsp. of green chile sauce.

c.    Roll the enchilada. For me, this ended up being more like folding it twice.

d.    Place ``rolled'' enchilada into 9 inch by 13 inch pan; fill pan so enchiladas are snug.

When all enchiladas are assembled, cover with remaining sauce and cheese and bake for 15 minutes (uncovered, until thoroughly heated). The Pink Adobe recommends serving the enchiladas with chopped lettuce and Corona Extra beer. A cookbook after my own heart.

\newpage
\recipe{Sopaipillas}
\equilpment{R \& V Works Cajun Fryer Model FF1.}
\ingredients{2 cups all-purpose flour;
2 teaspoons baking powder;
1 teaspoon salt;
2 tablespoons shortening or lard;
\half cup water;
peanut oil for frying.}
\textsc{Serve these hot from the fryer.  To eat, tear off one end of the sopaipilla and pour in  honey.}

Sift dry ingredients together then work in shortening and lukewarm water to make a soft dough. Place dough in refrigerator until chilled through. Roll out dough on a floured surface to about \quarter inch thickness and then cut into 3 inch squares. Deep fry at 400\deg F a few at the time until puffed being sure to brown on both sides.  Remove from oil and drain on paper towels.

\newpage
\recipe{Flour Tortillas}
\ingredients{455 grams all-purpose flour, plus more for rolling;
1 teaspoon Diamond Crystal kosher salt;
1 teaspoon baking powder;
\twothirds  vegetable shortening (may substitute for unsalted butter or lard), plus more for coating;
1\quarter cups hot water.}

Using your hands (or a whisk), combine the flour, salt and baking powder in a medium bowl.

In a large bowl, beat the vegetable shortening with your hand in circular motions to warm it up and spread it in the bottom of the bowl, until it is creamed and there are no lumps, about 1 minute. Add the flour mixture and mix it with the shortening, in circular motions, wiping the bowl as you mix, until the fat is evenly distributed through the flour, for about 1 or 2 minutes.

Distribute the water over the mixture and mix it into the flour in a circular motion, scraping from the bottom and folding and kneading the dough pressing it from the center out to the edges of the bowl. At first it will be very sticky and lumpy, but as you continue to knead, it will become more elastic, soft and homogenous, more light, less dense and springy to the touch, 3 to 4 minutes. Cover the bowl with a towel and let rest for 20 minutes.

Using your fingers, pinch off a heaping 1\half inch ball of the dough. (You should have about 12 pieces, each about 70 grams.) Roll each piece into a ball and place on a baking sheet or board. Rub a bit of vegetable shortening in the palm of your hands and roll each ball of dough between your palms to coat it with the shortening. You may need to repeat adding vegetable shortening to your hands about 4 to 5 times to go over the 12 balls. Cover with a towel and let rest for 20 minutes.

Heat your comal, griddle, or cast-iron or nonstick skillet over medium-low heat for at least 5 minutes.

Lightly flour your work surface and your rolling pin. Roll one ball into a 9-inch tortilla. You will need to rotate the tortilla on your work surface about 5 or 6 times as you roll it out, flip and add more flour as needed. Do not get discouraged if the tortilla does't make a perfect round; it takes lots of practice!

As soon as you are done rolling out a tortilla, using both hands, lay it on the hot comal, in a swift and determined way so it doesn't break. After 40 to 50 seconds there should be brown freckles on the bottom side and air bubbles on top. Using a spatula and your hand, flip the tortilla over and cook for another 40 to 50 seconds, until the other side is freckled and the tortilla puffs up even more. Transfer to a clean kitchen towel and keep covered.

Repeat with the remaining dough and as you cook the tortillas. If you don't eat all of them at once, let them cool then place them in a plastic bag and seal the bag. They will keep fresh, out of the refrigerator for at least 3 to 4 days. You can also store them in your refrigerator for up to a week. When ready to eat, take them out and reheat on a preheated comal, griddle or skillet over medium low heat, for a minute or so per side. (It is very important that you preheat the comal or skillet before adding the tortillas so that they don't stick or burn.)

Source:  Pati Jinich NYT Cooking

\chapter{Condiments and Components}



\recipe{Wine Vinegar}
\label{WineVinegar}
\equilpment{2 quart-size wide-mouth mason jars with rings; cheese cloth.}
\ingredients{750 ml bottle of ``over the hill'' red wine; \half teaspoon 3\% hydrogen peroxide solution; ; 750 ml purified water; 250 ml fluid ounces organic apple cider vinegar with mother.}

Add the hydrogen peroxide solution to the bottle of wine and give it a good shake. Wait for 10 minutes before proceeding with the rest of the recipe.  This step is necessary to neutralize the sulfides in the wine.

Pour half of the wine in each of the mason jars and then add an amount of purified water equal to the amount of wine in each bottle.  Each jar should contain a mixture of 375 ml of wine with 375 ml of water. Shake the bottle of cider vinegar to get the mother into solution then add 125 ml of vinegar to each jar.  Stir the contents of the jars and then cover the jars with cheese cloth.  The jar rings can be used to hold the cheese cloth in place.

Put the jars in a dark cupboard. Every week or two give the jars a shake and check the progress of the vinegar. The vinegar is done when almost all of the alcohol has been converted to acetic acid. This will take several months.

\newpage
\recipe{My Mother's Mayonnaise}

\ingredients{1 whole egg at room temperature;
1 teaspoon salt;
1 teaspoon sugar;
2 tablespoons lemon juice (juice from one large lemon);
1 tablespoon yellow ball park mustard;
Dash of paprika;
Dash of red pepper;
2 cups corn oil}

Install the balloon whisk on a stand mixer or put the steel blade in a food processor. Add all the ingredients except the oil to the food processor container or the bowl of the stand mixer. In mixer at high speed or in the food processor beat the mixture until it is very thick and creamy.

With mixer or processor running, add the oil very slowly at first until the mayonnaise begins to emulsify and then add in a slow stream until all the oil is incorporated.

\recipe{My Mayonnaise}
\label{MyMayonnaise}

\ingredients{1 whole egg at room temperature;
1 teaspoon salt;
3 tablespoons Pinot Grigio or other white wine vinegar;
3 tablespoon Dijon mustard;
2 cups canola oil}

Install the balloon whisk on a stand mixer or put the steel blade in a food processor. Add all the ingredients except the oil to the food processor container or the bowl of the stand mixer. In mixer at high speed or in the food processor beat the mixture until it is very thick and creamy.

With mixer or processor running, add the oil very slowly at first until the mayonnaise begins to emulsify and then add in a slow stream until all the oil is incorporated.

\recipe{Tartar Sauce}
\label{TartarSauce}
\ingredients{1 cup Mayonnaise;
6 tablespoons finely chopped dill pickles;
2 tablespoon fresh dill or fresh parsley, finely chopped (2 teaspoon dried dill in a pinch);
1 tablespoon lemon juice;
\quarter teaspoon freshly ground black pepper;
2 tablespoon finely chopped onion or shallot;
1 teaspoon Dijon mustard (optional);
2 tablespoon capers, chopped (optional);
1 teaspoon Worcestershire sauce (optional).}

Add ingredients to a medium mixing bowl and stir well to combine. Cover with plastic wrap and then refrigerate for a few hours to allow the flavors to combine.

\newpage
\recipe{Olive Salad}

\ingredients{4 cups undrained pimento stuffed green salad olives, slightly
crushed and well drained, 4 cups is approximately one and one half
14 oz. drained weight jars of salad olives;
1 cup pickled cauliflower, drained and sliced, approximately one half of a
16 oz. jar of Kroger hot pickled cauliflower;
2 small jars capers, drained;
\quarter  stalk celery, sliced diagonally;
1 large carrot, peeled and thinly sliced diagonally;
2 tablespoons celery seeds;
2 tablespoons dried oregano;
\quarter large head fresh garlic, peeled and minced;
\quarter teaspoon freshly ground black pepper;
\quarter of a 12 fluid oz. jar pepperoncini yielding 2 oz. drained weight;
drained (small salad peppers) left whole or sliced  to the
size of the cauliflower;
\quarter pound pitted kalamata olives, crushed to size of green olives;
\quarter of a 16 fluid oz. jar cocktail onions, drained and sliced in half;
\threehalves cups canola oil;
1\half cups extra virgin olive oil}

Combine all ingredients except the oils in a large bowl and mix well. Place
in a large jar and cover the extra virgin olive oil
and the canola oil. 

Store tightly covered in
refrigerator. Allow to marinate for at least 24 hours before
using.

Yields 8 cups.

Based on recipe in ``Cookery N'Orleans Style'' by Chiqui Collier.
Modified to reduce yield and to make ingredient amounts
more specific.


\chapter{Barbecue}
\recipe{Auburn Barbecue Sauce}

\ingredients{\half pound butter;
1 cup cider vinegar;
2 cups tomato catsup;
4 tablespoons worcestershire sauce;
1 tablespoon hot sauce;
1 tablespoon salt;
3 tablespoons prepared mustard;
Dash of red pepper;
juice of 1 lemon.}

In a saucepan melt butter over medium heat. When butter is melted, add all of the remaining ingredients and stir to blend well.

Bring to a boil, lower heat, and let simmer for a few minutes. Makes around 4 cups.

\newpage
\recipe{Arkansas Country Barbecue Sauce}

\ingredients{1 cup ketchup;
\half cup cider vinegar;
\quarter cup honey;
3 tablespoons freshly squeezed lemon juice;
1 tablespoon prepared mustard;
1 tablespoon Worcestershire sauce;
2 teaspoons soy sauce;
1 tablespoon lemon pepper;
1 tablespoon garlic powder;
1 teaspoon cayenne;
1 teaspoon salt;
1 teaspoon Mrs. Dash (optional);
1 teaspoon ground cumin;
\half teaspoon dry mustard;
\twothirds cup bourbon}

In a large saucepan, compose the ketchup, vinegar, honey, lemon juice, prepared mustard, Worcestershire sauce, and soy sauce. Bring to a simmer over medium-high heat.

In a small mixing bowl, compose the lemon pepper, garlic powder, cayenne, salt, Mrs. Dash, cumin and dry mustard. Stir well to thoroughly combine.

Whisk dry ingredients into sauce and then simmer for 20 minutes.

Add bourbon and stir well. Reduce the heat to low, cover, and cook for 1 hour until flavors are blended and sauce is thick.

Best used immediately but can be cooled to room temperature and them stored in the refrigerator for a week or two.  Makes around 3 cups.

Source: John Willingham from John Willingham's World Champion Bar-B-Q.

\newpage

\recipe{Florida Barbecue Sauce}
\label{FloridaBarbecueSauce}

\ingredients{\half cup butter; 1 cup cider vinegar; 1 cup catsup; 6-ounces prepared horseradish; 6 lemons or lines, juiced; 1 teaspoon salt; 1 tablespoon Worcestershire sauce; 1 teaspoon hot pepper sauce.}

\textsc{Use lemons for chicken and limes for beef or pork. Mainly used as a basting sauce for grilled ribs or chicken halves but reserved sauce can also be used at the table.}

Melt butter in a medium stainless steel or enameled cast iron saucepan, add remaining ingredients and stir. Bring to a boil and simmer uncovered for 25 minutes.  Makes 3\half cups of sauce.

Source: Jeanne Voltz from Barbecued Ribs, Smoked Butts and Other Great Feeds

\newpage

\recipe{Memphis Mop BBQ Sauce}

\label{MemphisMopBBQSauce}

\ingredients{2 cups ketchup;
\half cup prepared yellow mustard;
\half cup packed light brown sugar;
\half cup water;
\quarter cup cider vinegar;
3 tablespoons Worcestershire sauce;
1 tablespoon onion powder;
1 tablespoon chili powder;
1\half teaspoons freshly ground black pepper;
2 teaspoons granulated garlic;
\half teaspoon celery salt;
\half teaspoon salt;
1 tablespoon natural hickory liquid smoke.}

In a medium-sized saucepan, combine all the ingredients except the liquid smoke. Bring it to a gentle boil over medium heat, stirring to dissolve the sugar. Lower the heat to low and simmer until it's slightly thickened, 20 to 25 minutes, stirring occasionally.

With a whisk, blend in the liquid smoke until it's incorporated.

Let the sauce cool, transfer it to a jar and store it in the refrigerator for up to a month.

Makes about 3 cups of sauce.

Source: ``Award-Winning BBQ Sauces and How to Use The'' by Ray Sheehan, Page Street Publishing Co.

\newpage
\recipe{Bar-B-Que Rub For Pork and Beef}

\ingredients{1 cup salt;
2 tablespoons + 2 teaspoons freshly ground black pepper;
2 tablespoons + 2 teaspoons lemon pepper;
2 tablespoons + 2 teaspoons cayenne pepper;
2 tablespoons + 2 teaspoons chili powder;
2 tablespoons + 2 teaspoons dry mustard;
2 tablespoons + 2 teaspoons dark or light brown sugar;
1 tablespoons + 1 teaspoons garlic powder;
\half teaspoon ground cinnamon;
\half teaspoon Accent (optional).}

Combine all ingredients in a bowl and stir well to mix.

Use immediately or store in a glass jar in a cool dark place for several months. Yields 2 cups of rub.

Source: John Willingham from John Willingham's World Champion Bar-B-Q.


\recipe{Smoked Meatloaf}

\equilpment{Disposable aluminum or standard loaf pan large enough to hold meat loaf;
plastic wrap or wax paper;
frogmat non-stick grill mat, cookie cooling rack or small grill.}

\ingredients{
2 pounds  ground beef;
1 pound ground sausage  (breakfast or mild Italian);
\half cup finely chopped celery;
\half cup finely chopped onion;
1 tablespoon finely minced garlic;
2 tablespoon olive oil;
2 eggs;
2 cup bread crumbs, (more or less as needed to bind);
1 pound Bacon, regular slice (to wrap);
1 cup Cheddar cheese, grated
} 


Saut\'{e} garlic and onions in the olive oil until onions are just translucent. Mix together ground beef, sausage, celery, onion, garlic, eggs and just enough bread crumbs to bind the mixture together.

Remove bacon from package and set aside three slices. On a cutting board lay the remaining slices in a five by seven strip ``basket weave'' pattern. Place plastic wrap or wax paper over the bacon and invert the loaf pan on top. A standard five pound disposable aluminum pan  (11\threequarters" x\, 5\fiveeights" x  3\threesixteenths") works well for this application. Holding the loaf pan in place while inverting the cutting board so as to allow the plastic wrap or wax paper and bacon to drape into  the pan with the bacon running up the sides. Divide the loaf mixture in half. Place one half of the loaf mixture into the pan on top of the bacon. Flatten it out. Place cheese in center of the meat mixture running almost to the ends. Cover with remaining loaf mixture. Fold any loose ends of the bacon on top of the meat loaf. Cover with remaining three bacon slices.

Chill in the fridge for a couple of hours. Remove from the mold by inverting it on to a Frogmat if you have one or a small cookie sheet or grill to make it easier to handle.

Preheat smoker to 250\deg F. Insert a meat thermometer into the center of the loaf. Place loaf in smoker and cook until the internal temperature reaches 160\deg F. This will take two to two and a half hours. 

When done, remove and let stand for fifteen minutes before carving.


\chapter{Snacks and Party Food}

\recipe{Tamari Roasted Almonds}
\label{TamariRoastedNuts}

\ingredients{4 cups raw almonds;
scant \quarter cup tamari;
1 tablespoon toasted sesame oil}

Preheat oven to 300\deg F.  Place almonds in a single layer on a foil lined half-sheet pan and put in oven for 15 minutes. Stir almonds and continue roasting for another 15 minutes. Continue roasting for around 15 more minutes, checking every so often to be sure they do not burn. To check for doneness cut a nut in half. When done the nuts they should be golden to golden brown inside.  When done, turn off the oven and place the roasted nuts in a bowl. Pour the tamari over the nuts and stir. The nuts should sizzle/steam as you stir. Continue stirring until the liquid has evaporated then add the sesame oil. Return the nuts to the sheet pan and dry in the turned-off oven for around 10 minutes.  When dry store at room temperature. 

\newpage
\recipe{Hummus}
\label{Hummus}
\equilpment{14 cup heavy-duty food processor; 6 quart Dutch oven}
\ingredients{1 pound dried chickpeas;
2 teaspoons baking soda;
12 cups (3 quarts) water;
16 ounce jar Soom tahini;
8 tablespoons (4 ounces or \half cup) lemon juice;
8 cloves garlic, pealed;
1 tablespoon salt;
6 to 7 ounces ice-cold water}

The night before, put the chickpeas in a large bowl and cover them with cold water at least twice their volume. Leave to soak overnight.

The next day, drain the chickpeas. Place a Dutch oven over high heat and add the drained chickpeas and baking soda. Cook for about 3 minutes, stirring constantly. Add the water and bring to a boil. Cook at a rolling boil, skimming off any foam and any skins that float to the surface. The chickpeas will need to cook between 20 and 40 minutes, depending on the type and freshness. Once done, they should be very tender, breaking up easily when pressed between your thumb and finger, almost but not quite mushy.

Drain the chickpeas. Place the chickpeas in a food processor and process until you get a stiff paste. Then with the machine still running, add the tahini paste, lemon juice, garlic, and salt. Finally, slowly drizzle in the iced water and allow it to mix for about 5 minutes, until you get a very smooth and creamy paste.

Refrigerate or freeze in 1 or 2 cup portions. Cover the surface of the hummus with plastic wrap to avoid the formation of a skin. Defrost frozen hummus in the refrigerator overnight. Allow refrigerated hummus to come to room temperature before serving by leaving is on the counter for 30 minutes.

Makes approximately 8 cups.

Source: ``Jerusalem a Cookbook'' by Yotam Ottolenghi and Sami Tamimi.  Modified to use 1 pound of chickpeas and a whole 16 ounce jar of Soom tahini.

\newpage
\recipe{Pickled Eggs}


\ingredients{12 medium eggs, hard cooked and pealed;
2 cups cider vinegar;
2 tablespoons sugar;
1 teaspoon table salt;
4 peppercorns;
3 whole cloves;
1 teaspoon  whole celery seeds;
1 medium red beet, grated (optional);
1 small onion, sliced in rings;
2 cloves garlic, thinly sliced;
1 teaspoon caraway seeds;
sprig or two of fresh dill.
}

\textsc{My mother, Mallette Goggans, often made these eggs for parties.  Her recipe uses 15 dozen peewee eggs and a gallon of vinegar. I have scaled her recipe down to one dozen medium eggs so that it fits in a single one-quart canning jar. The beet in the recipe is just for color and can be omitted. When placing the eggs in the jar try to stand then up and place onions around then so that they do not touch the side of the jar.  That way the eggs will be uniformly colored and pickled. Pickled eggs may be kept several months in the pickling solution in the refrigerator.  The  eggs can be drained and served as is or sliced in half and used to make deviled eggs. }

Combine vinegar, sugar, salt peppercorns, cloves, celery seeds and grated beet in a sauce pan and bring to a simmer.  Simmer picking liquid for around 5 minutes then strain reserving liquid.

While pickling liquid is simmering, layer eggs, onions, garlic and dill in a one-quart wide-mouth canning jar.  Sprinkle caraway seeds over eggs in jar then pour strained liquid over eggs, onions, and garlic until covered. Place lid and ring on jar and tighten.  Place jar in the refrigerator  and marinate for at least 2 days although 2 weeks is better.

\chapter{ Charcuterie}
\recipe{Sausage and Chicken Liver P\^{a}t\'{e}}

\ingredients{1 pound raw thin sliced bacon strips;
1 pound chicken liver;
1 pound ground pork breakfast sausage;
2 large eggs;
2 or 3 cloves of garlic;
1 medium onion diced;
1 tablespoon diced parsley;
1 tablespoon brandy or cognac (Optional or use 3 tablespoons sherry);
1 teaspoon thyme;
1 teaspoon salt;
\eighth teaspoon black pepper;
\eighth teaspoon rosemary;
3 or 4 whole bay leaves for top of p\^{a}t\'{e}.}

Put a parchment paper sling in a 6 to 8 cup p\^{a}t\'{e}  mold (a bread pan works fine) and then line the mold with the
raw bacon strips letting the ends flop over the sides of the mold.  Use whole strips of bacon across the short side of the mold just overlapping the strip.  On the ends of the long side of the mold use bacon strip cut in half.  The sling is used to as an aid when unmolding the p\^{a}t\'{e} and is particularly important if a glass mold is used since in this case the mold can not be slammed against the platter without risk of breaking the glass.

Place all of the remaining ingredients except for the bay
leaves in a food processor or blender and process until liquefied.

Pour the liver mixture into the p\^{a}t\'{e} mold, top with the bay
leaves, and fold the end of the bacon over the mixture.

Cover the mold with foil and bake at 225 \deg F for 3 hours.  The internal temperature should be greater than 160 \deg F.

Remove from oven and pour off the excess fat from the p\^{a}t\'{e}.  As long as the specified internal temperature is reached, you should not be concerned if the excess fat you pour off is a little pink. Place a brick or
other heavy object on top of the foil to press the p\^{a}t\'{e} and leave
in the refrigerator overnight.

Unmold p\^{a}t\'{e} onto a platter
and garnish with greenery or strips of roasted red and/or yellow
bell pepper.  Serve with slices of \hyperref[SweetPickles]{sweet pickle} and Dijon mustard
on French bread.

\newpage
\recipe{Bratwurst}
\label{Bratwurst}
\equilpment{Microplane Classic Zester Grater; KitchenAid Mixer; LEM \#5 Electric Meat Grinder; LEM Mighty Bite Sausage Stuffer.}

\ingredients{2 kilograms pork shoulder, cut into rough 1-inch chunks, 30 grams kosher salt, 2 tablespoons garlic grated on a Microplane, 1 tablespoon freshly grated nutmeg, 1 teaspoon ground ginger, 2 teaspoons freshly ground black pepper, 1 cup sour cream, 4 3-foot sections of natural hog casings for sausage (e.g. Natural Hog Casings for Sausage by Oversea Casing). }

Combine the pork chunks, salt, grated garlic, nutmeg, ginger, black pepper and sour cream  in a large bowl and toss with clean hands until homogeneous. Transfer to a gallon-sized zipper-lock bag and refrigerate for 24 hours.

Rinse salt off casings and place in a bowl with cold water to soak.  Casings should soak for around 30 minutes.

Place bag with pork mixture in freezer for around 30 minutes then grind mixture using a 4.5 mm plate. Grind into a bowl other than the KitchenAid mixer bowl. Feed a piece of stale bread through the grinder to force out any remaining bits of sausage. 


Add 2 ounces of water and half of the ground meat to the mixer bowl. With the mixer's paddle attachment, mix the sausage on medium-low speed until homogeneous and tacky, about 2 minutes. Remove mixed sausage and repeat with an additional 2 ounces of water and the remaining ground meat. Load sausage into stuffer. 

Place open end of casing on the end of the kitchen faucet and inflate with water until fully inflated. Drain and place on \seveneights" (2.2 cm) diameter stuffer tube. Tie a knot in the end of the casing and evenly fill with sausage. Stop before the end of the casing and tie off with another knot.  Divide into links and twist to separate. 

Makes around 16 8-inch links.

Source: Modified from a recipe in ``The Food Lab'' by J. Kenji L\'{o}pez-Alt.

\newpage
\recipe{Texas German Sausage}
\label{TexasSausage}
\equilpment{KitchenAid Mixer; LEM \#5 Electric Meat Grinder; LEM Mighty Bite Sausage Stuffer; KBQ C-60 BBQ Smoker.}

\ingredients{1.7 kilograms beef brisket;
300 grams pork shoulder;
26.5 grams Kosher salt;
1 gram ground cayenne pepper;
15 grams fresh and coarsely ground black pepper;
3.5 grams cure \#1;
4 3-foot sections of natural hog casings for sausage (e.g. Natural Hog Casings for Sausage by Oversea Casing);
180 grams cold water;
28 grams dry milk powder;
 split oak logs for smoking.}

\textsc{This is my attempt to produce East Texas style German sausage like the Original Smoked Sausage made by Kreuz Market in Lockhart, Texas. According to their web site, Kreuz sausage is 85\% beef and 15\% pork, smoked using post oak, and seasoned using only salt, pepper and cayenne. This recipe has the same percentages of beef and pork as Kreuz sausage, is smoked using oak, and is seasoned with salt, black pepper and cayenne.}

Trim the brisket and port shoulder so the meat to be made into sausage is around 70\% lean and 30\% fat. Weigh out the required portions of brisket and pork shoulder and then
cut the brisket and pork shoulder into one-inch cubes. 

In a large metal or glass bowl, combine the meat cubes with the salt, cayenne pepper, black pepper, and cure \#1.

Mix thoroughly with gloved hands, making sure the seasoning is evenly distributed throughout the meat. Transfer to a two-gallon zipper-lock bag and refrigerate for 24 hours.

Rinse salt off casings and place in a bowl with cold water to soak.  Casings should soak for around 30 minutes.

Place meat grinder head and bag with beef and pork mixture in freezer for around 30 minutes then grind mixture using a 10 mm plate. Grind into a large metal bowl. Feed a piece of stale bread or a sheet of paper towel through the grinder to force out any remaining bits of sausage. Return meat grinder head and ground meat mixture to freezer for 15 minutes then grind the meat mixture a second time with the 4.5 mm plate.

Add half the water and half the milk powder to the mixer bowl containing half the ground meat. With the mixer's paddle attachment, mix the sausage on medium-low speed until homogeneous and tacky, about 2 minutes. Repeat with the remaining water, milk powder, and ground meat. Load sausage into stuffer. 

Place open end of casing on the end of the kitchen faucet and inflate with water until fully inflated. Drain and place on medium stuffer tube. Tie a knot in the end of the casing and evenly fill with sausage. Stop before the end of the casing and tie off with another knot.  Divide the 3-foot sections into 6 links and twist to separate. After twisting there should be approximately \threequarters inch between links. 


Arrange the sausage links on a half-sheet pan then place the pan in refrigerator on the bottom shelf to chill before smoking. If it is convenient you can leave the sausages in the refrigerator overnight and smoke them the next day. Pull the sausages out of the refrigerator an hour before smoking in order to allow the casings to dry at room temperature.

Start a fire in the KBQ C-60 smoker and preheat to 160\deg F. Half open the upper and lower smoke selector outlets on the  smoker.  When the temperature in the smoker is stable, place the sausage links on racks around the middle of the smoker. Cook until  the sausages reach an internal temperature of 154\deg F.  The sausages will take about 3 hours to reach the desired temperature.
   
Once the sausages are out of the smoker, it is important to cool them down as quickly as possible to avoid shriveled skins. Do this by placing the links in a cold water bath until they are room temperature or slightly cooler.


Blooming, or allowing the sausages to develop flavor at room temperature, is an important final step before enjoying your sausages. Hang them on a rack suspended between the backs of two chairs or any setup where they will have plenty of air exposure. The smoky flavor will continue to develop and spread throughout the sausage for a few hours. 

After blooming, you should store the finished sausage in a refrigerator where it will keep for 3-4 days. If you vacuum seal the sausages, they will last in a freezer for up to nine months.

Makes around 2 kilograms of links.

To serve the sausages, preheat oven to 275\deg F, place the sausages in oven and heat for 10 to 20 minutes. Or, even better, brown on a grill over lump charcoal. By the time the sausages are browned all over they will be heated through.

Source: Modified from a recipe found at https://blog.cavetools.com/texas-sausage/


\newpage
\recipe{Stove Top Cooked Link Sausage }
 
 \ingredients{link sausage; water to cover; butter or oil}
 
 Place link sausages in shallow pan or skillet and cover with water.  Place on cooktop over medium heat and cook until water comes to a bare simmer then remove from heat.  Let sausage poach until they reach 140\deg to 150\deg F in the center. 
 
 Remove sausage, pour off water and wipe dry pan or skillet.  Add butter or oil and then brown links on two sides.  Browning should only take a few minutes.  Remove sausage and let links rest for around 5 minutes.  The final internal temperature should be around 160\deg F.

Source: ``The Food Lab'' by J. Kenji L\'{o}pez-Alt.

\recipe{Andouille}

\ingredients{4 pounds boneless pork shoulder, cut in to large chunks; 
1 pound pork belly, cut into \half inch cubes;
45 grams kosher salt;
6 grams pink salt;
50 grams garlic;
8 grams black pepper; 
10 grams ground cayenne;
10 grams paprika;
6 grams celery seed;
6 grams dry mustard;
5 grams ground thyme;
2 grams oregano;
90 grams dried milk powder.}

Partially freeze pork shoulder then grind with a $ \SI{10}{\milli\meter} $ plate. Finish grinding by adding fresh garlic to grinder. Add spice mix and blend then add cubed pork belly and continue to mix. Add a a cup or two of water and continue to mix until every thing is nice and tacky. 

Place sausage mixture in stuffer and stuff into standard hog casings.  Make links 10 to 12 inches long and make sure they are well filled.

\recipe{Grinder Plate Sizes}
\textsc{This is not a recipe but rather a list of meat grinder plate hole sizes giving the hole diameter in mm and the approximate equivalent diameter as a rational number in inches.}


$$ \SI{3}{\milli\meter} \approx \frac{1}{8}{inch}$$
$$ \SI{4.5}{\milli\meter} \approx \frac{3}{16}{inch}$$
$$ \SI{6}{\milli\meter} \approx \frac{1}{4}{inch}$$
$$ \SI{10}{\milli\meter} \approx \frac{3}{8}{inch}$$
$$ \SI{16}{\milli\meter} \approx \frac{5}{8}{inch}$$

\chapter{Thanksgiving}

\recipe{Thanksgiving Dinner}
\ingredients{\hyperref[RoastTurkey]{Good Eats Roast Turkey};
\hyperref[CranberryRelish]{Cranberry Orange Relish};
\hyperref[OysterDressing]{Corn Bread Oyster Dressing};
\hyperref[PersimmonPudding]{Persimmon Pudding};
\hyperref[BrooklynCollards]{Brooklyn-Style Collard Greens};
\hyperref[TurkeyGravy]{Turkey Gravy}.}

\newpage
\recipe{Turkey Gravy}
\label{TurkeyGravy}
\ingredients{\half cup unsalted butter; scant \half cup all-purpose flour; turkey bone broth or chicken stock; drippings from roast turkey; Kitchen Bouquet Browning and Seasoning Sauce (optional); salt; pepper, freshly ground.}

In a small sauce pan, melt butter and then stir in an equal amount of flour.  Cook roux over medium heat until blond. As an alternative, you can melt the butter and cook the roux in a microwave in a glass measuring cup. When roux is cooked, whisk in stock until your desired consistency is achieved. Leave at room temperature until just before serving. 

When ready to serve, reheat the roux and add defatted drippings from the roast turkey. It this gravy isn't dark enough to suit you then add Kitchen Bouquet until it is. Season to taste with salt and pepper. Serve gravy in a gravy boat while it is still hot.

\recipe{Cranberry Orange Relish}
\label{CranberryRelish}
\ingredients{1 medium orange;
12 oz. package of fresh cranberries;
1 cup sugar;
1 cinnamon stick.}


Cut orange into wedges and remove seeds.

Pour \half of the cranberries and orange in blender or food processor container.
Cover; blend until chopped.  Repeat step.

Cook cranberry mixture, sugar and cinnamon stick, over medium heat, 
in large saucepan 5 to 10 minutes or until heated through.  
Remove cinnamon stick.  

Serve warm or chilled.

\newpage
\recipe{Good Eats Roast Turkey}
\label{RoastTurkey}

\equilpment{20 liter Yeti cooler (for brining); cooler large enough to hold quarter sheet pan (for holding cooked turkey), quarter sheet pan; half sheet pan with wire rack; 18 inch wide heavy-duty aluminium foil; Thermoworks ChefAlarm Thermometer.}

\ingredients{14 to 16 pound fresh turkey; 1 cups kosher salt; 1 gallon \hyperref[VegetableBroth]{vegetable broth};
\half cup light brown sugar;
1 tablespoon black peppercorns;
1\half teaspoons allspice berries; \half teaspoons chopped candied ginger; 1 gallons heavily iced water;
1 red apple, sliced; \half onion, sliced; 1 cup water; 4 sprigs rosemary; 6 leaves sage; Canola oil
}


To make the brine, combine the vegetable stock, salt, brown sugar, peppercorns, allspice berries, and candied ginger in a large stockpot over medium-high heat. Stir occasionally to dissolve solids and bring to a boil. Then remove the brine from the heat, cool to room temperature, and refrigerate.  

Combine the brine, water and ice in a 20 liter cooler. Place the turkey (with innards removed) breast side down in brine. If necessary, weigh down the bird to ensure it is fully immersed. Set cooler in cool area for 8 to 16 hours, turning the bird once half way through brining.

Set the oven rack to it lowest level and preheat the oven to 500 \deg F. Remove the bird from brine and rinse inside and out with cold water. Discard the brine. 

Place the bird on wire rack inside a half sheet pan and pat dry with paper towels.


Combine the apple, onion, cinnamon stick, and 1 cup of water in a microwave safe dish and microwave on high for 5 minutes. Add steeped aromatics to the turkey's cavity along with the rosemary and sage. Tuck the wings underneath the bird and coat the skin liberally with canola oil.  

Fold an 18'' foil square in half along the diagonal to form a right isosceles triangle. Coat one flat side of the triangle with Canola oil and then form to the breast of the turkey with the 90\deg angle pointed to the front of the bird and the 45\deg angles pointing at the wings.  Remove the foil triangle without bending it and set aside.


Put the turkey in the oven with legs first if it will fit this way. Roast the turkey on lowest level of the oven at 500 \deg F for 30 minutes. Reduce the oven temperature to 350 \deg F. Insert a probe thermometer into thickest part of the breast and place the foil triangle on the breast.  Set the thermometer alarm to 161 \deg F. A 14 to 16 pound bird should require a total of 2 to 2\half hours of roasting. Let the turkey rest, loosely covered with foil for at least 15 minutes before carving. Alternatively, the cooked turkey can be held for up to 90 minutes in a large closed cooler.  Put the turkey on a quarter sheet pan (or serving plater) and then place in  cooler. Cover the turkey with foil before closing the cooler.

\newpage
\recipe{Corn Bread Oyster Dressing}
\label{OysterDressing}
\ingredients{12 tbsp. butter;
1 large yellow onion finely chopped;
4 celery stalks finely chopped;
1/4 cup chopped fresh parsley;
1 tsp. dried sage;
1/2 tsp. dried tarragon;
1 recipe of \hyperref[cornbread]{corn bread} made in a skillet, crumbled;
2 eggs;
Pinch cayenne;
Salt and freshly ground black pepper;
Crisco;
Chicken Stock (less than 1 cup needed);
1 pt. to 1.5 pt. shucked Gulf oysters (3 dozen small shucked oysters).}

Make \hyperref[cornbread]{corn bread} in a cast iron skillet.
Grease skillet generously with Crisco and preheat in oven before adding 
batter so that the corn bread develops a thick brown crisp crust. 
Cool corn bread and then crumble. Two cups of corn meal mix made into
corn bread should yield around 6 cups of crumbled corn bread.

Preheat oven to 350 \deg F.  Melt 6 tbsp. of butter in a skillet over medium heat, 
add onions, celery, parsley, sage, tarragon, and cayenne.  Sweat the mixture 
until the vegetables are soft, about 20 minutes, then cool.

Drain and reserve the oyster liquid.  If using large oysters cut each oyster 
into 2 or 3 pieces.  Measure the oyster liquid and add enough chicken stock 
(or water) to bring the total to 1 cup.  Transfer liquid to a small sauce pan, 
add the remaining 6 tbsp. of butter, then bring to a simmer, stirring until 
butter has melted.

In a large mixing bowl, lightly beat 2 eggs then add the crumbled corn bread
and mix to combine.  Add vegetable mixture, season with salt (around 1 tbsp. 
if you use unsalted butter) and pepper (1 tsp. or more to taste) 
then mix to combine.  Pour the liquid/butter mixture over the corn bread 
mixture and combine by working with hands.  The mixture should be
uniformly moist but not wet. Add additional chicken stock if mixture is too dry.
Finally, gently mix oysters into corn bread mixture, taking care not 
to break up the oysters.

Transfer mixture to a greased approximately 64 sq. in. cast iron baking dish
or tall cast iron skillet.  Bake for approximately 40 minutes until firmly set 
and browned on top.  Remove from oven and cover with foil to keep 
warn until ready to serve.

Yields 8 generous servings.
Notes

I made a double recipe for Thanksgiving 2012.  
For the double recipe I used my blue Le Creuset baking pan both to make the 
corn bread and to make the dressing.  For the double recipe I increased the 
cooking time to 60 minutes and finished browning the top using the broiler 
on high for a minute or two.

I made a single recipe for Thanksgiving 2014. I used my black high sided Le Creuset baking pan. In this pan I cooked the recipe for 55 minute and then browned the top under the broiler on high.  I used 4 or 5 dozen large oysters cut in half. I had enough oyster liquid and so did not need the chicken stock.



\newpage
\recipe{Persimmon Pudding}
\label{PersimmonPudding}
\ingredients{2 cup persimmon pulp $\approx$ pulp from 5 or 6 Japanese-hybrid  persimmons;
3 eggs beaten;
1\half cups sugar;
2 cups whole milk;
2 cups unsifted all-purpose flour;
1 teaspoon baking soda;
1 teaspoon salt.}

\textsc{Around 1967 my father, Floyd Goggans, planted a dwarf Tanenashi
persimmon tree in the yard of the house at Kuderna Acres.  Every
year the persimmons ripened just in time to make persimmon pudding
for Thanksgiving dinner. My mother,  Mallette Goggans, often served this pudding with
a hard sauce made with bourbon.  I use the hard sauce recipe from
Joy of Cooking made with rye whisky and cream. Be sure to use
truly ripe persimmons.  A ripe persimmon is very soft. Its pulp is
like a thick gel and contains no solid bits.}

In mixing bowl combine all ingredients stirring until blended.

Pour mixture into a 8 cup metal mold. The pudding will
rise when cooked so do not completely fill the mold.

Place uncovered mold inside a large covered pot.

Pour boiling water in the pot to slightly below the top of the mold.

Adjust the stove top so that the water in the pot simmers slowly.

Cover the pot and then steam the pudding until it has a cake-pudding
texture.  This step will take around 3 hours.

Remove pudding and cool in refrigerator.

\chapter{Christmas}

\recipe{Christmas Eve Dinner}
\ingredients{\hyperref[PrimeRibRoastOven]{Prime Rib Roast -- Oven Roast} or \hyperref[PrimeRibRoastSousVide]{Prime Rib Roast -- Sous Vide};
\hyperref[CrispyRoastPotatoes]{Crispy Roast Potatoes};
\hyperref[HorseradishSauce]{Horseradish Sauce};
\hyperref[Chimichurri]{Chimichurri};
\hyperref[WarmSpringsEggnog]{Warm Springs Eggnog};
\hyperref[PearTart]{Pear and Almond Tart};
\hyperref[RomaineOrangePecanSalad]{Romaine, Orange, and Pecan Salad}.}

\newpage

\recipe{Prime Rib Roast -- Oven Roast}
\label{PrimeRibRoastOven}
\ingredients{Prime-grade prime rib roast, trimmed and tied (2-3 servings per bone); kosher salt; black pepper, freshly ground.}

Buy a bone-in, prime-grade prime rib, preferably dry-aged. Season well with salt and pepper, and let it rest, uncovered, on a rack in the refrigerator for at least overnight and up to four days. 

Roast in a 200\deg F oven until it hits 125\deg F at the center (around four to five hours for an average rib roast; your mileage may vary). Remove from the oven, tent lightly with aluminum foil, and let it rest at least 30 minutes and up to one and a half hours. 

Ten minutes before serving, remove the foil, pour off fat from pan and place it back into an oven preheated to its highest possible setting (500-550\deg F). Roast until well browned and crisp, about 10 minutes. Carve and serve immediately. 

To carve, grab bone with a clean kitchen towel and carve off bone from roast. Cut apart rib bones and carve roast into \half inch thick slices.

Notes from Christmas 2021: Cooked 4 bone 8.8 pound prime-grade trimmed and tied roast to serve 8 people. Even with large servings there was plenty left over. The roast went in a 200\deg F oven at 12:25 PM. The internal temperature when the roast went in the oven was 42.6\deg F. At 1:45 PM the internal temperature was 59.2\deg F.   At 2:15 PM the internal temperature was 71.1\deg F.  At 4:08 PM the internal temperature was 112.5\deg F. At 4:51 PM the internal temperature was 125.2\deg F and the roast was removed from the oven and tented with foil. The internal temperature was 135.5\deg F when the roast was served at 6:10 PM.


\newpage

\recipe{Prime Rib Roast -- Sous Vide}
\label{PrimeRibRoastSousVide}
\ingredients{7 pound prime-grade prime rib roast with \half inch fat cap (3 bones); kosher salt; black pepper, freshly ground; 1 tablespoon vegtable oil.}

To remove bones from roast, use sharp knife and run it down length of bones, following contours as closely as possible; set bones aside. Cut slits in surface layer of fat on roast, spaced 1 inch apart, in crosshatch pattern, being careful to cut down to, but not into, meat. Rub 2 tablespoons salt over entire roast and into slits. Place meat back on bones (to save space in refrigerator), transfer to plate, and refrigerate, uncovered, at least 24 hours or up to 96 hours.

Using sous vide circulator, bring water to 133\deg F/56\deg C in 12-quart container.

Separate meat and bones; set aside bones. Heat oil in 12-inch skillet over medium-high heat until just smoking. Sear sides and top of roast until browned, 6 to 8 minutes (do not sear side where roast was cut from bone). Place meat back on ribs so bones fit where they were cut, and let cool for 10 minutes. Tie meat to bones between ribs with 2 lengths of kitchen twine.
   
Season roast with pepper and place in 2-gallon zipper-lock freezer bag. Seal bag, pressing out as much air as possible. Gently lower bag into prepared water bath until roast is fully submerged, and then clip top corner of bag to side of water bath container, allowing remaining air bubbles to rise to top of bag. Reopen 1 corner of zipper, release remaining air bubbles, and reseal bag. Cover and cook for at least 16 hours or up to 24 hours.

Adjust oven rack to middle position and heat broiler. Set wire rack in aluminum foil-lined rimmed baking sheet and spray with vegetable spray. Transfer roast, fat side up, to prepared rack and let rest for 10 to 15 minutes. Pat roast dry with paper towels. Crumple 12-inch piece of foil into 3-inch ball and place under ribs to elevate fat cap. Broil until surface of roast is browned and crisp, 4 to 8 minutes.

Transfer roast to carving board and discard ribs. Slice meat into \threequarters inch thick slices. 

Source: America's Test Kitchen

Notes from Christmas 2022: Cooked a 3 bone 6.5 pound roast from Home Place Pastures to serve 9 people. With okay sized slices there was enough to serve everyone with very little left over. The roast was cooked Sous Vide using a recipe from America's Test Kitchen.

Notes from Christmas 2023: Cooked a 3 bone 4.9 pound roast from Porter Road to serve 7 people.  With good sized pieces there was enough to serve everyone with none left over.

\newpage

\recipe{Crispy Roast Potatoes}
\label{CrispyRoastPotatoes}

\ingredients{Kosher salt;
\half teaspoon (4 g) baking soda;
4 pounds (about 2 kg) russet or Yukon Gold potatoes, peeled and cut into quarters, sixths, or eighths, depending on size;
5 tablespoons (75 ml) extra-virgin olive oil, duck fat, goose fat, or beef fat;
Small handful picked fresh rosemary leaves, finely chopped;
3 medium cloves garlic, minced;
Freshly ground black pepper;
small handful fresh parsley leaves, minced.}

Adjust oven rack to center position and preheat oven to 450\deg F (230\deg C) (or 400\deg F (200\deg C) if using convection). Heat 2 quarts (2L) water in a large pot over high heat until boiling. Add 2 tablespoons kosher salt (about 1 ounce; 25 g), baking soda, and potatoes and stir. Return to a boil, reduce to a simmer, and cook until a knife meets little resistance when inserted into a potato chunk, about 10 minutes after returning to a boil.

Meanwhile, combine olive oil, duck fat, or beef fat with rosemary, garlic, and a few grinds of black pepper in a small saucepan and heat over medium heat. Cook, stirring and shaking pan constantly, until garlic just begins to turn golden, about 3 minutes. Immediately strain oil through a fine-mesh strainer set in a large bowl. Set garlic/rosemary mixture aside and reserve separately.

When potatoes are cooked, drain carefully and let them rest in the pot for about 30 seconds to allow excess moisture to evaporate. Transfer to bowl with infused oil, season to taste with a little more salt and pepper, and toss to coat, shaking bowl roughly, until a thick layer of mashed potato-like paste has built up on the potato chunks.

Transfer potatoes to a large rimmed baking sheet and separate them, spreading them out evenly. Transfer to oven and roast, without moving, for 20 minutes. Using a thin, flexible metal spatula to release any stuck potatoes, shake pan and turn potatoes. Continue roasting until potatoes are deep brown and crisp all over, turning and shaking them a few times during cooking, 30 to 40 minutes longer.

Transfer potatoes to a large bowl and add garlic/rosemary mixture and minced parsley. Toss to coat and season with more salt and pepper to taste. Serve immediately.

Serves 6 to 8.

Source: Serious Eats

Notes:

Russet potatoes will produce crisper crusts and fluffier centers. Yukon Golds will be slightly less crisp and have creamier centers, with a darker color and deeper flavor. You can also use a mix of the two.

The potatoes should be cut into very large chunks, at least 2 to 3 inches or so. For medium-sized Yukon Golds, this means cutting them in half crosswise, then splitting each half again to make quarters. For larger Yukon Golds or russets, you can cut the potatoes into chunky sixths or eighths.

\newpage

\recipe{Horseradish Sauce}
\label{HorseradishSauce}
\ingredients{\quarter cup Inglehoffer cream style horseradish; 1 cup sour cream; 1 teaspoon Worcestershire sauce; 1\half tablespoons Dijon mustard; 1 teaspoon champagne vinegar (optional); \half tablespoon fresh chopped chives (optional); salt and black pepper to taste.}

Whisk all ingredients together then adjust amounts to taste.

\recipe{Chimichurri}
\label{Chimichurri}
\ingredients{\half cup flat leaf parsley, roughly chopped; 1 shallot, peeled (optional); 4 garlic cloves, peeled; 1 teaspoon dried oregano; 3 tablespoons red wine vinegar; 1 small red chili, cleaned of seeds (optional); 1 teaspoon salt; \half teaspoon pepper; \twothirds cup extra virgin olive oil.}

Combine all ingredients except olive oil in a food processor and process until the finely chopped but not paste (don't over process). Transfer contents of food processor to a bowl then stir in olive oil by hand.


\newpage



\recipe{Warm Springs Eggnog}
\label{WarmSpringsEggnog}

\ingredients{6 eggs;
6 tablespoons sugar;
8 ounces heavy cream;
3 to 6 ounces bourbon;
nutmeg, freshly grated, for garnish.}

\textsc{A Goggans family Christmas tradition.  This eggnog is quite thick and more of a desert than a drink.
This recipe serves six.}

Separate the eggs.

Beat the egg whites to soft peaks.  Add the sugar and continue
beating until stiff peaks form.

In a separate bowl, whip cream until stiff.

In another bowl, beat egg yolks lightly and add bourbon.

Fold the egg white mixture, whipped cream, and egg yolk mixture
together.

Serve with a spoon in glasses garnished with grated nutmeg.



\chapter{ Herb and Spice Blends}

\recipe{Herb Rub for Pork, Lamb, or Beef}

\ingredients{2 tablespoons chopped fresh basil or 2 teaspoons dried;
2 teaspoons chopped fresh thyme or 1 teaspoons dried;
1 tablespoons chopped fresh rosemary or 2 teaspoons dried;
1 tablespoon fennel seeds;
2 teaspoons whole coriander or 1 teaspoon ground coriander;
2 teaspoons garlic powder or granulated garlic;
2 tablespoons salt;
1 tablespoon whole black pepper or 2 teaspoons ground black pepper.}

Add ingredients to work bowl of an electric spice grinder or  mini food
processor and grind until a course powder is obtained. Store sealed in the
freezer if any fresh herbs are used.

Yields \half cup.

Source: Bruce Aidells and Denis Kelly from The Complete Meat Cookbook.

\newpage
\recipe{Curry Powder}
\label{CurryPowder}
\ingredients{2 teaspoons cumin seeds; 
2 teaspoons cardamon seeds;
2 teaspoons coriander seeds;
4 teaspoons ground turmeric;
1 teaspoon dry mustard;
\quarter teaspoon cayenne.}

Toast the cumin, cardamon, and coriander seeds in a small, dry skillet over medium-low heat until the seeds are lightly browned and fragrant, 2 to 3 minutes. Transfer to a bowl and let cool completely.

Add the turmeric, mustard powder, and cayenne and mix to combine.

Grind the spices in a spice grinder or blade-type coffee-grinder reserved for grinding spices.

Store in an airtight container for up to 2 months.

Source: Donald Link from Down South.



\newpage
\recipe{Creole Seasoning}
\label{CreoleSeasoning}

% \ingredients{\third cup table salt; \quarter cup granulated garlic; \quarter cup freshly ground black pepper; 
% 2 tablespoons cayenne pepper; 2 tablespoons dried thyme; 2 tablespoons dried basil; 2 tablespoons dried oregano;
% \third cup paprika; 3 tablespoons granulated onion.}
% Makes 2 cups

\ingredients{4 teaspoons table salt;
1 tablespoon granulated garlic;
1 tablespoon freshly ground black pepper;
\half tablespoon cayenne pepper;
\half tablespoon dried thyme;
\half tablespoon dried basil;
\half tablespoon dried oregano;
4 teaspoons paprika;
2\quarter teaspoons granulated onion.}

Thoroughly combine all ingredients in a mixing bowl, and pour the mixture into a large glass jar.  Seal the jar so that it is airtight. Makes \half cup.

\newpage
\recipe{Coffee-Chile Dry Rub for Steaks}

\ingredients{2 teaspoons finely ground dark-roast coffee;
2 teaspoons ancho chile powder;
2 teaspoons dark brown sugar, tightly packed;
1 teaspoon smoked paprika;
1 teaspoon kosher salt;
\half teaspoon ground cumin.}


In a small bowl, mix all the ingredients thoroughly, massaging the mixture with your fingers to break down the dark brown sugar into fine crystals.

Liberally sprinkle a thin layer of the rub onto the steak, then pat it in with your fingers so it adheres.

Source: Matt Lee And Ted Lee from the New York Times, modified to make a reasonable quantity for 1 or 2 steaks.

\newpage
\recipe{Baking Powder}

\ingredients{Cream of tartar; Baking soda}

Mix two parts cream of tartar with one part baking soda.

\recipe{SPG}

\ingredients{\half cup course-ground black-pepper; \quarter cup kosher salt; 1 tablespoon granulated garlic.}

Thoroughly combine all ingredients in a mixing bowl, and pour the mixture into a small glass jar.  Seal the jar so that it is airtight.

\chapter{  Drinks}

\recipe{Last Word}
\label{LastWord}
\ingredients{\threequarters ounce London dry gin; \threequarters ounce green Chartreuse; \threequarters ounce Luxardo maraschino liqueur; \threequarters ounce fresh line juice.}

Shake all the ingredients with ice, then strain into a chilled coupe. Usually served without a garnish.

\recipe{Final Ward}
\label{FinalWard}
\ingredients{\threequarters ounce bonded rye whiskey; \threequarters ounce green Chartreuse; \threequarters ounce Luxardo maraschino liqueur; \threequarters ounce fresh lemon juice.}

Shake all the ingredients with ice, then strain into a chilled coupe. Usually served without a garnish.

\newpage

\recipe{Crop Top}
\label{CropTop}

\ingredients{\threequarters ounce London dry gin; \threequarters ounce Amaro Montenegro; \threequarters ounce Giffard Cr\'{e}me de Paplemousse; \threequarters ounce fresh lemon juice.}

Shake all the ingredients with ice, then strain into a chilled Nick \& Nora glass. Usually served without a garnish.

Devon Tarby, 2013

\recipe{Naked \& Famous}
\label{NakedAndFamous}
\ingredients{\threequarters ounce mezcal; \threequarters ounce yellow Chartreuse; \threequarters ounce Aperol; \threequarters ounce fresh line juice.}

Shake all the ingredients with ice, then strain into a chilled coupe. Usually served without a garnish.

\recipe{The Bird Is The Word No.2}
\ingredients{\third ounce chilled water (omit if using wet ice); \threequarters ounce Grappa; \threequarters ounce Green Chartreuse; \half ounce Luxardo maraschino liqueur; \half ounce fresh lime juice; 4 drops chocolate bitters.}

Shake all ingredients with ice and fine strain into a chilled glass.

\recipe{Sunflower}
\label{Sunflower}
\ingredients{\threequarters ounce gin; \threequarters St-Germain Elderflower Liqueur; \threequarters ounce Cointreau or Grand Marnier; \threequarters ounce fresh lemon juice; lemon twist.}

Shake all the liquid ingredients with ice, then strain into a chilled coupe. Garnish with a lemon twist. 

\recipe{Alpine Negroni}
\label{AlpineNegroni}

\ingredients{1  ounces gin;
\threequarters ounce Cocchi Americano;
\threequarters ounce Suze gentian liqueur;
\half ounce Genepi;
\sixth ounce white creme de menthe;
3 dashs lemon bitters;
\half pinch salt;
rosemary sprig.}

Add the gin, Cocchi Americano, Suze, Genepi, creme de menthe, bitters and salt into a mixing glass with ice and stir for 15 to 20 seconds until well-chilled. Strain into a rocks glass over fresh ice. Garnish with a rosemary sprig. 

\recipe{Aviation Cocktail}
\label{Aviation}

% \hyperref[CeruleanGin]{butterfly pea gin}
\ingredients{2 ounces \hyperref[CeruleanGin]{butterfly pea gin};
\threequarters ounce fresh lemon juice;
\half ounce Luxardo maraschino liqueur;
1 teaspoon cr\'{e}me de violette (optional). }

Shake all the ingredients with ice, then stain into a chilled coupe. Flame an orange zest strip, express over drink and then use as garnish.

\recipe{Sazerac}
\label{Sazerac}
\ingredients{absinthe; 1\half ounces rye; \half ounce Cognac; 4 dashes Peychaud's bitters; 1 dash Angostura bitters; lemon twist.}

Rinse an Old-Fashioned glass with absinthe and dump. Stir the remaining liquid ingredients over ice, then strain into the glass. Express the lemon twist over the drink and discard.

\recipe{The Smartest Man Alive}
\label{SmartestMan}
\ingredients{ \threequarters ounce rye whiskey;
\threequarters ounce Suze;
\threequarters ounce fresh lemon juice;
\half ounce 1:1 Honey Syrup. }

In a cocktail shaker, combine the rye, Suze, lemon juice and Honey Syrup. Fill the shaker with ice and shake well. Fine-strain into a chilled coupe.

Variation: In a small saucepan, heat the ingredients over moderate heat until hot, then stir in 3 ounce brewed black tea. Pour into a warmed mug and garnish with a lemon wheel.

\recipe{Hemingway Daiquiri}
\label{HemingwayDaiquiri}
\ingredients{1\half ounces aged white rum; \half ounce Luxardo maraschino liqueur; 1 ounce fresh grapefruit juice; \half ounce fresh lime juice; 1 teaspoon simple syrup; lime wedge.}

Shake all liquid ingredients with ice, then strain into a chilled coupe. Garnish with the lime wedge.

\recipe{Nuclear Daiquiri}
\label{NuclearDaiquiri}
\ingredients{1 ounce overproof rum;
    \threequarters ounce Green Chartreuse;
    \quarter ounce Velvet Falernum;
    1 ounce fresh lime juice; lime wheel.}

Shake all liquid ingredients with ice, then strain into a chilled Nick and Nora glass. Garnish with the lime wheel.

\recipe{Flying Dutchman}
\label{FlyingDutchman}
\ingredients{1\half ounces genever;
\half ounce Benedictine;
\half ounce yellow Chartreuse;
\half ounce lemon juice, freshly squeezed}

Add the genever, Benedictine, yellow Chartreuse and lemon juice into a shaker with ice and shake until well-chilled. Strain into a chilled cocktail glass. Garnish with a lemon twist if desired.

\newpage

\recipe{Caipirinha}
\label{Caipirinha}

\ingredients{3 ounces Cacha\c{c}a; 2 to 3 teaspoons white granulated sugar; 1 lime.}

Prepare lime by washing it and then cutting a thin slice off of the stem and opposite end.  Cut the lime in half along the axis and place the cut side down on a cutting board. Cut each half into 9 pieces using a \#  cut pattern.  Place lime pieces in a large rocks glass and sprinkle with sugar. Mash the sugar and lime pieces together using a muddler. Pour over Cacha\c{c}a, stir, and fill with crushed ice or small cubes of ice.

\recipe{Rabo de Galo}
\label{RaboDeGalo}

\ingredients{2 ounces Cacha\c{c}a ; \half ounce Cynar; \half ounce Carpano Antica Formula or other sweet red vermouth; strip of orange zest.}

Measure liquid ingredients directly into a tumbler. Stir and add ice. Express orange zest over tumbler then drop in for garnish.

\recipe{Macuna\'{i}ma}
\label{Macunaima}

\ingredients{1\twothirds ounces Cacha\c{c}a; \twothirds ounce lemon or lime juice; \half ounce rich sugar syrup; \quarter ounce Fernet Branca.}

Shake all ingredients with ice and strain into a chilled glass.

\newpage

\recipe{Premade Martini}
\label{Martini}

\ingredients{4 ounces (111 g) Beefeater gin; 1\half ounces (22 g) filtered water; \half ounce (14 g) Dolin dry vermouth.}

\textsc{This recipe makes two drinks. To make enough to fill a quart or liter bottle multiply amounts by five. The gram measurements assume the gin is 47\% ABV and the vermouth is 17.5\% ABV.}

Using a funnel combine all ingredients in a glass bottle. Shake to combine ingredients and then seal the bottle and store in the freezer overnight.

When mixture is fully chilled, pour 3 ounces into a cocktail glass and garnish with an olive or lemon twist.  



If there is ice in the bottle when you remove it from the freezer then give it a good shake and leave it out of the freezer for a minute or so before pouring.  



\recipe{Premade Manhattan}

\ingredients{11\quarter ounces rye; 5 ounces sweet vermouth; \half ounce Angostura bitters; 7\half ounces water.}

Using a funnel combine all ingredients in a glass bottle. Shake to combine ingredients and then seal the bottle and store in the freezer overnight.

When mixture is fully chilled, pour 4\half ounces into a cocktail glass and garnish with a lemon twist.  

\newpage

\recipe{Tuxedo No. 2}

\ingredients{2 ounces gin; \threequarters ounce dry vermouth; \quarter ounce Luxardo maraschino liqueur; 2 dashes absinthe; 1 dash orange bitters (optional); lemon twist; Luxardo maraschino cherry (optional).
}

Add all liquid ingredients to a mixing glass with ice. Stir until cold (10 to 30 seconds depending on ice size). Strain into a stemmed cocktail glass, express the oils of a lemon peel over the top and garnish with the peel and, if desired, a cocktail cherry.

\recipe{Montgomery's  Martini}

\ingredients{2\half ounces London dry gin; \sixth ounce extra dry vermouth;  1 dash orange bitters.}

Shake all ingredients with ice and strain into a chilled glass.  Garnish with an orange twist if desired.

\recipe{Gin Blossom Martini}

\ingredients{1\half ounces gin; 1\half ounces bianco vermouth or Lillet blanc or Cocchi Americano; \threequarters ounce apricot eau-de-vie;  2 dashes orange bitters}

Shake all ingredients with ice and strain into a chilled glass.  Garnish with an orange twist if desired.

\newpage

\recipe{Black Hawk}

\ingredients{1\half ounces Rye Whiskey; 1\half ounces Sloe Gin;  Luxardo maraschino cherry.}

Add all liquid ingredients to a mixing glass with ice. Stir until cold (10 to 30 seconds depending on ice size). Strain into a chilled stemmed cocktail glass and garnish with a cocktail cherry.

\recipe{Across the Pacific}

\ingredients{1 ounce Appleton Reserve rum; \half ounce Smith \& Cross Jamaican rum; \half ounce Averna amaro; \threequarters ounce lime juice, freshly squeezed; \threequarters ounce \hyperref[orgeat]{orgeat}; lime wheel (garnish); grated nutmeg (garnish)}


Add all ingredients into a shaker with one ice cube and shake until the ice melts and mixture is frothy. Pour into a rocks glass filled with crushed ice. Garnish with a lime wheel and grated nutmeg. 

Source: Meaghan Dorman via Liquor.com

\recipe{Southern Slipper}
\ingredients{1\half ounce \hyperref[CantaloupeLiqueur]{Cantaloupe Liqueur};
\half ounce Grand Marnier;
\half ounce Luxardo maraschino liqueur;
1 ounce fresh lemon juice;
Luxardo maraschino cherry.}

Shake all liquid ingredients with ice, then stain into a chilled coupe or Nick \& Nora glass. Garnish by dropping a Luxardo maraschino cherry in the glass and letting it sink to the bottom.

\newpage

\recipe{Cantaloupe Sour}
\ingredients{2 ounces \hyperref[CantaloupeLiqueur]{Cantaloupe Liqueur};
1 ounce fresh lemon juice;
1 egg white;
orange bitters.}

Combine cantaloupe liqueur, lemon juice and egg white in a shaker and dry shake. Add ice and shake again. Double strain into a chilled Nick and Nora glass. Top with a few drops of bitters.  For an attractive effect, use a toothpick to drag a line from the bitters drops.

\recipe{My Oh Mai}

\ingredients{\half ounce lime juice;
\quarter ounce orgeat;
\threequarters ounce  gentian liqueur (Suze);
1 ounce falernum;
1\half ounce bourbon;
1 tablespoon dark overproof rum.}
Shake all ingredients except rum in a shaker with ice. Strain into an old-fashioned glass filled with pebble ice. Top with a rum float.

Source: James Siegel via. Garden \& Gun

\recipe{Doctor No.4}

\ingredients{1\half ounce Swedish Punch liqueur; \half ounce Smith \& Cross Traditional Pot Still Navy Strength Jamaican rum; \threequarters fresh lime juice; orange zest twist.}

Shake all liquid ingredeints with ice and fine strain into a chilled coupe or Nick and Nora glass. Garnish with orange twist.

Source: Difford's Guide adapted from Doctor Cocktail by David A. Embury 

\newpage

\recipe{Sloe Gin Fizz}

\ingredients{2 ounces sloe gin;
1 ounce fresh lemon juice;
\half ounce semi-rich simple syrup;
2-3 ounces sparkling water;
Lemon wheel and cocktail cherry for garnish.}

\recipe{Army \& Navy}

\recipe{Southside}

\recipe{Grapefruit Wine}
\label{GrapefruitWine}

\ingredients{1 bottle of dry French ros\'e; 1 pink grapefruit; simple syrup to taste.}

Juice the grapefruit and mix the juice with the wine.  Add a splash of simple syrup to taste, or not. Serve over ice, or with frozen supremed grapefruit segments to keep the drink cold.

\recipe{Glow-Sour}
\label{GlowSour}
%\hyperref[GlowSour]{Glow-Sour}
\ingredients{2 ounces \hyperref[TumericGin]{Tumeric Infused Gin}; \threequarters ounce freshly strained lime juice; flat \threequarters ounce simple syrup; 3 drops saline solution; 1-2 dashes orange bitters.}

Combine the ingredients in a cocktail shaker, add ice, shake, and strain into a chilled coupe glass.

Source: Liquid Intelligence by Dave Arnold

\newpage

\recipe{Frozen Daiquiri}

\equilpment{Vitamix blender}

\ingredients{1\half tablespoons granulated sugar;
2 ounces light aged rum;
\threequarters ounce fresh lime juice;
5 drops 20\% saline solution;
\quarter teaspoon xantham gum;
170 grams pebble or crushed ice;
lime wheel for garnish.}

Add sugar, rum, lime juice and saline solution to the blender container, attached lid and blend to dissolve  sugar. With the blender off, remove the clear top plug from the lid and then, with the blender running, slowly add the xantham gum.  With the blender off, replace the clear plug in the lid,  add ice to the blender container, attach lid, and blend until smooth.  Pour drink into a frozen old fashioned glass and garnish with a lime wheel.

Source: 
Tropical Standard: Cocktail Techniques \& Reinvented Recipes
by Garret Richard and Ben Schaffer 
\newpage

\recipe{Original Chatham Artillery Punch}

\equilpment{punch bowl; alcoholic friends.}

\ingredients{8 lemons;
1 pound superfine sugar;
750-milliliter bottle bourbon or rye;
750-milliliter bottle Cognac;
750-milliliter bottle dark Jamaican rum;
3 bottles Champagne or other sparkling wine;
Nutmeg.}

Squeeze and strain the lemons to make 16 ounces of juice. Peel the lemons and muddle the peels with the sugar. Let the peels and sugar sit for an hour, then muddle again. Add the lemon juice and stir until sugar has dissolved. Strain out the peels.

Fill a 2- to 3-gallon punch bowl with crushed ice or ice cubes. Add the lemon-sugar mixture and the rum, the Cognac and the bourbon or rye. Stir and add the Champagne. Taste and adjust for sweetness. Grate nutmeg over the top and serve.

Yield: About 25 drinks.

Source: David Wondrich


\recipe{Campari Boilermaker}

\ingredients{6 ounces citrus Gose, sour or Berliner Weisse; 2 ounces Campari; orange wedge garnish.}

Combine beer and Campari in a 16 ounce glass, fill with ice, and top with orange wedge squeezed over glass.

\newpage
\recipe{Cherry Liqueur}
\label{CherryLiqueur}

\ingredients{2 pound Bing or other fresh cherries;
2 pound of sugar;
10 cups of good California brandy.} 

\textsc{If you start making this liqueur when fresh cherries are
available in the store it will be ready for Christmas.  This
recipe can also be made with vodka.}

Wash and stem the cherries and place in a towel to dry.

Pierce the cherries to the stone with a fork and place in a 2
quart wide mouth jar.

Pour the sugar over the cherries.

Pour the brandy (or vodka) over the sugar and cherries.  Cover
tightly with a lid and place out of the way.  It is not necessary
to shake or stir the brandy mixture.

Let the mixture stand without shaking or stirring for at least
3 months (6 months is better).  The sugar will be dissolved and
the liqueur will be a beautiful dark rose color.

Strain the liqueur into  bottles. Save the cherries to eat or use them
to garnish deserts.

\newpage

\recipe{Cantaloupe Liqueur}
\label{CantaloupeLiqueur}

\ingredients{1000 grams cantaloupe pulp, finely chopped; 500 ml  95\% ABV Neutral (Everclear); 500 ml purified water; 400 grams cane sugar.}

For this recipe you will need a 2-liter glass container with an airtight seal (two 1-quart Mason jars will also work). For the best results, use a very ripe cantaloupe and carefully remove every bit of peel before finely dicing the pulp.

Put the cantaloupe pulp in the glass container, cover with the alcohol and then the seal container. Macerate for 30 days in a cool dark place shaking the mixture every now and then.

After 30 days strain the cantaloupe pulp from the alcohol using a wire mesh strainer.  Press hard on the pulp to remove all the liquid and then discard the pulp. Clean the glass container and return the flavored alcohol to it. 

Prepare the syrup by bring the water and sugar to a boil then reducing heat to a simmer and stirring until the sugar is completely dissolved. Cool the syrup to room temperature before adding it to flavored alcohol. Stir well to combine, seal the jar and let it rest for 1 week before proceeding.

Filter the liqueur using a double layer of muslin moistened with alcohol. Filter again and returned to rinsed bottle. Seal the bottle and place it where it will not be disturbed to allow the fine particles to settle to the bottom.  When the liquid above a thin layer of particles on the bottom of the bottle is very clear, rack off the clear liquid and transfer to an attractive bottle.

Let the liqueur rest for 3 months before consuming.  Total time from start to consumption is around 4\half months.

\recipe{Genziana Amaro}
\label{GenzianaAmaro}

\ingredients{500 ml white wine; 200 grams gentian root; 100 ml purified water; 70 grams cane sugar; 100 ml 95\% ABV Neutral Spirits (Everclear).}

Combine wine and gentian root in a glass container with at tight fitting lid.  Place container in a dark place for 40 days.  Shake container every day for the first 10 days.  After 40 days, strain and filter the wine to remove the gentian root.

Make a syrup with the sugar and water and leave to cool.  When the syrup is room temperature add wine and stir to combine. Add neutral spirits, stir to combine and then bottle.

Yield: one 750 ml bottle

\recipe{Lime Cordial}
\label{limecordial}
\ingredients{250 grams sugar;
    8 fluid ounces/240 ml hot water;
    1\half fluid ounces/45 ml fresh lime juice; 
    1\half fluid ounces/45 ml freshly grated lime peel; 
    1  fluid ounce/30 ml citric acid.}

Combine all of the ingredients in a blender. Blend on medium speed for 30 seconds. Strain with a fine strainer. Bottle and refrigerate.

Recipe source: https://jeffreymorgenthaler.com/lime-cordial/

\newpage

\recipe{Chai Masala}
\ingredients{6 cups water;
20 grams English Breakfast tea, regular or decaffeinated;
40 grams fresh ginger, unpeeled but coarsely chopped;
25 grams turmeric, unpeeled but coarsely chopped; 
\half gram star anise;
1\half grams green cardamon pods;
3 grams Ceylon cinnamon stick;
\half gram whole black peppercorns;
2 cups whole milk;
2 tablespoons honey.}

Coarsely chop the fresh ginger and turmeric. Crack star anise, cardamon pods, and peppercorns against a cutting board using the end of a wooden rolling pin. Break up the cinnamon stick by the same method or by hand.  

Add water, tea, ginger, turmeric, anise, cardamon, cinnamon and pepper to a large sauce pan. Bring to boil and then cook at a boil for 3 minutes. While the tea is boiling, bring milk to near boiling temperature in the microwave.  At the end of the 3 minutes, add the milk.  When the mixture returns to a boil cook at a low boil for an additional 2 minutes.  Remove from heat, add honey and stir to combine. Strain out the solids and serve.

Yield: 6 to 8 servings

\recipe{Orgeat}
\label{orgeat}

\ingredients{200 grams unsweetened almond milk;
300 grams table sugar;
3.5 grams Cognac;
4.5 grams amaretto;
750 milligrams rose or orange blossom water.
}

Combine the almond milk and sugar in a saucepan over medium-low heat and cook, stirring occasionally, until the sugar has dissolved. Remove from the heat and stir in the Cognac, amaretto, and rose water. Let cool to room temperature, then transfer to a storage container and refrigerate until ready to use, up to 2 weeks.

\newpage
% Recipe from Everclear web site

\recipe{Aromatic Bitters}

\ingredients{\\
\\
\textsc{Bittering Jar --}
2 grams gentian root;
3.25 grams burdock root;
90 ml 95\% ABV neutral spirits.\\
\\
\textsc{Aromatic Jar --}
1 orange zest (15 grams);
2 cinnamon sticks (12 grams);
15 whole allspice berries (1.5 grams);
10 whole cloves (1 gram);
3 whole star anise (4 grams);
45 grams freshly chopped ginger (about a 2 inch nub);
300 ml 95\% ABV neutral spirits.\\
\\
\textsc{Brown sugar simple syrup --} 
\half cup brown sugar; \half cup water.}

For the bittering jar put chopped gentian root and burdock root in a half-pint-sized mason jar.
Fill the jar with 90 ml of 95\% ABV neutral spirits then seal and store in a cool, dry place away from direct sunlight. Shake once daily.

For the aromatic jar first roughly chop ginger root then
peel zest from a freshly washed orange and remove the pith.
Combine ginger root, orange zest, cinnamon sticks, allspice berries, cloves and star anise in pint-sized mason jar.
Fill the jar with 300 ml of 95\% ABV neutral spirits then seal and store in a cool, dry place away from direct sunlight. Shake once daily.

To make brown sugar simple syrup 
combine brown sugar and water in small sauce pan. Bring to boil, stirring continuously until sugar is dissolved.


After steeping both jars for 3 weeks, strain both jars into a  clean jar using a mesh strainer and a coffee filter. Stir the combined solutions. To the combined solutions, slowly add
brown sugar simple syrup to taste.

Bottle, store in a cool, dry place, and enjoy.

Source: Modified from a recipe on the Everclear website

\chapter{Gin}

\recipe{Making Gin}

\textsc{Gin is just neutral spirits that is flavored with juniper and other botanicals using a still. The first step in making gin is to make or buy neutral sprits. Making neutral spirits from scratch that is neutral enough in flavor to use in making quality gin requires a column still and hard won experience in using it. Because of this most people are better off purchasing neutral spirits from the liquor store. For these recipe you can use either bottom shelf 80 proof vodka that is already 40\% ABV or 95\% ABV neutral spirits (e.g. Everclear) diluted to 40\% ABV.}

\textsc{I don't crush the juniper berries in these recipe for two reasons. First, using crushed berries gives inconsistent results because is it difficult to crush uniformly and consistently. Second, I find that crushed berries release harsh flavors that whole berries do not. For similar reasons I don't macerate the botanicals in the spirits before distilling.}

\textsc{I put the fresh citrus zest in the gin basket because I like the fresh citrus flavor that results better than the marmalade like flavor that results from putting the zest into the still with the spirits. I also put fresh flowers in the gin basket so that more of their aroma transfers to the gin.}

\textsc{In these gin recipes the first 25 ml of distillate to emerge from the still is discarded to control the amount of juniper and citrus oil in the final product.  Without this step the gin is likely to louch (turn cloudy) when water is added to it and have too much harsh juniper and citrus flavor. }

\textsc{To give your gin additional ``mouth feel'' add 9 ml of USP glycerin to each 750 ml bottle of finished gin.}

\textsc{Online sources for botanicals include, Amazon, Mountain Rose Herbs, and San Francisco Herb Company.}

\textsc{For your first attempt at making gin try \hyperref[GinNumber1]{Gin \#1} or, for the more adventurous, \hyperref[GinNumber7]{Gin \#7}}. Both of these gins mimic well known commercial brands. 

\newpage

\recipe{Gin \#1}
\label{GinNumber1}

\ingredients{1 liter 40\% ABV neutral spirits;
15 grams juniper berries, whole;
7.5 grams coriander seeds, whole; 
1.5 grams angelica root, coarsely ground;
2 grams liquorice root, coarsely ground.
}

\textsc{These are the four botanicals used in Tanqueray gin.}

Do not crush the juniper berries or coriander seeds. Using an electric spice grinder, coarsely grind the angelica root and the liquorice root. Add all ingredients to the still.  Turn on still and after the first drip from the still at around 17 minutes collect the first 25 ml of distillate to emerge (discard or save the first 25 ml in a jar labeled ``Gin \#1 Feints'').  After this collect the next 425-450 ml of distillate. This distillate is used to make the gin. To increase the efficiency of future gin runs you can collect the next 200 ml and save it in the same feints jar as the first 25 ml. The contents of the feints jar can be added to the still with the botanicals the next time you make Gin \#1.

To make the gin, dilute the collected 450 ml of distillate with purified water to 45\% ABV. The result should be approximately 750 ml of gin. If you end up a little short of a bottle full then add 45\% ABV neutral spirits (can be made by diluting 95\% ABV neutral spirits) to make up the difference. After bottling it is best to  wait at least one week before drinking to allow flavors to stabilize.

If the gin has too much juniper flavor to suit you then adding 45\% ABV neutral spirits to the gin allows you to reduce the flavor intensity. To make your gin similar in intensity to Tanqueray gin this step is necessary.

\newpage

\recipe{Gin \#2}
\label{GinNumber2}

\ingredients{1 liter 40\% ABV neutral spirits;
15 grams juniper berries, whole;
7.5 grams coriander seeds, whole; 
1.5 grams angelica root, coarsely ground;
2 grams liquorice root, coarsely ground;
3-4 grams prepared lemongrass.
}

\textsc{This is \hyperref[GinNumber1]{Gin \#1} with citrus flavor added by including fresh lemongrass in the gin basket.}

\newpage

\recipe{Gin \#5}
\label{GinNumber5}
\equilpment{Still Spirits Air-Still with gin basket, alcohol hydrometer, Thermopen MK4 instant-read thermometer.}

\ingredients{
1 liter 40\% ABV neutral spirits;
20 grams juniper berries, whole;
5 grams coriander seeds, whole; 
1 gram angelica root, coarsely ground;
0.2 grams cardamom seeds, coarsely ground; 
0.1 grams vanilla bean;
0.5 grams fresh lemon thyme leaves;
2 grams hazelnut pieces, lightly toasted;
35 grams peeled and seeded ripe pear (or apple if pear is unavailable); 0.7 grams dried bitter orange peel, coarsely ground (optional);
\half teaspoon honey (optional)
2 gram fresh rose petals or 1 gram dried rose petals or 0.1 ml Nielsen-Massey Rose Water;
4 grams fresh lemon or orange zest.
} 

Do not crush the juniper berries or the coriander seeds. Add all ingredients except the rose petals and fresh orange or lemon zest to the still. For a very citrus forward gin you can use up to 8 grams of fresh citrus zest, however, you will likely have to correct the diluted gin for louching if you do.  Zest the orange or lemon with a sharp vegetable peeler and add zest strips and rose petals wrapped in cheese cloth to the gin basket.  Turn on still and after the first drip from the still at around 17 minutes collect and discard the first 25 ml of distillate to emerge. After this collect the next 425-450 ml of distillate for use in making the gin. Dilute the collected distillate with purified water to 45\% ABV. The result should be approximately 750 ml of gin. Bottle and wait at least one week before drinking to allow flavors to stabilize.

If the gin has too much juniper or citrus flavor to suit you then 
add 45\% ABV neutral spirits (can be made by diluting 95\% ABV neutral spirits) to the gin to reduce the flavor. Similarly, to correct louching, add 45\% ABV neutral spirits to your gin until it is clear.



\newpage

\recipe{Gin \#7}
\label{GinNumber7}

\ingredients{1 liter 40\% ABV neutral spirits;
15 grams juniper berries, whole;
7.5 grams coriander seeds, whole; 
1.5 grams angelica root, coarsely ground;
0.15 grams orris root, coarsely ground;
1.85 grams almonds, coarsely ground;
1 gram dried lemon peel;
1.5 grams liquorice root, coarsely ground;
1.5 grams cassia bark, coarsely ground;
1.5 grams cubeb pepper berries, whole;
1.5 grams grains of paradise, whole;
2 grams fresh lemon zest or a 2 gram mix of orange, grapefruit and lemon zest.
}

\textsc{Bombay Sapphire recipe according to an article in Wired Magazine.}

\textsc{Because this it the gin I most commonly make, I sometimes measure out 5 or 10 times the amount of botanical,  minus the fresh lemon zest, and store them in a sealed glass bottle in the freezer so I can quickly measure out the amount of botanicals to make single recipe at a later time.  To make measuring out the botanical mix easy it is important to very coarsely grind the ground botanicals.    Minus the fresh lemon zest, there are 33 grams of botanicals in this recipe.}

Do not crush the juniper berries, coriander seeds, cubeb pepper berries, or the grains of paradise. Add all ingredients except the fresh lemon zest to the still.   Zest a fresh clean lemon with a sharp vegetable peeler and add zest strips to the gin basket.  Turn on still and after the first drip from the still at around 17 minutes collect the first 25 ml of distillate to emerge (discard or save the first 25 ml in a jar labeled ``Gin \#7 Feints'').  After this collect the next 425-450 ml of distillate. This distillate is used to make the gin. To increase the efficiency of future gin runs you can collect the next 200 ml and save in the same feints jar as the first 25 ml. The contents of the feints jar can be added to the still with the botanicals the next time you make Gin \#7.

To make the gin, dilute the collected 425-450 ml of distillate with purified water to 45\% ABV. The result should be approximately 750 ml of gin. Bottle and wait at least one week before drinking to allow flavors to stabilize.

If the gin has too much juniper or citrus flavor to suit you then 
add 45\% ABV neutral spirits (can be made by diluting 95\% ABV neutral spirits) to the gin to reduce the flavor. Similarly, to correct louching, add 45\% ABV neutral spirits to your gin until it is clear.

\newpage
\recipe{Gin \#11}
\label{GinNumber11}

\equilpment{Still Spirits Air-Still with gin basket, alcohol hydrometer, Thermopen MK4 instant-read thermometer.}

\ingredients{
1 liter 40\% ABV neutral spirits;
14 grams juniper berries, whole;
5 grams coriander seeds, whole; 
1 gram angelica root, coarsely ground;
0.15 grams orris root, coarsely ground;
0.5 grams green cardamom, coarsely ground;
0.2 grams vanilla bean;
1.5 grams of fresh thyme leaves;
1.5 grams fresh lemon balm leaves;
1.5 grams licorices root, coarsely ground;
5 grams pecan pieces, lightly toasted;
35 grams fresh or frozen blackberries or mulberries;
0.7 grams dried bitter orange peel, coarsely ground;
2.0 grams fresh or frozen aromatic rose petals;
4 grams fresh blood-orange or lemon zest.
} 

Do not crush the juniper berries or coriander seeds. Add all ingredients except the rose petals and citrus zest to the still.  Zest citrus and add citrus zest strips and rose petals wrapped in cheese cloth to the gin basket.  Turn on still and after around 17 minutes collect and discard the first 25 ml of distillate to emerge. After this collect the next 450 ml of distillate for use in making the gin (this will take around 22 minutes after discarding the first distillate). Dilute the collected distillate with purified water to 45\% ABV. The result should be approximately 750 ml of gin. Bottle and wait at least one week before drinking to allow flavors to stabilize.


\newpage

\recipe{Mango Mulberry Gin}
\label{MangoMulberryGin}

\ingredients{1 liter 40\% ABV neutral spirits;
30 grams juniper berries, whole;
2 grams whole coriander seeds, whole;
0.75 grams angelica root, coarsely ground;
7.5 grams dried mango (no added sugar);
3 grams dried mulberries;
1.5 grams ground almonds;
2 pink peppercorns;
0.2 grams vanilla bean;
0.25 grams dried orange peel;
0.1 grams dried lemon peel.
}

Do not crush the juniper berries or the coriander seeds. Add all ingredients to the still.  Turn on still and after the first drip from the still at around 17 minutes collect and discard the first 25 ml of distillate to emerge. After this collect the next 425-450 ml of distillate for use in making the gin. Dilute the collected distillate with purified water to 45\% ABV. The result should be approximately 750 ml of gin. Bottle and wait at least one week before drinking to allow flavors to stabilize.



Source: Adapted from the recipe for ``Mango Mulberry Gin'' by Miss Brewbird from her YouTube channel. 
\newpage

\recipe{Breakfast Gin -- in development}

\ingredients{1 liter 40\% ABV neutral spirits;
whole juniper berries;
whole coriander seeds;
orris root;
fresh sage leaves;
bay leaves;
liquid smoke (or bacon fat wash the final gin);
fresh orange zest.}

\newpage
\recipe{Making Two or Three Bottles of Gin with One Run using an Air-Still}

\textsc{The maximum fill level of the Still Spirits Air-Still is 4 liters so you can scale the given recipe by 2 or 3 without exceeding the capacity of the still. For a times 2 or times 3 recipe run is it only necessary to collect and discard the first 25 ml of distillate.}  

\textsc{When starting with 2 liters of neural spirits for a times 2 run the time to first drip from the still is 32 minutes. After the first 25 ml, collect 850-900 ml of the distillate to make the gin.}

\textsc{When starting with 3 liters of neural spirits for a times 3 run, the time to first drip from the still is around 40 minutes. After the first 25 ml, collect 1,275-1,350 ml of the distillate to make the gin.}

\newpage
\recipe{Making Four Bottles of Gin with One Run using an Air-Still}
\label{ConcentrationMethod}
\textsc{Making four bottles of gin in one run requires a different approach since 4 liters of neutral plus botanicals for four bottles exceeds the capacity of an Air-Still. The approach used here is to make a concentrate}

\textsc{To make four bottles scale the botanicals in a one bottle recipe by a factor of four and place in still kettle and gin basket as directed in the recipe. Add 2 liters of 40\% neutral spirits to the kettle. Run the still as usual discarding the first 25 ml and then collecting 950 - 1000 ml of distillate to make the gin (let taste be your guide when collecting distillate). Dilute the collected distillate with purified water to 45\% ABV. The result should be approximately 1500 ml of very strong flavored gin. Add 1500 ml of 45\% ABV neural spirits to the strong gin to yield 3000 ml of gin, enough to fill four 750 ml bottles.} 



\newpage
\recipe{Turmeric Infused Gin}
\label{TumericGin}

\equilpment{Half liter iSi cream whipper; two 7.5-gram $N_2O$ chargers.}

\ingredients{500 ml \hyperref[GinNumber7]{Gin \#7}; 100 grams fresh turmeric, thinly sliced into 1.6 mm (\sixteenth inch) disks.}

\textsc{Wear nitrile exam gloves and cover your cutting board with plastic wrap to avoid staining your hands and cutting board more or less permanently yellow.}

Add gin and turmeric disks to the body of the whipper. Assemble whipper and  rapid infuse with two chargers for 2 minutes and 30 seconds (shake between adding chargers). Vent whipper by holding the dispenser nozzle up and holding the dispenser trigger.  When all the gas has escaped remove the top and listen to the bubbles.  Allow the bubbling to mellow out for a couple of minutes, then strain.  Press on the turmeric to extract the majority of the gin (don't forget to wear gloves). Bottle in glass and store in refrigerator or freezer.

Use this gin to make a \hyperref[GlowSour]{Glow-Sour}.

Source: Liquid Intelligence by Dave Arnold

\recipe{Cerulean Gin}
\label{CeruleanGin}
\ingredients{750 ml \hyperref[GinNumber1]{Gin \#1}; butterfly-pea tea teabag.}

Combine gin and teabag in a large glass measuring cup and cover with plastic wrap.  Let sit for around 60 minutes (gin should be quite blue) then pour into bottle. This gin turns purple when acid ingredients are added to it. Use it in a gin and tonic where it will change colors when your guest squeezes a lime into it. This gin is ideal for making an \hyperref[Aviation]{Aviation Cocktail}.

\newpage

 \recipe{Mulberry Gin}
 \label{MulberryGin}
 
 \ingredients{\\
\\
\textsc{Mulberry Macerate}
300 grams fresh ripe mulberries;
500 ml of approximately 74\% ABV gin.\\
\\
\textsc{Mulberry Syrup}
75 grams fresh ripe mulberries;
150 grams sugar;
75 ml water.
}
 

 
 In northern Mississippi mulberries usually ripen during the last weeks of May so this is the time to make this gin. Begin by making your favorite gin recipe.  Using the instructions given in the recipe for \hyperref[GinNumber7]{Gin \#7}, starting with 1 liter 40\% ABV neutral spirits in the still, and collect 500 ml of approximately 74\% ABV gin. To make the mulberry macerate, place the fresh mulberries and 74\% ABV gin in a 1 quart Mason jar. Put the jar in a dark pantry for two of months. 
 
 While fresh mulberries are available, make the mulberry syrup. Begin by making a rich simple syrup adding the fresh mulberries while the syrup is hot. When cool, strain the syrup to remove the mulberries and place strained syrup in the refrigerator until needed to sweeten and proof down the gin.
 
 After two months, strain the gin off of the mulberries pressing to extract as much gin and mulberry juice as possible.  Measure the volume of the extracted gin.  Assuming that no ethanol was lost in the maceration process, determine the volume of syrup and additional water that must be added so that the final mulberry gin is approximately 40\% ABV. Add all the syrup and then water to bring the added volume up that necessary to achieve 40\% ABV. Bottle in 325 ml clear bottles. Makes around 900 ml of mulberry gin.
 
\newpage 


\recipe{Muscadine  Gin Liqueur -- In Development}


\ingredients{\\
\\
\textsc{Muscidine Syrup}
300 grams hulls from very-ripe  dark-colored muscadines (e.g. Carlos); 300 ml water; 600 grams sugar.\\
\\
\textsc{Double-Strength Gin-Concentrate}
2000 ml 40\% ABV Neural Spirits;
60 grams juniper berries, whole;
30 grams coriander seeds, whole; 
6 grams angelica root, coarsely ground;
0.60 grams orris root, coarsely ground;
8 grams ground almonds;
4 gram dried lemon peel;
6 grams liquorice root, coarsely ground;
6 grams cassia bark, coarsely ground;
6 grams cubeb pepper berries, whole;
6 grams grains of paradise, whole;
8 grams fresh lemon zest.\\
\\
\textsc{Liqueur}
600 ml muscidine syrup;
1000 ml 75\% ABV gin concentrate;
1165 ml 8\% ABV unsweetened home-made muscadine wine, filtered with paper coffee filter;
435 ml 95\% ABV neural spirits.
}

\textsc{The unsweetened muscadine wine used in this recipe is easiest to obtained by making your own. The wine used here was made using red muscadines and no additional sugar hence the low 8\% ABV. The volume of syrup, wine, and neutral spirits given assume  8\% ABV  dry wine and a yield of 3.2 liters of 40\% ABV liqueur.}

\textsc{Most commercially available muscadine wine has residual sugar and is described on it's label as sweet or semi-sweet.  In addition, commercially available wine has a higher ABV than the wine used here. Commercially made muscadine wine can be used for this recipe, however, adjustments must be made to account for both the extra sugar and the extra alcohol in the wine.}

Begin by making the muscidine syrup. In a stainless steel sauce pan combine the muscidine hulls and the water and simmer for 5 minutes or so.  While the muscidine water is still warm filter it first through a fine metal strainer to remove most of the hulls and then through a paper coffee filter to remove the fine particles. Clean the sauce pan then add the sugar and filtered muscidine water to it. Bring to a simmer and stir until the sugar is completely dissolved. Reserve until needed to sweeten and proof down the liqueur. 

Using the instructions for making \hyperref[ConcentrationMethod]{four bottles of gin in one run using an Air Still} make 1000 ml of gin concentrate.  Combine 600 ml of the muscidine syrup,
the  75\% ABV gin concentrate, the muscadine wine, and the 95\% ABV neural spirits in a one gallon glass jug. Shake well to combine then wait for week for the liqueur to settle and for any sediment to form and drop out of solution. If required, filter to remove any sediment that has formed and then bottle.

The recipe above uses 600 ml of syrup for 1000 ml of gin concentrate. This ratio works well if the wine contains no additional sugar. If using sweetened wine then the amount of syrup should be reduced to account for the sugar in the wine.  Until you have some experience with the wine you are using you will have to guess at the amount of syrup to use. For wine labeled as sweet 400 ml of syrup for 1000 ml of gin concentrate is likely a good starting place. For wine labeled as semi-sweet try 500 ml of syrup for 1000 ml of gin concentrate. With the 1000 ml of gin concentrate at a known \%ABV and the chosen amount of syrup, the volume of wine at it's known \%ABV can be determine using the following equation:

$$V_W = \frac{V_L(P_N-P_L) - V_C(P_N-P_C) - V_S P_N}{P_N - P_W}.$$

The volume of neural spirits to add can then be determined using

$$V_N=V_L-V_C-V_S-V_W .$$

The symbols in these equations are defined as follows: 

$V_L$ -- volume of liqueur

$P_L$  --  \%ABV of liqueur

$V_C$  -- volume of gin concentrate

$P_C$  --  \%ABV  of gin concentrate

$V_W$ --  volume of wine

$P_W$  --  \%ABV  of wine

$V_N$ -- volume of neural spirits

$P_N$ --  \%ABV  of neural spirits

$V_S$ -- volume of syrup







\newpage

\recipe{1 Liter of 40\% ABV Neural Spirits}

\ingredients{420 ml 95\% ABV neutral spirits; 580 ml purified water.}

To make 1 liter of neural spirits at 40\% ABV mix 420 ml of 95\% ABV neutral spirits with 580 ml of purified water.

In Mississippi the cost with tax of 1000 ml 95\% ABV neutral spirits is \$18.77 so 1000 ml 40\% ABV neutral spirits cost \$7.88 to make.

\recipe{2 Liter of 40\% ABV Neural Spirits}

\ingredients{840 ml 95\% ABV neutral spirits; 1,160 ml purified water.}

To make 2 liter of neural spirits at 40\% ABV mix 420 ml of 95\% ABV neutral spirits with 580 ml of purified water.


\recipe{3 Liters of 40\% ABV Neural Spirits}

\ingredients{1,270 ml 95\% ABV neutral spirits; 1,730 ml purified water.}

To make 3 liters of neural spirits at 40\% ABV mix 1,270 ml of 95\% ABV neutral spirits with 1,730 ml of purified water.

\recipe{45\% ABV Neural Spirits}

\ingredients{355 ml 95\% ABV neutral spirits; 395 ml purified water.}

To make 750 ml of neural spirits at 45\% ABV mix 355 ml of 95\% ABV neutral spirits with 395 ml of purified water.


\newpage 
\recipe{Gin Recipe Idea \#1}

\ingredients{1 liter 40\% ABV neutral spirits;
juniper berries;
orris root;
angelica;
liquorice;
coriander;
lemon peel;
cardamom;
cubeb berries;
elderflower.
}

\recipe{Gin Recipe Idea \#2}
\ingredients{1 liter 40\% ABV neutral spirits;
15 grams juniper berries;
5 grams coriander;
0.15 grams orris root;
1 gram angelica root;
1 gram liquorice root;
2 grams fresh lemon peel;
cloves;
star anise;
0.4 grams Ceylon cinnamon;
0.5 grams fresh sage;
0.5 grams fresh rosemary;
wormwood herb;
1 gram bitter orange peel.
}

\recipe{Gin Recipe Idea \#3}
\ingredients{1 liter 40\% ABV neutral spirits;
35 grams juniper berries, whole;
17.5 grams coriander seeds, whole; 
3.5 grams angelica root, coarsely ground;
0.35 grams orris root, coarsely ground;
0.35 gram dried lemon peel;
0.35 gram dried sweet orange peel;
3.5 grams liquorice root, coarsely ground;
3.5 grams cassia bark, coarsely ground.
}

Mercury Gin recipe found in comments on YouTube.

\recipe{Gin Recipe Idea \#4}

Setup to make \hyperref[GinNumber7]{Gin \#7 } then add a shot of espresso to the kettle and put a vanilla bean in the gin basket.

\recipe{Gin Recipe Idea \#5}

Make \hyperref[GinNumber7]{Gin \#7 } then soak muscadine grape in gin to release color and flavor. Strain out the muscadine and then age the result with used oak staves for a few weeks.

\recipe{Gin Recipe Idea \#6}

Make \hyperref[GinNumber7]{Gin \#7 } using the least expensive 100\% agava white tequila you can find instead of neutral spirits.

\recipe{Gin Recipe Idea - Pink Peppercorn Gin}
\ingredients{1 liter 40\% ABV neutral spirits;
pink peppercorns;
juniper berries;
cardamom;
cassia bark, coarsely ground.
honey;
vanilla bean; 
angelica root, coarsely ground;
Earl Gray tea.
}

% \textsc{Pink Pepper Gin is made by the French company Audemus Spirits. This gin is inspired by theirs.}


\newpage

\recipe{Developing A New Gin Recipe}

Instructions for making 750 ml of gin is a small pot still.

To develop a new gin recipe first pick the botanicals you will use.  Use 3 botanicals from the essential category in Table \ref{tab:essential}: juniper berries, coriander seeds, and either orris root or angelica root as a fixative (or use both).  From 5 to 7 of the other the other categories chose one botanical. This results in 8 to 10 total botanicals. 

Do not crush any of the ingredients. When removing the zest from fresh citrus be careful not to remove any white pith along with the zest. Add all ingredients except the flower petals, fresh herbs and fresh citrus zest to the still.  Zest orange and add orange zest strips and rose petals wrapped in cheese cloth to the gin basket.  Turn on still and after around 25 minutes collect and discard the first 15 ml of distillate to emerge. After this collect the next 425 ml of distillate for use in making the gin. Dilute the collected distillate with purified water to 45 \% ABV. The result should be approximately 750 ml of gin. Bottle and wait at least one week before drinking to allow flavors to stabilize.

Botanical amounts for charging a still with 1 liter of 40\% ABV neutral spirit

% Poppy seeds 
% Lemon balm 
% Kombu (dried sea kelp)
% Roasted Sweet Potatoes 
% Cacao Nib
% Cucumber 
% Fresh Mint 
% Dill
% Bergamot
% Blue spruce
% Chamomile (dried flower)
% Waddel seed
% Douglas fir
% Quince 
% Shiso leaf (Perilla frutescens)
% Kaffir Lime leaf
% Galangal (Like ginger) galangal has a sharp citrusy, almost piney flavor, while ginger is fresh, pungently spicy, and barely sweet. 
% Yarrow 
% Allspice berries 
% Cloves 
% Nutmeg 
% TASMANIAN PEPPERBERRY
% Capers
% Citron
% Parmesan cheese 
% Passion fruit
% Kanéroku (Japanise smoked tea)
% Sumac (can be used for color particularly in Old Tom gin)
%\begin{table}[hb]

\begin{table}[H]
    \centering
    \caption{Essential Botanicals}
    \begin{tabular}{@{}p{1.25in}p{3in}p{1.25in}@{}}
        \toprule
      Botanical &  Comment & Amount per liter \\
        \midrule
      	Juniper Berries & It's not gin without it -- do not crush & 15-30 grams  \\
 		Coriander Seeds & Lemon and subtle spice -- do not crush  & 2-7.5 grams \\
		Orris Root & Light slightly perfumed violet or raspberry & 0.15 grams \\
		Angelica Root &  Deep, earthy, musky & 0.75-1.5 grams \\
        \bottomrule
    \end{tabular}
    \label{tab:essential}
\end{table}

\begin{table}[H]
    \centering
    \caption{Citrus Botanicals}
    \begin{tabular}{@{}p{1.25in}p{3in}p{1.25in}@{}}
        \toprule
      Botanical &  Comment & Amount per liter \\
        \midrule
        Dried Lemon Peel & Concentrated flavor use sparingly & 0.4-1 grams\\
        Dried Bitter Orange Peel & Concentrated flavor  use sparingly & 0.48-1.2 grams\\
        Fresh Lime Peel & Remove with a vegetable peeler & 2-4 grams\\
        Fresh Grapefruit Peel & Remove with a vegetable peeler & 2-4 grams\\
        Fresh Lemon Peel & Remove with a vegetable peeler & 2-4 grams\\
        Fresh Navel Orange Peel & Remove with a vegetable peeler & 2-4 grams\\
        Fresh Blood Orange Peel & Remove with a vegetable peeler & 2-4 grams\\
        \bottomrule
    \end{tabular}
    \label{tab:citrus}
\end{table}

\begin{table}[H]
    \centering
    \caption{Sweet Botanicals}
    \begin{tabular}{@{}p{1.25in}p{3in}p{1.25in}@{}}
        \toprule
      Botanical &  Comment & Amount per liter \\
        \midrule
		Liquorice root & Subtle sweetness & 1-2 grams \\
		Honey & Adds a lovely mouth finish  & \quarter to \half teaspoon \\
		Vanilla bean & Strong flavor use sparingly & 0.2 grams \\
\end{tabular}
    \label{tab:sweet}
\end{table}

\begin{table}[H]
    \centering
    \caption{Herbal Botanicals}
    \begin{tabular}{@{}p{1.25in}p{3in}p{1.25in}@{}}
        \toprule
      Botanical &  Comment & Amount per liter \\
        \midrule
		Fresh Rosemary & Savory and strong &  0.5-1 grams \\
		Fresh Sage & Savory and strong & 0.5-1 grams \\
		Lemon Balm & Lemony and herbal & 1-2 grams \\
		Lovage Seeds & Strong bold thyme flavor with hints of spice & 0.5-1 grams\\
		Fresh Mint & Refreshing and strong in flavor use sparingly & 0.1-0.5 grams\\
		Fresh Thyme & Delicate background herb & 0.5-1 grams \\
		Dried Hops & Strong in flavor use very sparingly & 1 or 2 compressed pieces\\
		Bay Leaf & Subtle savory flavor & \quarter to \half leaf \\
		Fresh Tarragon &  Aniseed notes and strong herbal flavor & 0.1-0.5 grams\\
		Dried Lemon Verbena & Lemony with herbal notes & Up to \half teaspoon \\
\end{tabular}
    \label{tab:herbal}
\end{table}

\begin{table}[H]
    \centering
    \caption{Floral Botanicals}
    \begin{tabular}{@{}p{1.25in}p{3in}p{1.25in}@{}}
        \toprule
      Botanical &  Comment & Amount per liter \\
        \midrule
		Fresh or Dried Rose Petals & Epitomizes floral & 0.1-3 grams \\
		Dried Chamomile Flowers & Subtle floral and dusty flavors & \quarter to \half teaspoon \\
		Dried Lavender Flowers & Fragrant and strong use sparingly & 0.1-0.2 grams \\
		Dried Elderflower & Floral but quite powerful use sparingly & 0.3-0.6 grams \\
		Fresh Marigold Flowers & Floral and mild in flavor & 1-2 flowers \\
		Heather Flowers & Floral and mild in flavor & \quarter to \half teaspoon \\
\end{tabular}
    \label{tab:floral}
\end{table}

\begin{table}[H]
    \centering
    \caption{Spicy Botanicals}
    \begin{tabular}{@{}p{1.25in}p{3in}p{1.25in}@{}}
        \toprule
      Botanical &  Comment & Amount per liter \\
        \midrule
		Ceylon Cinnamon & Subtle sweetness & 0.2-0.4 grams \\
		Cassia Bark & Grocery store cinnamon & 1-2 grams \\
		Decorticated Cardamom & Strong in flavor -- do not crush & 0.1-0.2 grams \\
		Green Cardamom Pods & Lightly crush or coarsely grind & 0.5 grams \\
		Ground Nutmeg & Warming and spicy with subtle sweetness & 0.2-0.4 grams \\
		Ground Mace & Milder version of nutmeg & 0.2-0.8 grams \\
		Fresh Ginger & Warming & 0.2-0.4 grams \\
		Black Peppercorn & Hotter than cubeb berries -- do not crush & 1-2 peppercorns \\
		Pink Peppercorn & Mild pepper spice -- do not crush & 1-3 peppercorns \\
		Cubeb Berries & Mild pepper spice, similar to allspice -- do not crush & 1-4 berries \\
		Whole Cumin & Strong fragrant spice -- do not crush & 0.1-0.2 grams \\
        Toasted Amburana Wood & Cinnamon and spice & 0.25-0.5 grams \\
\end{tabular}
    \label{tab:spicy}
\end{table}

\begin{table}[H]
    \centering
    \caption{Fruity Botanicals}
    \begin{tabular}{@{}p{1.25in}p{3in}p{1.25in}@{}}
        \toprule
      Botanical &  Comment & Amount per liter \\
        \midrule
		Dried Mulberries & Sweet and tart & 1-4 grams \\
		Fresh Blackberries & Rich and a little sharp &  3-8 berries \\
		Fresh Raspberries & Sweet and sharp & 2-4 berries \\
		Fresh Apple & Sweet and more subtle & A small slice \\
		Dried Currants & Sweet & 2-6 currants \\
		Dried Mango & Sweet and tropical & 6-9 grams \\
\end{tabular}
    \label{tab:fruity}
\end{table}

\begin{table}[H]
    \centering
    \caption{Nutty Botanicals}
    \begin{tabular}{@{}p{1.25in}p{3in}p{1.25in}@{}}
        \toprule
      Botanical &  Comment & Amount per liter \\
        \midrule
		Chestnut & Dry and nutty & 1 nut \\
		Walnut	& Savory, smooth, and buttery & 1 nut \\
		Pecan Pieces & Lightly toasted & 4-6 grams \\
		Hazelnut Pieces & Lightly toasted, Stands up well to strong flavors & 1-3 grams \\
		Ground Almonds & Sweet and nutty & Up to \half teaspoon \\
		Coconut & Adds mouth feel and texture rather than flavor & 1-3 grams \\

\end{tabular}
    \label{tab:nutty}
\end{table}	

\newpage

\recipe{Botanical Isotopes}

\textsc{An efficient method for developing and refining new gin recipes is to use botanical isotopes. A botanical isotope is just concentrated ``gin'' made with a single botanical (it is not really gin unless the single botanical is juniper berries.) Using this method, a new gin recipe is developed by combining botanical isotopes with a base gin until a pleasing combination is discovered. The base gin is typically made using the essential gin botanicals -- juniper berries and coriander seeds plus orris root and/or angelica root.} 

\textsc{To make a botanical isotope a large quantity of a single botanical is mixed with neutral spirit then the mixture is distilled and the collected distillate is proofed down to the chosen ABV of the base gin.}

\ingredients{500 ml 40\% ABV neutral spirits or 285 ml of 70\% ABV neutral spirits + 215 ml water; average recommended amount per liter times 10 of gin botanical.}

If you intend to macerate the botanical in the final gin recipe then macerate the botanical in 70\% ABV neutral spirits for 24 hours. After macerating for 24 hours add 215 ml water to the neutral spirits/botanical mixture to dilute it to 40\% ABV. Add the diluted mixture to the Air Still. If you will not macerate the botanical in the final gin recipe then add 500 ml of 40\% ABV neutral spirits and the botanical directly to the Air Still. Turn on the still and collect and discard the first 12 ml of distillate to emerge.  After this collect the next 225 ml of distillate for use in making the isotope. Dilute the collected distillate with purified water to the ABV chosen for the gin recipe. This should yield around 375 ml of botanical isotope. 

\recipe{Raspberry Geist}
\ingredients{1000 ml 40\% ABV neutral spirits; 240 grams frozen or fresh raspberries.}

\textsc{The original recipe calls for 2 pounds of fruit per gallon of 40\% ABV neutral spirits.}

Marinate the raspberries in the spirits for one hour at 100\deg F and then for another 23 hours at room temperature. Put raspberries and spirits into Air-Still and distill as usual.  As with gin, the first distillate to come out of the still may have bitter or off flavors so collect at least 25 ml before moving to another container to collect hearts (distillate for the
purpose of making your spirit), however, let taste be your guide in deciding when to start and stop collecting hearts. Expect to collect around 450 ml of hearts. Dilute the collected hearts with purified water to 45\% ABV. 
 
\chapter{Distilling}

\recipe{Turbo Yeast}
\label{Turbo Yeast}

\ingredients{
pinch Epsom salts;
\half tablespoon diamonioum phospate (DAP);
\half teaspoon citric acid;
2 tablespoons yeast nutrient; 
1 tablespoon distiller's yeast.}

Mix all ingredients together in a small stainless steel mixing bowl and store in an airtight container.

Makes enough for a sugar wash made with 5 gallons of water and 10 pounds corn sugar.
\newpage

\recipe{Ted's Fast Fermenting Vodka}
\label{TFFV}

\ingredients{4 kilograms sugar;
250 grams wheat bran;
1 multivitamin tablet, crushed;
pinch Epsom salts;
\half teaspoon diamonioum phospate (DAP);
\half teaspoon citric acid;
50 grams bakers yeast or 25 grams distiller's yeast.}

\textsc{This recipe makes 23 liters or 6.1 gallons.}

In a large pot (6  or more quarts) bring 3 quarts of water to the boil. Add the 250 g of bran, stir, bring back to the boil then simmer for 30 minutes stirring from time to time. When cooked the bran and water becomes a thin porridge.

Dissolve the sugar in warm water and add to the fermenter with cool water to bring quantity up to 20 litres. Add the crushed multivitamin tablet, Epsom salts and diamonioum phospate.

Once the bran has simmered for 30 minutes, add it to the fermenter.

Adjust the pH to around 5 with the citric acid. This usually takes about \half teaspoon of citric acid.

Rehydrate the yeast in 100 ml water and add to the mix. Making sure the start temp is below 30\deg C = 86 \deg F.

Stir well (I use a stick blender or paint mixer and drill to thoroughly aerate it.)

Leave plenty of headroom as this takes off like a rocket. I use a 30 liter fermenter.
Do NOT seal the fermenter for at least 24 hours as a thick foamy cap will form within an hour. I put it under an airlock only after 36 hrs so I can monitor progress.

This produces a fast ferment (normally dry to .990 within 3 to 4 days). The start specific gravity is around 1.070 so the yeast is not pushed hard to produce off-flavours and the resulting wash is approximately 10\% ABV.

Rack off and allow to stand for a couple of days to clear before distilling.

\newpage


\newpage
\recipe{Rice Wash Using Angel Yellow Label Yeast}
\label{RiceWash}

\equilpment{7 gallon brewing pail with air lock, Thermopen MK4 instant-read thermometer.}

\ingredients{9 or 10 pounds (4.1 or 4.5 kg) Goya rice flour; 
3\half or 4 gallons very-hot (around 125\deg F) tap water; 
32 or 36 grams (1.3 oz) Angel Yellow Label Yeast}

\textsc{Goya rice flour can be purchased packaged in either 5-pound or 1\half\hspace{-5pt}-pound bags. For this recipe you can use either 2 5-pound bags or 6 1\half\hspace{-5pt}-pound bags. I prefer to use 5-pound bags, however, the cost per pound of Goya rice flour is often significantly lower when buying 1\half\hspace{-5pt}-pound bags in bulk.}

Add the rice flour to the brewing pail then add the hot water one gallon at a time stirring well after each addition.  Loosely attach the pail's cover then let the wash cool naturally to about 90\deg F (32 \deg C), agitating occasionally to prevent settling. Note that it takes around 7 hours for the wash to cool from 125\deg F to 90\deg F with the brewing pail in a room at cool room temperature. When the wash is 90\deg F add the yeast directly to the wash and stir vigorously. Attach the pail top and the fill air lock.

During fermentation control the wash temperature to be 82-97\deg F (28-36\deg C). The optimal fermentation temperature is about 90\deg F (32\deg C).
The fermentation temperature lower limit is 79\deg F (26\deg C) and the upper limit 100\deg F (38\deg C). Agitate the wash twice every day for the first three days of fermentation. At 90\deg F the wash should ferment to dry in around three or four days. After fermentation is complete give the wash a vigorous stir and then let it settle for several days before siphoning off the clear wash into the still boiler. The fermented wash should yield around 10\% ABV. 
\newpage

\recipe{Barley Malt Wash}

\equilpment{}

\newpage
\recipe{Corn Wash Using High Temperature Alpha-Amylase}

\equilpment{Still Spirits Turbo 500 Boiler; PID temperature controller with thermocouple; AC power drill with steel paint mixer; improvised stand to hold drill and thermocouple; 7 gallon brewing pail with air lock; triplescale brewing hydrometer; Thermopen MK4 instant-read thermometer; pH meter.}

\ingredients{7.5 pounds ground dried corn kernels or ground corn meal; 
5.75 gallon cold tap water;
3.75 grams SEBstar HTL (liquid high temperature Alpha-Amylase);
3.75 grams SEBamyl GL (liquid GlucoAmylase);
2 tablespoons distiller's yeast (Red Star DADY);
citric acid and/or sodium carbonate as needed to adjust pH;
iodine for starch test.}



Add cold water to the boiler then position stirrer and thermocouple. Turn on boiler and set the PID controller to 185\deg F. With the stirrer running on its highest speed, slowly add the corn meal to the water. Measure the pH and then adjust the pH to between 5.6 and 6.5 (lower end recommended). After the pH is adjusted add SEBstar HTL to the wash. Monitor the temperature and when it hits 185\deg F start a timer. The stirrer speed can now be reduced to medium to reduce the sound level.  After 1 hour at 185\deg F do an iodine starch test.  If all the starch has converted then proceed to the next steps, otherwise hold the wash at 185\deg F for an additional 30 minutes and then do another iodine starch test.


When all the starch in the corn meal is converted dextrins, set the PID controller to 140\deg F and wait for the wash to cool to that temperature.  Turning the stirrer to its highest speed will aid in cooling. While the wash is cooling adjust the pH to between 2.8 and 5.5 (upper end recommended).  When the wash is 140\deg F add the SEBamyl GL.  Hold the wash at 140\deg F for 1 hour then test the specific gravity using a hydrometer. The specific gravity should be around 1.060 giving a potential ABV or around 8\%.

Turn off the boiler and allow the wash to cool to the pitching temperature for the yeast (100\deg F to 90\deg F). You can let the wash cool naturally or use ice in zipper lock bags to speed the cooling.

Once the temperature drops to below 100\deg F transfer the wash to the brewing pail. Check the pH again and adjust to around 5 if necessary. Add the yeast directly to the wash and stir vigorously. Attach the pail top and the fill air lock.

During fermentation control the wash temperature to be around 86\deg F.

Note: Five gallons of wash were obtained by letting the wash settle for a few weeks after fermentation ended and then pumping off the clear portion.  The remaining portion was placed in a brew bag placed in a colander and allowed to drain.  Finally the brew bag was squeezed to extract as much wash as possible.  This wash was placed in the T500 with the alembic pot still head for the stripping run. The stripping run ended at 5\% ABV and yielded around 3.5 liters of maybe 30\% ABV low wines.  The low wines were put in the AirStill for the spirit run.  The first 200 ml was discarded and then 200-300ml portions were collected.  Hard cuts were made to remove the heads. The tails tasted much of corn and all but the last portion of tails was included in the spirits.  The run resulted in 1000ml of spirits at 70.5\% ABV and 800ml of feints.


\newpage
\recipe{Ben's Panela Rum Wash}

\ingredients{25 kg Panela; 100 liters water; Fleischmann's bread yeast; A touch of yeast nutrient or some boiled baker's yeast.}
\ingredients{10-12 pounds Panela; 5 gallons water; 1\half teaspoons Fermaid O yeast nutrient; \half teaspoon glucoamylase; around 1 teaspoon citric acid; 20 grams Still Spirits Rum yeast or 40 grams bread yeast. }

Chop panela into small pieces and combine with around 3 gallons of near boiling water in fermenter. Add yeast nutrient to the fermenter. Stir until the panela is completely dissolved and then add 2 gallons of cold water.  Add glucoamylase then stir to incorporate. Test the PH and adjust to around 5.2 with citric acid.  Measure and record the specific gravity of the wash. When the temperature drops to below 90\deg F add yeast and stir vigorously. During fermentation hold the temperature of the wash between 85\deg F and 90\deg F.

Use double distillation with a stripping run and a spirit run. To get more flavor in the spirit run, reserve 10\% to 20\% of the wash before the stripping run and mix it with the low wines from the stripping run for the spirit run. Let the mixture rest for a week or so before making the spirit run. If you have them then add your rum feints to the still before the spirit run.

After selecting the cuts you will use in your final spirit, combine then and proof the result down to around 50\% ABV. Age the white spirit in a light toast, char 2 new American white oak barrel or using a toasted and charred American oak stave in a glass jar.


\newpage

\recipe{Oak-Aged Rice Spirits}

\ingredients{1 liter \hyperref[RiceWash]{rice spirits} at 55\% to 60\% ABV; 
6-inch long stave of medium-toasted and charred American white-oak;
1 or 2 one-inch cubes cut from a used bourbon barrel head (optional).}

Combine all ingredients in a large glass jar with a tightly fitting lid. Place jar in an out-of-the-way location, preferably in a storage room or shed where the temperature cycles from day to night.

Start checking the spirits for taste, smell and color after a three months of so of aging and then every month thereafter. The raw spirit taste and smell should be replaced by the smell and taste of whiskey.  Also, the spirit should turn a dark golden brown so that it is will be the color of commercial whiskey after it is proofed down to 45\% ABV and transferred to a 750 ml bottles. This transformation should take around 6 months.

When the spirits are ready, carefully pour off of the wood into a very fine-mesh stainless steel conical-strainer and collect the strained spirits. Over the course of a few days, proof the strained spirits down to 45\% ABV by adding purified water in timed increments. When diluting the spirits be sure that the spirits and the water are at the same temperature. Adding the water in increments and having the spirit and water at the same temperature reduces the chance that the proteins in the spirit will come out of solution, flocculate, and cloud your spirit. If your spirit does flocculate you will observe a white cloud usually at the bottom of your bottle after it rests for a few days on the shelf. The protein cloud does no harm but is unattractive and so should be avoided if possible. 

One sure way to avoid the problem of flocculation is to age your spirit at the \%ABV you wish to drink it at, say 50\% ABV.  For corn and barley spirits this is usually what I do, however, I find that rice based spirits benefits from aging on wood at a higher ABV. 



\newpage
\recipe{Bearded's Invert Sugar Vodka Wash} 

\ingredients{25  white sugar;
6\quarter teaspoons citric acid;
6\quarter teaspoons + 6\quarter teaspoons  yeast nutrient (Fermaid O);
\quarter teaspoon Epsom salt;
6\quarter tablespoons distiller's yeast (Red Star DADY);
1-2 tablespoons sodium carbonate as needed.}

Bring 3\threequarters gallons of water to boil then add sugar and citric acid. Stir to dissolve. Bring back to a boil and then boil for 20 minutes. Turn off fire and cool for 1 hour. Combine invert sugar syrup with 11\quarter gallons of water. Add 6\quarter teaspoons of yeast nutrient and Epson salt and stir.  Add distillers yeast and stir.  Check pH and adjust to between 4.4 and 5.4.  To raise pH add around 1 tablespoon of sodium carbonate. Measure pH and adjust with more sodium carbonate to get it into the correct range.  Measure and record the final pH and and SP. The SP should be close to 1.092 (12\% potential ABV). When SP drops by 30 percent add the second 6\quarter teaspoons of the yeast nutrient. 


The primary cost of this recipe is for sugar. Currently in Mississippi a 25 pound bag of white sugar cost \$20 with tax making the cost of sugar for this recipe \$16. Based on the cost of sugar only the cost per liter of 95\% ABV neutral is \$5.35. This is around a third of the cost to purchase a liter of 95\% ABV neutral grain spirits.

\newpage
\recipe{Absinthe -- In Development} 
\label{Absinthe}

\ingredients{\\
\\
\textsc{Infusion and Distillation}
28 grams of wormwood;
9 grams of hyssop leaves;
17 grams of calamus root (maybe not - replace with angelica root);
6 grams of melissa officinalis (lemon balm);
28 grams of aniseed;
25 grams of fennel;
10 grams of star anise;
3 grams of coriander.\\
\\
\textsc{Coloring}
4 grams of fresh mint;
1 gram of melissa;
3 grams of wormwood;
1 gram of lime zest;
4 grams of liquorice root.
}

    
\textsc{Absinthe is a gin without juniper but which contains, at a minimum, wormwood, aniseed and fennel. This recipe uses both infusion and distillation to make absinthe.}

Add all of the ingredients to 800 ml of 43\% ABV neutral alcohol, and steep for 10 days.

Add 600ml of distilled or reverse osmosis filtered water, and steep for 3 more days.

Filter out the solids from the macerated liquid and distill.

Dilute the recovered distillate to 50\% ABV with distilled water or reverse osmosis filtered water.

To color the absinthe green, steep the coloring ingredients in the diluted distillate for 3 to 4 days then filter through multiple coffee filters to remove all solids and sediment and bottle. As an alternative to steeping, sous vide the diluted distilate and coloring ingredients in a sealed bag for 30 minutes at 60\deg C.

Label bottle La F\'{e}e Verte (``The Green Fairy'').

% https://distillique.co.za/blogs/recipes/homemade-absinthe-recipe

\newpage

\recipe{Muscadine Eau-de-Vie -- In Development}
\equilpment{Still Spirits Air-Still, alcohol hydrometer, Thermopen MK4 instant-read thermometer, 7 or 8 pint Mason jars.}

\ingredients{1 gallon (5 bottles) muscadine wine}

\textsc{Distilling 5 bottles of commercially available 12\% ABV muscadine wine will yield approximately 1 bottle of 40\% ABV eau de vie. Most muscadine wine made to be consumed as wine is made with sugar added to the juice to boost the final ABV and also sweetened with sugar after fermentation. Wine of this sort is fine for making eau de vie, however, if you are making wine to be distilled rather that drunk as wine then there is no need to add sugar. Wine made from muscadine juice alone will be around 8 \% ABV so 2 gallons of wine will be required to make one bottle of eau de vie.} 

Add wine to an AirStill, attach cover, and turn on still. Position a pint jar to collect the still output For a gallon of wine expect the first drops to emerge from the still after around 60 minutes. Carefully watch the still output and collect 100 ml before switching to a new jar. The first jar contains mostly methanol and should be discarded. Collect 200 ml in the second, third, and forth jar. Collect 400 ml in the remaining jars

\chapter{Components}
Place for roux, stock, and other standard recipes that are used in many other recipes.

\chapter{  More of an Idea than a Recipe}

\recipe{Roasted Butternut Squash}

\ingredients{butternut squash; olive oil.}

Preheat oven to 400\deg F.  Line a sheet pan with foil. Cut a small slice off of stem end of squash then split in half along the axis using a heavy chef's knife. Using a large spoon, remove the stringy flesh and seeds from the center of the squash.  Move the squash to the aluminum foil-lined baking sheet, coat the squash halves with olive oil, then flip the squash cut-side down. Bake the squash for about 50 minutes at 400\deg F. When done remove from oven and flip squash over to cool.  When cool scoop out flesh to use in soup and other recipes.

\recipe{Fried Okra} 

\ingredients{Cut okra;
egg;
corn meal;
fat for frying;
salt.}

Put cut okra in a bowl. Crack one egg over the okra and then toss to coat. Add corn meal and toss.  The cut ends of the okra should be completely coated with corn meal but some green should be visible on the sides. Deep fat fry in an iron skillet. When brown remove from skillet with a slotted spoon and drain on paper towels. Season with salt and then serve immediately.

\recipe{Whipped Sour Cream}

\ingredients{sour cream.}

Whip the sour cream just as you would whip sweet cream.  Use as a topping for deserts.

\recipe{Whipped Bacon Fat}

\ingredients{well rendered bacon-fat; herbs of your choice.}

Heat bacon fat until it is in liquid form and then infuse it with herbs or salt and pepper. Take it off the heat and as it's cooling, pour it into a stand mixer and keep whipping it into a lardo-type texture. You can spread it on bread, or make a 50/50 ratio of whipped butter to whipped lardo and use it in place of whipped butter.

\recipe{Soft or Hard Cooked Eggs}

\ingredients{eggs; water.}
\label{SoftHardEggs}

The trick to cooking eggs in their shells is to steam rather than boil them. Start with a sauce pan with a tightly fitting lid. The pan should be large enough to hold the number of eggs you want to cook in a single layer.  Add one half inch of water to the pan, bring to a boil and then reduce to a simmer.  Add eggs to pan and cover.  Cook for 6 minutes for soft cooked eggs or 12 minutes for hard cooked eggs. If serving cold, transfer eggs to an ice bath after steaming.

\recipe{Oven Cooked Bacon}

\ingredients{sliced bacon.}

Preheat oven to 425\deg F. Place bacon strips on a foil covered half sheet pan. Cook in oven for 10 minutes then rotate pan and continue cooking until crisp, 5 to 10 minutes for thin cut or 10 to 15 minutes for thick cut. Remove bacon from oven and drain on paper towels

Yields: Up to as many strips as can be placed on a half sheet pan.



\recipe{Brown Sugar}
\label{BrownSugar}

\ingredients{1 cup white granulated sugar; 1 or 2 tablespoons molasses.}

Combine molasses and white sugar in a small bowl. Use 1 tablespoon of molasses for light brown sugar and 2 tablespoons for dark brown sugar. Mix ingredients with a fork until a uniform mixture is obtained.


\recipe{Warm Bacon-Fat Vinaigrette}
\label{BaconFatVinaigrette}

\ingredients{Well rendered bacon fat;
sherry vinegar.}

Warm the bacon fat until it melts to liquid form and is clear but is not too hot to touch.  Whisk in a splash of sherry vinegar and use to dress an asparagus or other vegetable salad.

\recipe{Miso Mayonnaise}

\ingredients{1 cup mayonnaise; 3 tablespoons white miso; a splash of mirin; a splash of soy sauce; a splash of sambal or chile-garlic sauce.}

 Whisked together ingredients.  The result should be a little less than thick.
 
 \recipe{Roasted and Peeled Sweet or Chili Peppers - To Include}
\textsc{To include}
 
\recipe{Sheet-Pan Caramelized Onions}

\ingredients{Onion or two; \quarter cup olive oil; salt; large splash of water}

Preheat oven to 450\deg F. Cut off ends of onions, peal, cut in half pole-to-pole, and then thinly slice onion halves by cutting perpendicular to the board from root end to stem end.

Line half sheet-pan with parchment paper and add sliced onions to paper. Pour olive oil over onions, season with salt and then toss to combine.  Add a generous splash of water to the onions to help them cook down, then spread out the onions and put them in the oven. 

Stir and check the onions every 10 to 15 minutes until they are browned to your satisfaction.  This should take around 45 minutes.

\newpage
\recipe{Curry Carrot Soup}

\ingredients{roast chicken carcass;  shallot; olive oil; onion; salt; 2 or 3 teaspoons \hyperref[CurryPowder]{curry powder}; 2 large carrots, chopped; 1 cup buttermilk; white pepper.}

Make a brown chicken stock using the chicken carcass and shallot. When cool remove excess fat from the top, strain out the chicken and shallot and reserve the stock. The result should be 2 or 3 cups of stock. 

In a medium dutch oven, cook the diced onion with a pinch of salt in olive oil until transparent and slightly browned.  Add the curry powder to the onion and cook for a minute or two to bloom the spices. Add the carrots and reserved stock and cook until the carrots are soft enough to pur\'{e}e  with an immersion blender. 

Using an immersion blender pur\'{e}e the carrot mixture and then blend in buttermilk and white pepper. Serve  hot or cold.
 
\chapter{  Non-Food Recipes}

\recipe{Isotonic Saline Solution}

\ingredients{0.5 liter = 1 pint = 2 cups water;
5 ml = 1 teaspoon salt;
2.5 ml = \half teaspoon baking soda.}

Do not use iodized salt or salt with anti-caking agents.

The water should be boiled and then cooled or bottled water should be used.

I usually make this by boiling the water in the microwave and then adding
the salt and soda to the just off boiling water.

Before using for nasal irrigation let the solution cool to room temperature.

\recipe{Glass Cleaner}

\equilpment{1 gallon plastic jug; funnel.}

\ingredients{\half cup sudsy ammonia; 2 cups rubbing alcohol; 1 teaspoon dish detergent; water to fill; blue food coloring.}

Using a funnel, add all ingredients to the jug except the water and food coloring.  
Slowly add water to fill and then add the food coloring a drop at at time until the desired color is obtained.

\newpage
\recipe{Polishing Silver}

\ingredients{1 gallon water, near boiling;
\half cup baking soda;
disposable aluminum casserole dish;
aluminum foil.}

Add baking soda to aluminum casserole dish and then add near boiling water.  Stir until baking soda is dissolved and then add silver pieces. Wearing rubber gloves, scrub silver pieces with aluminum foil as necessary then rinse and dry.  Add additional silver pieces to dish and repeat steps. This process is less effective as the solution cools. If you will be cleaning a large number of pieces then you can put the aluminum casserole dish in a cast iron casserole dish placed on a cook top to maintain a solution temperature just under boiling. Discard the aluminum casserole dish when finished.

\recipe{Cleaning Solution for Copper}
\label{CleaningSolution}

\ingredients{500 ml water;
50 ml 3\% Hydrogen Peroxide Solution;
1\half tablespoons Citric Acid.}

Mix all ingredients by shaking in a closed bottle with a top and then allow to settle and clear before using. To clean copper place the copper item in the solution for around 20 minutes.  After removing the copper item from the cleaning solution, clean with water and dish soap and then rinse with water to remove all dish soap.

\recipe{Cleaning Coffee Maker}

\ingredients{10 grams of either baking soda or sodium carbonate.}

If using baking soda first convert it to sodium carbonate by baking it in a 200\deg F oven for one hour.  Mix the sodium carbonate with 1 liter of water and pour into coffee maker's reservoir. Turn on the machine and complete the brew cycle.  After the cycle is complete, rinse all machine components that come into contact with the solution and then repeat the brew cycle twice with clean water.

\newpage

\recipe{Season a Cast Iron Pan}

\ingredients{Unseasoned or stripped cast iron pan; flaxseed oil.}

If the pan has been previously seasoned then strip it of seasoning by running it through the self-cleaning cycle of a self-cleaning oven. 

To begin seasoning, warm the unseasoned pan for 15 minutes in a 200\deg F oven. 

Remove the pan from the oven.  Place 1 tablespoon of flaxseed oil in the pan and, using tongs, rub the oil into the surface with paper towels. With fresh paper towels, thoroughly wipe out the pan to remove the excess oil. 

Place the oiled pan upside down in a cold oven, then set the oven to its maximum baking temperature. Once the oven reaches its maximum temperature, heat the pan for one hour.  Turn off the oven and cool the pan in the oven for at least two hours.

\recipe{Bubble Stuff for Giant Bubbles}

\ingredients{1 cup Joy or Dawn dishwashing liquid;
4 tablespoons glycerine (available at most drugstores);
10 to 15 cups cold water.}

Add 10 cups of water to a large pail and then pour in dishwashing liquid.  Add glycerine and then stir gently to avoid creating froth on top of the solution.  When solution is completely mixed, skim off any froth with your hand. On very dry days it may be necessary to add up to 5 cups of additional water to make truly huge bubbles.

\recipe{Four Wire Round Braid}

\ingredients{4 wires of different colors.}

Start with four wires side by side.
\begin{enumerate}
\item Take the leftmost wire and cross it OVER its TWO neighbors. 
\item Take the rightmost wire and cross it OVER its NEXT neighbor.
\item Take the leftmost wire and cross it UNDER its TWO neighbors.
\item Take the rightmost wire and cross it UNDER its NEXT neighbor.
\end{enumerate}
Repeat steps 1-4 until done.

\recipe{Fruit Fly Trap}

\ingredients{\quarter cup hot water; 1 tablespoon honey; \half cup apple cider vinegar; 1 teaspoon dishwashing liquid (unscented is best but not necessary)}

In a small bowl combine hot water and honey. Stir with a small whisk until the honey is completely dissolved. Add the cider vinegar and incorporate.  Add dishwashing liquid and whisk slowly to avoid making bubbles.  Add the solution to a small water glass and set on the counter in a area where fruit flies are a problem.



\end{document}

%%%%%%%%%%%%%%%%%%%%%%%%%%%%%%%%%%%%%%%%%%%% Recipe Template
\newpage
\recipe{Template}
\label{Template}

\textsc{Text describing and commenting on recipe.}

\equilpment{Pots; pans.}

\ingredients{Usually nothing is here unless it is used in all components.\\
\\
\textsc{Component 1 --}
2 jalape\~{n}os, seeded or not, thinly sliced;
1 gallon water.\\
\\
\textsc{Component 2 --}
\threequarters cup vegetable oil;
2 cups chopped onions
\half cup chopped red bell peppers;
salt; freshly ground pepper; ground cayenne; hot sauce; flat-leaf parsley, chopped;
green onions, thinly sliced.
}

Put a parchment paper sling in a 6 to 8 cup p\^{a}t\'{e}  mold. Pur\'{e}e with an immersion blender until smooth. Bake in a 350\deg F oven for one hour. Refer to another recipe with \hyperref[label]{in-line text}.


%%%%%%%%%%%%%%%%%%%%%%%%%%%%%%%%%%%%%%%%%%%%%%%%%%%%%%%%%%%%%%


\chapter{ Photographs}

\begin{figure}
\includegraphics[width=\textwidth]{PearAlmondTartPhoto.pdf}
\caption{Pear Almond Tart.}
\label{fig:PearAlmondTart}
\end{figure}

\begin{figure}
 \includegraphics[width=\textwidth]{PorkCollardPeaGumboPhoto.pdf}
\caption{Smoked Pulled Pork, Collard Greens, and Blackeye Peas Gumbo.}
\label{fig:PorkCollardPeaGumbo}
\end{figure}

\begin{figure}
 \includegraphics[width=\textwidth]{BrownRouxPhoto.pdf}
\caption{Brown Roux.}
\label{fig:BrownRoux}
\end{figure}

\begin{figure}
 \includegraphics[width=\textwidth]{OysterDressingPhoto.pdf}
\caption{Oyster Dressing.}
\label{fig:OysterDressing}
\end{figure}

\end{document}


%%%%%%%%%%%%%%%%%%%%%%%%%%%%%%%%%%%%%%%%%%%%%%%%%%%%%%%%%%%%%%%%

List of Recipes to Include

\newpage
\recipe{Pork Tenderloin Steak}

\ingredients{2 pork tenderloins;
more stuff}

Instructions go here.

Source: The Meat Cookbook

\newpage
\recipe{Bar-B-Que Rub}

\ingredients{1 cup salt;
2 tablespoons + 2 teaspoons freshly ground black pepper;
2 tablespoons + 2 teaspoons lemon pepper;
2 tablespoons + 2 teaspoons cayenne pepper;
2 tablespoons + 2 teaspoons chili powder;
2 tablespoons + 2 teaspoons dry mustard;
2 tablespoons + 2 teaspoons dark or light brown sugar;
1 tablespoons + 1 teaspoons garlic powder;
\half teaspoon ground cinnamon;
\half teaspoon Accent (optional).}

Combine all ingredients in a bowl and stir well to mix.

Use immediately or store in a glass jar in a cool dark place for several months. Yields 2 cups of rub.

Source: John Willingham from John Willingham's World Champion Bar-B-Q.



%  Junk Pile
%
%
\recipe{Simple Sugar Wash for Distilling}
\label{SimpleWash}
\equilpment{9.5 liter stainless steel stock pot; long handled stainless steel spoon; 12 liter (3 gallon) plastic fermenter with lid and airlock; triple-scale hydrometer; Thermapen digital thermometer; 2 liter (8 cup) Pyrex measuring cup; no-wash sanitizing product.}

\ingredients{1.75 kg sugar;
1.5 g citric acid;
1.8 g diamonioum phospate (DAP);
1.3 g food-grade gypsum (calcium sulphate);
150 mg Epsom salts;
16.8 g of distillers yeast or \quarter cup of dried bakers yeast.
}

\textsc{I found this recipe on the Home Distillers Forums under the name ``Wineos Plain Ol Sugar Wash.'' I have scaled the original recipe so that it works in a 12 liter (3 gallon) fermenter.  The target amount of sugar wash is 10.25 liters.  This volume should yield 8 liters of clear wash for distilling.} 

Mark 10.25 liters on the outside of your fermenter. Sanitize your fermenter and top, airlock, stock pot, hydrometer, thermometer probe, measuring cup and spoon before beginning.

In a stainless steel stock pot, bring around  4 liters of water to a simmer then dissolve the sugar in the hot water.  Mix well with a stainless or plastic spoon until the sugar is completely dissolved. In a drinking glass, add hot water and citric acid. Stir the solution until the citric acid is completely dissolved and then add to the sugar solution. Repeat this procedure with the diamonioum phospate, gypsum, and Epsom salts, individually. Stir the sugar solution until all the additions are incorporated.

Add the hot water solution to fermenter. Add cool water (and ice if you are in a hurry -- the ice should all melt) to make the total volume close to but under 10 liters. Using a hydrometer test the specific gravity (be sure to correct for the temperature of the solution). The specific gravity should be above 1.080. Add additional cold water to obtain a  specific gravity between 1.070 and 1.080. After fermentation this will yield between 9.5\% and 11\% alcohol by volume.

Check the temperature of the wash and once it is at or less than 95\deg F sprinkle the yeast on top. After 15-20 minutes, give the wash a good stir to incorporate the yeast and add some air to the wash. Place the cover on the fermenter and add water to the airlock. Place in a storage room or other similar out of the way place at warm room temperature.

Depending on the fermenting temperature this will work off in a week or two. It is important not to rush the fermentation. Let it finish fermenting to a specific gravity of 0.990 or less and then give it another week to clear before racking and running it.  It is important to rack the clear wash off the yeast sediment and only place the clear wash in the still so that a neutral and flavorless spirit is obtained.

% Citric acid 9.9g/T; Gypsum 8.7g/T; DAP 11.9g/T; Red Star distillers yeast 37.1g/\quarter cup; Epson Salts 13.7g/T

% My fermenter: Total Volume = 412 fluid oz = 3.2 gal; Volume below valve = 76 fluid ounces = 0.594 gal = 2.25 l;

% Desired amount of clear wash = 8 l. Target amount of wash to make = 10.25 l = 2.71 gal

\newpage
\recipe{Gin \#2 - In Development}

\ingredients{3\half cups 45\% ABV neutral spirits; 0.5 gram strip of blood orange zest; 15 grams juniper berries; 4 grams coriander seeds} 

Pour neutral spirits into a 1-quart Mason jar. 

Carefully remove zest from orange being certain that no white pith is included. Trim zest strip until it weighs 0.5 grams.

Using a mortar and pestle thoroughly crush the juniper berries and coriander seeds. 

Add the orange zest, juniper berries and coriander seeds to the neutral spirits, place the lid and ring on the jar, and shake.  Allow the mixture to macerate for around 7 days shaking the jar occasionally.

To distill, remove and discard orange zest then add contents of Mason jar (including juniper berries and coriander seeds) to the Air Still.  Add an additional quart of water to the still. Turn on still and after around 25 minutes collect and discard the first 20 ml of distillate to emerge. After this collect the next approximately 750 ml of distillate. Stop collecting distillate when the measured ABV of the collected distillate is around 45\%. Bottle and wait at least one week before drinking to allow flavors to stabilize. Some distillers suggest that aging the gin several months improves the flavor. 

\newpage

\recipe{Gin \#3 - In Development}

\ingredients{3\half cups 45\% ABV neutral spirits; 0.15 gram dried bitter orange zest; 15 grams juniper berries; 3 grams coriander seeds; 1.5 grams angelica root.} 

Pour neutral spirits into a 1-quart Mason jar. 


Using a mortar and pestle thoroughly crush the juniper berries and coriander seeds. 

Add the dried orange zest, juniper berries, coriander seeds, and angelica root to the neutral spirits, place the lid and ring on the jar, and shake.  Allow the mixture to macerate for around 7 days shaking the jar occasionally.

To distill, add contents of Mason jar plus one quart of water to the Air Still.  Turn on still and after around 25 minutes collect and discard the first 20 ml of distillate to emerge. After this collect the next approximately 750 ml of distillate. Stop collecting distillate when the measured ABV of the collected distillate is around 45\%. Bottle and wait at least one week before drinking to allow flavors to stabilize. Some distillers suggest that aging the gin several months improves the flavor. 

\recipe{Gin \#4 - In Development}

\ingredients{
750 ml 50\% ABV neutral spirits or vodka;
12 grams juniper berries;
3 grams coriander seeds; 
0.5 grams cardamom; 
0.75 grams angelica root;
8 grams orange zest
} 

Pour neutral spirits into a 1-quart Mason jar. 

Do not crush any of the ingredients. Add the  juniper berries, coriander seeds, angelica root, and cardamon to the neutral spirits, place the lid and ring on the jar, and shake.  Allow the mixture to macerate over night shaking the jar occasionally.

To distill, add contents of Mason jar plus one cup of water to the Air Still. Zest orange and add orange zest strips wrapped in cheese cloth to the gin basket.  Turn on still and after around 25 minutes collect and discard the first 17 ml of distillate to emerge. After this collect the next 375 ml of distillate for use in making the gin. Dilute the collected distillate with purified water to 44 \% ABV.

With this amount of orange zest your gin may louch. To correct louching, add 44 \% ABV neurtal spirits to your gin until it is clear.

Comment: For this recipe I used purchased 50 \% ABV vodka. The recipe yielded excellent if very orange forward gin after it was diluted to correct louching. The juniper flavor was not as harsh as previous recipes and I believe that is because the berries were not crushed.

recipe{Rice Wash}

\ingredients{6 to 12 kg (13 to 26 lb) of rice (I used 6 kg of medium grain rice); 
Water; 
36 grams (1.3 oz) Angel Yellow Label Yeast}

Using a grain mill or food processor, mill the rice to small pieces. If using a food processor, process in 1 kg batches for 5 minutes per batch (let the processor cool between processing batches). Measure the volume of the milled rice then put rice in brewing pail. Add a volume of almost boiling water equal to around three times the volume of the milled rice to the pail. Using a paint mixer and drill, agitate the mixture in the pail very well. 

Let the mixture sit for 1 hour then
add cold and/or warm water as needed to reach a total volume of 55 liters (14 gal) at 30\deg C (86\deg F).
Hydrate the yeast in 35\deg C (95\deg F) water. 
Add the hydrated yeast to the rice wash and agitate using a paint mixer.
Ferment at 30\deg C (86\deg F).
Wait for 3 days after fermentation has stopped then rack into still. 
Run three stripping runs of around 4 gallons.
Slowly distill the low wines from the three stripping runs in one spirit run 
making cuts based on flavor. 

Dunder and Lightning uses 6kg medium grain rice plus 20 liters of boiling water and 20 liters of cold water and 36 grams of yeast. He stirs every day for the first three days and then ferments to dry in 2 weeks. His yield was 1.4 liters of 94\% ABV. 


Instructions from Angel Yellow Label Yeast: For 100 kg of grain.
The finer the grind, the sooner it finishes.
Mix 100 kg with 250-300 kg of hot water.
Let cool naturally to about 90\deg F (32 \deg C), agitating to prevent settling.
Directly add 0.5-0.8 kg of the starter. Agitate.
Control the temperature to be 82-97\deg F (28-36\deg C), optimal fermentation temperature about 90\deg F (32\deg C).
Lower limit 79\deg F (26\deg C), upper limit 100\deg F (38\deg C).
Agitate twice every day in the first three days.
Ferment for 8-15 days.

According to the Angle Yellow Label Yeast instruction 6 kg (13.2 pounds) of rice would call for 18 liters (4.76 gallons) of water.  This is much less water than used in the recipes above. The 36 grams of yeast corresponds to 600 grams of yeast for 100 kg of rice. This is within the range recommended by the instructions.

\recipe{Gin Extract}
\label{GinExtract}

\ingredients{35 grams whole juniper berries;
2 grams whole coriander seeds;
2 grams green cardamom pods;
2 grams orris root;
400 ml water;
75 ml 95\% ABV neutral spirits.
}

Using an electric spice grinder, individually grind each botanical and add it to the still.  Add water to the still and then run still collecting the first 75 ml to emerge.
Add 75 ml neutral spirits to the 75 ml collected from the still and bottle.

To make gin using the gin extract, add 1 ml of extract to every 100 ml of 45\% ABV neutral spirits used.

This recipe is from the Barley and Hops YouTube channel.  The botanical amounts are roughly the same as the amounts used with 1 liter of 40\% ABV neutral spirits for recipes to be made in an Air-Still.  This suggests that any of the recipes in this chapter could be used to make Gin Extract using the method described in this recipe.
\recipe{Corn Wash -- In Development}

\ingredients{3 pounds (2 1.5-pound bags) Goya fine corn meal; 
11 quarts tap water;
\half teaspoon Amylase enzyme;
\quarter teaspoon Glucoamylase;
15 grams distiller's yeast.}

AMYLASE ENZYME BSG -- Amylase is easy to use.  When you have finished cooking your mash, allow it to cool to below 170 degrees and stir in one teaspoon (0.1 to 0.3 teaspoons per gallon) amylase per 5 gallons.  Amylase is a self-limiting glucoside there is nothing to be gained by upping the dosage.  Allow the mash to cool for an hour, but keep the temperature above 120 and stir occasionally.  Alternately if you are set up to hold the mash at 152 degrees for one hour that is ideal for highest conversion rates. 

Glucoamylase (10g) -- Sufficient for up to 20 pounds of dextrins (\half teaspoon per 5 gallons). For optimal activity add sachet contents to wash or wort once cooled to 131-140\deg F
(55-60\deg C). Stir to dissolve thoroughly. Pitch yeast once temperature is below 86\deg F
(30\deg C) and leave to ferment.
Enzyme is stable up to 149\deg F(65\deg C).
Optimum pH is pH 4.0 to 4.5 (enzyme is stable between pH 2.8 and pH 5.0).
Note: all enzymes require 50ppm calcium ions for maximum activity and stability.

Add water to a 5 gallon stock pot and place over high heat.  While the water is heating adjust its pH to 4.6. When the water reaches 180\deg F add the corn flour to the hot water in a slow stream using an immersion blender to fully incorporate the flour. Return temperature to 180\deg F and cook for 30 minutes stirring regularly to avoid scorching the corn meal. Add a little amylase enzyme to loosen the wash if it is too thick to stir effectively. Turn off the heat and cool the wash to 155\deg F.  You can let the wash cool naturally or use ice in zipper lock bags to speed the cooling. When the wash is 155\deg F add the amylase enzyme and stir well. Cover the pot and insulate with an old blanket to maintain the temperature and give around 90 minutes for the starch conversion to take place.

After 90 minutes do a starch test with iodine to be sure the starch has converted to sugar. Take a gravity reading to determine the potential alcohol.  Let the wash cool to 140\deg F and then add the glucoamylase and stir well.  Once the temperature drops to below 86\deg F transfer the wash to the brewing pail. Check the pH again and adjust to around 5 if necessary. Add the yeast directly to the wash and stir vigorously. Attach the pail top and the fill air lock.

During fermentation control the wash temperature to be around 86\deg F.

Non Alcoholic Campari  from https://www.theweepearl.com/

Ingredients:

    Lemon

    Grapefruit

    Cinnamon Sticks

    Star Anise

    Clove

    Rosemary

    Thyme

    Marjoram

    Turkish Rhubarb Root

    White Sugar

    Thujone Free Wormwood

    Orris Root

    Bitter Orange Peel

    Angelica Root

    Cochineal Bugs

    Glycerin

    Ginseng Root

    Gentian Root

    Sugar

    Water

     
    Method:

    Combine 4oz water with 2 oz glycerin, add 6g Angelica root, 6g Orris root, 1g grated ginseng root, 10g gentian root, 2g Turkey rhubarb root, and 25g bitter orange peel. Stir to combine and let infuse for 4 weeks. Shaking everyone in a while.

    Peel 20g of lemon peel and 20g of grapefruit peel. Add to a plastic bag and cover with 40g of sugar. Massage together until a syrup begins to form,. Set aside at room temperature for 24 hours. If any sugar crystals remain, massage them into the oil until they dissolve. Remove the peels and set aside.

    Add 6g wormwood, 2 grams marjoram, 2 grams thyme, 2 grams rosemary, 2 grams cinnamon, 1 star anise pod, 3 grams cloves,  to a French press, then cover them with 1.5 cups of hot water. Steep for 20 minutes. 

    Grind 2g of Cochineal bugs in a mortar and pestle until a fine dark red powder forms. Add 1/4 cup of water and 1/2 cup of sugar to a saucepan over medium low heat. Once sugar has dissolved remove from the heat. Whisk in the cochineal bug powder. Let it sit for 15 minutes, then strain out the bug particles using a fine strainer or nut milk bag.

    Combine all root tincture, oleo sacchrum, herbal tea and red syrup. Let the non alcoholic Campari rest for 2 weeks in a bottle, and enjoy! 

    Keep this refrigerated and use within 3 months.



% Recipe from Everclear web site

\recipe{AROMATIC BITTERS}
\ingredients{
Bittering Jar
2 tablespoons gentian root
2 tablespoons burdock root
1 \half cups Everclear (approx.)

Aromatic Jar
1 orange zest
2 cinnamon sticks
15 whole allspice berries
10 whole cloves
3 whole star anise
1 \half ounce freshly chopped ginger (about a 2” nub)
1 \half cups Everclear (approx.)

Brown sugar simple syrup
1 cup brown sugar
1 cup water}

Bittering Jar:
Put chopped gentian root and burdock root in a pint-sized mason jar.
Fill to the 12 oz. line with Everclear. Seal and store in a cool, dry place away from direct sunlight. Shake once daily.

Aromatic Jar:
Roughly chop ginger root.
Peel zest from a freshly washed orange and remove the pith.
Combine ginger root, orange zest, cinnamon sticks, allspice berries, cloves and star anise in pint-sized mason jar.
Fill to the 12 ounce mark with Everclear. Store in a cool, dry place away from direct sunlight. Shake once daily.

Brown Sugar Simple Syrup:
Combine brown sugar and water in small sauce pan. Bring to boil, stirring continuously until sugar is dissolved.

Mixing:
After steeping both jars for 3 weeks, strain into separate, clean jars using a mesh strainer and a coffee filter.
Mix 4 parts flavoring agent to 1 part bittering agent (or to taste)
Add brown sugar simple syrup to taste
Bottle, store in a cool, dry place, and enjoy.

\recipe{Spiral Prismatic Packing}

Diameter of SPP = 3.5 mm (0.1 inch)

Diameter of wire 0.2-0.3 mm (0.008 inch - 0.01 inch)

Number of turns = 11

Approximately 1 kg of 3.5 mm diameter SPP are required to fill a 44mm long 2 inch diameter pipe.

OR

Diameter of SPP = 5.5 mm (0.19 inch)

Diameter of wire 0.2-0.3 mm (0.008 inch - 0.01 inch)

Number of turns = 14

Doctor-gradus uses a triangular cross-section bar to form its SPP.

Approximately 650 grams of 5.5 mm diameter SPP are required to fill a 44mm long 2 inch diameter pipe.

Doctor-gradus uses a triangular cross-section bar to form its stainless steel SPP. Food grade M1 copper and a square cross-section bar is used for copper SPP.

Homemade Campari from https://www.theweepearl.com/recipes/homemade-campari

    25 grams of Bitter Orange Peel- Campari’s main discernible flavor is orange.

    15 grams lemon peel- I’m opting for fresh, but dried will work well too.

    15 grams grapefruit peel- I always get a lot of grapefruit notes from Campari. 

    6 grams Angelica root- to add some herbal notes

    6 grams Thujone free wormwood- the main bittering agent, as well as adding aromatics.

    2 grams marjoram

    1.5 grams sage

    2 grams thyme

    2 grams rosemary

    2 grams cinnamon- for some more spicy warmth.

    1 star anise pod- for that slight licorice taste.

    3 grams cloves

    6 grams Turkish rhubarb root -this adds some bitterness, tannic structure and a little smokiness.

    6 grams orris root- for a lovely sweet floral taste and aroma.

    3 grams of cochineal bugs- the original coloring agent for Campari. 

    1 cup of sugar

    1 cup of hot water- divided

    375ml bottle of vodka

I'm going to be making this in 3 parts: An herbal tea. A red colored syrup-  that uses the traditional Cochineal bugs, and a vodka infusion. 



First I’ll make an herbal tea. 

To my French press I am adding 2 grams of cinnamon, 1.5 g sage, 2 g marjoram,  3 g clove, 2 g thyme and 1 star anise pod. I’ll cover these herbs and spices with a half cup hot water and let them steep for 15 minutes. These ingredients  tend to get over extracted by alcohol, and thereby overpower the spirit. While experimenting with this recipe, I found that making a strong herbal tea worked best to capture the warm baking spice notes as well as the herbal flavors.

Next I’ll make the red colored syrup. Cochineal bugs are responsible for Campari’s brilliant red color- or at least they used to be. Campari in the US has been sold with artificial coloring since 2006. If you ever come across the naturally dyed version of Campari, you will find the color to have a bit more of an orange hue. In order to extract the red dye, I’ll have to grind the bugs in my mortar and pestle until they have become a fine dark red powder. 

Add one cup of sugar and a half cup of hot water and whisk until the sugar dissolves. Remove the rich simple syrup from the heat, and stir in the ground cochineal powder to combine. In the video version of this recipe I only used a half cup of sugar, but realized later on that it needs to be far sweeter to match the bitterness.

’ll let the syrup cool completely while I extract the herbal tea from the French press. Then I will filter the cochineal bug particles from the syrup. I used a nut milk bag to filter the syrup, but any fine filter should work. 

Once the syrup and tea have cooled, cover them and keep them in the refrigerator until the next day. 



For the vodka infusion I will be using 375ml of vodka, along with 15g grapefruit peel,  15g lemon peel, 25g bitter orange peel, 6g angelica root, 2g dried rosemary, 6g wormwood- which will act as a bittering agent as well as an aromatic, 6g Turkish rhubarb root and 6g orris root. I’ll  pour the vodka into the mixing container and make sure everything is submerged. This will all be infused for 12- 18 hours.  

By the next day, the botanical infusion can be filtered. The infusion will be very dark and taste very bitter. 

Now all that’s left to do is combine them all together. It wont look especially red just yet, but after about 2 weeks of bottle conditioning, it will clear a bit as the sediment falls to the bottom and the red color will be more pronounced. To achieve more of a red color, you can decant the liqueur through a fine filter, or use a fining agent like Chitosan and Kieselsol to really clarify the Campari- which I decided to do, as mine was taking ages to clear. 



This campariesque aperitivo makes an excellent Negroni. It subs out perfectly for Campari in any other cocktail recipe. 

This is the first time I’ve attempted to create a liqueur of this kind, and it was a lot of fun trying to get the recipe just right. Now that I have all of these ingredients on hand, I’m looking forward to testing out some new recipes- perhaps another amaro, or even my own vermouth!

This was my attempt at recreating the original recipe. If you are going to make your own attempt please use caution when sourcing your ingredients. Read and research as much as you can. Use your best judgement and be safe if you are going to try this. 
