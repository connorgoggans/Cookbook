\documentclass[letterpaper]{recipePMG}

%\usepackage{lmodern}
\usepackage{bookman}
\usepackage[T1]{fontenc}
\usepackage{nicefrac}
\usepackage {graphicx}
\usepackage{siunitx}
\usepackage{xfrac}
\sisetup{quotient-mode = fraction}
\usepackage[margin=1.25in]{geometry} 
\usepackage[pdftex]{hyperref}
\hypersetup{colorlinks=true,linkcolor=blue}

\newcommand{\bsi}[2]{%
  \fontencoding{T1}\fontfamily{cmr}\fontseries{m}\fontshape{n}%
  \fontsize{#1}{#2}\selectfont}

\renewcommand{\inghead}{\textbf{Ingredients}:\ }

\renewcommand{\equhead}{\textbf{Equipment}:\ }

\renewcommand{\rechead}{\centering\bsi{24.88pt}{30pt}}

\renewcommand{\deg}{$^\circ$}

\newcommand{\eighth}{\nicefrac{1}{8} \,}
\newcommand{\quarter}{\nicefrac{1}{4} \,}
\newcommand{\third}{\nicefrac{1}{3} \,}
\newcommand{\threeeights}{\nicefrac{3}{8} \,}
\newcommand{\half}{\nicefrac{1}{2} \,}
\newcommand{\twothirds}{\nicefrac{2}{3} \,}
\newcommand{\threequarters}{\nicefrac{3}{4} \,}
\newcommand{\threehalves}{\nicefrac{3}{2} \,}

\makeatletter
\renewcommand*\l@subsubsection{\@dottedtocline{3}{3em}{0em}}
\makeatother

\setlength\parindent{0pt}
\setlength\parskip{2ex plus 0.5ex}

\newcommand{\HRule}[1]{\rule{\linewidth}{#1}} 	% Horizontal rule

\makeatletter							% Title
\def\printtitle{%						
    {\centering \@title\par}}
\makeatother									

\makeatletter							% Author
\def\printauthor{%					
    {\centering \large \@author}}				
\makeatother	

\title{	\normalsize \textsc{Personal Cooking Notes} 	% Subtitle of the document
		 	\\[2.0cm]													% 2cm spacing
			\HRule{0.5pt} \\										% Upper rule
			\LARGE \textbf{\uppercase{Collected Recipes}}	% Title
			\HRule{2pt} \\ [0.5cm]								% Lower rule + 0.5cm spacing
			\normalsize \today									% Todays date
		}

\author{
		Paul M. Goggans\\	
        \texttt{paul.goggans@gmail.com} \\
}

\begin{document}
\frontmatter
\thispagestyle{empty}
\printtitle									% Print the title data as defined above
  	\vfill
\printauthor								% Print the author data as defined above

%
\chapter{Preface}
This is a book of recipes that I have collected from many sources and adapted to suit the equipment I own, the techniques I am comfortable with, the ingredients that are available where I live, and, most importantly, my own taste.  Some of these recipes are from cookbooks, some are from magazines and newspapers, some are from TV shows, and some are from family and friends.  In many cases I don't know or have forgotten the origin of the recipes. Where I know the source of a recipe I have included it. These are my recipes only in the sense that I like them and make them regularly enough so that I want to record them in a single place.  A few of the recipes are given exactly as I found them, but in most of the recipes I have changed the ingredients or proportions or times or techniques to suit myself.  I have also changed the instructions for some recipes to make them more specific and/or to make the result more pleasing to me. 

Mainly this is a place for me to record my cooking notes so they don't get lost. If you find it useful too, so much the better.

\tableofcontents\label{TOC}

\mainmatter
\chapter{Salads}
\recipe{Romaine, Orange, and Pecan Salad}


\ingredients{one head romaine lettuce;
3 tablespoon olive oil;
\half cup whole pecans;
4 oranges cut into supremes;
1 small red onion sliced thin and separated into rings;
\half cup olive oil;
2 tablespoon red wine vinegar;
1 tablespoon lemon juice;
2 teaspoon Dijon mustard;
\half teaspoon sherry;
salt and pepper to taste.}



Prepare the lettuce by separating and washing the leaves.  Place the wet leaves on a clean towel and then roll up the towel and place it in the refrigerator to allow the leaves to become firm and crisp.  When the lettuce is crisp (after 30 minutes or so in the refrigerator), tear the leaves from the center stem and rip the leaves into bite size pieces.  Do not use the stems in the salad.

Saut\'{e} the pecans in olive oil until they are medium brown and then cool on absorbent paper.
  
Prepare the orange slices and the onion slices. 

The remaining ingredients are mixed together to make the vinaigrette dressing. 

Arrange the salad on individual salad plates (or bowls) be placing the prepared romaine on the plates and topping with the orange slices, red onion rings, and saut\'{e}ed pecans.

Fresh herbs such as dill, tarragon, basil, thyme and oregano can be added to the vinaigrette as desired.  Pick one of the above and add 1\half teaspoons to the dressing. Grapefruit supremes can be submitted for orange supremes. 

Yields: Six servings.

%\hyperref[TOC]{Table of Contents}
\newpage
\recipe{Curried Carrot Raisin Salad}
\ingredients{10 tablespoons \hyperref[MyMayonnaise]{homemade mayonnaise};
2 teaspoons \hyperref[CurryPowder]{homemade curry powder};
2 teaspoons kosher salt;
\quarter teaspoon cayenne;
juice of one lemon;
1 pound carrots, peeled and shredded;
2 cups good-quality raisins;
1 cup loosely packed fresh flat-leaf parsley leaves, thinly sliced.}

In a medium bowl, mix the mayonnaise with the curry powder, salt, cayenne, and lemon juice. 

Fold in the carrots, raisins, and parsley. 

Refrigerate for an hour up to 4 hours to allow the flavors to develop.

Source: ``Down South'' by Donald Link.

%\hyperref[TOC]{Table of Contents}
\newpage
\recipe{Brownwood-Style Coleslaw}

\ingredients{1 cup \hyperref[PoppySeedDressing]{homemade poppy seed dressing};
1 cup apple cider vinegar;
1 teaspoon salt;
1 teaspoon freshly ground black pepper;
1 large head purple or green cabbage, cored, halved, sliced \half to \threequarters inch thick, and chopped;
2 cups of whole seedless grapes (black for green cabbage, green for purple cabbage);
2 Granny Smith apples, peeled and chopped;
\threequarters cup toasted pecans;
6 green onions with both the white and green parts chopped.}

Make dressing by whisking the poppy seed dressing with the vinegar and then seasoning with salt and pepper. Cover and refrigerate for at least 1 hour.

To make slaw, toss the cabbage, grapes, apples, and pecans in a large bowl. Pour the \twothirds of the the dressing over the slaw and mix gently. Add more dressing if necessary to completely dress the slaw but do not over dress.  Add the green onions and mix again.

Serve immediately or cover and refrigerate.

Source: John Willingham from John Willingham's World Champion Bar-B-Q.

%\hyperref[TOC]{Table of Contents}
\newpage
\recipe{Corky's Coleslaw}

\ingredients{1 medium-size green cabbage;
2 medium-size carrots;
1 green bell pepper;
2 tablespoons grated onion;
2 cups homemade mayonnaise;
\threequarters cup granulated sugar;
\quarter cup Dijon mustard;
\quarter cup cider vinegar;
2 tablespoons celery seeds;
1 teaspoon salt;
\eighth teaspoon white pepper.}

Remove tough outer leaves from cabbage, then core and shred. Peel and grate carrots. Core, seed, and finely dice green pepper.

Place cabbage, carrots, green pepper and onion in a large bowl.

In another bowl, mix together all remaining ingredients. Pour over the vegetables and toss well to combine.

Cover the coleslaw and refrigerate for 3 to 4 hours to allow the flavors to meld before serving.

Serves 6.

Source: The Commercial Appeal Memphis.


\newpage
\recipe{Hattie B's Black-Eyed Pea Salad}

\ingredients{\\
\\
\textsc{Pepper Vinaigrette --}
\quarter cup champagne vinegar;
\quarter cup malt vinegar;
\half cup extra virgin olive oil;
\half teaspoon minced fresh thyme;
\half teaspoon finely chopped fresh parsley;
1/half teaspoons black pepper, freshly ground;
\half teaspoon fine sea salt;
1 red bell pepper, diced small;
1 green bell pepper, diced small;
1 yellow bell pepper, diced small;
2 scallions, finely sliced;
\half teaspoon freshly roasted garlic, minced.\\
\\
\textsc{Black-Eyed Peas --}
2 cups dried black-eyed peas;
4 strips of bacon;
6 cups of chicken stock;
\half teaspoon fine sea salt.
}

Start by making the vinaigrette. Whisk together the vinegars, olive oil, thyme, parsley, black pepper, half the salt, all the peppers, scallions, and garlic.  Let sit in the refrigerator overnight.

In a large pot, add enough water to cover the peas by about 3-4 inches. Let sit overnight while the peppers are marinating. 

The next day, drain the peas in a colander and set aside.  In a large pot over medium-high heat, render the bacon, leaving it whole so it can be easily removed at the end of the recipe.  Once the bacon is rendered, add the chicken stock and bring to a simmer.

Once simmering, add the peas and cook on low for about 25 minutes. The peas should be tender but not mushy.  Strain the peas,  remove the strips of bacon, and place in a stainless steel or glass mixing bowl.

While the peas are hot, toss in the vinaigrette and add \half teaspoon of salt.  Taste and adjust seasonings.

Chill completely before serving.




%\hyperref[TOC]{Table of Contents}
\newpage
\recipe{Poppy-Seed Dressing}
\label{PoppySeedDressing}

\ingredients{1\half cups sugar;
    2 teaspoons dry mustard;
    2 teaspoons salt;
    \twothirds cup vinegar;
    3 tablespoons onion juice (see note after recipe);
    2 cups canola oil;
    3 tablespoons poppy seeds.}

Mix sugar, mustard, salt, and vinegar. Add onion juice and stir it in thoroughly. Add oil slowly, stirring constantly. Continue to beat until the mixture is thick. Add poppy seeds and mix for a few minutes. Store in a cool place, or in the refrigerator. Makes 3-1/2 cups.

Helen's Note: ``This recipe is easier to make when using an electric mixer or blender, using medium speed, but it can also be made by hand using a rotary beater. The onion juice is obtained by grating a large white onion on the fine side of a grater, or by putting the onion in a blender, then straining the juice. If the dressing separates, pour off the clear part and start over, adding the poppy-seed mixture slowly. It won't separate unless it becomes too cold or too hot. The dressing is delicious on fruit salads of any kind.'' 

Source: Helen Corbitt 

%\hyperref[TOC]{Table of Contents}
\newpage
\recipe{Crawfish Salad}
\label{CrawfishSalad}


\ingredients{
\half cup mayonnaise;
\quarter cup Creole mustard;
2 Tablespoons fresh lemon juice;
1 teaspoon minced parsley, plus more for garnish;
Hot sauce, preferably Tabasco, to taste;
Kosher salt and freshly ground black pepper, to taste;
1 pound cooked, peeled crawfish tails, roughly chopped;
1 cup frozen peas, defrosted;
6 scallions, minced;
2 hard-boiled eggs, peeled and diced;
2 stalks celery, minced.}

Whisk mayonnaise, mustard, lemon juice, parsley, hot sauce, salt, and pepper in a bowl. Stir in crawfish, peas, scallions, eggs, and celery. 

To serve as salad, fill four large bowls with torn tender lettuce leaves of your choice then top with crawfish salad and garnish with parsley.  Can also be used in a bread roll instead of lobster salad.

Serves 4

From the Saveur website


\chapter{Vegetables}

\recipe{Greens}

\ingredients{3 bunches mustard greens;
4 bunches turnip greens;
2 cups water;
\half pound salt pork;
\half stick of butter or 2 tablespoons of rendered bacon fat;
salt and black pepper to taste.}

Cut the salt pork into batons.  Add pork to water in the stock pot set over high heat.  When water comes to a boil reduce to a simmer and cook for 30 minutes or so.

While the pork stock is simmering, prepare the greens for cooking by removing crushed or yellow leaves and cutting off excess stems with the roots.  Wash greens thoroughly in plenty of cold water. Lift leaves from container of water and let sand and grit settle to bottom. Wash through several changes of water until greens are completely clean.

Bring pork stock to a boil then add the greens and stir.  When the greens are wilted, reduce heat and simmer until tender. When tender check the amount of liquid. Pour off and save all but around a cup of liquid.  Add butter or bacon fat and stir.

%\hyperref[TOC]{Table of Contents}
\newpage
\recipe{Brooklyn-Style Collard Greens}

\label{BrooklynCollards}


\ingredients{1 smoked turkey wing or 2 medium ham hocks;
3 to 4 bunches collard greens (about four pounds)
salt;
\quarter cup apple cider vinegar;
3 garlic cloves, smashed;
\quarter teaspoon sugar;
1 medium onion, sliced;
\half teaspoon crushed red pepper, more to taste.}

Place turkey wing or hocks in a very large pot and add just 
enough water to cover. Bring to a boil, turn heat to medium 
and simmer until water is reduced by about half.

Meanwhile, plunge greens into a sink full of lightly salted cold water, 
drain and then rinse well with cold fresh water. Trim or remove biggest stems. 
Place five or six leaves on top of one another and roll like a cigar, lengthwise. 
Cut collards into inch-wide ribbons, then, keeping collards rolled,
cut ribbons in half.

Turn heat under pot to high. Add vinegar, one teaspoon salt and the garlic,
then add as many greens as pot will hold. Wait until greens cook down, 
then add remaining greens. Turn heat to a simmer, cover and cook until greens
 for 30 minutes, stirring occasionally.

Add sugar, onion and crushed red pepper, cover again and continue 
to simmer until tender, another 15 to 30 minutes; 
time will vary depending on toughness of greens. Serve.

Yield: 4 to 6 servings.

Source: The New York Times, December 30, 2009.

%\hyperref[TOC]{Table of Contents}
\newpage
\recipe{White Beans with Collards}

\ingredients{1 pound large navy beans (washed and picked over for stones);
2 quarts water;
1 large onion chopped;
1 small (grocery store) bunch of collard greens;
1 pound smoked ham (cubed);
3 cloves garlic (chopped);
1 tablespoon table salt;
1 tablespoon white pepper;
1 whole red pepper;
\half cup olive oil;
1 bay leave.}

Placed washed beans, water, and chopped onion in a large Dutch oven
or small stock pot (at least 6 qts.). Bring beans to boil and then boil for 1 hour.

Remove the stems from the greens using a sharp knife to cut the
green portion of the leaves from both sides of the stem.  Wash the green
portions of the leaves very well to remove all sand and dirt (this may
require moving the leaves back an forth between sinks of clean water).
Take two or three leaf halves and roll into a cigar shape (roll from front
to back of leave). Cut the roll in half along the axis and then perpendicular
to the axis in one inch intervals.

When beans have cooked for 1 hour, add greens to beans along with ham,
garlic, salt, white pepper, and whole red pepper. Stir well, adding olive oil
and bay leave.

Cover pot and let simmer for 30 minutes until both greens and beans are tender.

Source: Leah Chase from The Dooky Chase Cookbook

Notes: In place of the whole red pepper I used a chopped red bell pepper
and added a few teaspoons of very hot hot-sauce. I also added around
3 tablespoons of cider vinegar because I decided it was needed to balance
the richness of the olive oil.  I used smoked ham but I think the dish would be
even better with the ham replaced with smoked pork shoulder.

%\hyperref[TOC]{Table of Contents}
\newpage
\recipe{Ale-Braised Collard Greens With Smoked Ham Hock}

\ingredients{2 tablespoons extra-virgin olive oil;
1 medium yellow onion, diced;
2 garlic cloves, minced;
Kosher salt, as needed;
\half teaspoon red pepper flakes;
1 tablespoon dark brown sugar;
12 ounces American amber ale;
\half cup apple cider vinegar;
1 smoked ham hock;
3 bunches (about 3 pounds) collard greens, thoroughly washed, stems removed, cut into 2-inch pieces;
Black pepper, as needed}

Heat oil in a large stockpot over medium heat. Add onion and garlic and saut\'{e}, stirring occasionally, until onion is softened and just starting to color, 10 to 12 minutes.

Add 1 teaspoon salt, the red pepper flakes and the brown sugar; stir to combine. Add beer and cook, scraping up any browned bits from bottom of pan. Raise heat to high and bring to a boil, then reduce heat to low and simmer for 3 minutes.

Add 2 cups water, the apple cider vinegar, the ham hock and the collard greens; stir to combine. Cover pot, raise heat to high, and bring to a rolling boil. Stir collards thoroughly to incorporate flavors, then reduce heat to low and simmer, stirring every 30 minutes, until collards reach desired tenderness, at least 30 minutes but preferably up to 2 hours. Remove ham hock; pull off and chop meat and return to pan, or discard if desired. Season with salt and pepper.

Source: The New York Times.

%\hyperref[TOC]{Table of Contents}
\newpage
\recipe{Spaghetti Squash Marinara}

\ingredients{1 whole spaghetti squash, about 1\threequarters pound;
kosher salt;
freshly ground black pepper;
\half cup extra-virgin olive oil;
4 spring of fresh thyme;
1 cup Marinara sauce;
4 ounces fresh mozzarella cheese;
chile oil}

Preheat the oven to 450\deg F. Cut the squash in half lengthwise and scoop out seeds with a sturdy spoon.

Arrange the squash halves, cut side up, on a rimmed baking sheet. Season squash with salt and pepper, drizzle with olive oil, and top each half with two sprigs of thyme.  Roast the  squash halves until fork tender, about 30 minutes.

Remove squash from the oven and discard the thyme. Fill squash cavities nearly to the top with marinara.  Using your fingers, pull apart the mozzarella into chunks and scatter them on top of the marinara. Return the squash to the oven and roast until marinara starts to bubble and cheese in melted, about 5 more minutes. Remove from the oven and drizzle with chili oil.

Source: Andrew Ticer and Michael Hudman from Collards \& Carbonara

%\hyperref[TOC]{Table of Contents}
\newpage
\recipe{Squash Casserole}

\equilpment{Blu Skillet Ironware 13 inch carbon-steel French-skillet;  Black Le Creuset \#30 cast-iron baking-dish}

\ingredients{1\half pounds summer squash;
2 tablespoons butter;
1 small onion, finely diced;
3 tablespoons all-purpose flour;
1 cup milk;
\half cup reserved squash cooking liquid;
\half teaspoon salt;
\eighth teaspoon freshly ground black pepper;
1 roasted and peeled red bell-pepper, diced;
2 eggs, beaten;
1 cup (3 ounces) graded extra sharp cheddar cheese;
\half cup dry homemade bread crumbs;
\half cup (1\half ounces) graded extra sharp cheddar cheese.}

Preheat oven to 350\deg F.

Wash and slice squash then add to a small amount of boiling salted water in a pan. Cover pan an bring water back to a boil. When water returns to a boil, lower heat and continue cooking until squash in tender.  Drain squash reserving cooking liquid.

While squash is cooking, melt butter in a skillet. Add onions and cook over low heat until transparent and tender. Add flour and stir to make a smooth paste. Heat milk in microwave until just below boiling but do not boil. Gradually add milk to paste, stirring with each addition of milk to prevent lumping. Add \half cup of reserved squash cooking liquid and stir until smooth.

Add salt, pepper, diced red bell pepper, drained squash, and beaten eggs to the mixture. Stir until well blended. Stir in 1 cup graded cheddar cheese.

Pour mixture into a buttered 1\half quart baking-dish or casserole. Bake for 30 minutes.

Mix remaining \half cup of cheese with bread crumbs and sprinkle over top. Return to oven and bake for 15 minutes longer.


Source: Based on recipe in the 1980 version of The Auburn Cookbook. 

%\hyperref[TOC]{Table of Contents}
\newpage
\recipe{Brown Rice For Future Use}

\ingredients{2-pound package brown rice (907 grams, around 5 cups); 
8 cups water;
3 tablespoon unsalted butter;
1 tablespoon kosher salt}

Preheat the oven to 375\deg F.

Place the rice into a 4 quart 15x10x2 inch Pyrex glass baking dish.

Bring the water, butter, and salt just to a boil in a kettle or covered saucepan. Once the water boils, pour it over the rice, stir to combine, and cover the dish tightly with heavy-duty aluminum foil. Bake on the middle rack of the oven for 1 hour.

After 1 hour, remove cover and fluff the rice with a fork. 

Makes around 15 cups of cooked rice that can be conveniently portioned into 5 3-cup freezer containers and frozen for future use.

Source: Based on a recipe by Alton Brown.

\newpage
\recipe{Beets Baked in Foil}

\ingredients{8 medium beets of nearly equal size and shape, 1\half to 2 pounds.}

Preheat oven to 400\deg F.

Trim greens and tap root from beets then wash well. Wrap beets individually in foil and put on a  half sheet pan. Beets should be tightly wrapped. 

Bake, undisturbed for 30 to 45 minutes. When done a thin bladed knife should slip easily through a test beet.

Let beets cool to room temperature, remove foil, and then remove peels from beets with a paper towel. To avoid staining hands wear kitchen gloves. To avoid staining cutting board, peel beets on the sheet pan.

To serve hot, slice beets and reheat in butter. Finish with a squeeze of lemon juice.

Source: Based on a recipe in  ``How to Cook Everything'' by Mark Bittman.

\newpage
\recipe{Roasted Broccoli}

\ingredients{1\half pounds broccoli crowns, around 2 large crowns;
3 tablespoons extra-virgin olive oil;
Kosher salt;
freshly ground black pepper.}

Position oven rack in center of oven and heat to 450\deg F. Line a half sheet pan with heavy-duty aluminium foil.

Prepare the broccoli by first removing any leaves from the crowns. Then cut away the individual florets so that the stem ends is no more that \half inch thick. Cut florets that are too large in half. Trim fibrous outer layer from stems and then cut in \half inch coins.

Place broccoli on sheet pan, drizzle with oil, season with salt and pepper, and toss to coat. Spread into an even layer placing cut side of any split florets down.  Roast for 15 minutes then flip pieces and roast for an additional 5 minutes.

Excellent served cold or room temperature with vinaigrette added just before serving.  

Source: Based on a recipe in ``All About Roasting'' by Molly Stevens.


\newpage
\recipe{French-Fried Potatoes}
\label{FrenchFriedPotatoes}
\equilpment{R \& V Works Cajun Fryer Model FF1.}
\ingredients{8 ounces Russet Potatoes; peanut oil to fill fryer; salt.}

Peel the potatoes and cut each into \quarter-inch thick slices. Stack a few slices at a time and cut the slices lengthwise into \quarter-inch-wide strips or sticks.  Drop the potatoes into cold water and drain almost immediately. Pat the potatoes completely dry.

Add potatoes to oil in fryer heated to 325\deg F.  Blanch them for 2 minutes.

Drain them well and transfer potatoes to a pan lined with absorbent paper.

Finish the potatoes in 375\deg F oil until they are golden brown and cooked through. Drain them well.

Salt to taste away from the fryer and serve immediately. 

Yields: Two servings

Source: Adapted from recipes in `` The New Professional Chef'' by The Culinary Institute of America and in the New York Times by Pierre Franey.
\newpage
\recipe{Spinach Cheddar Quiche}
\label{SpinichCheddarQuiche}

\ingredients{One recipe \hyperref[FlakyPieCrust]{Flaky Pie Crust}; 20 ounces frozen chopped spinach, thawed; 6 eggs; 6 cups shredded aged sharp cheddar cheese; 2 cups whole milk.}

Preheat oven to 350\deg F and Meanwhile, make one Flaky Pie Crust recipe.

To make the filling, squeeze as much excess water from the spinach as humanly possible (use dish towels) and then combine all remaining ingredients in a large bowl.


Roll out crusts for two pies. Divide the spinach and cheddar mixture between each crust and bake for 60 minutes.


\chapter{Soups, Stews and Gumbos}

\recipe{Beef, Bean, and Barley Stew}

\ingredients{ 1 tablespoon extra-virgin olive oil;
2 pounds beef chuck roast, cut into \half-inch cubes;
3 teaspoons Kosher salt;
\quarter teaspoon fresh ground black pepper;
1 medium onion, quartered and sliced;
1 stalk celery, diced;
4 cups water;
14.5-ounce can diced tomatoes;
\half cup pearled barley, uncooked;
15-ounce can cooked kidney beans, drained and rinsed;
2 bunches collards; cleaned and cut into \half-inch squares or 24-ounce bag frozen kale;
2 teaspoons finely chopped fresh rosemary;
1 teaspoon dried thyme.}

Preheat oven to 325\deg F.

Cut roast into cubes. If using fresh collards then clean, remove stems and chop leaves (see \hyperref[BrooklynCollards]{Brooklyn Style Collard Greens}.)

Toss beef cubes with salt and black pepper. Heat olive oil in cast iron gumbo pot until smoking.  Add half the beef cubes and brown on all sides. Remove browned beef to a plate.  Add additional oil and bring pot back to high heat and brown second half of beef cubes.  

Return beef cubes from plate to pot. 
Add onion, celery, water, tomatoes, barley, kidney beans, collards or kale, rosemary, and thyme.

Bring pot to boil then cover pot and place in oven.  Bake for 1 hour.


\newpage
 
\recipe{Coconut-Curry Chickpea Pumpkin Stew}

\ingredients{3 tablespoons neutral oil, such as sunflower or canola;
1 large onion, chopped;
2 jalape\~{n}os, seeded or not, thinly sliced;
1 bay leaf;
1 knob ginger (about 1 inch), minced;
4 garlic cloves, minced;
1 \half teaspoons garam masala;
1 teaspoon ground cumin;
\half teaspoon ground turmeric;
2 (15-ounce) cans chickpeas, rinsed;
1 (13.5-ounce) can full-fat coconut milk;
1 (13.5-ounce) can pumpkin pur\'{e}e;
1 \half teaspoons fine sea salt, more as needed;
\threequarters cup chopped cilantro, more for serving;
2 to 3 tablespoons fresh lime juice, plus wedges for serving;
Cooked rice or couscous, for serving}

Heat oil in a large skillet over medium-high heat. Stir in onion, jalape\~{n}os and bay leaf. Cook, stirring occasionally, until onion is golden on the edges, about 8 minutes.

Add ginger and garlic and cook until fragrant, about 2 minutes, stirring frequently. Stir in garam masala, cumin and turmeric; cook for an additional 30 seconds.

Stir in chickpeas, coconut milk, pumpkin, 1 \half cup water and 1 \half teaspoons salt. Bring to a simmer and continue to simmer for 10 minutes, stirring occasionally, to let the flavors meld. (Add more water if it starts to look too thick.) Stir in cilantro and lime juice to taste. Taste and add more salt if necessary.

Serve over rice or couscous, and top with more cilantro and lime wedges on the side.

Serves 6 and takes around an hour to make start to finish.

Source: Melissa Clark from the NYT

\newpage

\recipe{Red Lentil Soup with Lemon}

\ingredients{3 tablespoons olive oil, more for drizzling;
1 large onion, chopped;
2 garlic cloves, minced;
1 tablespoon tomato paste;
1 teaspoon ground cumin;
\quarter teaspoon kosher salt, more to taste;
\quarter teaspoon ground black pepper;
Pinch of ground chile powder or cayenne, more to taste;
1 quart chicken or vegetable broth;
1 cup red lentils;
1 large carrot, peeled and diced;
Juice of \half lemon, more to taste,
3 tablespoons chopped fresh cilantro.} 

In a large pot, heat 3 tablespoons oil over high heat until hot and shimmering. Add onion and garlic, and saut\'{e} until golden, about 4 minutes.

Stir in tomato paste, cumin, salt, black pepper and chili powder or cayenne, and saut\'{e} for 2 minutes longer.

Add broth, lentils and carrot. Bring to a simmer, then partially cover pot and turn heat to medium-low. Simmer until lentils are soft, about 30 minutes. Taste and add salt if necessary.

Using an immersion or regular blender or a food processor, pur\'{e}e half the soup then add it back to pot. Soup should be somewhat chunky.

Reheat soup if necessary, then stir in lemon juice and cilantro. Serve soup drizzled with good olive oil and dusted lightly with chili powder if desired.

Source: Melissa Clark from the NYT

%\hyperref[TOC]{Table of Contents}

\newpage
\recipe{Butter Bean and Ham Soup}
\label{ButterBeanAndHamSoup}
\equilpment{10-quart stock pot; 8-quart polycarbonate food-container;  immersion blender.}

\ingredients{1 ham bone from Honey Baked Ham; 4 quarts water; 2 pounds Camellia Brand dried large limas; 2 large yellow or white onions, chopped; 4 cloves garlic, finely chopped; 2 bay leaves; 1 tablespoon \hyperref[CreoleSeasoning]{Creole seasoning}; olive oil; salt and black pepper to taste.
}


\textsc{This recipe works best if the ham stock is made the day before the soup and allowed to cool overnight in the refrigerator.}

Cut most of the ham off of the ham bone and reserve.  Place the ham bone in the stock pot and add the water.  Bring to a boil and then simmer covered for 1 to 2 hours until the bone separates into two pieces and the attached ham falls off the bone. Remove pot from heat and fish out the bones.  Place the pot in the refrigerator overnight to cool.

In the morning skim off the congealed fat from the top of the ham stock. Strain the stock into  the food-container. Dump out the strainer onto a large cutting board then pick out the ham from the fat and gristle. Shred the ham and add it to the food-container.  

Add olive oil to cleaned stock pot and place over medium heat.  When oil is hot add onions and garlic with a good pinch of salt. Cook stirring occasionally until the onions are lightly browned. While the onions are cooking dice the reserved ham into \quarter inch cubes.  Add olive oil to cover the bottom of a large skillet and use to brown the ham cubes. Add the ham stock and ham cubes to the onions. Sort and rinse the beans and add to the stock along with the bay leaves and Creole seasoning. Bring pot to a boil for 10 minutes. Reduce heat, cover and simmer, stirring occasionally for about 1\half hours until the beans are tender. 

With a ladle, remove some beans and liquid and place in the container that came with the immersion blender.  Pur\'{e}e with the immersion blender until smooth and then add back to pot. Stir to combine and evaluate the thickness.  Continue to pur\'{e}e beans until the desired thickness is achieved. Season the soup to taste with salt and black pepper.

\newpage

\recipe{Vegetable Broth Base}
\label{VegetableBroth}

\ingredients{2 leeks, white and light green parts only, chopped and washed thoroughly (2\half cups or 5 ounces);
2 carrots, peeled and cut into \half-inch pieces (\twothirds cup or 3 ounces);
\half small celery root, peeled and cut into \half-inch pieces (\threequarters cup or 3 ounces);
\half cup (\half ounce) parsley leaves and thin stems;
3 tablespoons dried minced onions;
2 tablespoons Diamond Crystal kosher salt;
1\half tablespoons tomato paste;
3 tablespoons soy sauce.}

For the best balance of flavors, measure the prepped vegetables by weight. Kosher salt aids in grinding the vegetables. The broth base contains enough salt to keep it from freezing solid, making it easy to remove 1 tablespoon at a time. To make 1 cup of broth, stir 1 tablespoon of fresh or frozen broth base into 1 cup of boiling water. If particle-free broth is desired, let the broth steep for 5 minutes and then strain it through a fine-mesh strainer.

Process leeks, carrots, celery root, parsley, minced onions, and salt in food processor, scraping down sides of bowl frequently, until paste is as fine as possible, 3 to 4 minutes. Add tomato paste and process for 1 minute, scraping down sides of bowl every 20 seconds. Add soy sauce and continue to process 1 minute longer. Transfer mixture to airtight container and tap firmly on counter to remove air bubbles. Press small piece of parchment paper flush against surface of mixture and cover. Freeze for up to 6 months.

To make broth for use, mix 1 tablespoon of base for each cup of boiling water. For example, to make a quart of stock mix 4 tablespoons of base with 1 quart of boiling water.  To make 7 quarts of broth, mix an entire recipe of base with 7 quarts of boiling water.

Source: America's Test Kitchen


%\hyperref[TOC]{Table of Contents}
\newpage
\recipe{Buttermilk Chicken Stew}
\ingredients{1 fryer, cut into pieces or 6 boneless thighs and 3 boneless half breast;
1 tablespoon olive oil;
2 tablespoons butter;
1 medium onion, chopped fine;
1 pound can whole tomatoes, drained;
1\quarter cups buttermilk;
\quarter teaspoon sugar;
\half teaspoon salt;
\eighth teaspoon black pepper;
dash of pepper sauce;
\half cup finely chopped green onions;
\quarter cup flat leaf parsley, chopped;
1\half teaspoons dried dill weed;
\quarter teaspoon lemon juice;
1 cup sour cream.}

Add butter and oil to a large cast iron pot over medium heat.  When hot, add chicken pieces an cook until brown.  

When chicken is browned all over, add onion and saut\'{e} over low heat until soft, about 5 minutes.

Add tomatoes, buttermilk, sugar, salt, pepper, and pepper sauce and simmer mixture for about 25 minutes.

After 25 minutes, add the green onions, parsley, and dill then cook, uncovered for 5 minutes.  Add lemon juice and sour cream and continue to cook until just heated.

Serve with French bread and a glass of dry rose.  Serves 4.

Source: Leon E. Soniat, Jr. from La Bouche Creole.

%\hyperref[TOC]{Table of Contents}
\newpage
\recipe{Smoked Pulled Pork, Collard Greens, and Blackeye Peas Gumbo}

\ingredients{\\
\\
\textsc{Roux --}
\twothirds cup vegetable oil or bacon fat;
\threequarters cup flour;
2 medium onions, diced;
1 medium green bell pepper, diced;
1 rib celery, diced;
3 cloves garlic, minced;
7 or 8 cups chicken stock.\\
\\
\textsc{Gumbo --}
2 bay leaves;
3 slices hardwood smoked bacon;
1 bunch collard greens, washed and sliced into ribbons;
1 white onion, chopped;
1 teaspoon white vinegar;
2 teaspoons sugar;
2 teaspoons Louisiana-style hot sauce;
2 teaspoons salt;
1 teaspoon freshly ground black pepper;
2 cups cooked blackeye peas prepared from fresh, frozen or dryed blackeye peas;
1 pounds smoked, pulled pork (not with barbecue sauce), shredded;
1 tablespoon chopped fresh thyme;
1 tablespoon Creole seasoning.}

Cook blackeyed peas and then hold drained peas in the refrigerator. Dice vegtables and mince garlic for the roux.

In an 8 cup glass measuring cup combine the flour and the oil or bacon fat.  Make a very \hyperref[BrownRoux]{dark roux} using the microwave oven. When roux is a deep chocolate color, add the celery, onions, bell pepper, and garlic to the measuring cup and and saut\'{e} in the roux until tender, around 5 minutes. The roux will continue to darken, and the vegetables will caramelize.
 

Transfer roux to a large cast-iron Dutch oven. Add 7 cups of stock and bay leaves to the roux mixture. Simmer for an hour, stirring often. Skim any fat off the top. If too thick, add more stock.

While stock is simmering, fry bacon in a tall sided skillet or shallow Dutch oven until crisp. Remove bacon and drain on paper towels, leaving bacon drippings in skillet or oven.

Wash collard greens in several changes of water to remove all grit than remove stems and slice leaves into ribbons.  Chop remaining onion then add collard greens and chopped onion to bacon drippings in skillet, and saut\'{e} until wilted. Crumble bacon and combine it with cooked greens/onion mixture.

After the roux mixture has simmered for an hour, to Dutch oven, add vinegar, sugar, hot sauce, and salt and pepper to taste. Then add greens mixture, blackeye peas, pork, thyme, and Creole seasoning.

Return to a simmer and adjust seasonings to taste.

Serve over hot cooked rice with French bread.

Yield:  10 servings.

Source: Adapted from recipe found on Camellia brand beans website (http://www.camelliabrand.com).

%\hyperref[TOC]{Table of Contents}
\newpage
\recipe{Duck and Andouille Gumbo}

\ingredients{2 frozen ducks, thawed;
2 pounds andouille sausage cut into 1/4 inch rounds;
2 medium onions, finely chopped;
1 bell pepper, finely chopped;
4 stalks celery, finely chopped;
1 cup canola oil;
1 cup flour;
4 cloves garlic, finely chopped;
3 bay leaves;
\quarter teaspoon ground allspice;
\quarter teaspoon cayenne pepper;
1 teaspoon basil;
\half teaspoon powdered cloves;
\half teaspoon poultry seasoning;
1 tablespoon salt;
1 teaspoon ground black pepper;
4 tablespoons Worcestershire sauce;
\half teaspoon Tabasco sauce;
4 (10\half ounce) cans of Campbells beef consomm\'e;
3 consomm\'e cans of water.} 


Prepare duck by pricking skin all over and then cooking in a glass tabletop convection oven.  Brown ducks one at a time. Start with the oven at 300 \deg F for 30 minutes to render the fat under the skin and then increase to 450 \deg F and cook until browned. Let both ducks cool and then remove meat from the bones and reserve.
Be sure to save the rendered duck fat for another use or, if you live dangerously, to make the roux. This step can be done well in advance.

Prepare the sausage, onions, bell pepper, celery, and garlic.

Combine the consomm\'e and water in a stockpot and bring to boil. Reduce heat and let simmer.

In a heavy skillet brown the sausage then drain on paper towels.

In an 8 cup glass measuring cup combine the flour and the oil and/or duck fat.  Make a very \hyperref[BrownRoux]{dark roux} using the microwave oven. When roux is ready cook the celery, onions, and pepper in the roux until it is tender. Transfer roux to a large cast-iron Dutch oven. Very carefully add hot stock to the roux stirring vigorously until roux is smooth and thin.

Add duck, sausage, roux and all remaining ingredients except for the pepper sauce and green onions. Partially cover and continue simmering until duck is very tender around 2\half hours.

Remove from heat and add pepper sauce.  Taste for seasoning and add salt if needed.  Stir in green onions and serve over rice.

Source: Leon E. Soniat, Jr. from La Bouche Creole.

%\hyperref[TOC]{Table of Contents}
\newpage
\recipe{Leftover-Turkey Gumbo}

\ingredients{Leftover turkey carcass with enough leg and thigh meat to make 3 to 4 cups cubed -- a smoked turkey is best.\\
\\
\textsc{Turkey Bone Broth --}
3 ribs celery, roughly chopped;
2 medium onions, roughly chopped; 
4 garlic cloves, smashed; 
1 tablespoon whole black peppercorns;
1 tablespoon thyme leaves;
4 bay leaves;
1 gallon water.\\
\\
\textsc{Gumbo --}
\threequarters cup vegetable oil;
\threequarters cup all-purpose flour;
2 cups chopped onions
\half cup chopped red bell peppers;
\half cup chopped celery;
1 tablespoon minced garlic;
1 tablespoon fresh thyme leaves, chopped fine;
3 quarts turkey bone broth;
3 tablespoons or so Kitchen Bouquet, optional;
4 cups turkey meat, torn into bite-sized pieces from the turkey carcass and/or leftovers;
\half pound andouille;
salt; freshly ground pepper; ground cayenne; hot sauce; flat-leaf parsley, chopped;
green onions, thinly sliced.
}

Remove as much meat as possible from the turkey carcass; tear into bite-sized pieces. Refrigerate the meat until you are ready to make the gumbo.

To make the turkey bone broth, break up carcass and place pieces on a foil-lined half sheet-pan. Roast at 425 \deg F until bones are browned.

Place the browned carcass pieces in a large stockpot. Scrape all the browned bits and fond from the sheet-pan and add to pot. Add the celery, onions, garlic, thyme, peppercorns and bay leaves. Add one gallon of water.  Add more water if necessary to cover carcass pieces. 

Bring to a boil, then reduce the heat so that the broth is at a bare simmer. Cook for at least 2 hours and up to overnight. If simmering for an extended time, partially or fully cover the pot. 

Strain the broth through a large colander and discard the solids. Refrigerate until the fat has solidified and can be easily removed. Should yield 3 quarts of stock.  If yield is less that 3 quarts add water to make up the difference. Extra broth can be frozen and used for another purpose.

To make the gumbo, chop and have ready the onions, peppers, celery and garlic. While preparing the vegetables, heat 3 quarts of bone broth in a large saucepan. 

In a heavy skillet brown the sausage then drain on paper towels. When the sausage has cooled, cut the rounds into quarters.

In an 8 cup glass measuring cup combine the flour and the oil.  Using the recipe given elsewhere in these notes, make a very \hyperref[BrownRoux]{dark roux} using the microwave oven. When roux is a deep chocolate color, add the celery, onions, bell pepper, garlic and thyme to the measuring cup, stir until blended, and cook in the roux until the onion is tender. 

Transfer roux to a large cast-iron Dutch oven. Very carefully add hot broth to the roux stirring vigorously until roux is smooth and thin. Add Kitchen Bouquet to achieve desired color. Bring to a boil then reduce the heat to medium-low and simmer uncovered for 45 minutes. Add the reserved turkey meat and sausage and simmer for 20 minutes.

Season to taste with salt, pepper and cayenne; let simmer another 10 minutes. Check the seasoning and adjust if needed.

Place a mound of cooked rice in each soup bowl, surround with the gumbo, and sprinkle with parsley and scallions.

Source: http://illinoistimes.com/article-16457-turkey-bone-gumbo.html

%\hyperref[TOC]{Table of Contents}
\newpage

\recipe{Brown Roux}
\label{BrownRoux}

\ingredients{flour; canola oil or well rendered bacon fat; Kitchen Bouquet Browning and Seasoning Sauce (optional).}

Combine equal amounts  of flour and oil or fat in an 8 cup Pyrex measuring cup (typically \twothirds cup of each or use the amounts specified in your recipe). Stir well to combine with a long handled wooden spoon.

Microwave on high for 6 minutes.  Carefully remove measuring cup from microwave using a dish towel and stir well being very careful not to splatter any of the hot roux on yourself (it's called Cajun napalm for a reason). The roux should be light brown at this point. Return measuring cup to the microwave and cook on high for 2 minutes then remove again and stir. Return measuring cup to the microwave and cook on high for 1 minute then remove again and stir. Continue this process of cooking for 1 minute and stirring until roux is the desired color. Three or four 1 minute cooking intervals is usually enough to get a dark brown roux. If a very dark roux is desired it may be necessary to reduce the cooking interval to 30 seconds when the roux is dark but not dark enough to avoid burning the roux. If you burn your roux as evidenced by black specks and smoke then toss it out and start over.  Since a roux can be made so quickly in the microwave you haven't lost much time.  With a little practice with your microwave you will be able to turn out perfectly dark roux every time.

Transfer roux to a large cast iron Dutch oven and add the onions, bell peppers, celery, and garlic in the amounts specified in your recipe. Saut\'{e} vegetables for around 5 minutes. The roux will continue to darken and the vegetables will caramelize.

Add stock to roux mixture and stir well.  It the color is not dark enough to suit you then add a tablespoon or more of Kitchen Bouquet Browning and Seasoning Sauce to achieve the desired color. Professional kitchens add caramelized sugar to color their roux so this is not cheating. If they can do it then you can too. Kitchen Bouquet is made with caramel and caramelized vegetables so it is actually a better way to add color since it adds flavor too.


\chapter{Fish and Seafood}
\recipe{Grilled Pickled Shrimp}

\ingredients{1 to 1\half pound shrimp;
\twothirds cup of shrimp stock made from shrimp shells;
1 medium onion chopped into bite-size squares;
2 tablespoons crab boil;
\twothirds cup of olive oil;
\twothirds cup of white wine vinegar;
\quarter cup of lemon juice;
2 tablespoons capers;
1 teaspoon sugar;
\half teaspoon salt;
\quarter teaspoon hot sauce
1 cup pimiento-stuffed green olives.}
 
Peal shrimp and remove tails. Devein the shrimp if you are fussy.

Make shrimp stock by simmering shells and tails in \twothirds cup of water for 10 minutes. Strain out the shells and tails and reserve the stock.  

Combine the shrimp stock, the onion and the crab boil in a small sauce pan. Bring the liquid to a boil, reduce heat, and simmer for around 5 minutes. Remove from heat and add all remaining ingredients except the olives.  Pour the warm marinade over the shrimp, mix well, and marinate from 8 to 24 hours refrigerated.

Drain the shrimp and the onions and alternate them on skewers with the green olives. Skewer one end of the shrimp, slide on an olive, skewer the other end of the shrimp then place an onion square between shrimp on the skewer. The skewers can be returned to the marinade until just before cooking.

Grill shrimp skewers over medium-hot ashen-gray coals for a minute or two, until the shrimp are just firm. 

Serves 4 as a main course.

Source: Cheryl Alters Jamison and Bill Jamison from Texas Home Cooking.  

I like to serve these shrimp on top of yellow rice.  I strain the capers and onions from the marinade and add them the rice with a little of the marinade for dressing.  This recipe is ideal for entertaining because it requires very little last minute preparation.

\newpage
\recipe{Catfish Courtbouillon}

\ingredients{\half pound shrimp;
1\half pounds  catfish fillets;
5 tablespoons flour;
5 tablespoons olive oil;
2 cups chopped onion;
2 cups chopped celery;
\threequarters cup chopped bell peppers;
3 cloves garlic, minced;
\half cup chopped green onions;
\half cup chopped parsley;
1 cup dry white wine;
28 oz. can whole tomatoes, drained and broken up by hand;
\half small can tomato paste;
2 teaspoons salt;
\half teaspoon black pepper;
4 bay leaves;
\half teaspoon thyme;
2 tablespoons lemon juice;
2 teaspoons more or less hot sauce, to taste.}

Peal shrimp reserving heads, shells, and tails. Put shrimp in refrigerator and hold for use at the end of the recipe.  Simmer reserved shrimp heads, shells, and tails in 2 cups water for 10 minutes. Strain and reserve shrimp stock.

Combine flour and oil in glass measuring cup and proceed to make a dark brown roux in the microwave. Stir onions, celery, bell peppers, and garlic into the roux and then microwave for a few more minutes until the onions are soft.

Transfer roux mixture to a cast iron pot. Mix in tomatoes and tomato paste then cook, stirring constantly, about five minutes.  Add shrimp stock, wine, parsley, green onions, lemon juice, bay leaves, thyme, salt, pepper, and hot sauce. Simmer for about 45 minutes. Can stop at this point and then reheat courtbouillon before continuing recipe.

Add fish and simmer for 10 minutes or so until fish flakes when tested with a fork. If desired the shrimp can be added 5 or so minutes after the fish so that shrimp and fish are done at the same time. Serve with rice.

%\hyperref[TOC]{Table of Contents}
\newpage
\recipe{Catfish Pecan with Lemon Thyme Pecan Butter}


\ingredients{3 cups (10 ounces) pecan halves;
1\half cups all-purpose flour;
Creole seafood seasoning, to taste;
1 medium egg;
1 cup milk;
6 catfish fillets, 5 to 7 ounces each;
12 tablespoons (1\half sticks) butter;
3 lemons, cut in half;
1 tablespoon Worcestershire sauce;
6 large sprigs fresh thyme;
Kosher salt and freshly ground pepper to taste.}

Place half the pecans, the flour, and the Creole
seasoning in the work bowl of a food processor, and process until
finely ground. Transfer the pecan flour to a large bowl.

Whisk the egg in a large mixing bowl and add the milk. Season both
sides of the fish fillets with Creole seasoning. On e at a time,
place the fillets in the egg wash.

Remove one fillet from the egg wash, letting any excess fluid
drain back into the bowl. Dredge the fillet in the pecan flour and
coat both sides, shaking off any excess. Transfer to a dry sheet
pan, and repeat with the remaining fillets.

Place a large saut\'e an over high heat and add 2 tablespoons of the
butter. Heat for about 2 minute, or until the butter is completely
melted and starts to bubble. Place three fish fillets in the pan,
skin side up, and cook for 30 seconds. Reduce the heat to medium,
and cook for another 1\threequarters to 2 minutes, or until the fillets are
evenly brown and crisp. Turn the fish over and cook on the second
side for 2 to 2\half minutes, or until the fish is firm to the
touch and an even brown. The most important factor in determining
the ideal cooking time is the thickness of the fillets you are
using.

Remove the fish, place on a baking rack, wipe the pan clean with a
paper towel, and 2 tablespoons of butter, and repeat with the
remaining pieces of fish.

When a all the fish fillets are cooked, wipe the pan, and return
the heat to high. Melt the remaining 8 tablespoons of butter and,
just as the butter turns brown, and the remaining 1\half cups of
pecans and saut\'e or 2 to 3 minutes or until the nuts are toasted,
stirring occasionally. Put the lemons face down in the pan, first
squeezing a little juice from each piece. Add the Worcestershire
and the fresh thyme, season with salt and pepper, and cook for 30
seconds more or until thyme starts to wilt and becomes very
aromatic.

Place one fish fillet and a lemon piece on each of six dinner
plates, spoon some pecan better around each piece of the fish, and
use the wilted thyme to garnish the piece.

%Alternate instructions:
%
%For two large catfish filets, combine 1/2 C flour and 1/2 C pecan
%halves in the food processor with 1 t creole seasoning and grind
%until fine.  Mix one egg with milk.  Season both sides of the
%catfish and then put in egg wash.  From the egg wash press the
%filets into the pecan/flour mixture making sure the filets are
%well coated. Saut\'{e} catfish over medium heat in butter (not too hot
%or the crust will burn) for about 2 min. per side.  Take off fish
%and add more butter to the pan with whole pecan halves.  Toast the
%pecans. When brown add lemon halves cut side down to brown over
%low heat.  Add fresh thyme and season with salt and pepper. Finish
%pan sauce with Worcestershire sauce after it has cooked for a moment.
%To serve, put catfish on plate, put lemon halve cut side up on
%plate, put thyme sprig on top of catfish, scatter pecan halves
%over catfish, pour pan sauce around plate and scatter chopped
%green onions around plate.

%\hyperref[TOC]{Table of Contents}
\newpage
\recipe{Thin Fried Catfish with Slaw and Tartar Sauce}

\ingredients{catfish fillets;
buttermilk;
hot sauce;
yellow cornmeal;
flour;
peanut oil for frying;
Napa cabbage;
red cabbage;
red onion;
celery seeds;
salt;
black pepper;
cane vinegar;
\half cup homemade mayonnaise;
1 tablespoon prepared  horseradish;
2 tablespoons Creole mustard;
2 tablespoons chopped homemade sweet pickle;
1 teaspoons creole seasoning;
\quarter to \half cup chopped green onions.}

Cut 1 ounce portions of catfish.  Cut the catfish at an angle to get
thin pieces.  Put catfish in buttermilk and add hot sauce (2 or 3
tablespoons to 1 cup of buttermilk).  Make a cornmeal crust for the catfish by
combining equal parts of yellow cornmeal and shifted flour.  Shake
buttermilk off of catfish and press fish in cornmeal flour
mixture.  Press hard on both sides to flatten catfish pieces.  Fry
catfish pieces at 350\deg F in peanut oil. Season catfish with creole
seasoning when it comes out of the oil. 

Make a slaw with thinly cut Napa cabbage, red cabbage and red
onion.  Add celery seeds, salt, pepper and cane vinegar.

To make Tartar sauce, mix mayonnaise, horseradish,  Creole mustard,
chopped homemade sweet pickle, 
Creole seasoning and chopped green onions.  Season with salt and pepper.

To serve, put slaw on plate and top with catfish.  Drizzle tartar
sauce around plate.

%\hyperref[TOC]{Table of Contents}
\newpage
\recipe{Smoked Catfish P\^{a}t\'{e}}

\ingredients{1\half pounds catfish filets;
mesquite or pecan wood chips for smoking;
1 envelope unflavored gelatin;
\quarter cup cold water;
\half cup boiling water;
\half cup homemade mayonnaise;
1 tablespoon lemon juice;
1 tablespoon graded onion;
1 large clove garlic, pressed;
1 tablespoon hot sauce;
2 tablespoons finely chopped fresh basil;
1 teaspoon salt;
\quarter cup roasted red bell pepper strips;
1 cup heavy cream.}

Cook catfish on a barbecue grill using wood chips to obtain a strongly smoked flavor. The catfish should be cooked to the point where it flakes easily. Finely flake 2 cups of the smoked catfish to use in the recipe and set aside.

Soften the gelatin in the cold water in a large mixing bowl. Stir in the boiling water and whisk the mixture slowly until the gelatin dissolves. Cool to room temperature.

Whisk in the mayonnaise, lemon juice, graded onion, pressed garlic, hot sauce, basal and salt. Stir to blend completely and refrigerate  until mixture begins to thicken slightly, about 15 to 20 minutes.

Fold in the flaked catfish and bell pepper strips. In a separate bowl, whip the cream until is is thickened to peaks and fluffy.  Fold gently but completely into the catfish mixture. Transfer to a 6 to 8 cup mold (a bread pan works fine). Cover and chill for at least 4 hours.

Unmold on a platter and garnish with parsley, watercress, or other greenery. Serve with toast, black bread, or crackers.

%\hyperref[TOC]{Table of Contents}
\newpage
\recipe{Stove Top Smoked Salmon}

\ingredients{1 pound (more or less) farm raised salmon filet;
\twothirds cup light brown sugar;
\third cup Kosher salt;
2 or 3 Tablespoons cherry wood chips}

Combine brown sugar and salt in a glass casserole dish.  

Wash filet under running water and then pat completely dry using paper towels.

Spread a \quarter inch layer of the sugar and salt mixture on the bottom of glass casserole leaving the remaining mixture at the edges of the dish. Place the filet skin side down on the layer of curing mixture then completely cover the salmon flesh with the remaining mixture.  Cover the casserole with plastic wrap and refrigerate for at least one hour. 

Prepare the stove top smoker by adding cherry wood chips to bottom. Add filet to smoker rack, close cover of smoker and place smoker of medium high burner.  Smoke for 25 to 35 minutes depending on the thickness of the filet. When done the internal temperature should be around 140 \deg F and the filets should have a smokey brown appearance. 

Remove from smoker and cool for a few minutes if serving hot. Loosely wrap in foil and refrigerate if serving cold.

Excellent served warm over a green salad with grapefruit supremes and a creamy grapefruit vinaigrette dressing. Also excellent served in the traditional way as an appetizer with cream cheese, red onion, capers, and toasted bagels or sour dough bread.

%\hyperref[TOC]{Table of Contents}
\newpage
\recipe{Crawfish Monica}

\ingredients{1 pound crawfish tails; 
1 stick butter;
1 pint half and half;
1 bunch green onions, green and white parts chopped;
5 cloves garlic (or up to 10 to taste), chopped;
1-2 tablespoons Creole Seasoning;
1 pound Rotelli}

Cook pasta according to the directions on the package.  Drain,
then rinse under cool water.  Drain again, thoroughly.

Melt the butter in a large pot and saut\'{e} onions and garlic for 3
minutes.  Add the crawfish and saut\'{e} for 2 minutes.  Add the
half-and-half, then add several big pinches of Creole seasoning,
tasting before the next pinch until you think it's right.

Cook for 5 - 10 minutes over medium heat until the sauce thickens.
Add the pasta and toss well.  Let it sit for 10 minutes or so
over very low heat, stirring often.  Serve immediately, with lots
of French bread and a nice dry white wine.


%\hyperref[TOC]{Table of Contents}
\newpage

\recipe{Shrimp in Cashew-Yogurt Sauce}
\ingredients{1 to 1\quarter pounds large (16-20 count) frozen shrimp, shell-on and deveined;
5 tablespoons unsalted butter;
1 medium onion;
2 cloves garlic;
\half teaspoon chili powder, plus more as needed;
\half teaspoon ground cumin;
1 teaspoon ground turmeric;
\half teaspoon kosher salt, or more as needed;
1 teaspoon freshly ground black pepper; 
1 cup roasted, unsalted cashews;
2 cups plain Greek yogurt, preferably full-fat}

Place the shrimp (to taste) in a bowl of tap water and let them defrost a bit while you prep the butter. When you can, pull off the tails and reserve.

Melt the butter in the pot, over medium-low heat. As soon as white milk solids form on the surface, skim off and discard them. The butter should be mostly golden and clear; you have clarified it enough for this dish.

Add the reserved shrimp tails; cook for about 10 minutes, stirring once or twice. (You are infusing the butter with shrimp flavor!)

Meanwhile, peel the shrimp; it's okay if they are not fully defrosted. Cut the onion into small dice. Mince the garlic.

Remove the shrimp tails from the pot and discard, then stir in the onion and garlic. Increase the heat to medium; cook for 3 or 4 minutes until just softened, stirring a few times, then add the chili powder, cumin, turmeric, salt and pepper, stirring to incorporate. Add half the cashews.

Add the yogurt, stirring to form a thick sauce; cook for 2 or 3 minutes, then reduce the heat to low.
 
Use an immersion blender to puree the mixture right in the pot.

Drain the shrimp and add them to the pot, along with the remaining cashews. Increase the heat to medium; cook for about 4 minutes, stirring to make sure all the shrimp is pink and no longer translucent. Taste and add more salt and/or chili powder, as needed.

Divide among wide, shallow bowls. Sprinkle each portion with a little chili powder, and serve warm.

Makes four servings when served over rice. Attractive when topped with strips of roasted red bell pepper.

Source: Washington Post

%\hyperref[TOC]{Table of Contents}

\chapter{Poultry}
\recipe{French Chicken in a Pot}

\ingredients{1 small onion diced;
1 rib celery;
6 whole garlic cloves;
1 bay leave;
1 sprig rosemary;
4\half to 5 pound natural/free-range/organic chicken;
2 teaspoons Kosher salt;
fresh ground black pepper;
olive oil;
juice of \half lemon.}

Preheat oven to 250 \deg F. Dice onion and celery. Peal garlic cloves.

Fold back chicken's wing tips. Completely dry chicken with paper towels and
then season with salt and pepper.

Heat enough olive oil to cover the bottom of a Dutch oven till near smoking.
Using long tongs place the chicken breast side down in the Dutch oven.
Add bay leave, celery, rosemary, onion and pressed garlic cloves
on either side of chicken.

Brown breast side for around 5 minutes then turn to brown back for some what longer
(around 8 minutes). Occasionally stir the vegetables so they don't burn.

Turn off heat, put foil over Dutch oven to seal, put on lid and
place Dutch oven in preheated oven.

Bake for 80 to 110 minutes until breast is 160-165 \deg F and dark meat is 175 \deg F.

Let chicken rest under foil tent for 20 minutes before carving.

Strain juices from pan into a fat separator.

Pour fat-free juices into a small sauce pan and season with lemon juice,
salt and pepper. Reheat juices over low heat.

When chicken has rested, carve off wings and leg/thigh then separate
the leg and thigh. Remove breast halves each in one piece then
slice across from keel side to side side. Arrange  meat on a plater.

Pour some the heated juice over the meat and reserve some juice to serve on the side.

\newpage
\recipe{Korean Fried Chicken Wings}
\label{KoreanWings}
\equilpment{Microplane Classic Zester Grater; R \& V Works Cajun Fryer Model FF1.}
\ingredients{1 tablespoon toasted
sesame oil;
1 teaspoon garlic, grated to paste;
1 teaspoon grated fresh ginger;
1\threequarters cups water;
3 tablespoons sugar;
2-3 tablespoons gochujang;
1 tablespoon soy sauce;
1 cup all-purpose flour;
3 tablespoons cornstarch;
3 pounds chicken wings, cut at joints, wingtips discarded;
peanut oil to fill fryer.}

 For a complete meal, serve with steamed white rice and a slaw.

Combine sesame oil, garlic, and ginger in large bowl and microwave until mixture is bubbly and garlic and ginger are fragrant but not browned, 40 to 60 seconds. Whisk in \quarter cup water, sugar, gochujang, and soy sauce until smooth; set aside.

 Whisk flour, cornstarch, and remaining 1\half cups water in second large bowl until smooth.

Heat vegetable oil in fryer to 350\deg F. Place wings in batter and stir to coat. Using tongs, remove wings from batter one at a time, allowing any excess batter to drip back into bowl, and add to hot oil. When adding a wing to the oil, hold the wing in the oil with the tongs for a few seconds to allow the coating to set before releasing the wing into the fryer. Cook, stirring occasionally to prevent wings from sticking, until coating is light golden and beginning to crisp, about 7 minutes. Raise fryer basket and let wings rest for 5 minutes.

Heat oil to 375\deg F.  Return all wings to oil and cook, stirring occasionally, until deep golden brown and very crispy, about 7 minutes. Raise fryer basket and let stand for 2 minutes. Transfer wings to reserved sauce and toss until coated. Transfer to platter, let stand for 2 minutes to allow coating to set and then serve.

A generous meal for two.

Source: America's Test Kitchen recipe modified to use a R \& V Works Cajun Fryer. 

\newpage
\recipe{Marinated Game Hens in Mustard and Tarragon Sauce}

\ingredients{4 game hens;
\half cup olive oil;
\half teaspoon dried or 1\half teaspoons fresh tarragon;
2 tablespoons chopped shallot;
2 tablespoons chopped shallot
\half cup white wine or tarragon vinegar;
\half cup white vermouth;
2 tablespoons lemon or lime; 
juice strained roasting pan juices;
2 cups chicken stock;
1 cup heavy cream;
1 tablespoons Dijon mustard;
\half teaspoon tarragon or to taste;
3 or 4 tablespoons butter.}

Spilt the game hens with poultry shears (removing the backbone) and marinate overnight in a mixture of olive oil, tarragon, and shallot.

Arrange the split hens meat side up in a deep roasting pan.  Be sure that the hens do not touch.  Roast the hens for 35 to 45 minutes in a 375\deg F oven.  Check the drum sticks for doneness.

Transfer the hens to a covered warming pan and keep warm in the oven.  Strain the juice from the cooking pan and reserve for use in the sauce in the next step.

The remaining ingredients and the pan juices are used to make the sauce.  Because this is a reduction sauce, all the remaining ingredients should be prepared and measured before beginning the sauce.

In a wide, non-reactive skillet placed over high heat, add the vinegar, vermouth, and chopped shallots and boil until nearly dry.

Add the lemon or lime juice and the pan juices and reduce by one half.

Add the chicken stock to the pan and reduce by one half again.

Add the cream to the sauce and cook down to the consistency of maple syrup.

Finish the sauce by adding the mustard, tarragon, and butter.

Remove the hens from the warming pan and place on the serving platter.  Surround the hens with the broccoli and carrots and drizzle the sauce over the hens.  Serve the remaining sauce in a sauce boat.

I learned to make this from my friend Rosa Rajkovic during my time in Albuquerque, NM.  The reduction sauce that accompanies the game hens is very rich.  As a result, I usually serve this with a simple garnish of steamed baby carrots and/or broccoli on the serving platter and nothing but this and bread for the main course.  I have made several variations of this recipe.  These include replacing the vermouth with red port or Madeira and replacing the white vinegar with red.  Instead of roasting the marinated game hens they can be grilled and the sauce made without the pan juices.  Serves 8 (half a game hen per person is usually enough if a meal with a vegetable garnish, salad, bread, and desert is served).

\newpage
\recipe{Oven-Roasted Turkey Meatloaf}


\ingredients{1\quarter pound ground turkey;
\quarter pound bell peppers, diced;
\quarter pound yellow onions, diced;
\threequarters cup oatmeal;
\half cup breadcrumbs;
\quarter cup BBQ sauce;
\quarter cup garlic, minced;
2 eggs;
\half teaspoon + \quarter teaspoon chicken base;
\half teaspoon + \quarter teaspoon salt;
\half teaspoon + \quarter teaspoon freshly ground black pepper.}

Preheat oven to 350\deg F.

Combine all ingredients but do not over mix.

Line loaf pan with parchment paper and spray with non-stick cooking spray.

Fill pan with mixture, pack well to prevent air pockets. As an alternative, roll up mixture in parchment paper, twist the ends to make a compact bundle, and then place seam side down in loaf pan.

Place pan in oven cook 45 minutes to an hour.

Serve with Macaroni and cheese and Southern cooked greens.

From Isaac Hayes' restaurant in Memphis (Gone but great while it lasted). Makes 4 servings.

\newpage
\recipe{Roasted Chicken Proven\c{c}al}

\ingredients{4 chicken leg quarters;
2 teaspoons kosher salt;
1 teaspoon freshly ground black pepper;
\half to \threequarters cup all-purpose flour;
3 tablespoons olive oil;
2 tablespoons herbes de Provence;
1 lemon, quartered;
8-10 cloves garlic, peeled;
4-6 medium-size shallots, peeled and halved;
\third cup dry vermouth;
4 sprigs of thyme, for serving.}

Preheat oven to 400\deg F. Season the chicken with salt and pepper. Put the flour in a shallow pan, and lightly dredge the chicken in it, shaking the pieces to remove excess flour.

Swirl the oil in a large roasting pan and place the floured chicken in it. Season the chicken with the herbes de Provence. Arrange the lemons, garlic cloves and shallots around the chicken, and then add the vermouth to the pan.

Put the pan in the oven,and roast for 25 to 30 minutes,then baste it with the pan juices. Continue roasting for an additional 25 to 30 minutes, or until the chicken is very crisp and the meat cooked through.

Serve in the pan or on a warmed platter, garnished with the thyme.

Yield: 4 servings

Source: The New York Times Magazine, April 1, 2015

\newpage
\recipe{Gai Yang (Thai-Style Grilled Chicken)}
\label{GaiYang}
\ingredients{6 medium cloves garlic;
12 sprigs cilantro, including thick stalks;
3 tablespoons palm or light brown sugar;
2 teaspoons ground white pepper;
1 teaspoon ground coriander;
2 tablespoons fish sauce;
2 teaspoons dark soy sauce;
2 (3-inch) segments lemongrass (optional);
1 whole chicken, spine removed, split in half along the breast bone;
1 recipe \hyperref[ThaiDippingSauce]{Thai-Style Sweet Chili Dipping Sauce} (recipe elsewhere in book)}

Combine garlic, cilantro, sugar, white pepper, coriander, fish sauce, dark soy sauce, and lemongrass (if using) in the bowl of a food processor. Process until a rough paste is formed. Set aside.

Place chicken halves flat on a cutting board. Insert two metal skewers into each chicken, running parallel through the legs and the breasts. Transfer chicken to a casserole dish that just fits them. Rub on all surfaces with marinade. Cover, transfer to refrigerator, and allow to marinate for at least 2 hours and up to over night.

When ready to cook, light one chimney full of charcoal. When all the charcoal is lit and covered with gray ash, pour out and spread the coals evenly over half of coal grate. Set cooking grate in place, cover gill and allow to preheat for 5 minutes. Clean and oil the grilling grate.

Place chicken skin-side up on cooler side of grill with legs facing towards hotter side. Cover grill with vents on lid open and aligned over the chicken. Open bottom vents of grill. Cook until instant read thermometer inserted into deepest part of breast registers 140\deg F, about 45 minutes. Carefully flip chicken and place skin-side-down on cooler side of grill with breasts pointed towards hotter side. Cover and cook until skin is crisp and instant read thermometer inserted into deepest part of breast registers 145 to 150\deg F, about 3 minutes longer (be careful, the skin is very prone to burning). Do not leave the lid off for longer than it takes to check temperature or chicken will burn.

Transfer chicken to a cutting board and allow to rest for 5 minutes. Carve and serve with dipping sauce.

Yield: 3 or 4 servings

Source: SeriousEats.com

\recipe{Thai-Style Sweet Chili Dipping Sauce}
\label{ThaiDippingSauce} 
\ingredients{10 to 12 red Thai bird chilies, finely minced OR 3 tablespoons of Sambal Oelek;
4 cloves garlic, finely minced (about 4 teaspoons);
\quarter cup Asian fish sauce;
\half cup light brown sugar;
3 tablespoons distilled white vinegar;
2 tablespoons fresh lime juice}
 
Combine all ingredients in a small sealable container. Shake well, set aside, and allow sugar to dissolve, shaking every 10 minutes or so. When sugar is fully dissolved, store in refrigerator until ready to use. Sauce will keep in the refrigerator for up to 3 weeks.

Sauce used with \hyperref[GaiYang]{Gai Yang} (Thai-Style Grilled Chicken).

 
Source: SeriousEats.com

\newpage
\recipe{Slow Roasted Turkey Breasts}

\ingredients{whole turkey breast with bones attached;
1 cup kosher salt;
1 cup sugar}

If the turkey breast is frozen defrost it in the refrigerator for a day or two until completely thawed. Carefully cut the two breast off the bone while leaving the skin covering the breasts intact. Reserve the bones for stock making.  The breasts should weight from 3 to 4\half pounds each.  

In a sauce pan bring 2 quarts of water to boil with the salt and sugar. Pour brine into brining container and add ice to make the total volume 4 quarts.  Submerge the turkey breast in the brine and refrigerate from 12 to 24 hours.

Preheat the oven to 250\deg F.  Remove the breasts from the brine and wrap each one four times in plastic wrap and once in aluminum foil. Place the breast on a wire rack in a half sheet pan. Add water to reach just below the rack. Cook until the internal temperature reaches 135\deg F, 2 to 3 hours. Remove turkey from oven and then, without removing the wrapping, submerge the turkey in an ice bath for 5 minutes. Remove foil, plastic wrap, and skin.

One option for serving the roasted turkey breast is to cut 1 inch thick slices and brown both sides in butter using a cast iron skillet.

\newpage
\recipe{Garlicky Chicken With Lemon-Anchovy Sauce}

\ingredients{1\quarter pounds boneless, skinless chicken thighs (4 to 5 thighs);
1 teaspoon coarse kosher salt;
Freshly ground black pepper;
6 garlic cloves, smashed and peeled;
\quarter cup extra-virgin olive oil;
5 anchovy fillets;
2 tablespoons drained capers, patted dry;
1 large pinch chile flakes;
1 lemon, halved;
Fresh chopped parsley, for serving.}

Heat oven to 350\deg F. Season the chicken thighs with salt and pepper and let rest while you prepare the anchovy-garlic oil. Mince one of the garlic cloves and set it aside for later. In a large, ovenproof skillet over medium-high heat, add the oil. When the oil is hot, add the 5 smashed whole garlic cloves, the anchovies, capers and chile. Let cook, stirring with a wooden spoon to break up the anchovies, until the garlic browns around the edges and the anchovies dissolve, 3 to 5 minutes.

Add the chicken thighs and cook until nicely browned on one side, 5 to 7 minutes. Flip the thighs, place the pan in the oven and cook another 5 to 10 minutes, until the chicken is cooked through.

When chicken is done, transfer thighs to a plate (be careful, as the pan handle will be hot). Place skillet back on the heat and add minced garlic and the juice of one lemon half. Cook for about 30 seconds, scraping up the browned bits on the bottom of the pan. Return chicken to the pan and cook it in the sauce for another 15 to 30 seconds.

Source: Melissa Clark from the NYT

\chapter{Meat}

\recipe{Sichuan Pork Burgers}

\ingredients{\\
\\ 
\textsc{Saut\'{e}ed Onions --}
2 tablespoons canola oil;
2 large yellow onions, halved and thinly sliced (about 5 cups);\\
\\
\textsc{Burgers --}
1 tablespoon canola or other vegetable oil;
1 tablespoon garlic paste;
1 pound ground pork;
2 tablespoons gochujang;
\half tablespoon ground Sichuan peppercorns;
1 teaspoon gochugaru chili flakes;
1 tablespoon soy sauce;
1 tablespoon sugar;
Kosher salt;\\
\\
\textsc{Acompaniments --}
yellow prepared mustard;
planked dill pickles;\\
\\
\textsc{Buns --}
4 \hyperref[HamburgerBuns]{homemade hamburger buns};
8 tablespoons softened butter.
}

In a large skillet saut\'{e} onions in 2 tablespoons of oil for about 15 minutes until they are light golden.  Remove the onions from the skillet and wipe clean in preparation for cooking the burgers.

Grate garlic to paste using a Microplane. Grind Sichuan peppercorns in a spice grinder. 

In a bowl mix the burger ingredients (except oil) by hand.  Weigh and divide burger mixture into quarters then form into fairly thin burgers.  Add oil to hot skillet and then cook burgers until well done and browned on both sides.  When cooked set burgers aside to rest covered with foil.

Coat cut sides of buns with butter and then brown lightly in the skillet or a grill pan.

Serve burgers with saut\'{e}ed onions, yellow mustard and planked dill pickles.

Inspired by a Luck Peach recipe for Sichuan Pork Ragu -- thanks to Jack Williams for the idea of making burgers using the same ingredients.

\newpage
\recipe{Beef, Turkey, and Andouille Meat Loaf}

\ingredients{\\
\\
\textsc{Loaf --}
2 tablespoons olive oil;
1\half cups onion, finely chopped;
1 teaspoon salt;
\half cup chopped red bell peppers;
2 teaspoons minced garlic;
1 tablespoon Worcestershire sauce;
2 cups fresh bread crumbs;
\half cup milk;
2 large eggs, lightly beaten;
\half pound andouille, coarsely chopped in food processor;
1 pound 85\% lean ground beef;
1 pound ground turkey;
\half teaspoon dried sage;
1 teaspoon dried thyme;
1 teaspoon freshly ground black pepper;
\\
\\
\textsc{Glaze --}
6 tablespoons (3 ounces) Dijon mustard;
3 tablespoons dark brown sugar;
1 tablespoon cider vinegar.}

Preheat oven to 350\deg F. Heat oil in a large skillet over medium-high heat and add onions, bell peppers, garlic and a pinch of salt. Cover and cook  stirring occasionally until soft, around 5 minutes. Remove from heat and then add sage, thyme, pepper and salt. Stir to combine.  In as small bowl combine bread crumbs, milk, and Worcestershire sauce.  Once combined add beaten eggs and stir in.  Place chopped andouille, ground beef and ground turkey in a large bowl.  Mix the meats my hand until roughly homogeneous. Add onion mixture and bread crumb mixture to the meat mixture and knead by hand until well blended. Line a half sheet pan with foil.  Form the meat into a loaf about 10 by 4 by 4 inches.  Place loaf in oven and bake for 30 minutes.

While the loaf is in the oven combine the mustard, brown sugar and vinegar to make a glaze. After the first 30 minutes of baking, brush the glaze all over the loaf.

Bake the loaf for 30 to 45 minutes more until the internal temperature is 155\deg F.  Let loaf rest for 10 to 20 minutes, loosely covered with foil, before slicing and serving.

Source: Bruce Aidells and Denis Kelly from The Complete Meat Cookbook.

\newpage
\recipe{Eye Round Roast}

\ingredients{3\half to 4 pound eye round roast;
4 teaspoons kosher salt;
freshly ground black pepper;
vegetable oil.}

Rub salt over entire surface of roast then wrap tightly in plastic wrap and 
place in refrigerator for 18 to 24 hours.

Preheat oven to 225\deg F.

Dry the surface of the roast very well with paper towels then coat with olive oil and 
a generous amount of black pepper.

Heat a cast iron skillet to a high temperature then add roast. When the first side is
very brown, turn the roast and brown the next side.  Continue turning and 
browning until the entire roast is very brown.  The ends are browned by holding 
the roast upright with tongs.  If the browning is done correctly there is a good 
deal of smoke and splattering.

Transfer skillet with roast to the oven and cook until the roast is 130\deg F. Check temperature after 60 minutes. Total cook time is 75 to 90 minutes depending on size and initial internal temperature of the roast.

Remove roast and let rest on cutting board under a foil tent for 15 minutes 
before carving.


\newpage
\recipe{Beef Pot Roast With Tomato Sauce\ and Pearl Onions}
\ingredients{3 pound chuck roast;
salt and freshly ground black pepper;
2 tablespoons olive oil;
1 medium onion, finely chopped;
3 tablespoons of cider vinegar;
6 ounce can tomato paste;
1\half cups water;
\half teaspoon ground allspice;
\half teaspoon ground cumin seeds;
\quarter teaspoon sugar;
\threequarters cup frozen pealed pearl onions;
1 clove garlic, unpeeled;
2 bay leaves.}

Season roast with salt and pepper. Heat oil in a pressure cooker until very hot. Add roast and brown well on all sides.

Remove roast from cooker, add chopped yellow onions and cook until wilted.

Add vinegar, tomato paste, water, allspice, cumin and sugar to cooker then stir to mix well. Return roast to cooker and top with garlic, bay leaves, and pearl onions.

Close cooker and cook at 15 psi for 30 minutes. Release pressure and remove lid. Pick out garlic and bay leaves and discard. Remove roast from cooker and slice.

Serve roast with the tomato sauce and pearl onions. Excellent with spaghetti dressed with the tomato sauce and pearl onions then topped with slices of roast.

\newpage
\recipe{Garlic-Lime Grilled Pork Tenderloin Steaks}

\ingredients{2 (1-pound) pork tenderloins, trimmed;
1 tablespoon grated lime zest plus 1/4 cup juice (2 limes);
4 garlic cloves, minced;
4 teaspoons honey;
2 teaspoons fish sauce
\threequarters teaspoon salt;
\half teaspoon pepper; 
\half cup vegetable oil;
4 teaspoons mayonnaise;
1 tablespoon chopped fresh cilantro;
Flake sea salt (optional)}

Slice each tenderloin in half crosswise (perpendicular to the long axis) to create 4 steaks total. Pound each half to 3/4-inch thickness. Using sharp knife, cut 1/8-inch-deep slits spaced 1/2 inch apart in crosshatch pattern on both sides of steaks.

Whisk lime zest and juice, garlic, honey, fish sauce, salt, and pepper together in large bowl. Whisking constantly, slowly drizzle oil into lime mixture until smooth and slightly thickened. Transfer 1/2 cup lime mixture to small bowl and whisk in mayonnaise; set aside sauce. Add steaks to bowl with remaining marinade and toss thoroughly to coat; transfer steaks and marinade to large zipper-lock bag, press out as much air as possible, and seal bag. Let steaks sit at room temperature for 45 minutes.

Completely open the bottom vents of a charcoal grill. Light a large chimney starter filled with charcoal briquettes (6 quarts). When top coals are partially covered with ash, pour evenly over half of grill. Set cooking grate in place, cover, and open lid vent completely. Heat grill until hot, about 5 minutes.

Clean and oil cooking grate. Remove steaks from marinade (do not pat dry) and place over hotter part of grill. Cook, uncovered, until well browned on first side, 3 to 4 minutes. Flip steaks and cook until well browned on second side, 3 to 4 minutes. Transfer steaks to cooler part of grill, with wider end of each steak facing hotter part of grill. Cover and cook until meat registers 140 degrees, 3 to 8 minutes longer (remove steaks as they come to temperature). Transfer steaks to carving board and let rest for 5 minutes.

While steaks rest, microwave reserved sauce until warm, 15 to 30 seconds; stir in cilantro. Slice steaks against grain into 1/2-inch-thick slices. Drizzle with half of sauce; sprinkle with sea salt, if using; and serve, passing remaining sauce separately.

Yield: 4 to 6 servings

Source: America's Test Kitchen Season 15

\newpage
\recipe{Momofuku's Bo Ssam}

\ingredients{\textsc{Pork Butt --}
1 whole bone-in pork butt or picnic ham (8 to 10 pounds);
1 cup white sugar;
1 cup plus 1 tablespoon kosher salt;
7 tablespoons brown sugar;\\
\\
\textsc{Ginger-Scallion Sauce --}
2\half cups thinly sliced scallions, both green and white parts;
\half cup peeled, minced fresh ginger;
\quarter cup neutral oil;
1\half teaspoons light soy sauce;
1 scant teaspoon sherry vinegar;
\half teaspoon kosher salt, or to taste;\\
\\
\textsc{Ssam Sauce --}
2 tablespoons fermented bean-and-chili paste (ssamjang);
1 tablespoon chili paste (kochujang);
\half cup sherry vinegar;
\half cup neutral oil;\\
\\
\textsc{Accompaniments --}
2 cups plain white rice, cooked;
3 heads bibb lettuce, leaves separated, washed and dried;
1 dozen or more fresh oysters (optional);
Kimchi}


Place the pork in a large, shallow bowl. Mix the white sugar and 1 cup of the salt together in another bowl, then rub the mixture all over the meat. Cover it with plastic wrap and place in the refrigerator for at least 6 hours, or overnight.
    
When you're ready to cook, heat oven to 300\deg F. Remove pork from refrigerator and discard any juices. Place the pork in a roasting pan and set in the oven and cook for approximately 6 hours, or until it collapses, yielding easily to the tines of a fork. (After the first hour, baste hourly with pan juices.) At this point, you may remove the meat from the oven and allow it to rest for up to an hour.
    
Meanwhile, make the ginger-scallion sauce. In a large bowl, combine the scallions with the rest of the ingredients. Mix well and taste, adding salt if needed.
    
Make the ssam sauce. In a medium bowl, combine the chili pastes with the vinegar and oil, and mix well.
    
Prepare rice, wash lettuce and, if using, shuck the oysters. Put kimchi and sauces into serving bowls.
    
When your accompaniments are prepared and you are ready to serve the food, turn oven to 500\deg F. In a small bowl, stir together the remaining tablespoon of salt with the brown sugar. Rub this mixture all over the cooked pork. Place in oven for approximately 10 to 15 minutes, or until a dark caramel crust has developed on the meat. Serve hot, with the accompaniments.

Yield: 6 to 10 servings

Source: Adapted by the New York Times from ``Momofuku'' by David Chang and Peter Meehan.



\newpage
\recipe{Shepherd's Pie with Cauliflower Topping}
\equilpment{immersion blender; food processor, 7 quart cast-iron dutch oven; 9 inch by 12 inch baking dish (Le Creuset 40) or 6 5-inch ramekins;}

\ingredients{\\
\\
\textsc{Topping --}
1 medium head cauliflower, cut into large pieces;
2 tablespoons extra-virgin olive oil; 
\threequarters teaspoon salt; 
\quarter teaspoon white pepper;
15 ounce can cannellini beans, drained and rinsed.\\
\\
\textsc{Filling --}
1 large onion, pealed and quartered;
2 cloves garlic, pealed;
1 large fennel bulb, cut into large pieces;
1 6-ounce can tomato paste;
\half cup water;
1 \half teaspoons salt;
\quarter teaspoon black pepper;
1 tablespoon cayenne-based hot sauce;
1 tablespoon Worcestershire sauce;
2 teaspoons Maggi Liquid Seasoning;
1 teaspoon extra-virgin olive oil;
1 \half pounds 90\% lean ground beef;
8 ounces button or cremini mushrooms, sliced.
}

Preheat oven to 375\deg F.

To make topping, place cauliflower in a pot and then add water just to cover. Bring pot to boil over high heat, then reduce the heat to medium, and cook until cauliflower is very tender, about 10 minutes. Drain cauliflower, return it to the pot, and add the olive oil, salt, pepper, and beans.  Pur\'{e}e with an immersion blender until smooth. Set aside until assembly.

To make filling, first make two components that are added during the cooking phase. For the first component, place onion, garlic, and fennel in a food processor and pulse until finely chopped.  For the second component, stir the tomato paste and water together in a small bowl. Add the salt, pepper, hot sauce, Worcestershire sauce, Maggi seasoning and whisk to combine. Now heat olive oil in a large cast-iron dutch oven.  When hot add beef and brown.  When beef is well browned add onion mixture and mushrooms and cook until mushrooms are softened but not mushy.  Add tomato paste mixture and cook for a few more minutes to combine.

To assemble, transfer beef mixture to the baking dish and top with pureed cauliflower mixture.  Bake for 20 to 30 minutes until the casserole is bubbling.

Makes 6 servings. 

Source: Adapted from a recipe in Always Hungry? by David Ludwig







\chapter{Pasta}
\recipe{Lemon-Basil Orzotto}
\ingredients{2 tablespoons extra-virgin olive oil;
1 cup diced onion;
1\half cups orzo or pearl barley;
\half cup dry white wine;
3 cups chicken stock;
\half cup frozen petite green peas;
\third cup grated Parmesan cheese;
2 tablespoons chiffonade fresh basil (or 2 teaspoons pesto);
1 teaspoon lemon zest;
\quarter cup heavy cream;
Juice of 1 lemon;
Salt and freshly ground black pepper;
Optional: 2 boneless skinless chicken thighs cut in a \half inch dice.}

In a heavy-bottomed medium saucepan, heat the olive oil over medium-high heat. 
Add the onion and saut\'{e} until fragrant and translucent. Add the orzo and toast for 
2 minutes, stirring occasionally. Add the wine and cook until absorbed.

Gradually add the chicken stock, stirring frequently. Bring to a simmer, 
lower the heat, and cover. If using the optional chicken add it at this point. 
Cook for 8 to 10 minutes, until the liquid is almost absorbed and orzo is tender. 
Remove from the heat.

Stir in peas, Parmesan, fresh basil, lemon zest, heavy cream, and lemon juice. 
Season the orzo with salt and pepper, to taste, and serve.

\newpage
\recipe{Marinara}

\ingredients{260 grams yellow onion, large dice;
160 grams carrots, medium dice;
18 grams garlic;
20 grams (1\half tablespoons) olive oil;
1 large can (794 grams) Italian crushed tomatoes;
extra-virgin olive oil;
salt.}

Mince the onions, carrots, and garlic in food processor.

Add the olive oil to the base of a pressure cooker, and saut\'{e} the minced vegetables over medium heat until the onions become translucent, about 4 minutes.

Stir the tomatoes into the saut\'{e}ed vegetables.

Pressure-cook the vegetables at 15 psi for 45 minutes. Start timing as soon as full pressure has been reached.

Depressurize cooker, then season the sauce to taste with extra-virgin olive oil and plenty of salt.

From Modernist Cuisine at Home.

\newpage
\recipe{Butter Tomato Sauce}

\ingredients{2 cups pealed, seeded, and roughly chopped plum tomatoes, in addition to enough of their juices to cover or a 28-ounce can of San Marzano whole peeled tomatoes;
5 tablespoons butter;
1 onion, peeled and cut in half;
Salt.}

Combine the tomatoes, their juices, the butter and the onion halves in a saucepan. Add a pinch or two of salt.

Place over medium heat and bring to a simmer. Cook, uncovered, for about 45 minutes. Stir occasionally, mashing any large pieces of tomato with a spoon. Add salt as needed.

Discard the onion then emulsify sauce in pan using an immersion blender. 

Yields: 2 cups -- enough for one pound of pasta and 4 servings

Source: Marcella Hazan via the New York Times

\newpage
\recipe{Lemon-Butter Pasta with Parmesan}

\ingredients{
    8 tablespoons (1 stick) unsalted butter;
    Finely grated zest from 2 large lemons (2 tablespoons);
    Kosher salt;
    1 pound fettuccine;
    1 cup freshly grated Parmesan cheese, plus more for optional garnish;
    Freshly ground black pepper
}

In a Dutch oven or skillet large enough to hold all the pasta, melt the butter over medium heat. Add the lemon zest and swirl to mix. Remove from heat.

Bring a large pot of salted water to a boil, then add the pasta and cook until al dente, about 10 minutes. Drain the pasta, reserving 1\half cups of the pasta water.

Return the butter to medium heat, then whisk in the reserved pasta water until combined, about 2 minutes. Add the Parmesan in a couple of handfuls, whisking until emulsified, 3 to 4 minutes. Season to taste with salt and pepper. Add the pasta to the sauce; toss and stir until the noodles are glossed with sauce, 3 to 5 minutes.

Serve with more black pepper and Parmesan.

Yields: 4 to 6 servings

Source: Ali Slagle from The Washington Post

\chapter{Pizza}
\recipe{Neapolitan Pizza Dough}

\ingredients{500 grams Antico Caputo brand 00 wheat flour;
310 grams water;
10 grams salt;
10 grams agave syrup;
2.5 grams (1\half teaspoons) Bob's Red Mill brand vital wheat gluten;
2.5 grams (\threequarters teaspoon) Active dry yeast;
Neutral oil as needed.}

Mix flour, water, salt, agave syrup, and yeast in the bowl of a stand mixer with a dough hook until fully incorporated.

Mix on medium speed (4 on KitchenAid mixer) for 5 minutes.

Let the dough rest in the bowl for 10 minutes at room temperature, and then mix again for another 5 minutes at medium speed.
Allowing the gluten to relax between kneading creates an especially smooth, stretchy dough.

Transfer the dough to a well-floured surface, and cut into four equal weight chunks. Stretch and roll the dough into smooth, even balls.
Stretching the surface layer develops a network of gluten that traps air inside the ball, resulting in a lighter, more blistered crust.

Coat the balls lightly with oil, cover them with plastic wrap, and allow them to rise at room temperature for 1 hour before using. For a more complex flavor,
refrigerate the dough balls overnight, and then let them rest at room temperature for 1 hour before using.

Yields enough dough for four 12-14 inch pizzas.  The dough keeps for 3 days when refrigerated in plastic wrap or up to 3 months when frozen.

Source: Modernist Cuisine at Home.

\newpage
\recipe{Pizza Sauce}

\ingredients{50 grams (\quarter cup) chopped garlic;
50 grams (\half cup) olive oil;
1 large can (794 grams) Italian crushed tomatoes;
extra-virgin olive oil;
salt.}


Add the olive oil to the base of a pressure cooker and then fry garlic until golden brown, about 5 minutes.

Stir the tomatoes into cooker.

Pressure-cook the vegetables at 15 psi for 45 minutes. Start timing as soon as full pressure has been reached.

Depressurize cooker, then season the sauce to taste with extra-virgin olive oil and plenty of salt.

Source: Modernist Cuisine at Home.

\recipe{Basic Pizza}

\equilpment{pizza steel; pizza peal}

\ingredients{1 ball, approximately 200 grams, Neapolitan Pizza Dough; \half cup Pizza Sauce; 2 ounces fresh mozzarella; 1\third cup shredded low-moister whole-milk mozzarella; 2 tablespoons grated hard cheese; 2 teaspoons Mediterranean oregano, Antico Caputo brand 00 wheat flour, extra virgin olive oil or basil oil}
% Desert

\recipe{Basil Oil}

\equilpment{immersion blender}

\ingredients{\half cup extra virgin olive oil; 10 to 15 large fresh basil leaves, zest and juice from \half lemon, salt, freshly ground black pepper}

Combine all the ingredients except the salt an pepper in the container of the immersion blender.  Blend until completely liquified then season to taste with salt and pepper. Transfer to a squeeze bottle for use. Keeps in the refrigerator for several weeks.

\newpage
\recipe{Paprika Oil}

\ingredients{\third cup extra-virgin olive oil;
1\half teaspoons smoked (hot or sweet) paprika;
\half teaspoon red-pepper flakes.}

Heat olive oil in a small skillet over the lowest heat on your smallest burner. Add the paprika and red-pepper flakes and cook, stirring frequently, just until toasted and flavors bloom, 1 to 2 minutes. Strain oil, discarding flakes.

\chapter{Dessert}


\recipe{Buttermilk Chess Pie}

\ingredients{1/2 cup butter;
2 cup sugar;
2 tablespoons all-purpose flour;
1 teaspoon vanilla;
5 large egg, lightly beaten;
2/3 cup buttermilk;
1 unbaked 9-inch pastry shell}

\textsc{This pie was the winner of the first-ever Oxford, MS Mid-Town
Farmers' Market pie contest. The winning cook, Ms. Izetta
Barrington, submitted this recipe for her pie.}


In mixing bowl combine sugar and flour.
Add eggs and buttermilk, stirring until blended.
Stir in butter and vanilla.

Pour mixture into the pastry shell.
Bake in 350\deg F oven for 45 minutes, or until set.
Cool on a wire rack.


\newpage



\recipe{Chocolate Chess Pie}

\ingredients{\half cup butter;
1 cup sugar;
\quarter cup flour;
1 teaspoon vanilla;
\eighth teaspoon salt;
3 egg yolks;
1\half ounces unsweetened chocolate or 4\half tablespoons cocoa + 1\half tablespoons butter ;
\half cup evaporated milk;
1 pastry pie shell.}

In mixing bowl combine butter and sugar and beat well.

Add flour, vanilla, melted chocolate or cocoa plus additional melted butter, salt, and beat in egg yolks and milk.

Pour mixture in the pie shell and bake in a 425\deg F oven for 10 minutes.  Reduce the temperature to 300\deg F and cook for an additional 20 to 25 minutes.  The pie should be firm in the center so that a tooth pick inserted in the center comes out clean.

Serve cool topped with whipped cream. Can be made without the coco or chocolate for a traditional chess pie.

\newpage
\recipe{Blueberry Pie Filling}

\ingredients{10 to 11 tablespoons minute tapioca;
3 teaspoons ground cinnamon;
1 teaspoon salt;
2\half cups sugar;
8 cups blueberries (if frozen thaw temperature first if fresh, crush);  
8 tablespoons butter, melted;
3 teaspoons vanilla extract.}

Mix together flour, cinnamon, salt and sugar.

Add blueberries and toss lightly.

Stir in melted butter and vanilla, mixing until berries are well coated.

Make in a 9-inch pie pan with a bottom and top crust.

Line bottom  rack of oven with foil to catch drips.

Bake in preheated 450\deg F oven for 15 minutes, then reduce temperature to 375\deg F
and bake 25 to 30 minutes more.

Yields filling for two 9-inch pies.

\newpage
\recipe{Flaky Pie Crust}
\label{FlakyPieCrust}

\ingredients{3 cups unbleached all-purpose flour;
1 teaspoon salt;
\quarter teaspoon baking powder;
1\quarter cups shortening;
1 egg;
4 to 6 tablespoons ice cold water;
1 tablespoon white vinegar.}

Combine dry ingredients and cut in shortening.

In a separate bowl, beat together egg, 4 tablespoons water, and vinegar.

Add egg mixture to flower mixture and blend together well.

Add 1 to 2 more tablespoons of water if needed.

Wrap dough and chill before rolling out.

Divide dough into quarters and roll out on floured surface or between wax paper
to desired size.

Yields four 9-inch crust.

\newpage

\recipe{Pound Cake}

\ingredients{2 sticks butter;
2 cups sugar;
6 whole eggs;
2 cups all-purpose flour;
dash of salt;
2 teaspoons vanilla extract;
1 teaspoon almond extract}

Before beginning, bring all ingredients to room temperature. Preheat oven to 350\deg F. 

Beat the butter in the bowl of a stand mixer. Add the sugar and salt and beat thoroughly. Add \third cup flour and one egg at a time, beating thoroughly after each addition until all the flour and eggs are added.  Add vanilla and almond extracts and mix until blended.

Pour into a Bundt pan. Line the pan with slivered almonds if desired.

Bake for one hour.

Source: Lucie Krueger
\newpage


\recipe{Scandinavian Almond Cake}

\ingredients{1\quarter cup sugar;
1 egg;
1\half teaspoon almond extract;
\twothirds cups milk;
1\quarter cup flour;
\half teaspoon baking powder;
1 stick melted butter.}

\textsc{This recipe uses a special fluted half-cylinder pan.  Mine is aluminum 
and was made in Germany. Due to the ridges in cake pan, slicing is a 'piece of cake'.}

Beat together sugar, egg, extract and milk. Stir in flour and baking powder.
Then add butter and mix well.

Spray cake pan with cooking spray. Sprinkle slivered almonds into bottom of pan, if desired, then  pour batter into a greased pan. 

Bake at 350\deg F for 40 to 50 minutes or till edges are golden brown. 

Cool cake in pan completely before removing. To remove place inverted 
Almond Cake Plate to top of cake pan and then flip. Dust with confectionery 
sugar if desired.



 

\newpage
\recipe{Nanaimo Bars}

\ingredients{\half cup butter;
5 teaspoons cocoa;
2 teaspoons vanilla extract;
1 egg, beaten;
;
2 cups graham cracker crumbs;
\half cup nuts, chopped;
1 cup shredded cocoanut;
;
\half cup butter;
2 Tablespoons milk;
2 teaspoons vanilla extract;
2 cups confectioners sugar;
2 teaspoons vanilla pudding mix;
;
2 teaspoons butter;
4 squares semi sweet chocolate, melted}

Combine first five ingredients in a saucepan, mix and then heat until slightly thick. To this add the graham cracker crumbs, nuts and cocoanut.
Mix well and pack in a pan and chill in refrigerator until cool.

Make a second layer by combining second \half cup of butter, milk, 2 teaspoons of vanilla extract, vanilla pudding mix. Spread on first layer then return to refrigerator until chilled. Melt chocolate and butter together, mix and pour over second layer then chill until firm.

Cut into bars and serve.

\newpage
\recipe{Blueberry Crisp}
\ingredients{\\
\\
\textsc{Filling --} 4 generous cups blueberries, cleaned and stemmed;
\half cup sugar;
\quarter cup unbleached all-purpose flour;
\quarter teaspoon salt;
2 teaspoons lemon juice;\\
\\
\textsc{Topping --} 1\half cups unbleached all-purpose flour;
\quarter teaspoon salt;
\half cup sugar;
1\quarter sticks butter, melted;
1 cup pecans, chopped.}

Grease and flour a 9-inch pie pan. Preheat the oven to 350\deg F.

To make filling, put the berries in the pie pan. Mix the sugar, flour, salt, and lemon juice together, and sprinkle this mixture over the berries. 

For the topping, stir together the flour, salt, sugar, melted butter, and nuts in a medium mixing bowl. Sprinkle the topping over the fruit.

Bake the crisp for 45 to 50 minutes, until the top is golden and the filling is bubbly. Cool slightly, then serve warm, with vanilla ice cream or whipped cream.

Source: The King Arthur Flour Baker's Companion.

\newpage
\recipe{Pear and Almond Tart}
\label{PearTart}
\equilpment{Deep steel tart pan approximately 9" in diameter at the top and 2" deep}
\ingredients{\\
\\
\textsc{Poached Pears --} 3 ripe but firm pears;
100 grams sugar;
2 cups water.\\
\\
\textsc{Tart Pastry --} 1\quarter all purpose flour;
pinch of salt;
\half cup (1 stick) unsalted butter;
1 to 3 Tablespoons ice cold water.\\
\\
\textsc{Almond Filling --} 100 grams unsalted butter;
100 grams almond flour;
100 grams sugar;
1 egg;
1 tablespoon all purpose flour;
1 tablespoon dark rum or kirsch.}

To poach the pears first make a syrup by bring the water and sugar to boil in a saucepan. Peel pears, cut in half, then remove stem and seeds. Poach pears in simmering syrup until cooked through but still firm enough to hold together. Remove pear halves from syrup and allow to cool. Cook down syrup until reduced to a glaze then cool and reserve glaze. 

Preheat the oven to 425\deg F.

To make the tart pastry, place the flour and salt in the bowl of a food processor fitted with the steel blade. Cut in the butter in slices, and buzz several times until the mixture is uniform and resembles coarse meal.  Continue to process in quick spurts as you add water, 1 tablespoon at at time. As soon as the dough adheres to itself when pinched together, stop adding water, turn it out onto a floured surface, and push it together into a ball. 

Butter the tart pan then roll the dough into an 11-inch or so circle. Drape the dough over your rolling pin and the lift the dough into the tart pan, nudging it gently into the corners, and use your hands to form an even edge all the way around. Prick the dough all over with the tines of a fork. Wrap tightly and store in the refrigerator until assembly. 

To make the filling, mix butter, sugar, and almond flour, until the dough is smooth, add the rum, then the beaten egg. Work until the mixture is homogeneous, then add the flour and mix completely.

To assemble tart place poached pears radially around the center of the tart pan with the cut sides down and the stems pointed at the center. Pour the filling evenly around the pears leaving most of the pear exposed. Bake for 45 minutes then remove from oven and pour reserved glaze evenly over tart.  The tart will absorb the glaze after a few minutes so you can continue to pour over glaze until it is all used.  Let cool to room temperature before removing from the pan and serving.

\newpage

\recipe{Mighty Mousse}

\ingredients{1\half cups semisweet chocolate chips;
1 tablespoon rum or cognac;
2 large eggs;
1\half cups of half and half}

Heat half and half in sauce pan until scalding hot.

Combine all ingredients except half and half in a blender and mix briefly.

Add scalding hot half and half and blend thoroughly.

Pour into dishes, cover with plastic wrap or foil and chill until set.

Serve garnished with whipped cream and mint or shaved chocolate.  It is also good garnished with cherries that result from making \hyperref[CherryLiqueur]{Cherry Liqueur}.

Source: Too Busy to Cook column of Bon Appetite magazine

Serves: 6

\chapter{Jellies, Conserves, and Pickles}

\recipe{Sweet Pickles}
\label{SweetPickles}

\ingredients{7 pounds of cucumbers (or yellow squash or green tomatoes) sliced 3/8" thick;
2 gallons water;
1 cup pickling lime (food grade calcium hydroxide);
\half gallon White House vinegar;
6 pounds sugar;
2 tablespoons table salt (not iodized);
1 tablespoon pickling spice;
1 teaspoon whole cloves.}


Place cucumbers in a solution made from the water and pickling lime. 
Soak cucumbers in the solution for 24 hour stirring often.

Drain and rinse cucumbers and then soak for 4 hours in clean water.  
Drain and rinse cucumbers again.

Make a pickling solution by heating the remaining ingredients (do not boil).

Place the cucumbers in the pickling solution and heat to just simmering.  
Let cool for 4 hours then repeat simmering and cooling until pickles are clear.

Bring pickles to a simmer again and then hold at a simmer for 35 minutes.

Seal hot pickles in sterilized jars. Yields 12 pints.

\newpage
\recipe{Pear Brandy Conserve}

\ingredients{3\half cup ground pears;
\half cup slivered almonds;
\half cup seedless golden raisins;
\half teaspoon butter (to reduce foaming);
\quarter teaspoon cinnamon;
\quarter teaspoon grated lemon rind;
2 tablespoons lemon juice;
5 cups sugar;
1 box Sure Jell brand fruit pectin;
\half cup brandy.}


Peal and core pears then grind with a meat grinder or
grate using a food processor.

Combine ground pears, almonds, butter, raisins, lemon
rind, lemon juice and cinnamon in a tall 8-quart saucepan (6
inches tall and 9 inches in diameter).

Measure sugar and set it aside.

Mix Sure Jell into pear mixture then cook on high, stirring
constantly, until it reaches a full rolling boil. A full rolling
boil cannot be stirred down. It may be necessary to add some water
(up to a cup or so) to the mixture.

After the mixture has reached a full rolling boil, add the pre-measured sugar all at once.

While continuing to stir, bring the mixture back to a
rolling boil and then cook at a rolling boil for one additional
minute.

Remove mixture from heat, add brandy and seal in sterilized jars.

Makes seven half pints (7 cups).  Do not attempt to double this
recipe as it will not boil correctly.  If you want to make more
that seven half pints, boil the amount for seven half pints
repeatedly. Be sure to clean spoon and pan of all sugar between
batches.

\newpage
\recipe{Muscadine Hull Sauce}

\ingredients{very ripe muscadines to yield 19 \threequarters pounds of hulls and pulp once the seeds are removed;
18 pounds sugar;
1 quart cider vinegar;
6 tablespoons ground cinnamon;
3 tablespoons ground mace;
3 tablespoons ground cloves.}

Wash and pulp grapes, separating hulls and pulp.  Cook hulls until tender then grind hulls using a meat grinder with the coarse plate.  Cook pulp until tender and seeds start to separate. Completely the separate of the seeds from pulp by running the pulp through a colander to the remove seeds.  Mix ground hulls and deseeded pulp.  Weigh out 19 \threequarters pounds of hulls and pulp mixture.  Mix in vinegar, cinnamon, mace and cloves.

Boil the mixture until it reads 220\deg F on a jelly and candy thermometer.

Seal in sterilized jars. Yields 28 pints.

For preserves leave out vinegar and spices and add water to replace the vinegar.

\newpage
\recipe{Pepper Jelly}

\ingredients{2\half cups chopped bell peppers;
\half cup chopped hot peppers;
2 cups water;
6\half cups sugar;
2 cups apple cider vinegar;
\quarter teaspoon salt;
2 bottles Certo liquid pectin;
green or red liquid food coloring.}

Boil bell peppers and hot peppers until tender then mash the pulp and strain it through a jelly bag and reserve liquid. 

Combine 2 cups of reserved liquid, sugar, vinegar, and salt then boil for 5 minutes.

Take off heat and stir in 2 bottles of liquid pectin and green or red color.  Return to stove and bring to boil.  

Pour quickly into hot sterilized jelly glasses and cover with paraffin.

Yields 10-11 jelly glasses.


\chapter{Breads}
\recipe{Whole Wheat Sandwich Bread}

\equilpment{KitchenAid Mixer; Two 1\quarter quart Le Creuset Stoneware Deep-Dish Loaf Pans.}

\ingredients{6 to 6\half cups whole wheat flour;
2\half cups warm water between 105-110\deg F;
1\half tablespoons instant active dry yeast; 
\third cup honey;
\third cup oil;
2\half teaspoons salt.}

Combine water, yeast and 2 cups of the flour in the mixer's bowl. Set aside to sponge for 15-20 minutes, until risen and bubbly.

Add honey, oil, and  salt to the bowl.  Mix using the flat beater while adding a total of 4 cups of flour in \quarter cup increments. After adding the flour, continue mixing until the dough starts to clean sides of bowl. Change to dough hook and knead 6 to 7 minutes. Add a few tablespoons of flour at a time if dough sticks to sides.

Form into two loaves and place in greased loaf pans. Allow to rise in a warm place for about 60 minutes, until 1/2 to 1-inch above pans. Preheat oven to 350\deg F ten minutes before rising time is done.

Bake for 30 minutes, rotating halfway through if needed. Remove loaves from pans and return to oven for an additional 10 minutes to brown.

Allow  loaves to completely cool on racks before slicing.

Source: Adapted from a recipe on the website ``An Oregon Cottage.''

\newpage
\recipe{Sweet Potato Biscuits}

\ingredients{1 cup sweet potato puree;
2 cups self rising flour;
4 tablespoons of vegetable shortening or lard;
buttermilk.}

Preheat oven to 425\deg F.

Mix together sweet potato puree, flour, and shortening then add just enough buttermilk to make the right consistency for a soft dough.

Turn out dough and knead for about 30 seconds.

Roll out dough to the desired thickness on a lightly floured cloth or board.

Cut with a floured biscuit cutter and place on a greased baking sheet.

Cook until brown and done.  Makes around 10 biscuits.

\newpage
\recipe{Southern Cornbread Sticks}

\ingredients{\\
\textsc{Self-Rising White Cornmeal Mix} 
1\half cups regular white cornmeal;
6 Tablespoons all-purpose flour;
2 tablespoons baking powder
1 teaspoon granulated salt\\
\\
\textsc{Cornbread Batter}
1 egg, beaten;
1\threequarters cups buttermilk;
\quarter cup melted Crisco shortening;
2 cups Self-Rising White Cornmeal Mix}


Preheat oven to 450\deg F. Grease two cast iron cornstick pans very well with room 
temperature Crisco and place in the oven to heat.

In a large bowl, add cornmeal mix ingredients and combine. Instead of homemade cornmeal mix you can substitute 2 cups of Martha Whites Self-Rising White Cornmeal Mix.

Combine egg, buttermilk and corn meal mix and mix well.  
To this add the melted shortening and mix well again.

When the shortening in pans is beginning to smoke, remove pans from oven 
and fill each mold space \twothirds full.

Return pans to oven and bake cornsticks for 15 to 20 minutes
until they are golden brown.

Recipe will work with a cast iron muffin pan using the same instructions 
yielding 12 muffins.  The recipe will also work with a 8 or 9 inch cast iron 
skillet, however the baking time is then 20 to 25 minutes.

Yields 16 cornsticks.
\newpage

\recipe{Hamburger Buns}

\label{HamburgerBuns}

\ingredients{\threequarters cup lukewarm water;
2 tablespoons butter;
1 large egg;
3 \half cups King Arthur Unbleached All-Purpose Flour;
\quarter cup sugar;
1\quarter teaspoons salt;
1 tablespoon instant yeast}

In a heavy duty mixer with the dough hook attached, mix and then knead all of the dough ingredients to make a soft, smooth dough.

Cover the dough, and let it rise for 1 to 2 hours, or until it's nearly doubled in bulk.

Gently deflate the dough, and divide it into 8 pieces. Shape each piece into a round ball; flatten to about 3 inches across. Place the buns on a silicon baking mat in a half sheet pan, cover, and let rise for about an hour, until noticeably puffy.

Bake the buns in a preheated 375\deg F oven for 15 to 18 minutes, until golden.
Cool the buns on a rack.

Brushing buns with melted butter will give them a soft, light golden crust. Brushing with an egg-white wash (1 egg white beaten with 1/4 cup water) will give them a shinier, darker crust. For seeded buns, brush with the egg wash; it'll make the seeds adhere. And, feel free to add the extra yolk to the dough, reserving the white for the wash.

Yield: 8 large buns. 


\chapter{Breakfast}

\recipe{Banana Waffles}

\ingredients{2 cups sifted flour;
3 teaspoons baking powder;
1 teaspoon salt;
2 eggs;
1\quarter cups buttermilk;
\half cup melted butter;
1 cup mashed very ripe bananas.}

These are best with bananas that have become over ripe and then put in the freezer for a few days. Defrost bananas before using.

Preheat oven to 200\deg F. Preheat waffle iron. In the bowl of a stand mixer, sift together flour, baking powder, and salt. 

Beat eggs slightly and add buttermilk. Add egg mixture to flour mixture in bowl of stand mixer and mix well.

Add melted butter and well-mashed bananas and mix in completely.

Add 1 cup of batter to iron for Belgian waffles. Close iron and then cook until brown and crisp. Place cooked waffles on oven grates to further crisp and keep warm while making the remaining waffles.

Makes 5 waffles. 

\newpage
\recipe{Granola} 

\equilpment{2 half sheet-pans; wide heavy-duty aluminum-foil; heavy meat-pounder.}

\ingredients{\twothirds cup cane or maple syrup;
7 fluid ounces vegetable oil;
2 tablespoons vanilla extract;
1 teaspoons salt;
1 32-ounce bag regular rolled oats;
1 12-ounce bag chopped pecans or almonds.}


Preheat oven to 350\deg F. In a large bowl mix together the first five ingredients.  
To this add the oats and nuts and mix thoroughly. 

Line two half sheet-pans with  foil, distribute the mixture evenly between the pans, and then force the mixture into a compact layer using a 
large flat pressing tool such as a meat-pounder. Cook for 40 minutes reordering and turning the sheet pans once.

Cool the mixture for one hour then break into large pieces. At this point you can
add 3 cups of raisins or other dried fruit if desired. 

\newpage
\recipe{Yogurt}
\ingredients{5 cups 1 percent milk; \third cup nonfat dry milk; one package of yogurt starter or \third cup yogurt reserved from previous batch.}

Starting with milk from the refrigerator, microwave the milk in an 8 cup glass measuring cup for 10 to 12 minutes until it is 180\deg F. 
Remove from microwave and whisk in dry milk.

Cover the measuring cup with a dishtowel and leave to cool at room temperature for around one hour until the milk mixture cools to between 110 and 100\deg F. Whisk in starter or reserved yogurt. Place in yogurt maker and process for 8 hours. At the end of 8 hours transfer yogurt to storage container and refrigerate. 

\newpage

\recipe{Black Bean and Tofu Scramble - To Include}
\textsc{To include}



\chapter{ From New Mexico}

\recipe{Green Chile Sauce}
\ingredients{\quarter cup olive oil;
1 clove garlic, minced;
\half cup onion, minced;
1 tablespoon all purpose flour;
1 cup pork stock, chicken stock, or water;
1 cup diced green chile (2 4-ounce cans if fresh or frozen is not available);
salt.}

Saut\'{e} garlic and onions in olive oil using a heavy saucepan.  When onions and garlic are translucent but not browned, blend in flour and stir with a wooden spoon. Cook flour long enough to remove the raw taste but do not brown.  When flour is cooked add stock or water and diced green chile to the pan. Bring to boil and then reduce to a simmer. Stir frequently until sauce is thickened to your satisfaction, about 5 minutes. Add salt to taste.

\recipe{Red Chile Sauce}

\ingredients{4 New Mexican red chile pods;
3 tablespoons olive oil;
1 clove garlic, minced;
2 tablespoons all purpose flour;
2 cups pork stock, chicken stock, or water; salt.}

Begin by making red chile powder from the chile pods. Wash the pods to remove any dust or dirt from the exterior then tear open and remove the stem, seeds and veins. Tear the remain flesh into large pieces and toast in a heavy skillet.  Be careful not to burn the chile. Place toasted chile pieces in a spice grinder or blender and grind to a powder. Measure \half cup of powder to use in this recipe and save the rest for another use.

Saut\'{e} the garlic in olive oil for a minute or so then blend in flour with a  wooden spoon. Cook flour long enough to remove the raw taste but do not brown. When flour is cooked add the reserved \half cup of chili powder and blend in.  Blend in stock or water and cook to desired consistency. Add salt to taste.

\newpage
\recipe{Steaks with Green Chilies and Mushrooms}
\ingredients{1 medium onion finely chopped;
2 tablespoon olive oil;
1 cup roasted, peeled, and chopped fresh New Mexico green chilies;
\quarter teaspoon dried oregano;
\quarter teaspoon salt;
1 teaspoon hot sauce;
4 large fresh mushrooms sliced thin;
4 tablespoon butter;
2 steaks (as large as 14- to 15-ounce);
wood chips for smoking on the grill.}

The first 6 ingredients are for the green chili sauce.  To make the sauce, first saut\'{e} the onions in the olive oil.  When the onions are soft, add the remaining sauce ingredients and cook for 5 minutes.  Keep the sauce warm in oven set to low.

Saut\'{e} the mushrooms in butter until soft.  Remove from pan and keep warm with sauce.

Grill steaks using wood chips.

Transfer steaks to plates, divide mushrooms over the top and cover with green chili sauce.

From the Pink Adobe restaurant in Santa Fe, New Mexico.

\newpage

\recipe{Green Chili Stew - To Include}
\textsc{To include}

\newpage
\recipe{Sopaipillas}
\equilpment{R \& V Works Cajun Fryer Model FF1.}
\ingredients{2 cups all-purpose flour;
2 teaspoons baking powder;
1 teaspoon salt;
2 tablespoons shortening or lard;
\half cup water;
peanut oil for frying.}
\textsc{Serve these hot from the fryer.  To eat, tear off one end of the sopaipilla and pour in  honey.}

Sift dry ingredients together then work in shortening and lukewarm water to make a soft dough. Place dough in refrigerator until chilled through. Roll out dough on a floured surface to about \quarter inch thickness and then cut into 3 inch squares. Deep fry at 400\deg F a few at the time until puffed being sure to brown on both sides.  Remove from oil and drain on paper towels.

\newpage
\recipe{Flour Tortillas - Incomplete}
\ingredients{2 cups all-purpose flour;
1 teaspoon salt;
\half teaspoons baking powder;
3 tablespoons shortening;
\half cup warm water.}

\chapter{Condiments and Components}

\recipe{My Mother's Mayonnaise}

\ingredients{1 whole egg at room temperature;
1 teaspoon salt;
1 teaspoon sugar;
2 tablespoons lemon juice (juice from one large lemon);
1 tablespoon yellow ball park mustard;
Dash of paprika;
Dash of red pepper;
2 cups corn oil}

Install the balloon whisk on a stand mixer or put the steel blade in a food processor. Add all the ingredients except the oil to the food processor container or the bowl of the stand mixer. In mixer at high speed or in the food processor beat the mixture until it is very thick and creamy.

With mixer or processor running, add the oil very slowly at first until the mayonnaise begins to emulsify and then add in a slow stream until all the oil is incorporated.

\newpage
\recipe{My Mayonnaise}
\label{MyMayonnaise}

\ingredients{1 whole egg at room temperature;
1 teaspoon salt;
3 tablespoons Pinot Grigio wine vinegar;
3 tablespoon Dijon mustard;
2 cups canola oil}

Install the balloon whisk on a stand mixer or put the steel blade in a food processor. Add all the ingredients except the oil to the food processor container or the bowl of the stand mixer. In mixer at high speed or in the food processor beat the mixture until it is very thick and creamy.

With mixer or processor running, add the oil very slowly at first until the mayonnaise begins to emulsify and then add in a slow stream until all the oil is incorporated.

\newpage
\recipe{Olive Salad}

\ingredients{4 cups undrained pimento stuffed green salad olives, slightly
crushed and well drained, 4 cups is approximately one and one half
14 oz. drained weight jars of salad olives;
1 cup pickled cauliflower, drained and sliced, approximately one half of a
16 oz. jar of Kroger hot pickled cauliflower;
2 small jars capers, drained;
\quarter  stalk celery, sliced diagonally;
1 large carrot, peeled and thinly sliced diagonally;
2 tablespoons celery seeds;
2 tablespoons dried oregano;
\quarter large head fresh garlic, peeled and minced;
\quarter teaspoon freshly ground black pepper;
\quarter of a 12 fluid oz. jar pepperoncini yielding 2 oz. drained weight;
drained (small salad peppers) left whole or sliced  to the
size of the cauliflower;
\quarter pound pitted kalamata olives, crushed to size of green olives;
\quarter of a 16 fluid oz. jar cocktail onions, drained and sliced in half;
\threehalves cups canola oil;
1\half cups extra virgin olive oil}

Combine all ingredients except the oils in a large bowl and mix well. Place
in a large jar and cover the extra virgin olive oil
and the canola oil. 

Store tightly covered in
refrigerator. Allow to marinate for at least 24 hours before
using.

Yields 8 cups.

Based on recipe in ``Cookery N'Orleans Style'' by Chiqui Collier.
Modified to reduce yield and to make ingredient amounts
more specific.


\chapter{Barbecue}
\recipe{Auburn Barbecue Sauce}

\ingredients{\half pound butter;
1 cup cider vinegar;
2 cups tomato catsup;
4 tablespoons worcestershire sauce;
1 tablespoon hot sauce;
1 tablespoon salt;
3 tablespoons prepared mustard;
Dash of red pepper;
juice of 1 lemon.}

In a saucepan melt butter over medium heat. When butter is melted, add all of the remaining ingredients and stir to blend well.

Bring to a boil, lower heat, and let simmer for a few minutes. Makes around 4 cups.

\newpage
\recipe{Arkansas Country Barbecue Sauce}

\ingredients{1 cup ketchup;
\half cup cider vinegar;
\quarter cup honey;
3 tablespoons freshly squeezed lemon juice;
1 tablespoon prepared mustard;
1 tablespoon Worcestershire sauce;
2 teaspoons soy sauce;
1 tablespoon lemon pepper;
1 tablespoon garlic powder;
1 teaspoon cayenne;
1 teaspoon salt;
1 teaspoon Mrs. Dash (optional);
1 teaspoon ground cumin;
\half teaspoon dry mustard;
\twothirds cup bourbon}

In a large saucepan, compose the ketchup, vinegar, honey, lemon juice, prepared mustard, Worcestershire sauce, and soy sauce. Bring to a simmer over medium-high heat.

In a small mixing bowl, compose the lemon pepper, garlic powder, cayenne, salt, Mrs. Dash, cumin and dry mustard. Stir well to thoroughly combine.

Whisk dry ingredients into sauce and then simmer for 20 minutes.

Add bourbon and stir well. Reduce the heat to low, cover, and cook for 1 hour until flavors are blended and sauce is thick.

Best used immediately but can be cooled to room temperature and them stored in the refrigerator for a week or two.  Makes around 3 cups.

Source: John Willingham from John Willingham's World Champion Bar-B-Q.
\newpage
\recipe{Memphis Mop BBQ Sauce}



\ingredients{2 cups ketchup;
\half cup prepared yellow mustard;
\half cup packed light brown sugar;
\half cup water;
\quarter cup cider vinegar;
3 tablespoons Worcestershire sauce;
1 tablespoon onion powder;
1 tablespoon chili powder;
1\half teaspoons freshly ground black pepper;
2 teaspoons granulated garlic;
\half teaspoon celery salt;
\half teaspoon salt;
1 tablespoon natural hickory liquid smoke.}

In a medium-sized saucepan, combine all the ingredients except the liquid smoke. Bring it to a gentle boil over medium heat, stirring to dissolve the sugar. Lower the heat to low and simmer until it's slightly thickened, 20 to 25 minutes, stirring occasionally.

With a whisk, blend in the liquid smoke until it's incorporated.

Let the sauce cool, transfer it to a jar and store it in the refrigerator for up to a month.

Makes about 3 cups of sauce.

Source: ``Award-Winning BBQ Sauces and How to Use The'' by Ray Sheehan, Page Street Publishing Co.

\newpage
\recipe{Bar-B-Que Rub For Pork and Beef}

\ingredients{1 cup salt;
2 tablespoons + 2 teaspoons freshly ground black pepper;
2 tablespoons + 2 teaspoons lemon pepper;
2 tablespoons + 2 teaspoons cayenne pepper;
2 tablespoons + 2 teaspoons chili powder;
2 tablespoons + 2 teaspoons dry mustard;
2 tablespoons + 2 teaspoons dark or light brown sugar;
1 tablespoons + 1 teaspoons garlic powder;
\half teaspoon ground cinnamon;
\half teaspoon Accent (optional).}

Combine all ingredients in a bowl and stir well to mix.

Use immediately or store in a glass jar in a cool dark place for several months. Yields 2 cups of rub.

Source: John Willingham from John Willingham's World Champion Bar-B-Q.

 


\chapter{Snacks and Party Food}

\recipe{Tamari Roasted Almonds}
\label{TamariRoastedNuts}

\ingredients{4 cups raw almonds;
scant \quarter cup tamari;
1 tablespoon toasted sesame oil}

Preheat oven to 300\deg F.  Place almonds in a single layer on a foil lined half-sheet pan and put in oven for 15 minutes. Stir almonds and continue roasting for another 15 minutes. Continue roasting for around 15 more minutes, checking every so often to be sure they do not burn. To check for doneness cut a nut in half. When done the nuts they should be golden to golden brown inside.  When done, turn off the oven and place the roasted nuts in a bowl. Pour the tamari over the nuts and stir. The nuts should sizzle/steam as you stir. Continue stirring until the liquid has evaporated then add the sesame oil. Return the nuts to the sheet pan and dry in the turned-off oven for around 10 minutes.  When dry store at room temperature. 

\newpage
\recipe{Hummus}
\label{Hummus}
\equilpment{14 cup heavy-duty food processor; 6 quart Dutch oven}
\ingredients{1 pound dried chickpeas;
2 teaspoons baking soda;
12 cups (3 quarts) water;
16 ounce jar Soom tahini;
8 tablespoons (4 ounces or \half cup) lemon juice;
8 cloves garlic, pealed;
1 tablespoon salt;
6 to 7 ounces ice-cold water}

The night before, put the chickpeas in a large bowl and cover them with cold water at least twice their volume. Leave to soak overnight.

The next day, drain the chickpeas. Place a Dutch oven over high heat and add the drained chickpeas and baking soda. Cook for about 3 minutes, stirring constantly. Add the water and bring to a boil. Cook at a rolling boil, skimming off any foam and any skins that float to the surface. The chickpeas will need to cook between 20 and 40 minutes, depending on the type and freshness. Once done, they should be very tender, breaking up easily when pressed between your thumb and finger, almost but not quite mushy.

Drain the chickpeas. Place the chickpeas in a food processor and process until you get a stiff paste. Then with the machine still running, add the tahini paste, lemon juice, garlic, and salt. Finally, slowly drizzle in the iced water and allow it to mix for about 5 minutes, until you get a very smooth and creamy paste.

Refrigerate or freeze in 1 or 2 cup portions. Cover the surface of the hummus with plastic wrap to avoid the formation of a skin. Defrost frozen hummus in the refrigerator overnight. Allow refrigerated hummus to come to room temperature before serving by leaving is on the counter for 30 minutes.

Makes approximately 8 cups.

Source: ``Jerusalem a Cookbook'' by Yotam Ottolenghi and Sami Tamimi.  Modified to use 1 pound of chickpeas and a whole 16 ounce jar of Soom tahini.

\newpage
\recipe{Pickled Eggs}
\textsc{My mother, Mallette Goggans, often made these eggs for parties.  Her recipe uses 15 dozen peewee eggs and a gallon of vinegar. I have scaled her recipe down to one dozen medium eggs so that it fits in a single one-quart canning jar. The beet in the recipe is just for color and can be omitted. When placing the eggs in the jar try to stand then up and place onions around then so that they do not touch the side of the jar.  That way the eggs will be uniformly colored and pickled. Pickled eggs may be kept several months in the pickling solution in the refrigerator.  The  eggs can be drained and served as is or sliced in half and used to make deviled eggs. }

\ingredients{12 medium eggs, hard cooked and pealed;
2 cups cider vinegar;
2 tablespoons sugar;
1 teaspoon table salt;
4 peppercorns;
3 whole cloves;
1 teaspoon  whole celery seeds;
1 medium red beet, grated (optional);
1 small onion, sliced in rings;
2 cloves garlic, thinly sliced;
1 teaspoon caraway seeds;
sprig or two of fresh dill.
}

Combine vinegar, sugar, salt peppercorns, cloves, celery seeds and grated beet in a sauce pan and bring to a simmer.  Simmer picking liquid for around 5 minutes then strain reserving liquid.

While pickling liquid is simmering, layer eggs, onions, garlic and dill in a one-quart wide-mouth canning jar.  Sprinkle caraway seeds over eggs in jar then pour strained liquid over eggs, onions, and garlic until covered. Place lid and ring on jar and tighten.  Place jar in the refrigerator  and marinate for at least 2 days although 2 weeks is better.

\chapter{ Charcuterie}
\recipe{Sausage and Chicken Liver P\^{a}t\'{e}}

\ingredients{1 pound raw thin sliced bacon strips;
1 pound chicken liver;
1 pound ground pork breakfast sausage;
2 large eggs;
2 or 3 cloves of garlic;
1 medium onion diced;
1 tablespoon diced parsley;
1 tablespoon brandy or cognac (Optional or use 3 tablespoons sherry);
1 teaspoon thyme;
1 teaspoon salt;
\eighth teaspoon black pepper;
\eighth teaspoon rosemary;
3 or 4 whole bay leaves for top of p\^{a}t\'{e}.}

Put a parchment paper sling in a 6 to 8 cup p\^{a}t\'{e}  mold (a bread pan works fine) and then line the mold with the
raw bacon strips letting the ends flop over the sides of the mold.  Use whole strips of bacon across the short side of the mold just overlapping the strip.  On the ends of the long side of the mold use bacon strip cut in half.  The sling is used to as an aid when unmolding the p\^{a}t\'{e} and is particularly important if a glass mold is used since in this case the mold can not be slammed against the platter without risk of breaking the glass.

Place all of the remaining ingredients except for the bay
leaves in a food processor or blender and process until liquefied.

Pour the liver mixture into the p\^{a}t\'{e} mold, top with the bay
leaves, and fold the end of the bacon over the mixture.

Cover the mold with foil and bake at 225 \deg F for 3 hours.  The internal temperature should be greater than 160 \deg F.

Remove from oven and pour off the excess fat from the p\^{a}t\'{e}.  As long as the specified internal temperature is reached, you should not be concerned if the excess fat you pour off is a little pink. Place a brick or
other heavy object on top of the foil to press the p\^{a}t\'{e} and leave
in the refrigerator overnight.

Unmold p\^{a}t\'{e} onto a platter
and garnish with greenery or strips of roasted red and/or yellow
bell pepper.  Serve with slices of \hyperref[SweetPickles]{sweet pickle} and Dijon mustard
on French bread.

\newpage
\recipe{Bratwurst}
\label{Bratwurst}
\equilpment{Microplane Classic Zester Grater; KitchenAid Mixer; LEM \#5 Electric Meat Grinder; LEM Mighty Bite Sausage Stuffer.}

\ingredients{2 kilogram pork shoulder, cut into rough 1-inch chunks, 30 grams kosher salt, 2 tablespoons garlic grated on a Microplane, 1 tablespoon freshly grated nutmeg, 1 teaspoon ground ginger, 2 teaspoons freshly ground black pepper, 1 cup sour cream, 4 3-foot sections of natural hog casings for sausage (e.g. Natural Hog Casings for Sausage by Oversea Casing). }

Combine the pork chunks, salt, grated garlic, nutmeg, ginger, black pepper and sour cream  in a large bowl and toss with clean hands until homogeneous. Transfer to a gallon-sized zipper-lock bag and refrigerate for 24 hours.

Rinse salt off casings and place in a bowl with cold water to soak.  Casings should soak for around 30 minutes.

Place bag with pork mixture in freezer for around 30 minutes then ground mixture using a 4.5 mm plate. Grind directly into the bowl of a KitchenAid mixer. Feed a piece of stale bread through the grinder to force out any remaining bits of sausage. 

Add 4 ounces of water to the bowl. With the mixer's paddle attachment, mix the sausage on medium-low speed until homogeneous and tacky, about 2 minutes. Load sausage into stuffer. 

Place open end of casing on the end of the kitchen faucet and inflate with water until fully inflated. Drain and place on medium stuffer tube. Tie a knot in the end of the casing and evenly fill with sausage. Stop before the end of the casing and tie off with another knot.  Divide into links and twist to separate. 

Makes around 16 8-inch links.

Source: ``The Food Lab'' by J. Kenji L\'{o}pez-Alt.

\newpage
\recipe{Texas German Sausage}
\label{TexasSausage}
\equilpment{KitchenAid Mixer; LEM \#5 Electric Meat Grinder; LEM Mighty Bite Sausage Stuffer; KBQ C-60 BBQ Smoker.}

\ingredients{1.7 kilograms beef brisket;
300 grams pork shoulder;
26.5 grams Kosher salt;
1 gram ground cayenne pepper;
15 grams fresh and coarsely ground black pepper;
3.5 grams cure \#1;
4 3-foot sections of natural hog casings for sausage (e.g. Natural Hog Casings for Sausage by Oversea Casing);
180 grams cold water;
28 grams dry milk powder;
 split oak logs for smoking.}

\textsc{This is my attempt to produce East Texas style German sausage like the Original Smoked Sausage made by Kreuz Market in Lockhart, Texas. According to their web site, Kreuz sausage is 85\% beef and 15\% pork, smoked using post oak, and seasoned using only salt, pepper and cayenne. This recipe has the same percentages of beef and pork as Kreuz sausage, is smoked using oak, and is seasoned with salt, black pepper and cayenne.}

Trim the brisket and port shoulder so the meat to be made into sausage is around 70\% lean and 30\% fat. Weigh out the required portions of brisket and pork shoulder and then
cut the brisket and pork shoulder into one-inch cubes. 

In a large metal or glass bowl, combine the meat cubes with the salt, cayenne pepper, black pepper, and cure \#1.

Mix thoroughly with gloved hands, making sure the seasoning is evenly distributed throughout the meat. Transfer to a two-gallon zipper-lock bag and refrigerate for 24 hours.

Rinse salt off casings and place in a bowl with cold water to soak.  Casings should soak for around 30 minutes.

Place meat grinder head and bag with beef and pork mixture in freezer for around 30 minutes then grind mixture using a 10 mm plate. Grind into a large metal bowl. Feed a piece of stale bread or a sheet of paper towel through the grinder to force out any remaining bits of sausage. Return meat grinder head and ground meat mixture to freezer for 15 minutes then grind the meat mixture a second time with the 4.5 mm plate.

Add half the water and half the milk powder to the mixer bowl containing half the ground meat. With the mixer's paddle attachment, mix the sausage on medium-low speed until homogeneous and tacky, about 2 minutes. Repeat with the remaining water, milk powder, and ground meat. Load sausage into stuffer. 

Place open end of casing on the end of the kitchen faucet and inflate with water until fully inflated. Drain and place on medium stuffer tube. Tie a knot in the end of the casing and evenly fill with sausage. Stop before the end of the casing and tie off with another knot.  Divide the 3-foot sections into 6 links and twist to separate. After twisting there should be approximately \threequarters inch between links. 


Arrange the sausage links on a half-sheet pan then place the pan in refrigerator on the bottom shelf to chill before smoking. If it is convenient you can leave the sausages in the refrigerator overnight and smoke them the next day. Pull the sausages out of the refrigerator an hour before smoking in order to allow the casings to dry at room temperature.

Start a fire in the KBQ C-60 smoker and preheat to 160\deg F. Half open the upper and lower smoke selector outlets on the  smoker.  When the temperature in the smoker is stable, place the sausage links on racks around the middle of the smoker. Cook until  the sausages reach an internal temperature of 154\deg F.  The sausages will take about 3 hours to reach the desired temperature.
   
Once the sausages are out of the smoker, it is important to cool them down as quickly as possible to avoid shriveled skins. Do this by placing the links in a cold water bath until they are room temperature or slightly cooler.


Blooming, or allowing the sausages to develop flavor at room temperature, is an important final step before enjoying your sausages. Hang them on a rack suspended between the backs of two chairs or any setup where they will have plenty of air exposure. The smoky flavor will continue to develop and spread throughout the sausage for a few hours. 

After blooming, you should store the finished sausage in a refrigerator where it will keep for 3-4 days. If you vacuum seal the sausages, they will last in a freezer for up to nine months.

Makes around 2 kilograms of links.

To serve the sausages, preheat oven to 275\deg F, place the sausages in oven and heat for 10 to 20 minutes. Or, even better, brown on a grill over lump charcoal. By the time the sausages are browned all over they will be heated through.

Source: Modified from a recipe found at https://blog.cavetools.com/texas-sausage/


\newpage
\recipe{Stove Top Cooked Link Sausage }
 
 \ingredients{link sausage; water to cover; butter or oil}
 
 Place link sausages in shallow pan or skillet and cover with water.  Place on cooktop over medium heat and cook until water comes to a bare simmer then remove from heat.  Let sausage poach until they reach 140\deg to 150\deg F in the center. 
 
 Remove sausage, pour off water and wipe dry pan or skillet.  Add butter or oil and then brown links on two sides.  Browning should only take a few minutes.  Remove sausage and let links rest for around 5 minutes.  The final internal temperature should be around 160\deg F.

Source: ``The Food Lab'' by J. Kenji L\'{o}pez-Alt.

\recipe{Grinder Plate Sizes}
\textsc{This is not a recipe but rather a list of meat grinder plate hole sizes giving the hole diameter in mm and the approximate equivalent diameter as a rational number in inches.}

$$ \SI{3}{\milli\meter} \approx \SI[fraction-function=\sfrac]{1/8}{inch}$$
$$ \SI{4.5}{\milli\meter} \approx \SI[fraction-function=\sfrac]{3/16}{inch}$$
$$ \SI{6}{\milli\meter} \approx \SI[fraction-function=\sfrac]{1/4}{inch}$$
$$ \SI{10}{\milli\meter} \approx \SI[fraction-function=\sfrac]{3/8}{inch}$$
$$ \SI{16}{\milli\meter} \approx \SI[fraction-function=\sfrac]{5/8}{inch}$$
\chapter{Thanksgiving}

\recipe{Cranberry Orange Relish}

\ingredients{1 medium orange;
12 oz. package of fresh cranberries;
1 cup sugar;
1 cinnamon stick,}


Cut orange into wedges and remove seeds.

Pour \half of the cranberries and orange in blender or food processor container.
Cover; blend until chopped.  Repeat step.

Cook cranberry mixture, sugar and cinnamon stick, over medium heat, 
in large saucepan 5 to 10 minutes or until heated through.  
Remove cinnamon stick.  

Serve warm or chilled.

\newpage
\recipe{Good Eats Roast Turkey}

\ingredients{14 to 16 pound fresh turkey; 2 cups kosher salt; 2 gallons \hyperref[VegetableBroth]{vegetable broth};
1 cup light brown sugar;
2 tablespoons black peppercorns;
3 tsp. allspice berries; 3 tsp. chopped candied ginger; 2 gallons heavily iced water;
1 red apple, sliced; \half onion, sliced; 1 cup water; 4 sprigs rosemary; 6 leaves sage; Canola oil
}
\textsc{This version of the recipe has ingredient amounts to make twice the amount of brine called for in the original recipe.  You may be able to use half this amount of brine if you brine your turkey in a small cooler that tightly fits your turkey instead of using a 5-gallon plastic bucket.}

To make the brine, combine the vegetable stock, salt, brown sugar, peppercorns, allspice berries, and candied ginger in a large stockpot over medium-high heat. Stir occasionally to dissolve solids and bring to a boil. Then remove the brine from the heat, cool to room temperature, and refrigerate.  

Combine the brine, water and ice in the 5-gallon food-safe plastic bucket (a brewing pail for example). Place the turkey (with innards removed) breast side down in brine. If necessary, weigh down the bird to ensure it is fully immersed, cover, and refrigerate or set in cool area for 8 to 16 hours, turning the bird once half way through brining.

Set the oven rack to it lowest level and preheat the oven to 500 \deg F. Remove the bird from brine and rinse inside and out with cold water. Discard the brine. 

Place the bird on wire rack inside a half sheet pan and pat dry with paper towels.


Combine the apple, onion, cinnamon stick, and 1 cup of water in a microwave safe dish and microwave on high for 5 minutes. Add steeped aromatics to the turkey's cavity along with the rosemary and sage. Tuck the wings underneath the bird and coat the skin liberally with canola oil.  

Fold an 18'' foil square in half along the diagonal to form a right isosceles triangle. Coat one flat side of the triangle with Canola oil and then form to the breast of the turkey with the 90\deg angle pointed to the front of the bird and the 45\deg angles pointing at the wings.  Remove the foil triangle without bending it and set aside.


Put the turkey in the oven with legs first if it will fit this way. Roast the turkey on lowest level of the oven at 500 \deg F for 30 minutes. Reduce the oven temperature to 350 \deg F. Insert a probe thermometer into thickest part of the breast and place the foil triangle on the breast.  Set the thermometer alarm to 161 \deg F. A 14 to 16 pound bird should require a total of 2 to 2 \half hours of roasting. Let the turkey rest, loosely covered with foil for at least 15 minutes before carving. Alternatively, the cooked turkey can be held for up to an hour in a large closed cooler.  Place the turkey on a serving plater and then place the plater in the cooler. Cover the turkey with foil before closing the cooler.

\newpage
\recipe{Corn Bread Oyster Dressing}

\ingredients{12 tbsp. butter;
1 large yellow onion finely chopped;
4 celery stalks finely chopped;
1/4 cup chopped fresh parsley;
1 tsp. dried sage;
1/2 tsp. dried tarragon;
1 recipe of corn bread made with Martha White self-rising corn 
              meal mix (white corn), crumbled;
2 eggs;
Pinch cayenne;
Salt and freshly ground black pepper;
Crisco;
Chicken Stock (less than 1 cup needed);
1 pt. to 1.5 pt. shucked Gulf oysters (3 dozen small shucked oysters).}

Make corn bread in a cast iron skillet using melted Crisco in place of oil.
Grease skillet generously with Crisco and preheat in oven before adding 
batter so that the corn bread develops a thick brown crisp crust. 
Cool corn bread and then crumble. Two cups of corn meal mix made into
corn bread should yield around 6 cups of crumbled corn bread.

Preheat oven to 350 \deg F.  Melt 6 tbsp. of butter in a skillet over medium heat, 
add onions, celery, parsley, sage, tarragon, and cayenne.  Sweat the mixture 
until the vegetables are soft, about 20 minutes, then cool.

Drain and reserve the oyster liquid.  If using large oysters cut each oyster 
into 2 or 3 pieces.  Measure the oyster liquid and add enough chicken stock 
(or water) to bring the total to 1 cup.  Transfer liquid to a small sauce pan, 
add the remaining 6 tbsp. of butter, then bring to a simmer, stirring until 
butter has melted.

In a large mixing bowl, lightly beat 2 eggs then add the crumbled corn bread
and mix to combine.  Add vegetable mixture, season with salt (around 1 tbsp. 
if you use unsalted butter) and pepper (1 tsp. or more to taste) 
then mix to combine.  Pour the liquid/butter mixture over the corn bread 
mixture and combine by working with hands.  The mixture should be
uniformly moist but not wet. Add additional chicken stock if mixture is too dry.
Finally, gently mix oysters into corn bread mixture, taking care not 
to break up the oysters.

Transfer mixture to a greased approximately 64 sq. in. cast iron baking dish
or tall cast iron skillet.  Bake for approximately 40 minutes until firmly set 
and browned on top.  Remove from oven and cover with foil to keep 
warn until ready to serve.

Yields 8 generous servings.
Notes

I made a double recipe for Thanksgiving 2012.  
For the double recipe I used my blue Le Creuset baking pan both to make the 
corn bread and to make the dressing.  For the double recipe I increased the 
cooking time to 60 minutes and finished browning the top using the broiler 
on high for a minute or two.

I made a single recipe for Thanksgiving 2014. I used my black high sided Le Creuset baking pan. In this pan I cooked the recipe for 55 minute and then browned the top under the broiler on high.  I used 4 or 5 dozen large oysters cut in half. I had enough oyster liquid and so did not need the chicken stock.



\newpage
\recipe{Persimmon Pudding}

\ingredients{2 cup persimmon pulp $\approx$ pulp from 5 or 6 Japanese-hybrid  persimmons;
3 eggs beaten;
1\half cups sugar;
2 cups whole milk;
2 cups unsifted all-purpose flour;
1 teaspoon baking soda;
1 teaspoon salt.}

\textsc{Around 1967 my father, Floyd Goggans, planted a dwarf Tanenashi
persimmon tree in the yard of the house at Kuderna Acres.  Every
year the persimmons ripened just in time to make persimmon pudding
for Thanksgiving dinner. My mother,  Mallette Goggans, often served this pudding with
a hard sauce made with bourbon.  I use the hard sauce recipe from
Joy of Cooking made with rye whisky and cream. Be sure to use
truly ripe persimmons.  A ripe persimmon is very soft. Its pulp is
like a thick gel and contains no solid bits.}

In mixing bowl combine all ingredients stirring until blended.

Pour mixture into a 8 cup metal mold. The pudding will
rise when cooked so do not completely fill the mold.

Place uncovered mold inside a large covered pot.

Pour boiling water in the pot to slightly below the top of the mold.

Adjust the stove top so that the water in the pot simmers slowly.

Cover the pot and then steam the pudding until it has a cake-pudding
texture.  This step will take around 3 hours.

Remove pudding and cool in refrigerator.



\chapter{ Herb and Spice Blends}

\recipe{Herb Rub for Pork, Lamb, or Beef}

\ingredients{2 tablespoons chopped fresh basil or 2 teaspoons dried;
2 teaspoons chopped fresh thyme or 1 teaspoons dried;
1 tablespoons chopped fresh rosemary or 2 teaspoons dried;
1 tablespoon fennel seeds;
2 teaspoons whole coriander or 1 teaspoon ground coriander;
2 teaspoons garlic powder or granulated garlic;
2 tablespoons salt;
1 tablespoon whole black pepper or 2 teaspoons ground black pepper.}

Add ingredients to work bowl of an electric spice grinder or  mini food
processor and grind until a course powder is obtained. Store sealed in the
freezer if any fresh herbs are used.

Yields \half cup.

Source: Bruce Aidells and Denis Kelly from The Complete Meat Cookbook.

\newpage
\recipe{Curry Powder}
\label{CurryPowder}
\ingredients{2 teaspoons cumin seeds; 
2 teaspoons cardamon seeds;
2 teaspoons coriander seeds;
4 teaspoons ground turmeric;
1 teaspoon dry mustard;
\quarter teaspoon cayenne.}

Toast the cumin, cardamon, and coriander seeds in a small, dry skillet over medium-low heat until the seeds are lightly browned and fragrant, 2 to 3 minutes. Transfer to a bowl and let cool completely.

Add the turmeric, mustard powder, and cayenne and mix to combine.

Grind the spices in a spice grinder or blade-type coffee-grinder reserved for grinding spices.

Store in an airtight container for up to 2 months.

Source: Donald Link from Down South.



\newpage
\recipe{Creole Seasoning}
\label{CreoleSeasoning}
\ingredients{\third cup table salt;
\quarter cup granulated garlic;
\quarter cup freshly ground black pepper;
2 tablespoons cayenne pepper;
2 tablespoons dried thyme;
2 tablespoons dried basil;
2 tablespoons dried oregano;
\third cup paprika;
3 tablespoons granulated onion.}

Thoroughly combine all ingredients in a mixing bowl, and pour the mixture into a large glass jar.  Seal the jar so that it is airtight.

\newpage
\recipe{Coffee-Chile Dry Rub for Steaks}

\ingredients{2 teaspoons finely ground dark-roast coffee;
2 teaspoons ancho chile powder;
2 teaspoons dark brown sugar, tightly packed;
1 teaspoon smoked paprika;
1 teaspoon kosher salt;
\half teaspoon ground cumin.}


In a small bowl, mix all the ingredients thoroughly, massaging the mixture with your fingers to break down the dark brown sugar into fine crystals.

Liberally sprinkle a thin layer of the rub onto the steak, then pat it in with your fingers so it adheres.

Source: Matt Lee And Ted Lee from the New York Times, modified to make a reasonable quantity for 1 or 2 steaks.

\newpage
\recipe{Baking Powder}

\ingredients{Cream of tartar; Baking soda}

Mix two parts cream of tartar with one part baking soda.

\chapter{  Drinks}

\recipe{Caipirinha}
\label{Caipirinha}

\ingredients{3 ounces Cacha\c{c}a; 2 to 3 teaspoons white granulated sugar; 1 lime}

Prepare lime by washing it and then cutting a thin slice off of the stem and opposite end.  Cut the lime in half along the axis and place the cut side down on a cutting board. Cut each half into 9 pieces using a \#  cut pattern.  Place lime pieces in a large rocks glass and sprinkle with sugar. Mash the sugar and lime pieces together using a muddler. Pour over Cacha\c{c}a, stir, and fill with crushed ice or small cubes of ice.

\recipe{Rabo de Galo}
\label{RaboDeGalo}

\ingredients{2 ounces Cacha\c{c}a ; \half ounce Cynar; \half ounce Carpano Antica Formula or other sweet red vermouth; strip of orange zest.}

Measure liquid ingredients directly into a tumbler. Stir and add ice. Express orange zest over tumbler then drop in for garnish.

\newpage
\recipe{Premade Martini}
\label{Martini}

\ingredients{4 ounces (111 g) Beefeater gin; 1\half ounces (22 g) filtered water; \half ounce (14 g) Dolin dry vermouth.}

\textsc{This recipe makes two drinks. To make enough to fill a quart or liter bottle multiply amounts by five. The gram measurements assume the gin is 47\% ABV and the vermouth is 17.5\% ABV.}

Using a funnel combine all ingredients in a glass bottle. Shake to combine ingredients and then seal the bottle and store in the freezer overnight.

When mixture is fully chilled, pour 3 ounces into a cocktail glass and garnish with an olive or lemon twist.  



If there is ice in the bottle when you remove it from the freezer then give it a good shake and leave it out of the freezer for a minute or so before pouring.  

\recipe{Premade Manhattan}

\ingredients{11\quarter oz rye; 5 oz sweet vermouth; \half oz Angostura bitters; 7\half ounces water.}

Using a funnel combine all ingredients in a glass bottle. Shake to combine ingredients and then seal the bottle and store in the freezer overnight.

When mixture is fully chilled, pour 4\half ounces into a cocktail glass and garnish with a lemon twist.  



\recipe{Grapefruit Wine}
\label{GrapefruitWine}

\ingredients{1 bottle of dry French ros\'e; 1 pink grapefruit; simple syrup to taste.}

Juice the grapefruit and mix the juice with the wine.  Add a splash of simple syrup to taste, or not. Serve over ice, or with frozen supremed grapefruit segments to keep the drink icy cold.

\newpage
\recipe{Cherry Liqueur}
\label{CherryLiqueur}

\ingredients{2 pound Bing or other fresh cherries;
2 pound of sugar;
10 cups of good California brandy.} 

\textsc{If you start making this liqueur when fresh cherries are
available in the store it will be ready for Christmas.  This
recipe can also be made with vodka.}

Wash and stem the cherries and place in a towel to dry.

Pierce the cherries to the stone with a fork and place in a 2
quart wide mouth jar.

Pour the sugar over the cherries.

Pour the brandy (or vodka) over the sugar and cherries.  Cover
tightly with a lid and place out of the way.  It is not necessary
to shake or stir the brandy mixture.

Let the mixture stand without shaking or stirring for at least
3 months (6 months is better).  The sugar will be dissolved and
the liqueur will be a beautiful dark rose color.

Strain the liqueur into  bottles. Save the cherries to eat or use them
to garnish deserts.



\newpage
\recipe{Warm Springs Egg Nog}



\ingredients{6 eggs;
6 tablespoons sugar;
1 pint heavy cream;
6 tablespoons bourbon;
nutmeg for garnish.}

\textsc{A Goggans family Christmas tradition.  This egg nog is quite thick and
usually served with a spoon.  This recipe serves six.}

Separate the eggs.

Beat the egg whites to soft peaks.  Add the sugar and continue
to beat until stiff.

In a separate bowl, whip cream until stiff.

In another bowl, beat egg yolks lightly and add bourbon.

Fold the egg white mixture, whipped cream, and egg yolk mixture
together.

Serve in glasses garnished with nutmeg.



\newpage

\recipe{Original Chatham Artillery Punch}

\equilpment{punchbowl; alcoholic friends.}

\ingredients{8 lemons;
1 pound superfine sugar;
750-milliliter bottle bourbon or rye;
750-milliliter bottle Cognac;
750-milliliter bottle dark Jamaican rum;
3 bottles Champagne or other sparkling wine;
Nutmeg.}

Squeeze and strain the lemons to make 16 ounces of juice. Peel the lemons and muddle the peels with the sugar. Let the peels and sugar sit for an hour, then muddle again. Add the lemon juice and stir until sugar has dissolved. Strain out the peels.

Fill a 2- to 3-gallon punch bowl with crushed ice or ice cubes. Add the lemon-sugar mixture and the rum, the Cognac and the bourbon or rye. Stir and add the Champagne. Taste and adjust for sweetness. Grate nutmeg over the top and serve.

Yield: About 25 drinks.

Source: David Wondrich

\newpage

\recipe{Campari Boilermaker}

\ingredients{6 ounces citrus Gose, sour or Berliner Weisse; 2 ounces Campari; orange wedge garnish}

Combine beer and Campari in a 16 ounce glass, fill with ice, and top with orange wedge squeezed over glass.
\newpage

\recipe{Lime Cordial}

\ingredients{250 grams sugar;
    8 fluid ounces/240 ml hot water;
    1\half fluid ounces/45 ml fresh lime juice; 
    1\half fluid ounces/45 ml freshly grated lime peel; 
    1  fluid ounce/30 ml citric acid.}

Combine all of the ingredients in a blender.

Blend on medium speed for 30 seconds.
    
Strain with a fine strainer.

Bottle and refrigerate.

Recipe source: https://jeffreymorgenthaler.com/lime-cordial/

\newpage

\recipe{Chai Masala}
\ingredients{6 cups water;
20 grams English Breakfast tea, regular or decaffeinated;
40 grams fresh ginger, unpeeled but coarsely chopped;
25 grams turmeric, unpeeled but coarsely chopped; 
\half gram star anise;
1\half grams green cardamon pods;
3 grams Ceylon cinnamon stick;
\half gram whole black peppercorns;
2 cups whole milk;
2 tablespoons honey}

Coarsely chop the fresh ginger and turmeric. Crack star anise, cardamon pods, and peppercorns against a cutting board using the end of a wooden rolling pin. Break up the cinnamon stick by the same method or by hand.  

Add water, tea, ginger, turmeric, anise, cardamon, cinnamon and pepper to a large sauce pan. Bring to boil and then cook at a boil for 3 minutes. While the tea is boiling, bring milk to near boiling temperature in the microwave.  At the end of the 3 minutes, add the milk.  When the mixture returns to a boil cook at a low boil for an additional 2 minutes.  Remove from heat, add honey and stir to combine. Strain out the solids and serve.

Yield: 6 to 8 servings

\chapter{Distilling}

\recipe{Ted's Fast Fermenting Vodka}
\label{TFFV}

\ingredients{4 kg sugar;
250 g wheat bran;
1 Multivitamin tablet;
pinch Epsom salts;
\half teaspoon diamonioum phospate (DAP);
\half teaspoon Citric acid;
50 g bakers yeast.}

\textsc{This recipe makes 23 liters or 6.1 gallons.}

In a large pot (6  or more quarts) bring 3 quarts of water to the boil. Add the 250 g of bran, stir, bring back to the boil then simmer for 30 minutes stirring from time to time. When cooked the bran and water becomes a thin porridge.

Dissolve the sugar in warm water and add to the fermenter with cool water to bring quantity up to 20 litres. Add the crushed MV tab, Epsom salts and DAP.

Once the bran has simmered for 30 minutes, add it to the fermenter.

Adjust the pH to around 5 with the citric acid. This usually takes about \half teaspoon of citric acid.

Rehydrate the yeast in 100ml water and add to the mix. Making sure the start temp is below 30\deg C = 86 \deg F.

Stir well ( I use a stick blender to thoroughly aerate it.)

Leave plenty of headroom as this takes off like a rocket. I use a 30l fermenter
Do NOT seal for at least 24 hours as a thick foamy cap will form within an hour. I put it under an airlock only after 36 hrs so I can monitor progress.

This produces a fast ferment (normally dry to .990 within 3 to 4 days). The start specific gravity is around 1.070 so the yeast is not pushed hard to produce off-flavours and the resulting wash is approximately 10\% ABV.

Rack off and allow to stand for a couple of days to clear before distilling.

\newpage

\recipe{Gin \#2 - In Development}

\ingredients{3\half cups 45\% ABV neutral spirits; 0.5 gram strip of blood orange zest; 15 grams juniper berries; 4 grams coriander seeds} 

Pour neutral spirits into a 1-quart Mason jar. 

Carefully remove zest from orange being certain that no white pith is included. Trim zest strip until it weighs 0.5 grams.

Using a mortar and pestle thoroughly crush the juniper berries and coriander seeds. 

Add the orange zest, juniper berries and coriander seeds to the neutral spirits, place the lid and ring on the jar, and shake.  Allow the mixture to macerate for around 7 days shaking the jar occasionally.

To distill, remove and discard orange zest then add contents of Mason jar (including juniper berries and coriander seeds) to the Air Still.  Add an additional quart of water to the still. Turn on still and after around 25 minutes collect and discard the first 20 ml of distillate to emerge. After this collect the next approximately 750 ml of distillate. Stop collecting distillate when the measured ABV of the collected distillate is around 45\%. Bottle and wait at least one week before drinking to allow flavors to stabilize. Some distillers suggest that aging the gin several months improves the flavor. 

\newpage

\recipe{Gin \#3 - In Development}

\ingredients{3\half cups 45\% ABV neutral spirits; 0.15 gram dried bitter orange zest; 15 grams juniper berries; 3 grams coriander seeds; 1.5 grams angelica root.} 

Pour neutral spirits into a 1-quart Mason jar. 


Using a mortar and pestle thoroughly crush the juniper berries and coriander seeds. 

Add the dried orange zest, juniper berries, coriander seeds, and angelica root to the neutral spirits, place the lid and ring on the jar, and shake.  Allow the mixture to macerate for around 7 days shaking the jar occasionally.

To distill, add contents of Mason jar plus one quart of water to the Air Still.  Turn on still and after around 25 minutes collect and discard the first 20 ml of distillate to emerge. After this collect the next approximately 750 ml of distillate. Stop collecting distillate when the measured ABV of the collected distillate is around 45\%. Bottle and wait at least one week before drinking to allow flavors to stabilize. Some distillers suggest that aging the gin several months improves the flavor. 

\recipe{Gin \#4 - In Development}

\ingredients{
750 ml 50\% ABV neutral spirits or vodka;
12 grams juniper berries;
3 grams coriander seeds; 
0.5 grams cardamom; 
0.75 grams angelica root;
8 grams orange zest
} 

Pour neutral spirits into a 1-quart Mason jar. 

Do not crush any of the ingredients. Add the  juniper berries, coriander seeds, angelica root, and cardamon to the neutral spirits, place the lid and ring on the jar, and shake.  Allow the mixture to macerate over night shaking the jar occasionally.

To distill, add contents of Mason jar plus one cup of water to the Air Still. Zest orange and add orange zest strips wrapped in cheese cloth to the gin basket.  Turn on still and after around 25 minutes collect and discard the first 17 ml of distillate to emerge. After this collect the next 375 ml of distillate for use in making the gin. Dilute the collected distillate with purified water to 44 \% ABV.

With this amount of orange zest your gin may louch. To correct louching, add 44 \% ABV neurtal spirits to your gin until it is clear.

Comment: For this recipe I used purchased 50 \% ABV vodka. The recipe yielded excellent if very orange forward gin after it was diluted to correct louching. The juniper flavor was not as harsh as previous recipes and I believe that is because the berries were not crushed.

\recipe{Gin \#5 - In Development}

\ingredients{
1 liter 40\% ABV neutral spirits or vodka;
25 grams juniper berries;
5 grams coriander seeds; 
1 gram angelica root;
0.2 grams cardamom; 
0.1 grams vanilla bean;
0.5 grams fresh thyme leaves;
2 grams hazelnut;
35 grams pear, peeled and seeded;
1 gram fresh rose petals;
2 grams orange zest
} 

Do not crush any of the ingredients. Add all ingredients except the rose petals and orange zest to the still.  Zest orange and add orange zest strips and rose petals wrapped in cheese cloth to the gin basket.  Turn on still and after around 25 minutes collect and discard the first 15 ml of distillate to emerge. After this collect the next 425 ml of distillate for use in making the gin. Dilute the collected distillate with purified water to 44 \% ABV. The result should be approximately 750 ml of gin.


Comment: For this recipe I used purchased 40 \% ABV vodka. The recipe yielded a complex gin but it had too much juniper flavor and not enough orange flavor to suit me.
I added 44 \% ABV neurtal spirits to the gin to reduce the juniper flavor but in the next iteration I will use 12 grams of juniper berries and 4 grams of orange zest.


\recipe{Gin \#6 - In Development}

\ingredients{
1 liter 40\% ABV neutral spirits or vodka;
14 grams juniper berries;
5 grams coriander seeds; 
1 gram angelica root;
0.5 grams cardamom; 
0.2 grams vanilla bean;
0.5 grams dried thyme leaves;
2 pecan halves;
4 dried Mulberries;
2 gram fresh rose petals;
4 grams orange zest
} 

Do not crush any of the ingredients. Add all ingredients except the rose petals and orange zest to the still.  Zest orange and add orange zest strips and rose petals wrapped in cheese cloth to the gin basket.  Turn on still and after around 25 minutes collect and discard the first 15 ml of distillate to emerge. After this collect the next 425 ml of distillate for use in making the gin. Dilute the collected distillate with purified water to 44 \% ABV. The result should be approximately 750 ml of gin.

\chapter{  More of an Idea than a Recipe}

\recipe{Fried Okra} 

\ingredients{Cut okra;
egg;
corn meal;
fat for frying;
salt.}

Put cut okra in a bowl. Crack one egg over the okra and then toss to coat. Add corn meal and toss.  The cut ends of the okra should be completely coated with corn meal but some green should be visible on the sides. Deep fat fry in an iron skillet. When brown remove from skillet with a slotted spoon and drain on paper towels. Season with salt and then serve immediately.

\recipe{Whipped Sour Cream}

\ingredients{sour cream.}

Whip the sour cream just as you would whip sweet cream.  Use as a topping for deserts.

\recipe{Whipped Bacon Fat}

\ingredients{well rendered bacon-fat; herbs or your choice.}

Heat bacon fat until it is in liquid form and then infuse it with herbs or salt and pepper. Take it off the heat and as it's cooling, pour it into a stand mixer and keep whipping it into a lardo-type texture. You can spread it on bread, or make a 50/50 ratio of whipped butter to whipped lardo and use it in place of whipped butter.

\recipe{Soft or Hard Cooked Eggs}

\ingredients{eggs; water.}

The trick to cooking eggs in their shells is to steam rather than boil them. Start with a sauce pan with a tightly fitting lid. The pan should be large enough to hold the number of eggs you want to cook in a single layer.  Add one half inch of water to the pan, bring to a boil and then reduce to a simmer.  Add eggs to pan and cover.  Cook for 6 minutes for soft cooked eggs or 12 minutes for hard cooked eggs.

\recipe{Oven Cooked Bacon}

\ingredients{sliced bacon.}

Preheat oven to 425\deg F. Place bacon strips on a foil covered half sheet pan. Cook in oven for 10 minutes then rotate pan and continue cooking until crisp, 5 to 10 minutes for thin cut or 10 to 15 minutes for thick cut. Remove bacon from oven and drain on paper towels

Yields: Up to as many strips as can be placed on a half sheet pan.



\recipe{Brown Sugar}
\label{BrownSugar}

\ingredients{1 cup white granulated sugar; 1 or 2 tablespoons molasses.}

Combine molasses and white sugar in a small bowl. Use 1 tablespoon of molasses for light brown sugar and 2 tablespoons for dark brown sugar. Mix ingredients with a fork until a uniform mixture is obtained.


\recipe{Warm Bacon-Fat Vinaigrette}
\label{BaconFatVinaigrette}

\ingredients{Well rendered bacon fat;
sherry vinegar.}

Warm the bacon fat until it melts to liquid form and is clear but is not too hot to touch.  Whisk in a splash of sherry vinegar and use to dress an asparagus or other vegetable salad.

\recipe{Miso Mayonnaise}

\ingredients{1 cup mayonnaise; 3 tablespoons white miso; a splash of mirin; a splash of soy sauce; a splash of sambal or chile-garlic sauce.}

 Whisked together ingredients.  The result should be a little less than thick.
 
 \recipe{Roasted and Peeled Sweet or Chili Peppers - To Include}
\textsc{To include}
 
 
\chapter{  Non-Food Recipes}

\recipe{Isotonic Saline Solution}

\ingredients{0.5 liter = 1 pint = 2 cups water;
5 ml = 1 teaspoon salt;
2.5 ml = \half teaspoon baking soda.}

Do not use iodized salt or salt with anti-caking agents.

The water should be boiled and then cooled or bottled water should be used.

I usually make this by boiling the water in the microwave and then adding
the salt and soda to the just off boiling water.

Before using for nasal irrigation let the solution cool to room temperature.

\recipe{Glass Cleaner}

\equilpment{1 gallon plastic jug; funnel.}

\ingredients{\half cup sudsy ammonia; 2 cups rubbing alcohol; 1 teaspoon dish detergent; water to fill; blue food coloring.}

Using a funnel, add all ingredients to the jug except the water and food coloring.  
Slowly add water to fill and then add the food coloring a drop at at time until the desired color is obtained.

\newpage
\recipe{Polishing Silver}

\ingredients{1 gallon water, near boiling;
\half cup baking soda;
disposable aluminum casserole dish;
aluminum foil.}

Add baking soda to aluminum casserole dish and then add near boiling water.  Stir until baking soda is dissolved and then add silver pieces. Wearing rubber gloves, scrub silver pieces with aluminum foil as necessary then rinse and dry.  Add additional silver pieces to dish and repeat steps. This process is less effective as the solution cools. If you will be cleaning a large number of pieces then you can put the aluminum casserole dish in a cast iron casserole dish placed on a cooktop to maintain a solution temperature just under boiling. Discard the aluminum casserole dish when finished.

\recipe{Cleaning Coffee Maker}

\ingredients{10 grams of either baking soda or sodium carbonate.}

If using baking soda first convert it to sodium carbonate by baking it in a 200\deg F oven for one hour.  Mix the sodium carbonate with 1 liter of water and pour into coffee maker's reservoir. Turn on the machine and complete the brew cycle.  After the cycle is complete, rinse all machine components that come into contact with the solution and then repeat the brew cycle twice with clean water.

\recipe{Season a Cast Iron Pan}

\ingredients{Unseasoned or stripped cast iron pan; flaxseed oil.}

If the pan has been previously seasoned then strip it of seasoning by running it through the self-cleaning cycle of a self-cleaning oven. 

To begin seasoning, warm the unseasoned pan for 15 minutes in a 200\deg F oven. 

Remove the pan from the oven.  Place 1 tablespoon of flaxseed oil in the pan and, using tongs, rub the oil into the surface with paper towels. With fresh paper towels, thoroughly wipe out the pan to remove the excess oil. 

Place the oiled pan upside down in a cold oven, then set the oven to its maximum baking temperature. Once the oven reaches its maximum temperature, heat the pan for one hour.  Turn off the oven and cool the pan in the oven for at least two hours.

\recipe{Bubble Stuff for Giant Bubbles}

\ingredients{1 cup Joy or Dawn dishwashing liquid;
4 tablespoons glycerine (available at most drugstores);
10 to 15 cups cold water.}

Add 10 cups of water to a large pail and then pour in dishwashing liquid.  Add glycerine and then stir gently to avoid creating froth on top of the solution.  When solution is completely mixed, skim off any froth with your hand. On very dry days it may be necessary to add up to 5 cups of additional water to make truly huge bubbles.

\recipe{Four Wire Round Braid}

\ingredients{4 wires of different colors.}

Start with four wires side by side.
\begin{enumerate}
\item Take the leftmost wire and cross it OVER its TWO neighbors. 
\item Take the rightmost wire and cross it OVER its NEXT neighbor.
\item Take the leftmost wire and cross it UNDER its TWO neighbors.
\item Take the rightmost wire and cross it UNDER its NEXT neighbor.
\end{enumerate}
Repeat steps 1-4 until done.

\recipe{Fruit Fly Trap}

\ingredients{\quarter cup hot water; 1 tablespoon honey; \half cup apple cider vinegar; 1 teaspoon dishwashing liquid (unscented is best but not necessary)}

In a small bowl combine hot water and honey. Stir with a small whisk until the honey is completely dissolved. Add the cider vinegar and incorporate.  Add dishwashing liquid and whisk slowly to avoid making bubbles.  Add the solution to a small water glass and set on the counter in a area where fruit flies are a problem.

\end{document}

%%%%%%%%%%%%%%%%%%%%%%%%%%%%%%%%%%%%%%%%%%%% Recipe Template
\newpage
\recipe{Template}
\label{Template}

\textsc{Text describing and commenting on recipe.}

\equilpment{Pots; pans.}

\ingredients{Usually nothing is here unless it is used in all components.\\
\\
\textsc{Component 1 --}
2 jalape\~{n}os, seeded or not, thinly sliced;
1 gallon water.\\
\\
\textsc{Component 2 --}
\threequarters cup vegetable oil;
2 cups chopped onions
\half cup chopped red bell peppers;
salt; freshly ground pepper; ground cayenne; hot sauce; flat-leaf parsley, chopped;
green onions, thinly sliced.
}

Put a parchment paper sling in a 6 to 8 cup p\^{a}t\'{e}  mold. Pur\'{e}e with an immersion blender until smooth. Bake in a 350\deg F oven for one hour. Refer to another recipe with \hyperref[label]{in-line text}.


%%%%%%%%%%%%%%%%%%%%%%%%%%%%%%%%%%%%%%%%%%%%%%%%%%%%%%%%%%%%%%


\chapter{ Photographs}

\begin{figure}
\includegraphics[width=\textwidth]{PearAlmondTartPhoto.pdf}
\caption{Pear Almond Tart.}
\label{fig:PearAlmondTart}
\end{figure}

\begin{figure}
 \includegraphics[width=\textwidth]{PorkCollardPeaGumboPhoto.pdf}
\caption{Smoked Pulled Pork, Collard Greens, and Blackeye Peas Gumbo.}
\label{fig:PorkCollardPeaGumbo}
\end{figure}

\begin{figure}
 \includegraphics[width=\textwidth]{BrownRouxPhoto.pdf}
\caption{Brown Roux.}
\label{fig:BrownRoux}
\end{figure}

\begin{figure}
 \includegraphics[width=\textwidth]{OysterDressingPhoto.pdf}
\caption{Oyster Dressing.}
\label{fig:OysterDressing}
\end{figure}

\end{document}


%%%%%%%%%%%%%%%%%%%%%%%%%%%%%%%%%%%%%%%%%%%%%%%%%%%%%%%%%%%%%%%%

List of Recipes to Include

\newpage
\recipe{Pork Tenderloin Steak}

\ingredients{2 pork tenderloins;
more stuff}

Instructions go here.

Source: The Meat Cookbook

\newpage
\recipe{Bar-B-Que Rub}

\ingredients{1 cup salt;
2 tablespoons + 2 teaspoons freshly ground black pepper;
2 tablespoons + 2 teaspoons lemon pepper;
2 tablespoons + 2 teaspoons cayenne pepper;
2 tablespoons + 2 teaspoons chili powder;
2 tablespoons + 2 teaspoons dry mustard;
2 tablespoons + 2 teaspoons dark or light brown sugar;
1 tablespoons + 1 teaspoons garlic powder;
\half teaspoon ground cinnamon;
\half teaspoon Accent (optional).}

Combine all ingredients in a bowl and stir well to mix.

Use immediately or store in a glass jar in a cool dark place for several months. Yields 2 cups of rub.

Source: John Willingham from John Willingham's World Champion Bar-B-Q.



%  Junk Pile
%
%
\recipe{Simple Sugar Wash for Distilling}
\label{SimpleWash}
\equilpment{9.5 liter stainless steel stock pot; long handled stainless steel spoon; 12 liter (3 gallon) plastic fermenter with lid and airlock; triple-scale hydrometer; Thermapen digital thermometer; 2 liter (8 cup) pyrex measuring cup; no-wash sanitizing product.}

\ingredients{1.75 kg sugar;
1.5 g citric acid;
1.8 g diamonioum phospate (DAP);
1.3 g food-grade gypsum (calcium sulphate);
150 mg Epsom salts;
16.8 g of distillers yeast or \quarter cup of dried bakers yeast.
}

\textsc{I found this recipe on the Home Distillers Forums under the name ``Wineos Plain Ol Sugar Wash.'' I have scaled the original recipe so that it works in a 12 liter (3 gallon) fermenter.  The target amount of sugar wash is 10.25 liters.  This volume should yield 8 liters of clear wash for distilling.} 

Mark 10.25 liters on the outside of your fermenter. Sanitize your fermenter and top, airlock, stock pot, hydrometer, thermometer probe, measuring cup and spoon before beginning.

In a stainless steel stock pot, bring around  4 liters of water to a simmer then dissolve the sugar in the hot water.  Mix well with a stainless or plastic spoon until the sugar is completely dissolved. In a drinking glass, add hot water and citric acid. Stir the solution until the citric acid is completely dissolved and then add to the sugar solution. Repeat this procedure with the diamonioum phospate, gypsum, and Epsom salts, individually. Stir the sugar solution until all the additions are incorporated.

Add the hot water solution to fermenter. Add cool water (and ice if you are in a hurry -- the ice should all melt) to make the total volume close to but under 10 liters. Using a hydrometer test the specific gravity (be sure to correct for the temperature of the solution). The specific gravity should be above 1.080. Add additional cold water to obtain a  specific gravity between 1.070 and 1.080. After fermentation this will yield between 9.5\% and 11\% alcohol by volume.

Check the temperature of the wash and once it is at or less than 95\deg F sprinkle the yeast on top. After 15-20 minutes, give the wash a good stir to incorporate the yeast and add some air to the wash. Place the cover on the fermenter and add water to the airlock. Place in a storage room or other similar out of the way place at warm room temperature.

Depending on the fermenting temperature this will work off in a week or two. It is important not to rush the fermentation. Let it finish fermenting to a specific gravity of 0.990 or less and then give it another week to clear before racking and running it.  It is important to rack the clear wash off the yeast sediment and only place the clear wash in the still so that a neutral and flavorless spirit is obtained.

% Citric acid 9.9g/T; Gypsum 8.7g/T; DAP 11.9g/T; Red Star distillers yeast 37.1g/\quarter cup; Epson Salts 13.7g/T

% My fermenter: Total Volume = 412 fluid oz = 3.2 gal; Volume below valve = 76 fluid ounces = 0.594 gal = 2.25 l;

% Desired amount of clear wash = 8 l. Target amount of wash to make = 10.25 l = 2.71 gal

